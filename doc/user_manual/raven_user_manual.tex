%
% This is an example LaTeX file which uses the SANDreport class file.
% It shows how a SAND report should be formatted, what sections and
% elements it should contain, and how to use the SANDreport class.
% It uses the LaTeX article class, but not the strict option.
% ItINLreport uses .eps logos and files to show how pdflatex can be used
%
% Get the latest version of the class file and more at
%    http://www.cs.sandia.gov/~rolf/SANDreport
%
% This file and the SANDreport.cls file are based on information
% contained in "Guide to Preparing {SAND} Reports", Sand98-0730, edited
% by Tamara K. Locke, and the newer "Guide to Preparing SAND Reports and
% Other Communication Products", SAND2002-2068P.
% Please send corrections and suggestions for improvements to
% Rolf Riesen, Org. 9223, MS 1110, rolf@cs.sandia.gov
%
\documentclass[pdf,12pt]{INLreport}
% pslatex is really old (1994).  It attempts to merge the times and mathptm packages.
% My opinion is that it produces a really bad looking math font.  So why are we using it?
% If you just want to change the text font, you should just \usepackage{times}.
% \usepackage{pslatex}
\usepackage{times}
\usepackage{longtable}
\usepackage[FIGBOTCAP,normal,bf,tight]{subfigure}
\usepackage{amsmath}
\usepackage{amssymb}
\usepackage[labelfont=bf]{caption}
\usepackage{pifont}
\usepackage{enumerate}
\usepackage{listings}
\usepackage{fullpage}
\usepackage{xcolor}          % Using xcolor for more robust color specification
\usepackage{ifthen}          % For simple checking in newcommand blocks
\usepackage{textcomp}
%\usepackage{authblk}         % For making the author list look prettier
%\renewcommand\Authsep{,~\,}

% Custom colors
\definecolor{deepblue}{rgb}{0,0,0.5}
\definecolor{deepred}{rgb}{0.6,0,0}
\definecolor{deepgreen}{rgb}{0,0.5,0}
\definecolor{forestgreen}{RGB}{34,139,34}
\definecolor{orangered}{RGB}{239,134,64}
\definecolor{darkblue}{rgb}{0.0,0.0,0.6}
\definecolor{gray}{rgb}{0.4,0.4,0.4}

\lstset {
  basicstyle=\ttfamily,
  frame=single
}

\setcounter{secnumdepth}{5}
\lstdefinestyle{XML} {
    language=XML,
    extendedchars=true,
    breaklines=true,
    breakatwhitespace=true,
%    emph={name,dim,interactive,overwrite},
    emphstyle=\color{red},
    basicstyle=\ttfamily,
%    columns=fullflexible,
    commentstyle=\color{gray}\upshape,
    morestring=[b]",
    morecomment=[s]{<?}{?>},
    morecomment=[s][\color{forestgreen}]{<!--}{-->},
    keywordstyle=\color{cyan},
    stringstyle=\ttfamily\color{black},
    tagstyle=\color{darkblue}\bf\ttfamily,
    morekeywords={name,type},
%    morekeywords={name,attribute,source,variables,version,type,release,x,z,y,xlabel,ylabel,how,text,param1,param2,color,label},
}
\lstset{language=python,upquote=true}

\usepackage{titlesec}
\newcommand{\sectionbreak}{\clearpage}
\setcounter{secnumdepth}{4}

%\titleformat{\paragraph}
%{\normalfont\normalsize\bfseries}{\theparagraph}{1em}{}
%\titlespacing*{\paragraph}
%{0pt}{3.25ex plus 1ex minus .2ex}{1.5ex plus .2ex}

%%%%%%%% Begin comands definition to input python code into document
\usepackage[utf8]{inputenc}

% Default fixed font does not support bold face
\DeclareFixedFont{\ttb}{T1}{txtt}{bx}{n}{9} % for bold
\DeclareFixedFont{\ttm}{T1}{txtt}{m}{n}{9}  % for normal

\usepackage{listings}

% Python style for highlighting
\newcommand\pythonstyle{\lstset{
language=Python,
basicstyle=\ttm,
otherkeywords={self, none, return},             % Add keywords here
keywordstyle=\ttb\color{deepblue},
emph={MyClass,__init__},          % Custom highlighting
emphstyle=\ttb\color{deepred},    % Custom highlighting style
stringstyle=\color{deepgreen},
frame=tb,                         % Any extra options here
showstringspaces=false            %
}}


% Python environment
\lstnewenvironment{python}[1][]
{
\pythonstyle
\lstset{#1}
}
{}

% Python for external files
\newcommand\pythonexternal[2][]{{
\pythonstyle
\lstinputlisting[#1]{#2}}}

\lstnewenvironment{xml}
{}
{}

% Python for inline
\newcommand\pythoninline[1]{{\pythonstyle\lstinline!#1!}}

% Named Colors for the comments below (Attempted to match git symbol colors)
\definecolor{RScolor}{HTML}{8EB361}  % Sonat (adjusted for clarity)
\definecolor{DPMcolor}{HTML}{E28B8D} % Dan
\definecolor{JCcolor}{HTML}{82A8D9}  % Josh (adjusted for clarity)
\definecolor{AAcolor}{HTML}{8D7F44}  % Andrea
\definecolor{CRcolor}{HTML}{AC39CE}  % Cristian
\definecolor{RKcolor}{HTML}{3ECC8D}  % Bob (adjusted for clarity)
\definecolor{DMcolor}{HTML}{276605}  % Diego (adjusted for clarity)
\definecolor{PTcolor}{HTML}{990000}  % Paul

%\def\DRAFT{} % Uncomment this if you want to see the notes people have been adding
% Comment command for developers (Should only be used under active development)
\ifdefined\DRAFT
  \newcommand{\nameLabeler}[3]{\textcolor{#2}{[[#1: #3]]}}
\else
  \newcommand{\nameLabeler}[3]{}
\fi
\newcommand{\alfoa}[1] {\nameLabeler{Andrea}{AAcolor}{#1}}
\newcommand{\cristr}[1] {\nameLabeler{Cristian}{CRcolor}{#1}}
\newcommand{\mandd}[1] {\nameLabeler{Diego}{DMcolor}{#1}}
\newcommand{\maljdan}[1] {\nameLabeler{Dan}{DPMcolor}{#1}}
\newcommand{\cogljj}[1] {\nameLabeler{Josh}{JCcolor}{#1}}
\newcommand{\bobk}[1] {\nameLabeler{Bob}{RKcolor}{#1}}
\newcommand{\senrs}[1] {\nameLabeler{Sonat}{RScolor}{#1}}
\newcommand{\talbpaul}[1] {\nameLabeler{Paul}{PTcolor}{#1}}
% Commands for making the LaTeX a bit more uniform and cleaner
\newcommand{\TODO}[1]    {\textcolor{red}{\textit{(#1)}}}
\newcommand{\xmlAttrRequired}[1] {\textcolor{red}{\textbf{\texttt{#1}}}}
\newcommand{\xmlAttr}[1] {\textcolor{cyan}{\textbf{\texttt{#1}}}}
\newcommand{\xmlNodeRequired}[1] {\textcolor{deepblue}{\textbf{\texttt{<#1>}}}}
\newcommand{\xmlNode}[1] {\textcolor{darkblue}{\textbf{\texttt{<#1>}}}}
\newcommand{\xmlString}[1] {\textcolor{black}{\textbf{\texttt{'#1'}}}}
\newcommand{\xmlDesc}[1] {\textbf{\textit{#1}}} % Maybe a misnomer, but I am
                                                % using this to detail the data
                                                % type and necessity of an XML
                                                % node or attribute,
                                                % xmlDesc = XML description
\newcommand{\default}[1]{~\\*\textit{Default: #1}}
\newcommand{\nb} {\textcolor{deepgreen}{\textbf{~Note:}}~}

%%%%%%%% End comands definition to input python code into document

%\usepackage[dvips,light,first,bottomafter]{draftcopy}
%\draftcopyName{Sample, contains no OUO}{70}
%\draftcopyName{Draft}{300}

% The bm package provides \bm for bold math fonts.  Apparently
% \boldsymbol, which I used to always use, is now considered
% obsolete.  Also, \boldsymbol doesn't even seem to work with
% the fonts used in this particular document...
\usepackage{bm}

% Define tensors to be in bold math font.
\newcommand{\tensor}[1]{{\bm{#1}}}

% Override the formatting used by \vec.  Instead of a little arrow
% over the letter, this creates a bold character.
\renewcommand{\vec}{\bm}

% Define unit vector notation.  If you don't override the
% behavior of \vec, you probably want to use the second one.
\newcommand{\unit}[1]{\hat{\bm{#1}}}
% \newcommand{\unit}[1]{\hat{#1}}

% Use this to refer to a single component of a unit vector.
\newcommand{\scalarunit}[1]{\hat{#1}}

% \toprule, \midrule, \bottomrule for tables
\usepackage{booktabs}

% \llbracket, \rrbracket
\usepackage{stmaryrd}

\usepackage{hyperref}
\hypersetup{
    colorlinks,
    citecolor=black,
    filecolor=black,
    linkcolor=black,
    urlcolor=black
}

\newcommand{\wiki}{\href{https://github.com/idaholab/raven/wiki}{RAVEN wiki}}

% Compress lists of citations like [33,34,35,36,37] to [33-37]
\usepackage{cite}

% If you want to relax some of the SAND98-0730 requirements, use the "relax"
% option. It adds spaces and boldface in the table of contents, and does not
% force the page layout sizes.
% e.g. \documentclass[relax,12pt]{SANDreport}
%
% You can also use the "strict" option, which applies even more of the
% SAND98-0730 guidelines. It gets rid of section numbers which are often
% useful; e.g. \documentclass[strict]{SANDreport}

% The INLreport class uses \flushbottom formatting by default (since
% it's intended to be two-sided document).  \flushbottom causes
% additional space to be inserted both before and after paragraphs so
% that no matter how much text is actually available, it fills up the
% page from top to bottom.  My feeling is that \raggedbottom looks much
% better, primarily because most people will view the report
% electronically and not in a two-sided printed format where some argue
% \raggedbottom looks worse.  If we really want to have the original
% behavior, we can comment out this line...
\raggedbottom
\setcounter{secnumdepth}{5} % show 5 levels of subsection
\setcounter{tocdepth}{5} % include 5 levels of subsection in table of contents

% ---------------------------------------------------------------------------- %
%
% Set the title, author, and date
%
\title{RAVEN User Manual}
%\author{%
%\begin{tabular}{c} Author 1 \\ University1 \\ Mail1 \\ \\
%Author 3 \\ University3 \\ Mail3 \end{tabular} \and
%\begin{tabular}{c} Author 2 \\ University2 \\ Mail2 \\ \\
%Author 4 \\ University4 \\ Mail4\\
%\end{tabular} }


\author{
\textbf{\textit{Project Manager:}}
 \\Cristian Rabiti\\
 \textbf{\textit{Principal Investigator and Technical Leader:}}
 \\Andrea Alfonsi\\
\textbf{\textit{Main Developers:}}
\\Andrea Alfonsi
\\Diego Mandelli
\\Joshua Cogliati
\\Congjian Wang
\\Paul W. Talbot
\\Daniel P. Maljovec
\\Robert Kinoshita
\\Mohammad G. Abdo\\
\textbf{\textit{Former Developers:}} \\Sonat Sen
\\Jun Chen\\
\textbf{\textit{Contributors:}}
\\Alessandro Bandini (Post-Processor)
\\Ivan Rinaldi (documentation)
\\Claudia Picoco (new external code interface)
\\James B. Tompkins (new external code interface)
\\Matteo Donorio (new external code interface)
\\Fabio Giannetti (new external code interface)
\\Jia Zhou (conjugate gradient optimizer)
}
% \\James B. Tompkins}   Just people who actually ``developed'' a significant capability in the code should be placed here. Andrea
%\author{\textbf{\textit{Main Developers:}}  \\Andrea Alfonsi}
%\author{\\Joshua Cogliati}
%\author{\\Diego Mandelli}
%\author{\\Robert Kinoshita}
%\author{\\Sonat Sen}
%
%\author{Cristian Rabiti}
%\author{Andrea Alfonsi}
%\author{Joshua Cogliati}
%\author{Diego Mandelli}
%\author{Robert Kinoshita}
%\author{Sonat Sen}
%\affil{Idaho National Laboratory, Idaho Falls, ID 83402}
%\\\{cristian.rabiti, andrea.alfonsi, joshua.cogliati, diego.mandelli, robert.kinoshita, ramazan.sen\}@inl.gov}

% There is a "Printed" date on the title page of a SAND report, so
% the generic \date should [WorkingDir:]generally be empty.
\date{}


% ---------------------------------------------------------------------------- %
% Set some things we need for SAND reports. These are mandatory
%
\SANDnum{INL/EXT-15-34123}
\SANDprintDate{\today}
\SANDauthor{Cristian Rabiti, Andrea Alfonsi, Joshua Cogliati, Diego Mandelli,
Robert Kinoshita, Sonat Sen, Congjian Wang, Paul W. Talbot, Daniel P. Maljovec}
\SANDreleaseType{Revision 7}

% ---------------------------------------------------------------------------- %
% Include the markings required for your SAND report. The default is "Unlimited
% Release". You may have to edit the file included here, or create your own
% (see the examples provided).
%
% \include{MarkOUO} % Not needed for unlimted release reports

\def\component#1{\texttt{#1}}

% ---------------------------------------------------------------------------- %
\newcommand{\systemtau}{\tensor{\tau}_{\!\text{SUPG}}}

% Added by Sonat
\usepackage{placeins}
\usepackage{array}

\newcolumntype{L}[1]{>{\raggedright\let\newline\\\arraybackslash\hspace{0pt}}m{#1}}
\newcolumntype{C}[1]{>{\centering\let\newline\\\arraybackslash\hspace{0pt}}m{#1}}
\newcolumntype{R}[1]{>{\raggedleft\let\newline\\\arraybackslash\hspace{0pt}}m{#1}}

% end added by Sonat
% ---------------------------------------------------------------------------- %
%
% Start the document
%

\begin{document}
    \maketitle

    % ------------------------------------------------------------------------ %
    % An Abstract is required for SAND reports
    %
%    \begin{abstract}
%    \input abstract
%    \end{abstract}


    % ------------------------------------------------------------------------ %
    % An Acknowledgement section is optional but important, if someone made
    % contributions or helped beyond the normal part of a work assignment.
    % Use \section* since we don't want it in the table of context
    %
%    \clearpage
%    \section*{Acknowledgment}



%	The format of this report is based on information found
%	in~\cite{Sand98-0730}.


    % ------------------------------------------------------------------------ %
    % The table of contents and list of figures and tables
    % Comment out \listoffigures and \listoftables if there are no
    % figures or tables. Make sure this starts on an odd numbered page
    %
    \cleardoublepage		% TOC needs to start on an odd page
    \tableofcontents
    %\listoffigures
    %\listoftables


    % ---------------------------------------------------------------------- %
    % An optional preface or Foreword
%    \clearpage
%    \section*{Preface}
%    \addcontentsline{toc}{section}{Preface}
%	Although muggles usually have only limited experience with
%	magic, and many even dispute its existence, it is worthwhile
%	to be open minded and explore the possibilities.


    % ---------------------------------------------------------------------- %
    % An optional executive summary
    %\clearpage
    %\section*{Summary}
    %\addcontentsline{toc}{section}{Summary}
    %\input{Summary.tex}
%	Once a certain level of mistrust and skepticism has
%	been overcome, magic finds many uses in todays science



%	and engineering. In this report we explain some of the
%	fundamental spells and instruments of magic and wizardry. We
%	then conclude with a few examples on how they can be used
%	in daily activities at national Laboratories.


    % ---------------------------------------------------------------------- %
    % An optional glossary. We don't want it to be numbered
%    \clearpage
%    \section*{Nomenclature}
%    \addcontentsline{toc}{section}{Nomenclature}
%    \begin{description}
%          \item[alohomoral]
%           spell to open locked doors and containers
%          \item[leviosa]
%           spell to levitate objects
%    \item[remembrall]
%           device to alert you that you have forgotten something
%    \item[wand]
%           device to execute spells
%    \end{description}


    % ---------------------------------------------------------------------- %
    % This is where the body of the report begins; usually with an Introduction
    %
    \SANDmain		% Start the main part of the report

\input{introduction.tex}
\input{nomenclature.tex}
\input{Installation/main.tex}
\input{HowToRun.tex}
\input{ravenStructure.tex}
\input{runInfo.tex}
\input{files.tex}
\input{variablegroups.tex}
\input{ProbabilityDistributions.tex}
\section{Samplers}
\label{sec:Samplers}

%%%%%%%%%%%%%%%%%%%%%%%%%%%%%%%%%%%%%%%%%%%%%%%%%%%%%%%%%%%%%%%%%%%%%%%%%%%%%%%%
% If you are confused by the input of this document, please make sure you see
% these defined commands first. There is no point writing the same thing over
% and over and over and over and over again, so these will help us reduce typos,
% by just editing a template sentence or paragraph.
\renewcommand{\nameDescription}
{
  \xmlAttr{name},
  \xmlDesc{required string attribute}, user-defined name of this sampler.
  \nb As with other objects, this identifier can be used to reference this
  specific entity from other input blocks in the XML.
}
\newcommand{\shapeVariableDescription}
{
  \xmlAttr{shape},
  \xmlDesc{comma-separated integers, optional field},
  determines the number of samples and shape of samples
  to be taken.  For example, \xmlAttr{shape}=``2,3'' will provide a 2 by 3
  matrix of values, while \xmlAttr{shape}=``10'' will produce a vector of 10 values.
  Omitting this optional attribute will result in a single scalar value instead.
  Each of the values in the matrix or vector will be the same as the single sampled value.
  \nb A model interface must be prepared to handle non-scalar inputs to use this option.
}
\newcommand{\shapeConstantDescription}
{
  \xmlAttr{shape},
  \xmlDesc{comma-separated integers, optional field},
  determines the shape of samples of the constant value.
  For example, \xmlAttr{shape}=``2,3'' will shape the values into a 2 by 3
  matrix, while \xmlAttr{shape}=``10'' will shape into a vector of 10 values.
  Unlike the \xmlNode{variable}, the constant requires each value be entered; the number
  of required values is equal to the product of the \xmlAttr{shape}.
  \nb A model interface must be prepared to handle non-scalar inputs to use this option.
}

\renewcommand{\specBlock}[2]
{
  The specifications of this sampler must be defined within #1 \xmlNode{#2} XML
  block.
}
\newcommand{\variableChildIntro}
{
 This \xmlNode{variable} recognizes the following child node:
}

\newcommand{\variableChildrenIntro}
{
  This \xmlNode{variable} recognizes the following child nodes:
}

\newcommand{\variableIntro}[1]
{
  In the \xmlNode{#1} input block, the user
  needs to specify the variables to sample.
  %
  As already mentioned, these variables are specified within consecutive
  \xmlNode{variable} XML blocks:
}

\newcommand{\constructionGridDescriptionOnlyCustom}
{
Based on the \xmlAttr{construction} type, the content of the \xmlNode{grid}
XML node and the requirements for other attributes change. In this case, only the following is available:
\begin{itemize}
  \item \xmlAttr{construction}\textbf{\texttt{=}}\xmlString{custom}.
    The grid will be directly specified by the user.
    This construction type requires that the \xmlNode{grid} node contains
    the actual mesh bins.
    For example, if the grid \xmlAttr{type} is \xmlString{CDF}, in the body
    of \xmlNode{grid}, the user will specify the CDF probability thresholds
    (nodalization in probability).
      All the bins are checked against the associated
      \xmlNode{distribution} bounds.
      If one or more of them falls outside the distribution's bounds, the
      code will raise an error.
      No additional attributes are needed.
\end{itemize}
}

\newcommand{\constructionGridDescription}
{
Based on the \xmlAttr{construction} type, the content of the \xmlNode{grid}
XML node and the requirements for other attributes change:
\begin{itemize}
  \item \xmlAttr{construction}\textbf{\texttt{=}}\xmlString{equal}.
    The grid is going to be constructed equally-spaced
    (\xmlAttr{type}\textbf{\texttt{=}}\xmlString{value}) or equally probable
    (\xmlAttr{type}\textbf{\texttt{=}}\xmlString{CDF}).
    This construction type requires the definition of additional attributes:
      \begin{itemize}
         \item \xmlAttr{steps}, \xmlDesc{required integer attribute}, number
           of equally spaced/probable discretization steps.
      \end{itemize}
      This construction type requires that the content of the \xmlNode{grid}
      node represents the lower and upper bounds (either
      in probability or value). Two values need to be specified; the lowest one
     will be considered as the $lowerBound$, the largest, the $upperBound$.
      The lower and upper bounds are checked against the associated
      \xmlNode{distribution} bounds.
      If one or both of them falls outside the distribution's bounds, the
      code will raise an error.
      The $stepSize$ is determined as follows:
      \\ $stepSize=(upperBound - lowerBound)/steps$
  \item \xmlAttr{construction}\textbf{\texttt{=}}\xmlString{custom}.
    The grid will be directly specified by the user.
    No additional attributes are needed.
    This construction type requires that the \xmlNode{grid} node contains
    the actual mesh bins.
    For example, if the grid \xmlAttr{type} is \xmlString{CDF}, in the body
    of \xmlNode{grid}, the user will specify the CDF probability thresholds
    (nodalization in probability).
      All the bins are checked against the associated
      \xmlNode{distribution} bounds.
      If one or more of them falls outside the distribution's bounds, the
      code will raise an error.
\end{itemize}
}
\newcommand{\variableDescription}
{
  \xmlNode{variable}, \xmlDesc{XML node, required parameter} can specify the following attribute:
  \begin{itemize}
    \item \xmlAttr{name}, \xmlDesc{required string attribute}, user-defined name
      of this variable.
    \item \shapeVariableDescription
  \end{itemize}
}

\newcommand{\constantVariablesDescription}
{
  \xmlNode{constant}, \xmlDesc{XML node, optional parameter} the user is able to input variables that need to be
  kept constant.
  For doing this, as many \xmlNode{constant} nodes as needed can be
  input.
  %
  There are options for setting a constant. To simply set the constant's value,
  the body of the node contains the constant value, and the \xmlNode{constant}
  node has the following attributes:
  \begin{itemize}
    \item \xmlAttr{name}, \xmlDesc{required string attribute}, user-defined name
      of this constant.
    \item \shapeConstantDescription
  \end{itemize}

  Alternatively, the constant value can
  be read from a DataObject that has been identified as a \xmlNode{ConstantSource} for
  this Sampler. In this case, the body of the \xmlNode{constant} node is the name of the
  variable that needs to be read from the \xmlNode{ConstantSource}, and the \xmlNode{constant}
  node has the following additional attributes in addition to the those above:
  \begin{itemize}
    \item \xmlAttr{source}, \xmlDesc{required string attribute}, the name of the DataObject
      containing the value to be used for this constant. This must be the name of one of
      the \xmlNode{ConstantSource} DataObjects.
    \item \xmlAttr{index}, \xmlDesc{optional integer attribute}, the index of the realization
      in the source DataObject that contains the value to use for the constant. \default{the last entry}
  \end{itemize}

  By way of example, consider the following Sampler definition. The constant will be named
  \xmlString{C} in the Sampler, and its value is taken from the DataObject \xmlString{MyConstant},
  which is identified in the \xmlNode{ConstantSource} node. To find the value of the constant in
  \xmlString{MyConstant}, the Sampler will look at the realization with index \xmlString{3} for the
  value of variable \xmlString{A} to use as the constant value.
  \lstinputlisting[style=XML]{const_example.xml}

}

\newcommand{\distributionDescription}
{
  \xmlNode{distribution}, \xmlDesc{string,
  required field}, name of the distribution that is associated to this variable.
  Its name needs to be contained in the \xmlNode{Distributions} block explained
  in Section \ref{sec:distributions}. In addition, if NDDistribution is used,
  the attribute \xmlAttr{dim} is required. \nb{Alternatively, this node must be omitted
  if the \xmlNode{function} node is supplied.}
}
\newcommand{\functionDescription}
{
  \xmlNode{function}, \xmlDesc{string, required field}, name of the function that
  defines the calculation of this variable from other distributed variables.  Its name
  needs to be contained in the \xmlNode{Functions} block explained in Section
  \ref{sec:functions}. This function must implement a method named ``evaluate''.
 \nb{Alternatively, this node must be ommitted
  if the \xmlNode{distribution} node is supplied.}
}




\newcommand{\gridDescriptionOnlyCustom}
{
  \xmlNode{grid}, \xmlDesc{space separated floats, required
  field}, the content of this XML node depends on the definition of the
  associated attributes:
  \begin{itemize}
  \itemsep0em
    \item \xmlAttr{type}, \xmlDesc{required string attribute}, user-defined
      discretization metric type: 1) \xmlString{CDF}, the grid will be
      specified based on cumulative distribution function probability
      thresholds, and 2) \xmlString{value}, the grid will be provided
      using variable values.
    \item \xmlAttr{construction}, \xmlDesc{required string attribute}, how
      the grid needs to be constructed, independent of its type (i.e.
      \xmlString{CDF} or \xmlString{value}).
  \end{itemize}
  \constructionGridDescriptionOnlyCustom
  \nb{The \xmlNode{grid} node is only required if a \xmlNode{distribution}
  node is supplied.  In the case of a \xmlNode{function} node, no grid
  information is requested.}
}



\newcommand{\gridDescription}
{
  \xmlNode{grid}, \xmlDesc{space separated floats, required
  field}, the content of this XML node depends on the definition of the
  associated attributes:
  \begin{itemize}
  \itemsep0em
    \item \xmlAttr{type}, \xmlDesc{required string attribute}, user-defined
      discretization metric type: 1) \xmlString{CDF}, the grid will be
      specified based on cumulative distribution function probability
      thresholds, and 2) \xmlString{value}, the grid will be provided
      using variable values.
    \item \xmlAttr{construction}, \xmlDesc{required string attribute}, how
      the grid needs to be constructed, independent of its type (i.e.
      \xmlString{CDF} or \xmlString{value}).
  \end{itemize}
  \constructionGridDescription
  \nb{The \xmlNode{grid} node is only required if a \xmlNode{distribution}
  node is supplied.  In the case of a \xmlNode{function} node, no grid
  information is requested.}
}

\newcommand{\convergenceDescription}
{
\xmlNode{Convergence}, \xmlDesc{float, required field}, Convergence
    tolerance.
    %
    The meaning of this tolerance depends on the definition of other attributes
    that might be defined in this XML node:
    \begin{itemize}
      \item \xmlAttr{limit}, \xmlDesc{optional integer attribute}, the
        maximum number of adaptive samples (iterations).
        %
        \default{infinite}.
      \item \xmlAttr{forceIteration}, \xmlDesc{optional boolean attribute},
        this attribute controls if at least a number of iterations equal to
        \textbf{limit} must be performed.
        %
        \default{False}.
      \item \xmlAttr{weight}, \xmlDesc{optional string attribute (case insensitive)}, defines on
        what the convergence check needs to be performed.
        \begin{itemize}
          \item \xmlString{CDF}, the convergence is checked in terms
            of probability (Cumulative Distribution Function). From a practical point of view,
            this means that full uncertain domain
            is discretized in a way that the probability volume of each cell is going to be equal to
           the tolerance specified in the body of the node \xmlNode{Convergence}
          \item \xmlString{value}, the convergence is checked on the
            hyper-volume in terms of variable values.From a practical point of view,
            this means that full uncertain domain
            is discretized in a way that the ``volume'' fraction of each cell is going to be equal to
           the tolerance specified in the body of the node \xmlNode{Convergence}. In other words,
           each cell volume is going to be equal to the total volume times the tolerance.
        \end{itemize}
        \default{CDF}.
      \item \xmlAttr{persistence}, \xmlDesc{optional integer attribute},
        offers an additional convergence check.
        %
        It represents the number of times the computed error needs to be
        below the inputted tolerance before convergence is reported.
        %
        \default{5}.
        \item \xmlAttr{subGridTol}, \xmlDesc{optional float attribute},
            this attribute is used to activate the multi-grid approach (adaptive meshing)
            of the constructed evaluation grid (see attribute \xmlAttr{weight}).
            In case this attribute is specified, the final grid discretization (cell's ``volume content''
             aka convergence confidence) is represented by the
            value here specified. The sampler converges on the initial coarse grid, defined by
            the tolerance specified in the body of the node \xmlNode{Convergence}.
            When the Limit Surface has been identified on the coarse grid, the sampler starts
            refining the grid until the ``volume content'' of each cell is equal to the value
            specified in this attribute (Multi-grid approach).
           \default{None}.
    \end{itemize}
    In summary, this XML node contains the information that is needed in order
    to control this sampler's convergence criterion.
}

\newcommand{\assemblerDescription}[1]
{
  \textbf{Assembler Objects} These objects are either required or optional
    depending on the functionality of the #1 Sampler.
    %
    The objects must be listed with a rigorous syntax that, except for the XML
    node tag, is common among all the objects.
    %
    Each of these nodes must contain 2 attributes that are used to identify them
    within the simulation framework:
    \begin{itemize}
      \item \xmlAttr{class}, \xmlDesc{required string attribute}, the main
        ``class'' of the listed object.
        %
        For example, it can be \xmlString{Models}, \xmlString{Functions}, etc.
      \item \xmlAttr{type},  \xmlDesc{required string attribute}, the object
        identifier or sub-type.
        %
        For example, it can be \xmlString{ROM}, \xmlString{External}, etc.
    \end{itemize}
    The \textbf{#1} approach requires or optionally accepts the
    following object types:

}
\newcommand{\ROMDescription}[1]
{
    \begin{itemize}
      \item \xmlNode{ROM}, \xmlDesc{string, required field}, the
        body of this XML node must contain the name of an appropriate ROM defined in the
        \xmlNode{Models} block (see Section~\ref{subsec:models_ROM}).
    \end{itemize}
}

\newcommand{\restartDescription}[1]
{
    \begin{itemize}
      \item \xmlNode{Restart}, \xmlDesc{string, optional field}, the
        body of this XML node must contain the name of an appropriate \textbf{DataObject} defined in the
        \xmlNode{DataObjects} block (see Section~\ref{sec:DataObjects}).  It is used as a
        ``restart'' tool, where it accepts pre-existing solutions in the PointSet instead
        of recalculating solutions.
    \end{itemize}

    The following node is an additional option when a restart DataObject is
    provided:

    \begin{itemize}
      \item \xmlNode{restartTolerance}, \xmlDesc{float, optional field}, the
        body of this XML node must contain a valid floating point value.  If a \xmlNode{Restart} node is
        supplied for this \xmlNode{Sampler}, this node offers a way to determine how strictly matching points
        are determined.  Given a point in the input space, if that point is within a relative Euclidean
        distance (equal to the tolerance) of a restart point, the nearest restart point will be used.
        \default{1e-15}
    \end{itemize}
}
\newcommand{\constantSourceDescription}[1]
{
    \begin{itemize}
      \item \xmlNode{ConstantSource}, \xmlDesc{string, optional field}, the
        body of this XML node must contain the name of an appropriate \textbf{DataObject} defined in the
        \xmlNode{DataObjects} block (see Section~\ref{sec:DataObjects}).  It is used as a
        source from which constants can take values.
    \end{itemize}
}

\newcommand{\variablesTransformationDescription}[1]
{
    \begin{itemize}
      \item \xmlNode{variablesTransformation}, \xmlDesc{optional field}. this XML node accepts one attribute:
      \begin{itemize}
        \item \xmlAttr{distribution}, \xmlDesc{required string attribute}, the name for the distribution defined in the XML node \xmlNode{Distributions}.
        This attribute indicates the values of \xmlNode{manifestVariables} are drawn from \xmlAttr{distribution}.
      \end{itemize}
      In addition, this XML node also accepts three childen nodes:
      \begin{itemize}
        \item \xmlNode{latentVariables}, \xmlDesc{comma separated string, required field}, user-defined latent variables that
        are used for the variables transformation. All the variables listed under this node should be also mentioned in \xmlNode{variable}.
        \item \xmlNode{manifestVariables}, \xmlDesc{comma separated string, required field}, user-defined manifest variables
        that can be used by the \xmlAttr{model}.
        \item \xmlNode{manifestVariablesIndex}, \xmlDesc{comma separated string, optional field}, user-defined manifest variables indices paired with \xmlNode{manifestVariables}.
        These indices indicate the position of manifest variables associated with multivariate normal distribution defined in the XML node \xmlNode{Distributions}.
        The indices should be postive integer. If not provided, the code will use the positions of manifest variables listed in \xmlNode{manifestVariables} as the indices.
        \item \xmlNode{method}, \xmlDesc{string, required field}, the method that is used for the variables transformation. The currently available method is '\textbf{pca}'.
      \end{itemize}
    \end{itemize}
}

\newcommand{\convergenceStudyDescription}
{
    \xmlNode{convergenceStudy}, \xmlDesc{optional node},
    if included, triggers writing state points at particular numbers of model solves for the purpose of
    a convergence study.  The study is performed by writing XML output files as described in the
    OutStreams for ROMs at the state points requested, using \xmlString{all} as the requested
    \xmlNode{what} values.
    The state points are identified when a certain
    number of model runs is passed, as specified by the \xmlNode{runStatePoints} node.
    This node has the following sub-nodes to define its parameters:
    \begin{itemize}
      \item \xmlNode{runStatePoints}, \xmlDesc{list of integers, required node},
        lists the number of model runs at which state points should be written. Note that these will be
        written when the requested number of runs is met or passed, so the actual value is often somewhat more
        than the requested value, and the exact value will be listed in the XML output.
      \item \xmlNode{baseFilename}, \xmlDesc{string, optional node},
        if specified determines the base file name for the state point outputs.  If not specified, defaults to
        \xmlString{out\_}.
      \item \xmlNode{pickle}, \xmlDesc{no text, optional node},
        if this node is included, serialized (pickled) versions of the ROM at each of the run states is also
        created in the working directory, with the format \texttt{<baseFilename><numRuns>.pk}, such as
        \texttt{out\_100.pk}.
    \end{itemize}
}

\newcommand{\convergenceDescriptionAMC}
{
  \xmlNode{Convergence} recognizes the following child nodes:
    \begin{itemize}
      \item \xmlNode{limit}, \xmlDesc{integer required field}, the
        maximum number of adaptive samples (iterations).
        %
        \default{infinite}.
      \item \xmlNode{forceIteration}, \xmlDesc{boolean optional field},
        this attribute controls if at least a number of iterations equal to
        \textbf{limit} must be performed.
        %
        \default{False}.
      \item \xmlNode{persistence}, \xmlDesc{integer optional field},
        offers an additional convergence check.
        %
        It represents the number of times the computed error needs to be
        below the inputted tolerance before convergence is reported.
        %
        \default{5}.

      \item \xmlNode{"metric"}, \xmlDesc{comma separated string list, required field},
        specifications for the aggregate metrics on which \xmlNode{AdaptiveMonteCarlo} will attempt to converge. The name of each node is the requested metric. The text of the node is a comma-separated list of the
        parameters for which the metric should be calculated. See the example below.\\


        \xmlNode{AdaptiveMonteCarlo} will attempt to converge the standard errors of the requested metrics. Currently the metrics available are:
        \begin{itemize}
          \item \textbf{expectedValue}: expected value or mean
          \item \textbf{median}: median
          \item \textbf{variance}: variance
          \item \textbf{sigma}: standard deviation
          \item \textbf{skewness}: skewness
          \item \textbf{kurtosis}: excess kurtosis (also known as Fisher's kurtosis)
        \end{itemize}



      The nodes containing metrics need to contain the following attributes:
      \begin{itemize}
        \itemsep0em
        \item \xmlAttr{prefix}, \xmlDesc{required string attribute}, user-defined prefix for the given \textbf{metric}.
          For scalar quantifies, RAVEN will define a variable with name defined as:  ``prefix'' + ``\_'' + ``parameter name''.
          For example, if we define ``mean'' as the prefix for \textbf{expectedValue}, and parameter ``x'', then variable
          ``mean\_x'' will be defined by RAVEN.
          For matrix quantities, RAVEN will define a variable with name defined as: ``prefix'' + ``\_'' + ``target parameter name'' + ``\_'' + ``feature parameter name''.
          For example, if we define ``sen'' as the prefix for \textbf{sensitivity}, target ``y'' and feature ``x'', then
          variable ``sen\_y\_x'' will be defined by RAVEN.
          \nb These variable will be used by RAVEN for the internal calculations. It is also accessible by the user through
          \textbf{DataObjects} and \textbf{OutStreams}.

          \item \xmlAttr{tol}, \xmlDesc{required float attribute}, convergence tolerance for the standard error of the metric.

      \end{itemize}
      %
      \vspace{12pt}
      RAVEN will define a variable with name defined as: ``prefix for given \textbf{metric}'' + ``\_ste\_'' + ``parameter name'' to
      store the standard error of the given \textbf{metric} with respect to the given parameter. This variable needs to be included in the \xmlNode{TargetEvaluation} \xmlNode{DataObject} which is an output of the \xmlNode{Step} in which the \xmlNode{AdaptiveMonteCarlo} is used. This variable is also available for output to the \xmlNode{SolutionExport} \xmlNode{DataObjec}.  \\

      \nb When defining the metrics to use, it is possible to have multiple nodes with the same name.  For
      example, if a problem has inputs $X1$, and $X2$, and the responses are $Y1$, $Y2$, it is possible that
      the desired metrics are the \xmlNode{sigma} of $Y1$,and $Y2$ on same tolerance, and \xmlNode{expectedValue}
      of $Y1$,and $Y2$ on different tolerance. The first has the parameters $Y1,Y2$ in the same node with one
      tolerance attribute, while the second need to divide into two nodes. One has target $Y1$ and another one
      has target $Y2$ instead.  This could reduce some computation effort in problems with many responses or inputs.
      An example of this is shown below.\\

    \end{itemize}

    In summary, the \xmlNode{convergence} node contains the information that is needed in order
    to control the \xmlNode{AdaptiveMonteCarlo} sampler's convergence criteria.
}
%


%%%%%%%%%%%%%%%%%%%%%%%%%%%%%%%%%%%%%%%%%%%%%%%%%%%%%%%%%%%%%%%%%%%%%%%%%%%%%%%%

The sampler is probably the most important entity in the RAVEN framework.
%
It performs the driving of the specific sampling strategy and, hence, determines
the effectiveness of the analysis, from both an accuracy and computational point
of view.
%
The samplers, that are available in RAVEN, can be categorized into three main
classes:
\begin{itemize}
\item \textbf{Forward} (see Section~\ref{subsec:onceThroughSamplers})
\item \textbf{Dynamic Event Tree (DET)} (see Section~\ref{subsec:DETSamplers})
\item \textbf{Adaptive} (see Section~\ref{subsec:AdaptSamplers})
\end{itemize}
Before analyzing each sampler in detail, it is important to mention that each
type has a similar syntax to input the variables to be ``sampled''.
%
In the example below, the variable \xmlString{variableName} is going to be
sampled by the Sampler \xmlString{whatever} using the distribution named\\
\xmlString{aDistribution}.
\begin{lstlisting}[style=XML]
<Simulation>
  ...
  <Samplers>
    ...
    <WhatEverSampler name='whatever'>
      ...
     <variable name='variableName'>
       ...
       <distribution>aDistribution</distribution>
       ...
     </variable>
      ...
    </WhatEverSampler>
    ...
  </Samplers>
  ...
</Simulation>
\end{lstlisting}

As reported in section \ref{sec:existingInterface}, the variable naming syntax,
for external driven codes, depends on the way the ``code interface'' has been
implemented.
%
For example, if the code has an input structure like the one reported below (YAML), the
variable name might be\xmlString{I-Level|II-Level|variable}.
%
In this way, the relative code interface (and input parser) will know which
variable needs to be perturbed and the ``recipe'' to access it.
%
As reported in \ref{sec:existingInterface}, its syntax is chosen by the
developer of the ``code interface'' and is implemented in the interface only
(no modifications are needed in the RAVEN code).

%\maljdan{Where does this type of input come from? Should the user care?}
%\alfoa{Dan, this is an example (in this case, a YAML structure)}

Example YAML based Input:
\begin{lstlisting}
[I-Level]
  [./II-Level]
    variable = xxx
  [../]
[]
\end{lstlisting}

Example XML block to define the variables and associated distributions:
\begin{lstlisting}[style=XML]
<variable name='I-Level|II-Level|variable'>
  <distribution>exampleDistribution</distribution>
</variable>
\end{lstlisting}

If the variable is associated to a multi-dimensional ND distribution, it is needed to specify which dimension of the ND distribution is associated to such variable. An example is shown below: the variable  ``variableX'' is associated to the third dimension of the ND distribution ``NDdistribution''.

\begin{lstlisting}[style=XML]
<variable name='variableX'>
     <distribution dim='3'>NDdistribution</distribution>
</variable>
\end{lstlisting}

For most codes, it is prudent that there are no redundant inputs; however there are
cases where this is not reality.  For example, if there is a variable \xmlString{inner\_radius} and
a variable \xmlString{outer\_radius}, there may be a third variable \xmlString{thickness} that
is actually derived from the previous two, as \xmlString{thickness} = \xmlString{outer\_radius} - \xmlString{inner\_radius}.
RAVEN supports this type of redundant input through a Function entity.  In this case,
instead of a \xmlNode{distribution} node in the \xmlNode{variable} block, there is a
\xmlNode{function} node, specifying the name of the function (defined in the \xmlNode{Functions} block).
In order to work properly, this function must have a method named ``evaluate''
that returns a single python float object. In this way, multiple variables can be associated with the same function.  For example,
\begin{lstlisting}[style=XML]
...
<Functions>
  <External name='torus_calcs' file='torus_calcs.py'>
    <variable>outer_radius</variable>
    <variable>inner_radius</variable>
  </External>
<Functions>
...
<Samplers>
  <WhatEverSampler name='myExampleSampler'>
    <variable name='inner_radius'>
      <distribution>inner_dist</distribution>
    </variable>
    <variable name='outer_radius'>
      <distribution>outer_dist</distribution>
    </variable>
    <variable name='thickness'>
      <function>torus_calcs</function>
    </variable>
  </WhatEverSampler>
</Samplers>
\end{lstlisting}
The corresponding function file \xmlString{torus\_calcs.py} needs the following method:
\begin{lstlisting}
def evaluate(self):
  return self.outer_radius - self.inner_radius
\end{lstlisting}
The \xmlString{thickness} parameter will still be treated as an input for the sake of csv
printing and DataObjects storage.
\\\nb It is important to notice that if the user use variables with no-Python compatible names (e.g. parenthesis, etc.),
the \xmlNode{alias} system needs to be used to alias the variables.

In the sampler class a special node exists: the \xmlNode{sampler\textunderscore init} node.
This node contains specific parameters that characterize each particular sampler.
In addition, \xmlNode{sampler\textunderscore init} might contain the information regarding the random generator function for each $N$-Dimensional distribution (specified in the \xmlNode{dist\textunderscore init} node):
\begin{itemize}
\item initial\textunderscore grid\textunderscore disc
\item tolerance
\end{itemize}

An example of \xmlNode{dist\textunderscore init} node is provided below:

\begin{lstlisting}[style=XML]
<distInit>
    <distribution name= 'ND_dist_name'>
         <initialGridDisc>5</initialGridDisc>
          <tolerance>0.2</tolerance>
     </distribution>
  </distInit>
\end{lstlisting}

In the \xmlNode{sampler\textunderscore init}  node it is possible to add also the subnode \xmlNode{globalGrid}.
The \xmlNode{globalGrid} can be used in two cases:
\begin{itemize}
\item 1D distributions: an identical grid that is associated to several distributions
\item ND distribution: a grid associated to a single ND distribution. This is the case when a stratified sampling is performed on the CDF of an ND distribution: the  \xmlNode{globalGrid} is  shared among the variables associated to the Nd distribution
\end{itemize}

%%%%%%%%%%%%%%%%%%%%%%%%%
%%%      Forward Samplers      %%%
%%%%%%%%%%%%%%%%%%%%%%%%%
\subsection{Forward Samplers}
\label{subsec:onceThroughSamplers}
The Forward sampler category collects all the strategies that perform the
sampling of the input space without exploiting, through dynamic learning
approaches, the information made available from the outcomes of calculations
previously performed (adaptive sampling) and the common system evolution
(patterns) that different sampled calculations can generate in the phase space
(dynamic event tree).
%
In the RAVEN framework, several different ``Forward'' samplers
are available:
\begin{itemize}
\item \textbf{Monte Carlo (MC)}
\item \textbf{Stratified}
\item \textbf{Grid Based}
\item \textbf{Sparse Grid Collocation}
\item \textbf{Sobol Decomposition}
\item \textbf{Response Surface Design of Experiment}
\item \textbf{Factorial Design of Experiment}
\item \textbf{Ensemble Forward Sampling strategy}
\item \textbf{Custom Sampling strategy}
\end{itemize}

From a practical point of view, these sampling strategies represent different
ways to explore the input space.
%
In the following paragraphs, the input requirements and a small explanation of
the different sampling methodologies are reported.


%%% Forward Samplers: MonteCarlo
\subsubsection{Monte Carlo}
\label{subsubsubsec:MC}
The \textbf{Monte-Carlo} sampling approach is one of the most well-known and
widely used approaches to perform exploration of the input space.
%
The main idea behind MonteCarlo sampling is to  randomly perturb the input space according
to uniform or parameter-based probability density functions.
%

\specBlock{a}{MonteCarlo}
%
\attrsIntro
\vspace{-5mm}
\begin{itemize}
\itemsep0em
\item \textbf{name}, \textit{required string attribute}, user-defined name of this Sampler. N.B. As for the other objects, this is the name that can be used to refer to this specific entity from other input blocks (xml);
%\item \textbf{limit}, \textit{required integer attribute}, number of MonteCarlo samples needs to be generated;
%\item \textbf{initial\_seed}, \textit{optional integer attribute}, initial seeding of random number generator. \textit{Default = random seed};
%\item \textbf{reseedEachIteration}, \textit{optional boolean/string attribute}, perform a re-seeding for each sample generated (True values = True, yes, y, t). \textit{Default = False};
\end{itemize}
\vspace{-5mm}

In the \textbf{MonteCarlo} input block, the user needs to specify the variables need to be sampled. As already mentioned, these variables are inputted within consecutive xml blocks called \xmlNode{variable}. In addition, the settings for this sampler need to be specified in the \xmlNode{samplerInit} XML block:
\begin{itemize}
\item \xmlNode{samplerInit},  \textit{\textbf{XML node, required parameter}}. In this xml-node,the following xml sub-nodes need to be specified:
  \begin{itemize}
    \item \xmlNode{limit}, \textit{\textbf{integer,required field}}, number of MonteCarlo samples needs to be generated;
    \item \xmlNode{initialSeed}, \textit{\textbf{integer, optional field}}, initial seeding of random number generator
    \item \xmlNode{reseedEachIteration},  \textit{\textbf{boolean/string(case insensitive), optional field}}, perform a re-seeding for each sample generated (True values = True, yes, y, t). \default{False};
    \item \xmlNode{distInit},  \textit{\textbf{integer, optional field}}, in this node the user specifies the initialization of the random number generator function for each N-Dimensional Probability Distributions (see Section~\ref{subsec:NdDist}).
    \item \xmlNode{samplingType}, \textit{\textbf{string, optional field}}, sub-type of sampling \default{None}. the user can choose to perform a Monte-Carlo sampling where the location of the samples in
the input space is uniformly distributed and not generated accordingly to the specific set of distributions. This can be specificed
in the \xmlNode{samplingType} with the kewyword ``uniform''. This option works only if all the distributions have an upper and lower
bound specified (i.e., \xmlNode{lowerBound} and \xmlNode{upperBound}). Allowed fields for this node are ``None'' and ``uniform''.
  \end{itemize}
\end{itemize}
\begin{itemize}
\item \variableDescription
 \variableChildrenIntro
 \begin{itemize}
    \item \distributionDescription
    \item \functionDescription
 \end{itemize}
 \item \constantVariablesDescription
\end{itemize}

If the input parameters are correlated, the \textbf{MonteCarlo} sampling approach can be also used if the user specified a
multivariate distributions inside the \xmlNode{Distributions} (see Section \ref{subsec:NdDist}). Furthermore, if the
covariance matrix is provided and the input parameters is assumed to have the multivariate normal distribution, one can also use
\textbf{MonteCarlo} approach to sample the input parameters in the transformed space (aka subspace, reduced space). If this is
the case, the user needs to provide additional information, i.e. the \xmlNode{transformation} under \xmlNode{MultivariateNormal} of
\xmlNode{Distributions} (more information can be found in Section \ref{subsec:NdDist}). In addition, the node
\xmlNode{variablesTransformation} is also required for \textbf{MonteCarlo} sampling. This node is used to tranform the variables
specified by \xmlNode{latentVariables} in the transformed space of input into variables spefified by \xmlNode{manifestVariables}
in the input space. The variables listed in \xmlNode{latentVariables} should be predefined in \xmlNode{variable}, and the variables
listed in \xmlNode{manifestVariables} are used by the \xmlNode{Models}.

\variablesTransformationDescription{MonteCarlo}
\assemblerDescription{MonteCarlo}
\restartDescription{MonteCarlo}
\constantSourceDescription{MonteCarlo}

Example:
\begin{lstlisting}[style=XML]
<Samplers>
  ...
  <MonteCarlo name='MCname'>
    <samplerInit>
      <limit>10</limit>
      <initialSeed>200286</initialSeed>
      <reseedEachIteration>false</reseedEachIteration>
      <distInit>
        <distribution name= 'ND_InverseWeight_P'>
          <initialGridDisc>10</initialGridDisc>
          <tolerance>0.2</tolerance>
        </distribution>
      </distInit>
    </samplerInit>
    <variable name='var1'>
      <distribution>aDistributionNameDefinedInDistributionBlock
      </distribution>
    </variable>
    <Restart class='DataObject' type='PointSet'>data</Restart>
  </MonteCarlo>
  ...
</Samplers>
  ...
  <PointSet name="data">
    <Input>var1</Input>
    <Output>ans</Output>
  </PointSet>
  ...
\end{lstlisting}

%%% Forward Samplers: Grid
\subsubsection{Grid}
\label{subsubsubsec:Grid}
The \textbf{Grid} sampling approach is probably the simplest exploration
approach that can be employed to explore an uncertain domain.
%
The idea is to construct an $N$-dimensional grid where each dimension is
represented by one uncertain variable.
%
This approach performs the sampling at each node of the grid.
%
The sampling of the grid consists in evaluating the answer of the system under
all possible combinations among the different variables' values with respect to
a predefined discretization metric.
%
In RAVEN two discretization metrics are available: 1) cumulative distribution
function, and 2) value.
%
Thus, the grid meshing can be input via probability or variable values.
%
Regarding the N-dimensional distributions, the user can specify for each dimension the type of grid to be used (i.e., value or CDF). Note the discretization of the CDF, only for the grid sampler, is performed on the marginal distribution for the specific variable considered.

\specBlock{a}{Grid}
%
\attrIntro
\begin{itemize}
\itemsep0em
\item \nameDescription
\end{itemize}
\variableIntro{Grid}
\begin{itemize}
\item \variableDescription
 \variableChildrenIntro
 \begin{itemize}
    \item \distributionDescription
    \item \functionDescription
    \item \gridDescription
  \end{itemize}
\item \constantVariablesDescription
\end{itemize}

If the input parameters are correlated, the \textbf{Grid} sampling approach can be also used if the user specified a
multivariate distributions inside the \xmlNode{Distributions} (see Section \ref{subsec:NdDist}). Furthermore, if the
covariance matrix is provided and the input parameters is assumed to have the multivariate normal distribution, one can also use
\textbf{Grid} approach to sample the input parameters in the transformed space (aka subspace, reduced space). This means one creates
the grids of variables listed by \xmlNode{latentVariables} in the transformed space. If this is the case, the user needs to
provide additional information, i.e. the \xmlNode{transformation} under \xmlNode{MultivariateNormal} of \xmlNode{Distributions}
(more information can be found in Section \ref{subsec:NdDist}). In addition, the node \xmlNode{variablesTransformation} is also
required for \textbf{Grid} sampling. This node is used to tranform the variables specified by \xmlNode{latentVariables} in the
transformed space of input into variables spefified by \xmlNode{manifestVariables} in the input space. The variables listed
in \xmlNode{latentVariables} should be predefined in \xmlNode{variable}, and the variables listed in \xmlNode{manifestVariables}
are used by the \xmlNode{Models}.

\variablesTransformationDescription{Grid}

\assemblerDescription{Grid}
\restartDescription{Grid}
\constantSourceDescription{Grid}

Example:
\begin{lstlisting}[style=XML,morekeywords={construction,steps,lowerBound,upperBound}]
<Samplers>
  ...
  <Grid name='Gridname'>
    <variable name='var1'>
      <distribution>aDistributionNameDefinedInDistributionBlock1
      </distribution>
      <grid type='value' construction='equal' steps='100' >0.2 10</grid>
    </variable>
    <variable name='var2'>
      <distribution>aDistributionNameDefinedInDistributionBlock2
      </distribution>
      <grid type='CDF' construction='equal' steps='5' >0.2 0.8</grid>
    </variable>
    <variable name='var3'>
      <distribution>aDistributionNameDefinedInDistributionBlock3
      </distribution>
      <grid type='value' construction='equal' steps='100' >0.2 21.0</grid>
    </variable>
    <variable name='var4'>
      <distribution>aDistributionNameDefinedInDistributionBlock4
      </distribution>
      <grid type='CDF' construction='equal' steps='5' >0.2 1.0</grid>
    </variable>
    <variable name='var5'>
      <distribution>aDistributionNameDefinedInDistributionBlock5
      </distribution>
      <grid type='value' construction='custom'>0.2 0.5 10.0</grid>
    </variable>
    <variable name='var6'>
      <distribution>aDistributionNameDefinedInDistributionBlock6
      </distribution>
      <grid type='CDF' construction='custom'>0.2 0.5 1.0</grid>
    </variable>
    <Restart class='DataObject' type='PointSet'>data</Restart>
    <restartTolerance>1e-6</restartTolerance>
  </Grid>
  ...
</Samplers>
  ...
  <PointSet name="data">
    <Input>var1,var2,var3,var4,var5,var6</Input>
    <Output>ans</Output>
  </PointSet>
  ...
\end{lstlisting}
\nb A restart example is included here but is not necessary in general.

%%% Forward Samplers: Sparse Grid Collocation
\subsubsection{Sparse Grid Collocation}
%\talbpaul{Work in progress.}
%\senrs{Assembler section updated}
\label{subsubsubsec:SparseGridCollocation}
\textbf{Sparse Grid Collocation} builds on generic \textbf{Grid} sampling by selecting evaluation points based on characteristic quadratures as part of
stochastic collocation for generalized polynomial chaos uncertainty quantification.  In collocation you construct an N-dimensional grid, with each uncertain
variable providing an axis.  Along each axis, the points of evaluation correspond to quadrature points necessary to integrate polynomials
(see \ref{subsubsec:GaussPolynomialRom}).  In the simplest (and most  naive) case, a N-Dimensional tensor product of all possible combinations of points from
each dimension's quadrature is constructed as sampling points.  The number of necessary samples can be reduced by employing Smolyak-like sparse grid algorithms,
which use reduced combinations of polynomial orders to reduce the necessary sampling space.  \specBlock{a}{SparseGridCollocation}.

\begin{itemize}
\itemsep0em
\item \nameDescription
\item \xmlAttr{parallel}, \xmlDesc{optional string attribute}, option to disable parallel construction of the sparse grid.  Because of increasing computational expense with increasing input space dimension, RAVEN will default to parallel construction of the sparse grid. %\talbpaul{Is this what we want?}
\item \xmlAttr{outfile}, \xmlDesc{optional string attribute}, option to allow the generated sparse grid points and weights to be printed to a file with the given name.
\default{True}
\end{itemize}
\variableIntro{SparseGridCollocation}
\begin{itemize}
\item \variableDescription
 In the variable node, the following xml-node needs to be specified:
 \begin{itemize}
    \item \distributionDescription
    \item \functionDescription
 \end{itemize}
 \item \constantVariablesDescription
\end{itemize}
Because of the tight coupling between the Sampler and the ROM in stochastic collocation for generalized polynomial chaos, the Sampler needs access to the ROM via the assembler do determine the polynomials, quadratures, and importance weights to use in each dimension (see \ref{subsubsec:GaussPolynomialRom}).

  % Assembler Objects
  \assemblerDescription{SparseGridCollocation}
  \ROMDescription{SparseGridCollocation}
  \restartDescription{SparseGridCollocation}
  \constantSourceDescription{SparseGridCollocation}

\footnotesize
\begin{lstlisting}[style=XML]
Example:
<Samplers>
  ...
  <SparseGridCollocation name="mySG" parallel="0">
    <variable name="x1">
      <distribution>myDist1</distribution>
    </variable>
    <variable name="x2">
      <distribution>myDist2</distribution>
    </variable>
    <ROM class = 'Models' type = 'ROM' >SCROM</ROM>
    <Restart class = 'DataObjects' type = 'PointSet' >solns</Restart>
  </SparseGridCollocation>
  ...
</Samplers>
  ...
  <PointSet name="solns">
    <Input>x1,x2</Input>
    <Output>y</Output>
  </PointSet>
  ...
\end{lstlisting}
 \normalsize

In general, \textbf{SparseGridCollocation} requires uncorrelated input parameters. If the input parameters are correlated, one can transform the
correlated parameters into uncorrelated parameters; the \textbf{SparseGridCollocation} can also be used with the uncorrelated parameters
in the transformed space. Like in the \textbf{Grid} sampler, if the covariance matrix is provided
and the input parameters are assumed to have the multivariate normal distribution, the \textbf{SparseGridCollocation} can be used.
This means one creates the sparse grids of variables listed by \xmlNode{latentVariables} in the transformed space. If this is
the case, the user needs to provide additional information, i.e. the \xmlNode{transformation} under \xmlNode{MultivariateNormal}
of \xmlNode{Distributions} (more information can be found in Section \ref{subsec:NdDist}). In addition, the node
\xmlNode{variablesTransformation} is also required for \textbf{SparseGridCollocation} sampler. This node is used to tranform
the variables specified by \xmlNode{latentVariables} in the transformed space of input into variables spefified by
\xmlNode{manifestVariables} in the input space. The variables listed in \xmlNode{latentVariables} should be predefined
in \xmlNode{variable}, and the variables listed in \xmlNode{manifestVariables}
are used by the \xmlNode{Models}.

\variablesTransformationDescription{SparseGridCollocation}


\begin{lstlisting}[style=XML,morekeywords={ND,grid}]
...
<Models>
    ...
    <ExternalModel ModuleToLoad="lorentzAttractor_noK" name="PythonModule" subType="">
        <variables>sigma,rho,beta,x,y,z,time,z0,y0,z0</variables>
    </ExternalModel>
    <ROM name="SCROM" subType="GaussPolynomialRom">
        <Target>and</Target>
        <Features>x1,y1,z1</Features>
        <IndexSet>TensorProduct</IndexSet>
        <PolynomialOrder>1</PolynomialOrder>
    </ROM>
    ...
</Models>

<Distributions>
    ...
    <MultivariateNormal name='MVNDist' method='pca'>
        <transformation>
            <rank>3</rank>
        </transformation>
        <mu>0.0 1.0 2.0</mu>
        <covariance type="abs">
            1.0       0.6      -0.4
            0.6       1.0      0.2
            -0.4      0.2      0.8
        </covariance>
    </MultivariateNormal>
    ...
</Distributions>

<Samplers>
  ...
  <SparseGridCollocation name='SC'>
        <variable name='x0'>
            <distribution dim='1'>MVNDist</distribution>
        </variable>
        <variable name='y0'>
            <distribution dim='2'>MVNDist</distribution>
        </variable>
        <variable name='z0'>
            <distribution dim='3'>MVNDist</distribution>
        </variable>
        <variablesTransformation model="PythonModule">
            <latentVariables>x1,y1,z1</latentVariables>
            <manifestVariables>x0,y0,z0</manifestVariables>
            <method>pca</method>
        </variablesTransformation>
        <ROM class = 'Models' type = 'ROM' >SCROM</ROM>
        <Restart class="DataObjects" type="PointSet">solns</Restart>
  </SparseGridCollocation>
  ...
</Samplers>
  ...
  <PointSet name="solns">
    <Input>x0,y0,z0</Input>
    <Output>ans</Output>
  </PointSet>
  ...
\end{lstlisting}

 %%% Forward Samplers: Sobol
\subsubsection{Sobol}
%\senrs{Assembler section updated}
\label{subsubsubsec:SobolSampler}
The \textbf{Sobol} sampler uses high-density model reduction (HDMR) a.k.a. Sobol decomposition to approximate a function as the sum of increasing-complexity
interactions.  At its lowest level (order 1), it treats the function as a sum of the reference case plus a functional of each input dimesion separately.  At
order 2, it adds functionals to consider the pairing of each dimension with each other dimension.  The benefit to this approach is considering several functions
of small input cardinality instead of a single function with large input cardinality.  This allows reduced order models like generalized polynomial chaos
(see \ref{subsubsec:GaussPolynomialRom}) to approximate the functionals accurately with few computations runs.  This Sobol sampler uses the associated HDMRRom
(see \ref{subsubsec:HDMRRom}) to determine at what points the input space need be evaluated. Since Sobol sampler relies on SparseGridCollocation, it is also compatible with
multivariate normal distribution objects. The \xmlNode{Sobol} node supports the following attributes:

\begin{itemize}
\itemsep0em
\item \nameDescription
\item \xmlAttr{parallel}, \xmlDesc{optional string attribute}, option to disable parallel construction of the sparse grid.  Because of increasing computational expense with increasing input space dimension, RAVEN will default to parallel construction of the sparse grid.
\default{True}
\end{itemize}
\variableIntro{Sobol}
\begin{itemize}
\item \variableDescription
 In the variable node, the following xml-node needs to be specified:
 \begin{itemize}
    \item \distributionDescription
    \item \functionDescription
 \end{itemize}
\item \constantVariablesDescription
\end{itemize}

Like the \textbf{SparseGridCollocation}, if multivariate normal distribution is provided, the following node need to be specified:
\variablesTransformationDescription{Sobol}

Because of the tight coupling between the Sobol sampler and the HDMRRom, the Sampler needs access to the ROM via the assembler do determine the polynomials, quadratures, Sobol order, and importance weights to use in each dimension (see \ref{subsubsec:HDMRRom}).

  % Assembler Objects
  \assemblerDescription{Sobol}
  \ROMDescription{Sobol}
  \restartDescription{Sobol}
  \constantSourceDescription{Sobol}

\footnotesize
\begin{lstlisting}[style=XML]
Example:
<Samplers>
  ...
  <Sobol name="mySobol" parallel="0">
    <variable name="x1">
      <distribution>myDist1</distribution>
    </variable>
    <variable name="x2">
      <distribution>myDist2</distribution>
    </variable>
    <ROM class = 'Models' type = 'ROM' >myHDMR</ROM>
    <Restart class="DataObjects" type="PointSet">solns</Restart>
  </Sobol>
  ...
</Samplers>
  ...
  <PointSet name="solns">
    <Input>x1,y2</Input>
    <Output>ans</Output>
  </PointSet>
  ...
\end{lstlisting}
 \normalsize

%%% Forward Samplers: Stratified
\subsubsection{Stratified}
\label{subsubsubsec:Stratified}
The \textbf{Stratified} sampling approach is a method for the exploration of the
input space that consists of dividing the uncertain domain into subgroups before
sampling.
%
In the ``stratified'' sampling, these subgroups must be:
\begin{itemize}
 \item mutually exclusive: every element in the population must be assigned to
   only one stratum (subgroup);
 \item collectively exhaustive: no population element can be excluded.
\end{itemize}

Then simple random sampling or systematic sampling is applied within each
stratum.
%
It is worthwhile to note that the well-known Latin hypercube sampling represents
a specialized version of the stratified approach, when the domain strata are
constructed in equally-probable CDF bins.

\specBlock{a}{Stratified}
%
\attrIntro
\begin{itemize}
\itemsep0em
\item \nameDescription
\end{itemize}
\variableIntro{Stratified}
\begin{itemize}
\item \variableDescription
 \variableChildrenIntro
 \begin{itemize}
    \item \distributionDescription
    \item \functionDescription
    \item \gridDescription
  \end{itemize}
\item \constantVariablesDescription
\end{itemize}
In addition, the settings for this sampler need to be specified in the \xmlNode{samplerInit} XML block:
\begin{itemize}
\item \xmlNode{samplerInit},  \textit{\textbf{XML node, required parameter}}. In this xml-node,the following xml sub-nodes need to be specified:
  \begin{itemize}
    \item \xmlNode{initialSeed}, \textit{\textbf{integer, optional field}}, initial seeding of random number generator
    \item \xmlNode{distInit},  \textit{\textbf{integer, optional field}}, in this node the user specifies the initialization of the random number generator function for each N-Dimensional Probability Distributions (see Section~\ref{subsec:NdDist}).
  \end{itemize}
\end{itemize}

As one can see, the input specifications for the \textbf{Stratified} sampler are
similar to that of the \textbf{Grid} sampler.
%
It is important to mention again that for each zone (grid mesh) only a point,
randomly selected, is picked and not all the nodal combinations (like in the
\textbf{Grid} sampling).

\assemblerDescription{Stratified}
\restartDescription{Stratified}
\constantSourceDescription{Stratified}

Example:
\begin{lstlisting}[style=XML,morekeywords={construction,steps,lowerBound,upperBound}]
<Samplers>
  ...
  <Stratified name='StratifiedName'>
    <variable name='var1'>
      <distribution>aDistributionNameDefinedInDistributionBlock1
      </distribution>
      <grid type='CDF' construction='equal' steps='5' >0.2 0.8</grid>
    </variable>
    <variable name='var2'>
      <distribution>aDistributionNameDefinedInDistributionBlock2
      </distribution>
      <grid type='value' construction='equal' steps='100' >0.2 21.0</grid>
    </variable>
    <variable name='var3'>
      <distribution>aDistributionNameDefinedInDistributionBlock3
      </distribution>
      <grid type='CDF' construction='custom'>0.2 0.5 1.0</grid>
    </variable>
  </Stratified>
  ...
</Samplers>
\end{lstlisting}

For N-dimensional  (ND) distributions, there are two different approahes to perform the stratified sampling. In the first approach,
the subgroups is determined by the joint CDF of given multivariate distributions. If this approach is used, the sampling is performed
on a grid on a CDF, while the user is required to specify the same CDF grid for all the dimensions of the ND distribution.
This is possible by defining a \xmlNode{globalGrid} node and associate such \xmlNode{globalGrid} to each variable belonging to the
ND distribution as follows.

\begin{lstlisting}[style=XML,morekeywords={ND,grid}]
<Samplers>
  ...
  <Stratified name='StratifiedName'>
        <variable name='x0'>
            <distribution dim='1'>ND_InverseWeight_P</distribution>
            <grid type='globalGrid'>name_grid1</grid>
        </variable>
        <variable name='y0,z0'>
            <distribution dim='2'>ND_InverseWeight_P</distribution>
            <grid type='globalGrid'>name_grid1</grid>
        </variable>
        <globalGrid>
            <grid name='name_grid1' type='CDF' construction='custom'>0.1 1.0 0.2</grid>
        </globalGrid>
  </Stratified>
  ...
</Samplers>
...
\end{lstlisting}

The second approach is different than the first approach. Like in the \textbf{Grid} sampling, if the covariance matrix is provided
and the input parameters is assumed to have the multivariate normal distribution, one can also use \textbf{Stratified} approach to
sample the input parameters in the transformed space (aka subspace, reduced space). This means one creates
the grids of variables listed by \xmlNode{latentVariables} in the transformed space. If this is the case, the user needs to
provide additional information, i.e. the \xmlNode{transformation} under \xmlNode{MultivariateNormal} of \xmlNode{Distributions}
(more information can be found in Section \ref{subsec:NdDist}). In addition, the node \xmlNode{variablesTransformation} is also
required for \textbf{Stratified} sampler. This node is used to tranform the variables specified by \xmlNode{latentVariables} in the
transformed space of input into variables spefified by \xmlNode{manifestVariables} in the input space. The variables listed
in \xmlNode{latentVariables} should be predefined in \xmlNode{variable}, and the variables listed in \xmlNode{manifestVariables}
are used by the \xmlNode{Models}. In addition, \xmlNode{globalGrid} will be not used for approach.

\variablesTransformationDescription{Stratified}


\begin{lstlisting}[style=XML,morekeywords={ND,grid}]
...
<Models>
    ...
    <ExternalModel ModuleToLoad="lorentzAttractor_noK" name="PythonModule" subType="">
        <variables>sigma,rho,beta,x,y,z,time,z0,y0,z0</variables>
    </ExternalModel>
    ...
</Models>

<Distributions>
    ...
    <MultivariateNormal name='MVNDist' method='pca'>
        <transformation>
            <rank>3</rank>
        </transformation>
        <mu>0.0 1.0 2.0</mu>
        <covariance type="abs">
            1.0       0.6      -0.4
            0.6       1.0      0.2
            -0.4      0.2      0.8
        </covariance>
    </MultivariateNormal>
    ...
</Distributions>

<Samplers>
  ...
  <Stratified name='StratifiedName'>
        <variable name='x0'>
            <distribution dim='1'>MVNDist</distribution>
            <grid type='CDF' construction='equal' steps='3'>0.1 0.9</grid>
        </variable>
        <variable name='y0'>
            <distribution dim='2'>MVNDist</distribution>
            <grid type='value' construction='equal' steps='3'>0.1 0.9</grid>
        </variable>
        <variable name='z0'>
            <distribution dim='3'>MVNDist</distribution>
            <grid type='CDF' construction='equal' steps='3'>0.2 0.8</grid>
        </variable>
        <variablesTransformation model="PythonModule">
            <latentVariables>x1,y1,z1</latentVariables>
            <manifestVariables>x0,y0,z0</manifestVariables>
            <method>pca</method>
        </variablesTransformation>
  </Stratified>
  ...
</Samplers>
...
\end{lstlisting}

%%% Forward Samplers: Response Surface Design
\subsubsection{Response Surface Design}
\label{subsubsubsec:RespSurfDOE}
The \textbf{Response Surface Design}, or Response Surface Modeling (RSM),
approach is one of the most common Design of Experiment (DOE) methodologies
currently in use.
%
It explores the relationships between several explanatory variables and one or
more response variables.
%
The main idea of RSM is to use a sequence of designed experiments to obtain an
optimal response.
%
RAVEN currently employs two different algorithms that can be classified within
this family of methods:
\begin{itemize}
 \item \textbf{Box-Behnken}: This methodology aims to achieve the following
  goals:
  \begin{itemize}
    \item Each factor, or independent variable, is placed at one of three
      equally spaced values, usually coded as -1, 0, +1. (At least three levels
      are needed for the following goal);
    \item The design should be sufficient to fit a quadratic model, that is, one
      squared term per factor and the products of any two factors;
    \item The ratio of the number of experimental points to the number of
      coefficients in the quadratic model should be reasonable (in fact, their
      designs keep it in the range of 1.5 to 2.6);
    \item The estimation variance should more or less depend only on the
      distance from the center (this is achieved exactly for the designs with 4
      and 7 factors), and should not vary too much inside the smallest
      (hyper)cube containing the experimental points.
  \end{itemize}
  Each design can be thought of as a combination of a two-level (full or
  fractional) factorial design with an incomplete block design.
  In each block, a certain number of factors are put through all combinations
  for the factorial design, while the other factors are kept at the central
  values.
 \item \textbf{Central Composite}: This design consists of three distinct sets
  of experimental runs:
  \begin{itemize}
    \item A factorial (perhaps fractional) design in the factors are studied,
      each having two levels;
    \item A set of center points, experimental runs whose values of each factor
      are the medians of the values used in the factorial portion.
      This point is often replicated in order to improve the precision of the
      experiment;
    \item A set of axial points, experimental runs identical to the centre
      points except for one factor, which will take on values both below and
      above the median of the two factorial levels, and typically both outside
      their range.
      All factors are varied in this way.
  \end{itemize}
  This methodology is useful for building a second order (quadratic) model for
  the response variable without needing to use a complete three-level factorial
  experiment.
\end{itemize}
All the parameters, needed for setting up the algorithms reported above, must be
defined within a \xmlNode{ResponseSurfaceDesign} block.
%
\attrIntro
\begin{itemize}
\itemsep0em
\item \nameDescription
\end{itemize}

\variableIntro{ResponseSurfaceDesign}
\begin{itemize}
\item \variableDescription
 \variableChildrenIntro
 \begin{itemize}
    \item \distributionDescription
    \item \functionDescription
     \item \gridDescriptionOnlyCustom
     \nb{Only the construction ``custom'' is available. In the \xmlNode{grid} body only the lower and upper bounds can be inputted (2 numbers only).}
 \end{itemize}
 \item \constantVariablesDescription
\item \xmlNode{ResponseSurfaceDesignSettings}, \xmlDesc{required},
In this sub-node, the user needs to specify different settings depending on the
algorithm being used:
 \begin{itemize}
  \item \xmlNode{algorithmType}, \xmlDesc{string, required field}, this XML node
    will contain the name of the algorithm to be used.
    Based on the chosen algorithm, other nodes need to be defined:
    \begin{itemize}
      \item \xmlNode{algorithmType}\texttt{BoxBehnken}\xmlNode{algorithmType/}. If Box-Behnken
        is specified, the following additional node is recognized:
     \begin{itemize}
      \item \xmlNode{ncenters}, \xmlDesc{integer, optional field}, the
        number of center points to include in the box.
        If this parameter is not specified, then a pre-determined number of
        points are automatically included.
        \default{Automatic Generation}.
     \end{itemize}
     \nb In order to employ the ``Box-Behnken'' design, at least 3 variables
     must be used.
     \item \xmlNode{algorithmType}\texttt{CentralComposite}\xmlNode{algorithmType/}. If
       Central Composite is specified, the following additional nodes will
       be recognized:
     \begin{itemize}
      \item \xmlNode{centers}, \xmlDesc{comma separated integers, optional
        field}, the number of center points to be included.
        This block needs to contain 2 integers values separated by a comma.
        The first entry represents the number of centers to be added for the
        factorial block; the second one is the one for the star block.
        \default{4,4}.
      \item \xmlNode{alpha}, \xmlDesc{string, optional field}, in this node,
        the user decides how an $\alpha$ factor needs to be determined.
        Two options are available:
        \begin{description}
          \item[\texttt{orthogonal}] for orthogonal design.
          \item[\texttt{rotatable}] for rotatable design.
        \end{description}
        \default{orthogonal}.
      \item \xmlNode{face}, \xmlDesc{string, optional field}, in this node,
        the user defines how faces should be constructed.
        Three options are available:
        \begin{description}
          \item[\texttt{circumscribed}] for circumscribed facing
          \item[\texttt{inscribed}] for inscribed facing
          \item[\texttt{faced}] for faced facing.
        \end{description}
        \default{circumscribed}.
     \end{itemize}
  \end{itemize}
  \nb In order to employ the ``Central Composite'' design, at least 2
  variables must be used.
\end{itemize}
\end{itemize}

Furthermore, if the covariance matrix is provided and the input parameters are assumed to have a multivariate normal distribution, one can use
\textbf{ResponseSurfaceDesign} approach to sample the input parameters in the transformed space (aka subspace, reduced space).
In this case, the user needs to provide additional information, i.e. the \xmlNode{transformation} under \xmlNode{MultivariateNormal} of \xmlNode{Distributions}
(more information can be found in Section \ref{subsec:NdDist}). In addition, the node \xmlNode{variablesTransformation} is also
required for \textbf{ResponseSurfaceDesign} sampling. This node is used to tranform the variables specified by \xmlNode{latentVariables} in the
transformed space of input into variables spefified by \xmlNode{manifestVariables} in the input space. The variables listed
in \xmlNode{latentVariables} should be predefined in \xmlNode{variable}, and the variables listed in \xmlNode{manifestVariables}
are used by the \xmlNode{Models}.

\variablesTransformationDescription{ResponseSurfaceDesign}
%\maljdan{Is it weird that one of these uses ncenters and the other uses centers?}
%\alfoa{The names of those parameters are different in order to avoid confusion, since the meaning (and the way ) to input them is different}

%\maljdan{This is the first example where type is an attribute and a node...This
%is confusing.}
%\alfoa{Changed.}

Example:
\begin{lstlisting}[style=XML,morekeywords={}]
<Samplers>
  ...
    <ResponseSurfaceDesign name='BoxBehnkenRespDesign'>
        <ResponseSurfaceDesignSettings>
            <algorithmType>BoxBehnken</algorithmType>
            <ncenters>1</ncenters>
        </ResponseSurfaceDesignSettings>
        <variable name='var1' >
            <distribution >Gauss1</distribution>
           <grid type='CDF' construction='custom'  >0.2 0.8</grid>
        </variable>
        <!-- N.B. at least 3 variables need to inputted
                in order to employ this algorithm
         -->
    </ResponseSurfaceDesign>
    <ResponseSurfaceDesign name='CentralCompositeRespDesign'>
        <ResponseSurfaceDesignSettings>
            <algorithmType>CentralComposite</algorithmType>
            <centers>1, 2</centers>
            <alpha>orthogonal</alpha>
            <face>circumscribed</face>
        </ResponseSurfaceDesignSettings>
        <variable name='var4' >
            <distribution >Gauss1</distribution>
            <grid type='CDF' construction='custom'  >0.2 0.8</grid>
        </variable>
        <!-- N.B. at least 2 variables need to inputted
                in order to employ this algorithm
         -->
    </ResponseSurfaceDesign>
    <ResponseSurfaceDesign name='transformedSpaceSampling'>
        <ResponseSurfaceDesignSettings>
            <algorithmType>BoxBehnken</algorithmType>
            <ncenters>1</ncenters>
        </ResponseSurfaceDesignSettings>
        <variable name='var1' >
            <distribution >Gauss1</distribution>
           <grid type='CDF' construction='custom'  >0.2 0.8</grid>
        </variable>
        ...
        <variablesTransformation model="givenModel">
          <latentVariables>var1,...</latentVariables>
          <manifestVariables>...</manifestVariables>
          <method>pca</method>
        </variablesTransformation>
    </ResponseSurfaceDesign>
  ...
</Samplers>
\end{lstlisting}

%%% Forward Samplers: Factorial Design
\subsubsection{Factorial Design}
\label{subsubsubsec:FactorialDOE}
The \textbf{Factorial Design} method is an important method to determine the
effects of multiple variables on a response.
%
A factorial design can reduce the number of samples one has to perform by
studying multiple factors simultaneously.
%
Additionally, it can be used to find both main effects (from each independent
factor) and interaction effects (when both factors must be used to explain the
outcome).
%
A factorial design tests all possible conditions.
%
Because factorial designs can lead to a large number of trials, which can
become expensive and time-consuming, they are best used for small numbers of
variables with only a few domain discretizations (1 to 3).
%
Factorial designs work well when interactions between variables are strong and
important and where every variable contributes significantly.
%
RAVEN currently employs three different algorithms that can be classified within
this family of techniques:
\begin{itemize}
  \item \textbf{General Full Factorial} explores the input space by
    investigating all possible combinations of a set of factors (variables).
  \item \textbf{2-Level Fractional-Factorial} consists of a carefully chosen
    subset (fraction) of the experimental runs of a full factorial design.
    %
    The subset is chosen so as to exploit the sparsity-of-effects principle
    exposing information about the most important features of the problem
    studied, while using a fraction of the effort of a full factorial design in
    terms of experimental runs and resources.
  \item \textbf{Plackett-Burman} identifies the most important factors early in
    the experimentation phase when complete knowledge about the system is
    usually unavailable.
    %
    It is an efficient screening method for identifying the active factors
    (variables) using as few samples as possible.
    %
    In Plackett-Burman designs, main effects have a complicated confounding
    relationship with two-factor interactions.
    %
    Therefore, these designs should be used to study main effects when it can be
    assumed that two-way interactions are negligible.
\end{itemize}
All the parameters needed for setting up the algorithms reported above must be
defined within a \xmlNode{FactorialDesign} block.
%
\attrIntro
\begin{itemize}
\itemsep0em
\item \nameDescription
\end{itemize}
\variableIntro{FactorialDesign}
\begin{itemize}
  \item \variableDescription
    \variableChildrenIntro
    \begin{itemize}
      \item \distributionDescription
      \item \functionDescription
      \item \gridDescription
    \end{itemize}
  \item \constantVariablesDescription
\end{itemize}

The main \xmlNode{FactorialDesign} block needs to contain an additional sub-node
called\\\xmlNode{FactorialSettings}.
%
In this sub-node, the user needs to specify different settings depending on the
algorithm being used:
   \begin{itemize}
    \item \xmlNode{algorithmType}, \xmlDesc{string, required field}, specifies the
      algorithm to be used.
      %
      Based on the chosen algorithm, other nodes may be defined:
      \begin{itemize}
        \item \xmlNode{algorithmType}\texttt{full}\xmlNode{algorithmType/}. Full factorial design.
          If \texttt{full} is specified, no additional nodes are necessary.
          \\
          \nb The full factorial design does not have any limitations on the
          number of discretization bins that can be used in the \xmlNode{grid}
          XML node for each \xmlNode{variable} specified.
        \item \xmlNode{algorithmType}\texttt{2levelFract}\xmlNode{algorithmType/}. Two-level
          Fractional-Factorial design.
          %
          If \\\texttt{2levelFract}  is specified, the following additional
          nodes must be specified:
          \begin{itemize}
            \item \xmlNode{gen}, \xmlDesc{space separated strings, required
              field}, specifies the confounding mapping.
              %
              For instance, in this block the user defines the decisions on a
              fraction of the full-factorial by allowing some of the factor main
              effects to be compounded with other factor interaction effects.
              %\maljdan{compounded?} \alfoa{Right Dan.}
              %
              This is done by defining an alias structure that defines,
              symbolically, these interactions.
              %
              These alias structures are written like “C = AB” or “I = ABC”, or
              “AB = CD”, etc.
              %
              These define how a column is related to the others.
            \item \xmlNode{genMap}, \xmlDesc{space separated strings, required
              field}, defines the mapping between the \xmlNode{gen} symbolic
              aliases and the variables that have been inputted in the
              \xmlNode{FactorialDesign} main block.
          \end{itemize}
          \nb The Two-levels Fractional-Factorial design is limited to 2
          discretization bins in the \xmlNode{grid} node for each
          \xmlNode{variable}.
       \item \xmlNode{algorithmType}\texttt{pb}\xmlNode{algorithmType/}. Plackett-Burman design.
         If \texttt{pb} is specified, no additional nodes are necessary.
         \\
         \nb The Plackett-Burman design does not have any limitations on the
         number of discretization bins allowed in the \xmlNode{grid} node for
         each \xmlNode{variable}.
      \end{itemize}

  \end{itemize}
Example:
\begin{lstlisting}[style=XML,morekeywords={construction,upperBound,steps}]
<Samplers>
  ...
  <FactorialDesign name='fullFactorial'>
    <FactorialSettings>
      <algorithmType>full</algorithmType>
    </FactorialSettings>
    <variable name='var1' >
      <distribution>aDistributionNameDefinedInDistributionBlock1
      </distribution>
      <grid type='value' construction='custom' >0.02 0.03 0.5</grid>
    </variable>
    <variable name='var2' >
      <distribution>aDistributionNameDefinedInDistributionBlock2
      </distribution>
      <grid type='CDF' construction='custom'>0.5 0.7 1.0</grid>
    </variable>
  </FactorialDesign>
  <FactorialDesign name='2levelFractFactorial'>
    <FactorialSettings>
      <algorithmType>2levelFract</algorithmType>
      <gen>a,b,ab</gen>
      <genMap>var1,var2,var3</genMap>
    </FactorialSettings>
    <variable name='var1' >
      <distribution>aDistributionNameDefinedInDistributionBlock3
      </distribution>
      <grid type='value' construction='custom' >0.02 0.5</grid>
    </variable>
    <variable name='var2' >
      <distribution>aDistributionNameDefinedInDistributionBlock
      </distribution>
      <grid type='CDF' construction='custom'>0.5 1.0</grid>
    </variable>
    <variable name='var3'>
      <distribution>aDistributionNameDefinedInDistributionBlock5
      </distribution>
      <grid type='value' upperBound='4' construction='equal' steps='1'>0.5</grid>
    </variable>
  </FactorialDesign>
  <FactorialDesign name='pbFactorial'>
    <FactorialSettings>
      <algorithmType>pb</algorithmType>
    </FactorialSettings>
    <variable name='var1' >
      <distribution>aDistributionNameDefinedInDistributionBlock6
      </distribution>
      <grid type='value' construction='custom' >0.02 0.5</grid>
    </variable>
    <variable name='VarGauss2' >
      <distribution>aDistributionNameDefinedInDistributionBlock7
      </distribution>
      <grid type='CDF' construction='custom'>0.5 1.0</grid>
    </variable>
  </FactorialDesign>
  ...
</Samplers>
\end{lstlisting}

%%% Forward Samplers: EnsembleForward
\subsubsection{Ensemble Forward Sampling strategy}
\label{subsubsubsec:EnsembleSampler}
The \textbf{Ensemble Forward} sampling approach allows the user to combine multiple Forward sampling strategies
into one single strategy. For example, it can happen that a variable is more suitable for a particular sampling strategy (e.g. a
stochastic event
modeled with a Monte Carlo approach) and a second variable is more suitable for another sampling method (e.g. because part of a parametric space modeled with a Grid-based approach).
\specBlock{a}{EnsembleForward}
%
\attrsIntro
\vspace{-5mm}
\begin{itemize}
\itemsep0em
\item \textbf{name}, \textit{required string attribute}, user-defined name of this Sampler. N.B. As for the other objects, this is the name that can be used to refer to this specific entity from other input blocks (xml);
\end{itemize}
\vspace{-5mm}

In the \textbf{EnsembleForward} input block, the user needs to specify the sampling strategies that he wants to combine together.
\\Currently, only the following strategies can be combined:
\begin{itemize}
  \item \xmlNode{MonteCarlo}
  \item \xmlNode{Grid}
  \item \xmlNode{Stratiefied}
  \item \xmlNode{FactorialDesign}
  \item \xmlNode{ResponseSurfaceDesign}
  \item \xmlNode{CustomSampler}
\end{itemize}
For each of the above samplers, the input specifications can be found in the relative sections.

Example:
\begin{lstlisting}[style=XML]
<Samplers>
  ...
    <EnsembleForward name="testEnsembleForward">
        <MonteCarlo name = "theMC">
            <samplerInit> <limit>4</limit> </samplerInit>
            <variable name="sigma">
                <distribution>norm</distribution>
            </variable>
        </MonteCarlo>
        <Grid name = "theGrid">
            <variable name="x0">
                <distribution>unif</distribution>
                <grid construction="custom" type="value">0.02 0.5 0.6</grid>
            </variable>
        </Grid>
        <Stratified name = "theStratified">
            <variable name="z0">
                <distribution>tri</distribution>
                <grid construction="equal" steps="2" type="CDF">0.2 0.8</grid>
            </variable>
            <variable name="y0">
                <distribution>unif</distribution>
                <grid construction="equal" steps="2" type="value">0.5 0.8</grid>
            </variable>
        </Stratified>
        <ResponseSurfaceDesign name = "theRSD">
            <ResponseSurfaceDesignSettings>
                <algorithmType>CentralComposite</algorithmType>
                <centers>1,2</centers>
                <alpha>orthogonal</alpha>
                <face>circumscribed</face>
            </ResponseSurfaceDesignSettings>
            <variable name="rho">
                <distribution>unif</distribution>
                <grid construction="custom" type="CDF">0.0 1.0</grid>
            </variable>
            <variable name="beta">
                <distribution>tri</distribution>
                <grid construction="custom" type="value">0.1 1.5</grid>
            </variable>
        </ResponseSurfaceDesign>
    </EnsembleForward>
  ...
</Samplers>
\end{lstlisting}

Care should be used when using deterministic random seeds for EnsembleForward sampling.  The EnsembleForward
sample will ignore any seeds set in any of its subset samplers; however, the global random seed can be set by
adding a \xmlNode{samplerInit} block with the \xmlNode{initialSeed} block therein, with an integer value
providing the seed.  For example,
\begin{lstlisting}[style=XML]
  <Samplers>
    ...
    <EnsembleForward name='testEnsembleForward'>
      <samplerInit>
        <initialSeed>42</initialSeed>
      </samplerInit>
    ...
     </EnsembleForward>
    ...
  </Samplers>
\end{lstlisting}
Because RAVEN has a single global random number generator, this will set the seed for the full calculation
when the Step containing a run using this ForwardSampler is begun.

Note also variables that are defined from functions, as well as constants, need to be defined outside the
samplers of the ensemble sampler. An example is shown below.

Example:
\begin{lstlisting}[style=XML]
  <Samplers>
    <EnsembleForward name='testEnsembleForward'>
      <variable name='x3'>
          <function>funct1</function>
      </variable>
      <variable name='x4,x5'>
          <function>funct2</function>
      </variable>
      <constant name='pi'>3.14159</constant>
      <MonteCarlo name='notNeeded'>
        <samplerInit>
          <limit>3</limit>
        </samplerInit>
        <variable name='x1'>
          <distribution>norm</distribution>
        </variable>
      </MonteCarlo>
      <Grid name='notNeeded'>
        <variable name='x2'>
          <distribution>unif</distribution>
          <grid construction='custom' type='value'>0.02 0.6</grid>
        </variable>
      </Grid>
     </EnsembleForward>
  </Samplers>
\end{lstlisting}

In this example note that:
\begin{itemize}
  \item variables $x1$ and $x2$ are generated by the two samplers (Monte-Carlo and Grid respectively)
  \item variable $x3$ is generated from the function $funct1$
  \item variables $x4$ and $x5$ are generated from the function $funct2$
  \item variables $x3$, $x4$ and $x5$ are defined outside the Monte-Carlo and Grid
\end{itemize}

%%% Forward Samplers: Custom Sampler
\subsubsection{Custom Sampling strategy}
\label{subsubsubsec:CustomSampler}
The \textbf{Custom} sampling approach allows the user to specify a predefined set of coordinates (in the input space) that RAVEN should use to inquire the model. For example, the user can provide a CSV file containing a list of samples that RAVEN should use.
\specBlock{a}{CustomSampler}
%
\attrsIntro
\vspace{-5mm}
\begin{itemize}
\itemsep0em
\item \textbf{name}, \textit{required string attribute}, user-defined name of this Sampler. N.B. As for the other objects, this is the name that can be used to refer to this specific entity from other input blocks (xml);
\end{itemize}
\vspace{-5mm}

In the \textbf{CustomSampler} input block, the user needs to specify the variables need to be sampled. As
already mentioned, these variables are inputted within consecutive XML blocks called \xmlNode{variable}.  Note
that if any variables are dependent on other dimensions (e.g. ``time''), the dependent dimensions need to be
listed as variables as well.

In addition, the \xmlNode{Source} from which the samples need to be retrieved needs to be specified:
\begin{itemize}
  \item \xmlNode{variable}, \xmlDesc{XML node,
    required parameter} can specify the following attribute:
    \begin{itemize}
      \item \xmlAttr{name}, \xmlDesc{required string attribute}, user-defined name of this variable.
      \item \xmlAttr{nameInSource}, \xmlDesc{optional string attribute}, name of the variable to read from in
        \xmlNode{Source}.  \default Same as \xmlAttr{name}.
      \item \shapeVariableDescription
    \end{itemize}
 \item \xmlNode{Source}, \xmlDesc{XML node,
  required parameter} will specify the following attributes:
  \begin{itemize}
    \item \xmlAttr{class}, \xmlDesc{required string attribute}, class entity of the source where the samples need to be retrieved from.
     It can be either \textbf{Files} or \textbf{DataObjects}.
     \item \xmlAttr{type}, \xmlDesc{required string attribute}, type of the source withing the previously explained ``class''.
      If \xmlAttr{class} is  \textbf{Files}, this attribute needs to be kept empty; otherwise it must be one
      of the \textbf{DataSet} objects: PointSet, HistorySet, or DataSet.
      \\ \nb If the \xmlNode{Source} \xmlAttr{class} is  \textbf{Files}, the File needs to be a standard CSV file, specified in the
      \xmlNode{Files} XML block in the RAVEN input.
      \\ In addition, it is important to notice that if in the \xmlNode{Source}  the \textbf{PointProbability} and
      \textbf{ProbabilityWeight} quantities are not found, the samples are assumed to come from a MonteCarlo (from a  statistical
      post-processing prospective).
  \end{itemize}
 \item \xmlNode{index}, \xmlDesc{comma-separated integer, optional parameter} indexes to use from the
   \xmlNode{Source}. If provided, then only the listed indexes will be used. Indexes are zero-based; that is,
   the first realization is indexed at 0, the second at 1, and so forth. Default is for all indices in the
   source to be used.
 \item \constantVariablesDescription
\end{itemize}

Example:
\begin{table}[h!]
\centering
\caption{samples.csv}
\begin{tabular}{ccccc}
\textbf{y}  & \textbf{x}  & \textbf{z}  & \textbf{PointProbability} & \textbf{ProbabilityWeight} \\
0.725675246 & 0.031099304 & 0.984988317 & 0.1                       & 0.2                        \\
0.565949127 & 0.028589754 & 1.13186372  & 0.1                       & 0.2                        \\
0.72567754  & 0.031099304 & 0.967209238 & 0.1                       & 0.2                        \\
0.565951633 & 0.028589754 & 1.111431662 & 0.1                       & 0.2                        \\
0.725968307 & 0.031100307 & 0.98498835  & 0.1                       & 0.2
\end{tabular}
\end{table}
\begin{lstlisting}[style=XML]
<Samplers>
  ...
  <Samplers>
    <CustomSampler name="customSamplerDataObject">
      <Source   class="DataObjects"  type="PointSet">outCustomSamplerFromFile</Source>
      <variable name="x"/>
      <variable name="y"/>
      <variable name="z"/>
    </CustomSampler>
  </Samplers>
  <Samplers>
    <CustomSampler name="customSamplerFile">
      <Source   class="Files"  type="">samples.csv</Source>
      <variable name="x"/>
      <variable name="y"/>
      <variable name="z"/>
    </CustomSampler>
  </Samplers>
  ...
</Samplers>
\end{lstlisting}

%%%%%%%%%%%%%%%%%%%%%%%%%%%%
%%% Dynamic Event Tree Samplers %%%
%%%%%%%%%%%%%%%%%%%%%%%%%%%%
\subsection{Dynamic Event Tree (DET) Samplers}
\label{subsec:DETSamplers}
The \textbf{Dynamic Event Tree} methodologies are designed to take the timing of
events explicitly into account, which can become very important especially when
uncertainties in complex phenomena are considered.
%
Hence, the main idea of this methodology is to let a system code determine the
pathway of an accident scenario within a probabilistic environment.
%
In this family of methods, a continuous monitoring of the system evolution in
the phase space is needed.
%
In order to use the DET-based methods, the generic driven code needs to have, at
least, an internal trigger system and, consequently, a ``restart'' capability.
%
In the RAVEN framework, 4 different DET samplers are available:
\begin{itemize}
\item \textbf{Dynamic Event Tree (DET)}
\item \textbf{Hybrid Dynamic Event Tree (HDET)}
\item \textbf{Adaptive Dynamic Event Tree (ADET)}
\item \textbf{Adaptive Hybrid Dynamic Event Tree (AHDET)}
\end{itemize}

The ADET and the AHDET methodologies represent a hybrid between the DET/HDET and adaptive sampling
approaches.
%
For this reason, its input requirements are reported in the Adaptive Samplers'
section (\ref{subsec:AdaptSamplers}).

%%%%%%%%% Dynamic Event Tree Samplers: Dynamic Event Tree
\subsubsection{Dynamic Event Tree}
\label{subsubsubsec:DET}
The \textbf{Dynamic Event Tree} sampling approach is a sampling strategy that is
designed to take the timing of events, in transient/accident scenarios,
explicitly into account.
%
From an application point of view, an $N$-Dimensional grid is built on the CDF
space.
%
A single simulation is spawned and a set of triggers is added to the system code
control logic.
%
Every time a trigger is activated (one of the CDF thresholds in the grid is
exceeded), a new set of simulations (branches) is spawned.
%
Each branch carries its conditional probability.
%
In the RAVEN code, the triggers are defined by specifying a grid using a
predefined discretization metric in the mode input space.
%
RAVEN provides two discretization metrics: 1) CDF, and 2) value.
%
Thus, the trigger thresholds can be entered either in probability or value
space.
%

\specBlock{a}{DynamicEventTree}
%
\attrsIntro
\begin{itemize}
  \itemsep0em
  \item \nameDescription
  \item \xmlAttr{printEndXmlSummary}, \xmlDesc{optional string/boolean attribute},
    controls the dumping of a ``summary'' of the DET performed into an external
    XML.
    %
    \default{False}.
  \item \xmlAttr{maxSimulationTime}, \xmlDesc{optional float attribute}, this
    attribute controls the maximum ``mission'' time of the simulation
    underneath.
    %
    \default{None}.
\end{itemize}
\variableIntro{DynamicEventTree}
\begin{itemize}
\item \variableDescription
  \variableChildrenIntro
  \begin{itemize}
    \item \distributionDescription
    \item \functionDescription
    \item \gridDescription
  \end{itemize}
  \item \constantVariablesDescription
\end{itemize}

Example:
\begin{lstlisting}[style=XML]
<Samplers>
  ...
  <DynamicEventTree name='DETname'>
    <variable name='var1'>
      <distribution>aDistributionNameDefinedInDistributionBlock1 </distribution>
      <grid type='value' construction='equal' steps='100' >1.0 201.0</grid>
    </variable>
    <variable name='var2'>
      <distribution>aDistributionNameDefinedInDistributionBlock2 </distribution>
      <grid type='CDF' construction='equal' steps='5'>0 1</grid>
    </variable>
    <variable name='var3'>
      <distribution>aDistributionNameDefinedInDistributionBlock3 </distribution>
      <grid type='value' construction='equal' steps='10' >11.0 21.0</grid>
    </variable>
    <variable name='var4'>
      <distribution>aDistributionNameDefinedInDistributionBlock4 </distribution>
      <grid type='CDF' construction='equal' steps='5' >0.0 1.0</grid>
    </variable>
    <variable name='var5'>
      <distribution>aDistributionNameDefinedInDistributionBlock5 </distribution>
      <grid type='value' construction='custom'>0.2 0.5 10.0</grid>
    </variable>
    <variable name='var6'>
      <distribution>aDistributionNameDefinedInDistributionBlock6 </distribution>
      <grid type='CDF' construction='custom'>0.2 0.5 1.0</grid>
    </variable>
  </DynamicEventTree>
  ...
</Samplers>
\end{lstlisting}

%%%%%%%%% Dynamic Event Tree Samplers: Hybrid Dynamic Event Tree
\subsubsection{Hybrid Dynamic Event Tree}
\label{subsubsubsec:HDET}
The \textbf{Hybrid Dynamic Event Tree} sampling approach is a sampling strategy
that represents an evolution of the Dynamic Event Tree method for the
simultaneous exploration of the epistemic and aleatory uncertain space.
%
In similar approaches, the uncertainties are generally treated by employing a
Monte-Carlo sampling approach (epistemic) and DET methodology (aleatory).
%
The HDET methodology, developed within the RAVEN code, can reproduce the
capabilities employed by this approach, but provides additional sampling
strategies to the user.
%
The epistemic or epistemic-like uncertainties can be sampled through the
following strategies:

\begin{itemize}
  \item Monte-Carlo;
  \item Grid sampling;
  \item Stratified (e.g., Latin Hyper Cube).
\end{itemize}

From a practical point of view, the user defines the parameters that need to be
sampled by one or more different approaches.
%
The HDET module samples those parameters creating an $N$-dimensional grid
characterized by all the possible combinations of the input space coordinates
coming from the different sampling strategies.
%
Each coordinate in the input space represents a separate and parallel standard
DET exploration of the uncertain domain.
%
The HDET methodology allows the user to explore the uncertain domain
employing the best approach for each variable kind.
%
The addition of a grid sampling strategy among the usable approaches allows the
user to perform a discrete parametric study under aleatory and epistemic
uncertainties.

Regarding the input requirements, the HDET sampler is a ``sub-type'' of the\\
\xmlNode{DynamicEventTree} sampler.
%
For this reason, its specifications must be defined within a
\xmlNode{DynamicEventTree} block.
%
\attrsIntro

\begin{itemize}
  \itemsep0em
  \item \nameDescription
  \item \xmlAttr{printEndXmlSummary}, \xmlDesc{optional string/boolean attribute},
    controls the dumping of a ``summary'' of the DET performed into an external
    XML.
    %
    \default{False}.
  \item \xmlAttr{maxSimulationTime}, \xmlDesc{optional float attribute}, this
    attribute controls the maximum ``mission'' time of the simulation
    underneath.
    %
    \default{None}.
\end{itemize}

\variableIntro{DynamicEventTree}

\begin{itemize}
  \item \variableDescription
  \variableChildrenIntro
  \begin{itemize}
    \item \distributionDescription
    \item \functionDescription
    \item \gridDescription
  \end{itemize}
 \item \constantVariablesDescription
\end{itemize}

In order to activate the \textbf{Hybrid Dynamic Event Tree}  sampler, the main
\xmlNode{DynamicEventTree} block needs to contain, at least, an additional
sub-node called \xmlNode{HybridSampler}.
%
As already mentioned, the user can combine the Monte-Carlo, Stratified, and Grid
approaches in order to create a ``pre-sampling'' $N$-dimensional grid, from
whose nodes a standard DET method is employed.
%
For this reason, the user can specify a maximum of three
\xmlNode{HybridSampler} sub-nodes (i.e. one for each of the available
Forward samplers).
%
This sub-node needs to contain the following attribute:
\begin{itemize}
  \item \xmlAttr{type}, \xmlDesc{required string attribute}, type of
    pre-sampling strategy to be used.
    %
    Available options are \xmlString{MonteCarlo}, \xmlString{Grid}, and
    \xmlString{Stratified}.
 \end{itemize}

Independent of the type of ``pre-sampler'' that has been specified, the
\xmlNode{HybridSampler} must contain the variables that need to be sampled.
%
As already mentioned, these variables are specified within consecutive
\xmlNode{variable} XML blocks:

\begin{itemize}
  \item \variableDescription
    \variableChildrenIntro
    \begin{itemize}
      \item \distributionDescription
      \item \functionDescription
    \end{itemize}
  \item \constantVariablesDescription
 \end{itemize}

If a pre-sampling strategy \xmlAttr{type} is either \xmlString{Grid} or
\xmlString{Stratified}, within the \xmlNode{variable} blocks, the user needs to
specify the sub-node \xmlNode{grid}.
%
As with the standard DET, the content of this XML node depends on the definition
of the associated attributes:
\begin{itemize}
\itemsep0em
\item \xmlAttr{type}, \xmlDesc{required string attribute}, user-defined
  discretization metric type:
  \begin{itemize}
    \item \xmlString{CDF}, the grid is going to be specified based on the
      cumulative distribution function probability thresholds
    \item \xmlString{value}, the grid is going to be provided using variable
      values.
  \end{itemize}
  \item \xmlAttr{construction}, \xmlDesc{required string attribute}, how the
    grid needs to be constructed, independent of its type (i.e. \xmlString{CDF}
    or \xmlString{value}).
\end{itemize}
\constructionGridDescription

Example:
\begin{lstlisting}[style=XML]
<Samplers>
  ...
  <DynamicEventTree name='HybridDETname' print_end_XML="True">
    <HybridSampler type='MonteCarlo' limit='2'>
      <variable name='var1' >
        <distribution>aDistributionNameDefinedInDistributionBlock1 </distribution>
      </variable>
      <variable name='var2' >
        <distribution>aDistributionNameDefinedInDistributionBlock2 </distribution>
        <grid type='CDF' construction='equal' steps='1' lowerBound='0.1'>0.1</grid>
      </variable>
    </HybridSampler>
    <HybridSampler type='Grid'>
      <!-- Point sampler way (directly sampling the variable) -->
      <variable name='var3' >
        <distribution>aDistributionNameDefinedInDistributionBlock3 </distribution>
        <grid type='CDF' construction='equal' steps='1' lowerBound='0.1'>0.1</grid>
      </variable>
      <variable name='var4' >
        <distribution>aDistributionNameDefinedInDistributionBlock4 </distribution>
        <grid type='CDF' construction='equal' steps='1' lowerBound='0.1'>0.1</grid>
      </variable>
    </HybridSampler>
    <HybridSampler type='Stratified'>
      <!-- Point sampler way (directly sampling the variable ) -->
      <variable name='var5' >
        <distribution>aDistributionNameDefinedInDistributionBlock5 </distribution>
        <grid type='CDF' construction='equal' steps='1' lowerBound='0.1'>0.1</grid>
      </variable>
      <variable name='var6' >
        <distribution>aDistributionNameDefinedInDistributionBlock6 </distribution>
        <grid type='CDF' construction='equal' steps='1' lowerBound='0.1'>0.1</grid>
      </variable>
    </HybridSampler>
    <!-- DYNAMIC EVENT TREE INPUT (it goes outside an inner block like HybridSamplerSettings) -->
      <Distribution name='dist7'>
        <distribution>aDistributionNameDefinedInDistributionBlock7 </distribution>
        <grid type='CDF' construction='custom'>0.1 0.8</grid>
      </Distribution>
  </DynamicEventTree>
  ...
</Samplers>
\end{lstlisting}

%%%%%%%%%%%%%%%%%%%%%%%%%
%%% Adaptive Samplers %%%
%%%%%%%%%%%%%%%%%%%%%%%%%
\subsection{Adaptive Samplers}
\label{subsec:AdaptSamplers}
The Adaptive Samplers' family provides the possibility to perform smart sampling
(also known as adaptive sampling) as an alternative to classical “Forward”
techniques.
%
The motivation is that system simulations are often computationally expensive,
time-consuming, and high dimensional with respect to the number of input
parameters.
%
Thus, exploring the space of all possible simulation outcomes is infeasible
using finite computing resources.
%
During simulation-based probabilistic risk analysis, it is important to discover
the relationship between a potentially large number of input parameters and the
output of a simulation using as few simulation trials as possible.

The description above characterizes a typical context for performing adaptive
sampling where a few observations are obtained from the simulation, a reduced
order model (ROM) is built to represent the simulation space, and new samples
are selected based on the model constructed.
%
The reduced order model (see section \ref{subsec:models_ROM}) is then updated
based on the simulation results of the sampled points.
%
In this way, an attempt is made to gain the most information possible with a
small number of carefully selected sample points, limiting the number of
expensive trials needed to understand features of the system space.
%

Currently, RAVEN provides support for the following adaptive algorithms:

\begin{itemize}
  \item Limit Surface Search
  \item Adaptive Monte Carlo
  \item Adaptive Dynamic Event Tree
  \item Adaptive Hybrid Dynamic Event Tree
  \item Adaptive Sparse Grid
  \item Adaptive Sobol Decomposition
\end{itemize}

In the following paragraphs, the input requirements and a small explanation of
the different sampling methods are reported.

%%% Adaptive Samplers: Adaptive Sampling for Limit Surface search
\subsubsection{Limit Surface Search}
\label{subsubsubsec:LimitSurfaceSearch}
The \textbf{Limit Surface Search} approach is an advanced methodology that employs
a smart sampling around transition zones that determine a change in the status
of the system (limit surface).
%
To perform such sampling, RAVEN uses ROMs for predicting, in the input space,
the location(s) of these transitions, in order to accelerate the exploration of
the input space in proximity of the limit surface.
%

\specBlock{an}{LimitSurfaceSearch}
%
\attrIntro

\begin{itemize}
  \itemsep0em
  \item \nameDescription
\end{itemize}

\variableIntro{LimitSurfaceSearch}

\begin{itemize}
    \item \variableChildrenIntro
    \begin{itemize}
      \item \distributionDescription
      \item \functionDescription
    \end{itemize}
\end{itemize}

In addition to the \xmlNode{variable} nodes, the main XML node
\xmlNode{Adaptive} needs to contain two supplementary sub-nodes:

\begin{itemize}
  \item \convergenceDescription
  \item \xmlNode{batchStrategy}, \xmlDesc{string, optional field}, defines how
    points should be selected within a batch of size $n$ where $n$ is given by
    the \xmlNode{maxBatchSize} parameter below.
    Four options are available:
    \begin{itemize}
       \item \xmlString{none} If this is specified then the
       \xmlNode{maxBatchSize} parameter below will be ignored and the
       functionality will replicate the LimitSurfaceSearch, in that the limit
       surface will be rebuilt and the points will be re-scored after each trial
       is completed.
       \item \xmlString{naive} The top $n$ candidates will be queued for
       adaptive sampling before retraining the limit surface and re-scoring the
       new candidate set.
       \item \xmlString{maxP} The topology of the limit surface given the
       scoring function values will be decomposed and the top $n$ highest
       topologically persistent features (local maxima) will be queued for
       adaptive sampling before retraining and re-scoring the new candidate set.
       \item \xmlString{maxV} The topology of the limit surface given the
       scoring function values will be decomposed and the top $n$ highest
       topological features (local maxima) will be queued for adaptive sampling
       before retraining and re-scoring the new candidate set.
    \end{itemize}
  \default{none}.
  \item \xmlNode{maxBatchSize}, \xmlDesc{integer, optional field}, specifies
  the number of points to select for adaptive sampling before retraining the
  limit surface and re-scoring the candidates. This is the equivalent of the
  $n$ parameter used in the \xmlNode{batchStrategy} description.
  \default{1}.
  \item \xmlNode{scoring}, \xmlDesc{string, optional field}, defines the scoring
    function to use on the candidate limit surface points in order to select the
    next adaptive point.
    Two options are available:
    \begin{itemize}
       \item \xmlString{distance} will scoring the candidate points by their
       distance to the closest realized point, in this way preference is given
       to unexplored regions of the limit surface.
       \item \xmlString{distancePersistence} augments the distance above by
       multiplying it with the inverse persistence of a candidate point which
       measures how many times the label of the candidate point has changed
       throughout the lifespan of the algorithm.
    \end{itemize}
  \default{distancePersistence}.
  \item \xmlNode{simplification}, \xmlDesc{float in the range [0,1], optional
  field}, specifies the percent of the scoring function range (on the candidate
  set) as the amount of topological simplification to do before extracting the
  topological features from the candidate set (local maxima). This only applies
  when the \xmlNode{batchStrategy} is set to \xmlString{maxP} or
  \xmlString{maxV}. Thus, one may end up with a batch size less than that
  specified by \xmlNode{maxBatchSize}.
  \default{0}.
  \item \xmlNode{thickness}, \xmlDesc{positive integer, optional field},
  specifies how much the limit surface should be expanded (in terms of grid
  distance) when constructing a candidate set. A value of 1 implies only the
  points bounding the limit surface.
  \default{1}.
  \item \xmlNode{threshold}, \xmlDesc{float in the range [0,1], optional field},
  once the candidates have been ranked and selected, before queueing them for
  adaptive sampling, this value is used to threshold any points whose score is
  less than this percentage of the scoring function range (on the candidate
  set). Thus, one may end up with a batch size less than that specified by
  \xmlNode{maxBatchSize}.
  \default{0}
  % Limit Surface Search Objects
  \item \assemblerDescription{LimitSurfaceSearch}
    \begin{itemize}
      \item \xmlNode{Function}, \xmlDesc{string, required field},  the
        body of this XML block needs to contain the name of an external
        function object defined within the \xmlNode{Functions} main block (see
        Section~\ref{sec:functions}).
        %
        This object represents the boolean function that defines the transition
        boundaries.
        %
        This function must implement a method called
        \texttt{\_\_residuumSign(self)}, that returns either -1 or 1, depending
        on the system conditions (see Section \ref{sec:functions}.
      \item \xmlNode{ROM}, \xmlDesc{, string, optional  field}, if used, the
        body of this XML node must contain the name of a ROM defined in the
        \xmlNode{Models} block (see Section~\ref{subsec:models_ROM}). The ROM
        here specified is going to be used as ``acceleration model'' to speed up the
        convergence of the sampling strategy. The \xmlNode{Target} XML node in the ROM
        input block (within the \xmlNode{Models} section) needs to match the name of the goal
        \xmlNode{Function} (e.g. if the goal function is named ``transitionIdentifier'', the \xmlNode{Target} of the
        ROM needs to report the same name: \xmlNode{Target}\textbf{transitionIdentifier}\xmlNode{Target}).
      \item \xmlNode{TargetEvaluation}, \xmlDesc{string, required field},
        represents the container where the system evaluations are stored.
        %
        From a practical point of view, this XML node must contain the name of
        a data object defined in the \xmlNode{DataObjects} block (see
        Section~\ref{sec:DataObjects}). The object here specified must be
        input as  \xmlNode{Output} in the Steps that employ this sampling strategy.
        %
        The Limit Surface Search sampling accepts ``DataObjects'' of type
        ``PointSet'' only.
    \end{itemize}
\end{itemize}

Example:
\begin{lstlisting}[style=XML,morekeywords={class,limit,subGridTol,weight,persistence}]
<Samplers>
  ...
  <LimitSurfaceSearch name='LSSName'>
    <ROM class='Models' type='ROM'>ROMname</ROM>
    <Function class='Functions' type='External' >FunctionName</Function>
    <TargetEvaluation class='DataObjects' type='PointSet'>DataName</TargetEvaluation>
    <Convergence limit='3000'  forceIteration='False' weight='CDF'  subGridTol='1e-4' persistence='5'>
      1e-2
    </Convergence>
    <variable name='var1'>
      <distribution>aDistributionNameDefinedInDistributionBlock1 </distribution>
    </variable>
    <variable name='var2'>
      <distribution>aDistributionNameDefinedInDistributionBlock2 </distribution>
    </variable>
    <variable name='var3'>
      <distribution>aDistributionNameDefinedInDistributionBlock3 </distribution>
    </variable>
  </LimitSurfaceSearch>
  ...
</Samplers>
\end{lstlisting}

Batch sampling Example:
\begin{lstlisting}[style=XML,morekeywords={class,limit,subGridTol,weight,persistence}]
<Samplers>
  ...
  <LimitSurfaceSearch name='LSBSName'>
    <ROM class='Models' type='ROM'>ROMname</ROM>
    <Function class='Functions' type='External' >FunctionName</Function>
    <TargetEvaluation class='DataObjects' type='PointSet'>DataName</TargetEvaluation>
    <Convergence limit='3000'  forceIteration='False' weight='CDF'  subGridTol='1e-4' persistence='5'>
      1e-2
    </Convergence>
    <scoring>distancePersistence</scoring>
    <batchStrategy>maxP</batchStrategy>
    <thickness>1</thickness>
    <maxBatchSize>4</maxBatchSize>
    <variable name='var1'>
      <distribution>aDistributionNameDefinedInDistributionBlock1 </distribution>
    </variable>
    <variable name='var2'>
      <distribution>aDistributionNameDefinedInDistributionBlock2 </distribution>
    </variable>
    <variable name='var3'>
      <distribution>aDistributionNameDefinedInDistributionBlock3 </distribution>
    </variable>
  </LimitSurfaceSearch>
  ...
</Samplers>
\end{lstlisting}

Associated External Python Module:
\begin{lstlisting}[language=python]
def __residuumSign(self):
  if self.whatEverValue < self.OtherValue :
    return  1
  else:
    return -1
\end{lstlisting}

%%% Adaptive Samplers: ADMC
\subsubsection{Adaptive Monte Carlo}
\label{subsubsubsec:ADMC}
The \xmlNode{AdaptiveMonteCarlo} approach is an extension of the \xmlNode{MonteCarlo} sampler.
However, instead of having a predefined number of samples, the \xmlNode{AdaptiveMonteCarlo} sampler continues sampling until the standard error of all the desired metrics are less than the specified tolerance.
%

\specBlock{an}{AdaptiveMonteCarlo}
%
\begin{itemize}
  \itemsep0em
  \item \nameDescription
\end{itemize}

\variableIntro{AdaptiveMonteCarlo}
\begin{itemize}
\item \variableDescription
  \variableChildrenIntro
  \begin{itemize}
    \item \distributionDescription
    \item \functionDescription
  \end{itemize}
  \item \constantVariablesDescription
\end{itemize}

 In addition to the \xmlNode{variable} nodes, the main
\xmlNode{AdaptiveMonteCarlo} node needs to contain the following supplementary
sub-nodes:

\begin{itemize}
  \item \convergenceDescriptionAMC
  \item \xmlNode{initialSeed}, \xmlDesc{integer, optional field},
  initial seeding of random number generator for Monte Carlo sampler. By default, RAVEN uses an internal static seed.
  \default{20021986}
  % Assembler Objects
  \item \assemblerDescription{AdaptiveMonteCarlo}
    \begin{itemize}
      \item \xmlNode{TargetEvaluation}, \xmlDesc{string, required field},
        represents the container where the system evaluations are stored.
        %
        From a practical point of view, this XML node must contain the name of
        a data object defined in the \xmlNode{DataObjects} block (see
        Section~\ref{sec:DataObjects}).
        %
        The adaptive sampling accepts ``DataObjects'' of type
        ``PointSet'' only.
    \end{itemize}
\end{itemize}


Example:
\begin{lstlisting}[style=XML]
<Samplers>
  ...
  <AdaptiveMonteCarlo name = 'AdaptiveName'>
    <TargetEvaluation class = 'DataObjects' type = 'PointSet'>DataName</TargetEvaluation>
    <Convergence>
      <forceIteration>False</forceIteration>
      <limit>30</limit>
      <persistence>6</persistence>
      <expectedValue prefix="mean" tol="1e-1">y1,y2</expectedValue>
      <sigma prefix="sigma" tol="6e-2">y1</sigma>
      <sigma prefix="sigma" tol="5e-2">y2</sigma>
    </Convergence>
    <variable name = 'var1'>
        <distribution>
         aDistributionNameDefinedInDistributionBlock1
        </distribution>
    </variable>
    <variable name = 'var2'>
        <distribution>
          aDistributionNameDefinedInDistributionBlock2
        </distribution>
    </variable>
    <variable name = 'var3'>
        <distribution>
          aDistributionNameDefinedInDistributionBlock3
        </distribution>
    </variable>
  </AdaptiveMonteCarlo>
  ...
</Samplers>
\end{lstlisting}

%%% Adaptive Samplers: ADET
\subsubsection{Adaptive Dynamic Event Tree}
\label{subsubsubsec:ADET}
The \textbf{Adaptive Dynamic Event Tree} approach is an advanced methodology
employing a smart sampling around transition zones that determine a change in
the status of the system (limit surface), using the support of a Dynamic Event
Tree methodology.
%
The main idea of the application of the previously explained adaptive sampling
approach to the DET comes from the observation that the DET, when evaluated from
a limit surface perspective, is intrinsically adaptive.
%
For this reason, it appears natural to use the DET approach to perform a
goal-function oriented pre-sampling of the input space.

RAVEN uses ROMs for predicting, in the input space,
the location(s) of these transitions, in order to accelerate the exploration of
the input space in proximity of the limit surface.

\specBlock{an}{AdaptiveDynamicEventTree}
%
\attrIntro

\begin{itemize}
  \itemsep0em
  \item \nameDescription
  \item \xmlAttr{printEndXmlSummary}, \xmlDesc{optional string/boolean attribute},
    this attribute controls the dumping of a ``summary'' of the DET performed in
    to an external XML.
    %
    \default{False}.
  \item \xmlAttr{maxSimulationTime}, \xmlDesc{optional float attribute}, this
    attribute controls the maximum ``mission'' time of the simulation
    underneath.
    %
    \default{None}.
  \item \xmlAttr{mode}, \xmlDesc{optional string attribute}, controls when the
    adaptive search needs to begin.
    %
    Two options are available:
    \begin{itemize}
       \item \xmlString{post}, if this option is activated, the sampler first
         performs a standard Dynamic Event Tree analysis. At end of it, it uses
         the outcomes to start the adaptive search in conjunction with the DET
         support.
       \item \xmlString{online}, if this option is activated, the adaptive
         search starts at the beginning, during the initial standard Dynamic
         Event Tree analysis.
         %
         Whenever a transition is detected, the
         \textbf{Adaptive Dynamic Event Tree} starts its goal-oriented search
         using the DET as support;
    \end{itemize}
      \default{post}.
  \item \xmlAttr{updateGrid}, \xmlDesc{optional boolean attribute}, if true,
    each adaptive request is going to update the meshing of the initial DET
    grid.
    %
    \default{True}.
\end{itemize}
\variableIntro{AdaptiveDynamicEventTree}
\begin{itemize}
\item \variableDescription
  \variableChildrenIntro
 \begin{itemize}
    \item \distributionDescription
    \item \functionDescription
    \item \gridDescription
  \end{itemize}
  \item \constantVariablesDescription
\end{itemize}

 In addition to the \xmlNode{variable} nodes, the main
\xmlNode{AdaptiveDynamicEventTree} node needs to contain two supplementary
sub-nodes:

\begin{itemize}
  \item \convergenceDescription
  % Assembler Objects
  \item \assemblerDescription{AdaptiveDynamicEventTree}
    \begin{itemize}
      \item \xmlNode{Function}, \xmlDesc{string, required field},  the
        body of this XML block needs to contain the name of an external
        function object defined within the \xmlNode{Functions} main block (see
        Section~\ref{sec:functions}).
        %
        This object represents the boolean function that defines the transition
        boundaries.
        %
        This function must implement a method called
        \texttt{\_\_residuumSign(self)}, that returns either -1 or 1, depending
        on the system conditions (see Section \ref{sec:functions}.
      \item \xmlNode{ROM}, \xmlDesc{, string, optional  field}, if used, the
        body of this XML node must contain the name of a ROM defined in the
        \xmlNode{Models} block (see Section~\ref{subsec:models_ROM}). The ROM
        here specified is going to be used as ``acceleration model'' to speed up the
        convergence of the sampling strategy. The \xmlNode{Target} XML node in the ROM
        input block (within the \xmlNode{Models} section) needs to match the name of the goal
        \xmlNode{Function} (e.g. if the goal function is named ``transitionIdentifier'', the \xmlNode{Target} of the
        ROM needs to report the same name: \xmlNode{Target}\textbf{transitionIdentifier}\xmlNode{Target}).
      \item \xmlNode{TargetEvaluation}, \xmlDesc{string, required field},
        represents the container where the system evaluations are stored.
        %
        From a practical point of view, this XML node must contain the name of
        a data object defined in the \xmlNode{DataObjects} block (see
        Section~\ref{sec:DataObjects}).
        %
        The adaptive sampling accepts ``DataObjects'' of type
        ``PointSet'' only.
    \end{itemize}
\end{itemize}


Example:
\begin{lstlisting}[style=XML]
<Samplers>
  ...
  <AdaptiveDynamicEventTree name = 'AdaptiveName'>
    <ROM class = 'Models' type = 'ROM'ROMname</ROM>
    <Function class = 'Functions' type = 'External'>FunctionName</Function>
    <TargetEvaluation class = 'DataObjects' type = 'PointSet'>DataName</TargetEvaluation>
    <Convergence limit = '3000' subGridTol= '0.001' forceIteration = 'False' weight = 'CDF' subGriTol='''1e-5' persistence = '5'>
      1e-2
    </Convergence>
    <variable name = 'var1'>
        <distribution>
         aDistributionNameDefinedInDistributionBlock1
        </distribution>
        <grid type='CDF' construction='custom'>0.1 0.8</grid>
    </variable>
    <variable name = 'var2'>
        <distribution>
          aDistributionNameDefinedInDistributionBlock2
        </distribution>
        <grid type='CDF' construction='custom'>0.1 0.8</grid>
    </variable>
    <variable name = 'var3'>
        <distribution>
          aDistributionNameDefinedInDistributionBlock3
        </distribution>
        <grid type='CDF' construction='custom'>0.1 0.8</grid>
    </variable>
  </AdaptiveDynamicEventTree>
  ...
</Samplers>
\end{lstlisting}

Associated External Python Module:
\begin{lstlisting}[language=python]
def __residuumSign(self):
  if self.whatEverValue < self.OtherValue:
    return  1
  else:
    return -1
\end{lstlisting}


%%% Adaptive Samplers: AHDET
\subsubsection{Adaptive Hybrid Dynamic Event Tree}
\label{subsubsubsec:AHDET}
The \textbf{Adaptive Hybrid Dynamic Event Tree} approach is an advanced methodology
employing a smart sampling around transition zones that determine a change in
the status of the system (limit surface), using the support of the Hybrid Dynamic Event
Tree methodology. Practically, this methodology represents a conjunction between the previously
described Adaptive DET and the Hybrid DET method for the treatment of the epistemic variables.

Regarding the input requirements, the AHDET sampler is a ``sub-type'' of the\\
\xmlNode{AdaptiveDynamicEventTree} sampler.
%
For this reason, its specifications must be defined within a
\xmlNode{AdaptiveDynamicEventTree} block.

\specBlock{an}{AdaptiveDynamicEventTree}
%
\attrIntro

\begin{itemize}
  \itemsep0em
  \item \nameDescription
  \item \xmlAttr{printEndXmlSummary}, \xmlDesc{optional string/boolean attribute},
    this attribute controls the dumping of a ``summary'' of the DET performed in
    to an external XML.
    %
    \default{False}.
  \item \xmlAttr{maxSimulationTime}, \xmlDesc{optional float attribute}, this
    attribute controls the maximum ``mission'' time of the simulation
    underneath.
    %
    \default{None}.
  \item \xmlAttr{mode}, \xmlDesc{optional string attribute}, controls when the
    adaptive search needs to begin.
    %
    Two options are available:
    \begin{itemize}
       \item \xmlString{post}, if this option is activated, the sampler first
         performs a standard Dynamic Event Tree analysis. At end of it, it uses
         the outcomes to start the adaptive search in conjunction with the DET
         support.
       \item \xmlString{online}, if this option is activated, the adaptive
         search starts at the beginning, during the initial standard Dynamic
         Event Tree analysis.
         %
         Whenever a transition is detected, the
         \textbf{Adaptive Dynamic Event Tree} starts its goal-oriented search
         using the DET as support;
    \end{itemize}
      \default{post}.
  \item \xmlAttr{updateGrid}, \xmlDesc{optional boolean attribute}, if true,
    each adaptive request is going to update the meshing of the initial DET
    grid.
    %
    \default{True}.
\end{itemize}

\variableIntro{AdaptiveDynamicEventTree}
\begin{itemize}
\item \variableDescription
  \variableChildrenIntro
 \begin{itemize}
    \item \distributionDescription
    \item \functionDescription
    \item \gridDescription
  \end{itemize}
  \item \constantVariablesDescription
\end{itemize}

In addition to the \xmlNode{variable} nodes, the main
\xmlNode{AdaptiveDynamicEventTree} node needs to contain two supplementary
sub-nodes:

\begin{itemize}
  \item \convergenceDescription
  % Assembler Objects
  \item \assemblerDescription{AdaptiveDynamicEventTree}
    \begin{itemize}
      \item \xmlNode{Function}, \xmlDesc{string, required field},  the
        body of this XML block needs to contain the name of an external
        function object defined within the \xmlNode{Functions} main block (see
        Section~\ref{sec:functions}).
        %
        This object represents the boolean function that defines the transition
        boundaries.
        %
        This function must implement a method called
        \texttt{\_\_residuumSign(self)}, that returns either -1 or 1, depending
        on the system conditions (see Section \ref{sec:functions}.
      \item \xmlNode{ROM}, \xmlDesc{, string, optional  field}, if used, the
        body of this XML node must contain the name of a ROM defined in the
        \xmlNode{Models} block (see Section~\ref{subsec:models_ROM}). The ROM
        here specified is going to be used as ``acceleration model'' to speed up the
        convergence of the sampling strategy. The \xmlNode{Target} XML node in the ROM
        input block (within the \xmlNode{Models} section) needs to match the name of the goal
        \xmlNode{Function} (e.g. if the goal function is named ``transitionIdentifier'', the \xmlNode{Target} of the
        ROM needs to report the same name: \xmlNode{Target}\textbf{transitionIdentifier}\xmlNode{Target}).
      \item \xmlNode{TargetEvaluation}, \xmlDesc{string, required field},
        represents the container where the system evaluations are stored.
        %
        From a practical point of view, this XML node must contain the name of
        a data object defined in the \xmlNode{DataObjects} block (see
        Section~\ref{sec:DataObjects}).
        %
        The adaptive sampling accepts ``DataObjects'' of type
        ``PointSet'' only.
    \end{itemize}
\end{itemize}

As it can be noticed, the basic specifications of the Adaptive Hybrid Dynamic Event Tree
method are consistent with the ones for the ADET methodology.
In order to activate the \textbf{Adaptive Hybrid Dynamic Event Tree}  sampler, the main
\xmlNode{AdaptiveDynamicEventTree} block needs to contain an additional
sub-node called \xmlNode{HybridSampler}.
This sub-node needs to contain the following attribute:
\begin{itemize}
  \item \xmlAttr{type}, \xmlDesc{required string attribute}, type of
    pre-sampling strategy to be used.
    %
    Up to now only one option is available:
    \begin{itemize}
      \item \xmlString{LimitSurface}. With this option, the epistemic variables here listed are going to be part of the LS search.
                                                        This means that the discretization of the domain of these variables is determined by the
                                                        \xmlNode{Convergece} node.
    \end{itemize}
 \end{itemize}
Independent of the type of HybridSampler that has been specified, the
\xmlNode{HybridSampler} must contain the variables that need to be sampled.
%
As already mentioned, these variables are specified within consecutive
\xmlNode{variable} XML blocks:
\begin{itemize}
  \item \variableDescription
    \variableChildrenIntro
    \begin{itemize}
      \item \distributionDescription
      \item \functionDescription
    \end{itemize}
  \item \constantVariablesDescription
 \end{itemize}


Example:
\begin{lstlisting}[style=XML]
<Samplers>
  ...
  <AdaptiveDynamicEventTree name = 'AdaptiveName'>
    <ROM class = 'Models' type = 'ROM'ROMname</ROM>
    <Function class = 'Functions' type = 'External'>FunctionName</Function>
    <TargetEvaluation class = 'DataObjects' type = 'PointSet'>DataName</TargetEvaluation>
    <Convergence limit = '3000' subGridTol= '0.001' forceIteration = 'False' weight = 'CDF' subGriTol='''1e-5' persistence = '5'>
      1e-2
    </Convergence>
    <HybridSampler type='LimitSurface'>
       <variable name = 'epistemicVar1'>
          <distribution>
            aDistributionNameDefinedInDistributionBlock1
          </distribution>
      </variable>
       <variable name = 'epistemicVar2'>
          <distribution>
            aDistributionNameDefinedInDistributionBlock2
          </distribution>
      </variable>
    </HybridSampler>
    <variable name = 'var1'>
        <distribution>
         aDistributionNameDefinedInDistributionBlock3
        </distribution>
        <grid type='CDF' construction='custom'>0.1 0.8</grid>
    </variable>
    <variable name = 'var2'>
        <distribution>
          aDistributionNameDefinedInDistributionBlock4
        </distribution>
        <grid type='CDF' construction='custom'>0.1 0.8</grid>
    </variable>
    <variable name = 'var3'>
        <distribution>
          aDistributionNameDefinedInDistributionBlock5
        </distribution>
        <grid type='CDF' construction='custom'>0.1 0.8</grid>
    </variable>

  </AdaptiveDynamicEventTree>
  ...
</Samplers>
\end{lstlisting}

Associated External Python Module:
\begin{lstlisting}[language=python]
def __residuumSign(self):
  if self.whatEverValue < self.OtherValue:
    return  1
  else:
    return -1
\end{lstlisting}


%%% Adaptive Samplers: Adaptive Sparse Grid Collocation
\subsubsection{Adaptive Sparse Grid}
\label{subsubsubsec:AdaptiveSparseGrid}
The \textbf{Adaptive Sparse Grid} approach is an advanced methodology that employs
an intelligent search for the most suitable sparse grid quadrature to characterize a model.
%
To perform such sampling, RAVEN adaptively builds an index set and generates sparse grids
in a similar manner to Sparse Grid Collocation samplers.  In each iterative step, the adaptive
index set determines the next possible quadrature orders to add in each dimension, and
determines the index set point that would offer the largest impact to one of the convergence
metrics.  This process continues until the total impact of all the potential index set points is
less than tolerance.  For many models, this function converges after fewer runs than a traditional
Sparse Grid Collocation sampling.  However, it should be noted that this algorithm fails
in the event that the partial derivative of the response surface with respect to any single
input dimension is zero at the origin of the input domain.  For example, the adaptive
algorithm fails for the model $f(x)=x\cdot y$.
%

\specBlock{an}{Adaptive Sparse Grid}
%
\attrIntro

\begin{itemize}
  \itemsep0em
  \item \nameDescription
\end{itemize}

\variableIntro{Adaptive Sparse Grid}

\begin{itemize}
  \item \variableDescription
    \variableChildrenIntro
    \begin{itemize}
      \item \distributionDescription
    \item \functionDescription
    \end{itemize}
    \item \constantVariablesDescription
\end{itemize}

In addition to the \xmlNode{variable} nodes, the main XML node
\xmlNode{AdaptiveSparseGrid} needs to contain the following supplementary sub-nodes:

\begin{itemize}
  \item \xmlNode{Convergence}, \xmlDesc{float, required field}, Convergence
    tolerance.
    %
    The meaning of this tolerance depends on the \xmlAttr{target} attribute of this node.
    \begin{itemize}
      \item \xmlAttr{target}, \xmlDesc{required string attribute}, the metric for convergence.
        The following metrics are available: \xmlString{variance}, which
        converges the sparse quadrature integration of the second moment of the model.%; and
        %\xmlString{coeffs}, which integrates the L2 norm of the coefficients of the polynomial
        %moments from a GaussPolynomialRom construction using the sparse grid.
        %
      \item \xmlAttr{maxPolyOrder}, \xmlDesc{optional integer attribute},
        limits the maximum size equivalent polynomial for any one dimension.
        %
        \default{10}.
      \item \xmlAttr{persistence}, \xmlDesc{optional integer attribute}, defines the number of
        index set points that are required to be found before calculation can exit.  Setting this to a higher
        value can help if the adaptive process is not finding significant indices on its own.
        %
        \default{2}.
    \end{itemize}
    In summary, this XML node contains the information that is needed in order
    to control this sampler's convergence criterion.
  \item \convergenceStudyDescription
  \item \xmlNode{logFile}, \xmlDesc{optional node},
    if included, the log file onto which the adaptive step progress can be printed.  The log includes the
    values of included polynomial coefficients as well as the expected impacts of polynomial coefficients not
    yet included.  This is different from
    the convergenceStudy print, which will give statistical moments at certain steps.
  \item \xmlNode{maxRuns}, \xmlDesc{optional node},
    if included, the adaptive sampler will track the number of computational solves necessary to construct the
    associated GaussPolynomialROM.  If at any point the number of solves exceeds the value given, it will not
    initiate any additional solves, and will exit when existing solves finish.
\end{itemize}
  % Adaptive Sparse Grid Objects
  %\assemblerDescription{Adaptive Sparse Grid}
  %\ROMDescription{Adaptive Sparse Grid}
  \assemblerDescription{Adaptive Sparse Grid}
       \ROMDescription{Adaptive Sparse Grid}
        %
        \begin{itemize}
      \item \xmlNode{TargetEvaluation}, \xmlDesc{string, required field},
        represents the container where the system evaluations are stored.
        %
        From a practical point of view, this XML node must contain the name of
        a data object defined in the \xmlNode{DataObjects} block (see
        Section~\ref{sec:DataObjects}).
        %
        The Adaptive Sparse Grid sampling accepts ``DataObjects'' of type
        ``PointSet'' only.
   % \end{itemize}
\end{itemize}

Example:
\begin{lstlisting}[style=XML,morekeywords={class,limit,subGridTol,weight,persistence}]
<Samplers>
  ...
  <AdaptiveSparseGrid name="ASG" verbosity='debug'>
    <Convergence target='coeffs'>1e-2</Convergence>
    <variable name="x1">
      <distribution>UniDist</distribution>
    </variable>
    <variable name="x2">
      <distribution>UniDist</distribution>
    </variable>
    <ROM class = 'Models' type = 'ROM'>gausspolyrom</ROM>
    <TargetEvaluation class = 'DataObjects' type = 'PointSet'>solns</TargetEvaluation>
  </AdaptiveSparseGrid>
  ...
</Samplers>
\end{lstlisting}

Like in the \textbf{SparseGridCollocation} sampler, if the covariance matrix is provided
and the input parameters are assumed to have the multivariate normal distribution, the \textbf{AdaptiveSparseGrid} can be also used.
This means one creates the sparse grids of variables listed by \xmlNode{latentVariables} in the transformed space. If this is
the case, the user needs to provide additional information, i.e. the \xmlNode{transformation} under \xmlNode{MultivariateNormal}
of \xmlNode{Distributions} (more information can be found in Section \ref{subsec:NdDist}). In addition, the node
\xmlNode{variablesTransformation} is also required for \textbf{AdaptiveSparseGrid} sampler. This node is used to tranform
the variables specified by \xmlNode{latentVariables} in the transformed space of input into variables spefified by
\xmlNode{manifestVariables} in the input space. The variables listed in \xmlNode{latentVariables} should be predefined
in \xmlNode{variable}, and the variables listed in \xmlNode{manifestVariables}
are used by the \xmlNode{Models}.

\variablesTransformationDescription{AdaptiveSparseGrid}


\begin{lstlisting}[style=XML,morekeywords={ND,grid}]
...
<Models>
    ...
    <ExternalModel ModuleToLoad="lorentzAttractor_noK" name="PythonModule" subType="">
        <variables>sigma,rho,beta,x,y,z,time,x0,y0,z0</variables>
    </ExternalModel>
    <ROM name="gausspolyrom" subType="GaussPolynomialRom">
        <Target>ans</Target>
        <Features>x1,y1,z1</Features>
        <IndexSet>TensorProduct</IndexSet>
        <PolynomialOrder>1</PolynomialOrder>
    </ROM>
    ...
</Models>

<Distributions>
    ...
    <MultivariateNormal name='MVNDist' method='pca'>
        <transformation>
            <rank>3</rank>
        </transformation>
        <mu>0.0 1.0 2.0</mu>
        <covariance type="abs">
            1.0       0.6      -0.4
            0.6       1.0      0.2
            -0.4      0.2      0.8
        </covariance>
    </MultivariateNormal>
    ...
</Distributions>

<Samplers>
  ...
  <AdaptiveSparseGrid name='ASC'>
        <variable name='x0'>
            <distribution dim='1'>MVNDist</distribution>
        </variable>
        <variable name='y0'>
            <distribution dim='2'>MVNDist</distribution>
        </variable>
        <variable name='z0'>
            <distribution dim='3'>MVNDist</distribution>
        </variable>
        <variablesTransformation model="PythonModule">
            <latentVariables>x1,y1,z1</latentVariables>
            <manifestVariables>x0,y0,z0</manifestVariables>
            <method>pca</method>
        </variablesTransformation>
        <ROM class = 'Models' type = 'ROM'>gausspolyrom</ROM>
        <TargetEvaluation class = 'DataObjects' type = 'PointSet'>solns</TargetEvaluation>
  </AdaptiveSparseGrid>
  ...
</Samplers>
...
\end{lstlisting}

\subsubsection{Adaptive Sobol Decomposition}
\label{subsubsubsec:AdaptiveSobol}
The \textbf{Adaptive Sobol Decomposition} approach is an advanced methodology that decomposes an uncertainty
space into subsets and adaptively includes the most influential ones.  For example, for a response function
$f(a,b,c)$, the full list of subsets include $(a), (b), (c), (a,b), (a,c), (b,c), (a,b,c)$.  A Gauss Polynomial ROM is
constructed for each included subset using the Adaptive Sparse Grid sampler.  The importance of each subset is
estimated based on the importance of preceding subsets; that is, the impact of $(a,b)$ on the representation
of $f$ is estimated using the impact of $(a)$ and $(b)$.  Because of the excellent performance of Gauss
Polynomial ROMs for small-dimension spaces, this sampler used to construct an HDMR ROM can be very efficient.
Note that the ROM specified for this sampler \emph{must} be an HDMRRom specified in the Models block.
%

\specBlock{an}{Adaptive Sobol}
%
\attrIntro

\begin{itemize}
  \itemsep0em
  \item \nameDescription
\end{itemize}

\variableIntro{Adaptive Sobol}

\begin{itemize}
  \item \variableDescription
    \variableChildrenIntro
    \begin{itemize}
      \item \distributionDescription
    \item \functionDescription
    \end{itemize}
    \item \constantVariablesDescription
\end{itemize}

In addition to the \xmlNode{variable} nodes, the main XML node
\xmlNode{AdaptiveSobol} needs to contain the following supplementary sub-nodes:

\begin{itemize}
  \item \xmlNode{Convergence}, \xmlDesc{required node}, Convergence
    properties.
    This node contains the following properties that can be set by sub-nodes:
    %
    \begin{itemize}
      \item \xmlNode{relTolerance}, \xmlDesc{required float}, the relative tolerance to converge.
        This will compare to the estimate of subset polynomial errors and additional subset polynomials over
        the variance of the expansion so far to determine convergence.
      \item \xmlNode{maxRuns}, \xmlDesc{optional integer field},
        a limit for the number of model calls.  Once this limit is reached, no additional subsets
        will be generated or considered; however, existing subsets will continue to be trained.  If not
        specified, no limit on solves is imposed.
      \item \xmlNode{maxSobolOrder}, \xmlDesc{optional integer field},
        the largest polynomials orders to use in subset GaussPolynomialRom objects.  If specified, polynomial
        indices with a value larger than the value given will be rejected during adaptive construction.
      \item \xmlNode{progressParam}, \xmlDesc{optional float field}, a favoritism parameter ranging between
        0 and 2.  At 0, the algorithm will always prefer adding polynomials to adding new subsets in the HDMR
        expansion.  At 2, the opposite is true.  Default is 1.
      \item \xmlNode{logFile}, \xmlDesc{optional string field},
        a file to which adaptive progress is recorded.  If specified, each adaptive step will trigger printing
        progress to the file given, including the estimated error at the step, the next adaptive step to take,
        the coefficient of each polynomial within each gPC expansion, and the actual and expected Sobol
        sensitivities of each HDMR subset. Default is no printing.
      \item \xmlNode{subsetVerbosity}, \xmlDesc{optional string field}, the verbosity for components
        constructed during the adaptive HDMR process.  Options are \emph{silent}, \emph{quiet}, \emph{all}, or
        \emph{debug}, in order of
        verbosity.  If an invalid entry is provided, will resort to default.  Default is \emph{quiet}.
        %
    \end{itemize}
    In summary, this XML node contains the information that is needed in order
    to control this sampler's convergence criterion.
  \item \convergenceStudyDescription
  Like the \textbf{Sobol}, if multivariate normal distribution is provided, the following node need to be specified:
  \item \variablesTransformationDescription{AdaptiveSobol}

\end{itemize}
  % Adaptive Sobol
  \assemblerDescription{AdaptiveSobol}
       \ROMDescription{AdaptiveSobol}
        %
        \begin{itemize}
      \item \xmlNode{TargetEvaluation}, \xmlDesc{string, required field},
        represents the container where the system evaluations are stored.
        %
        From a practical point of view, this XML node must contain the name of
        a data object defined in the \xmlNode{DataObjects} block (see
        Section~\ref{sec:DataObjects}).
        %
        The Adaptive Sobol sampling accepts ``DataObjects'' of type
        ``PointSet'' only.
   % \end{itemize}
\end{itemize}

Example:
\begin{lstlisting}[style=XML,morekeywords={class,limit,subGridTol,weight,persistence}]
<Samplers>
  ...
  <AdaptiveSobol name="AS" verbosity='debug'>
    <Convergence>
      <relTolerance>1e-5</relTolerance>
      <maxRuns>150</maxRuns>
      <maxSobolOrder>3</maxSobolOrder>
      <progressParam>1</progressParam>
      <logFile>progress.txt</logFile>
      <subsetVerbosity>silent</subsetVerbosity>
    </Convergence>
    <variable name="x1">
      <distribution>UniDist</distribution>
    </variable>
    <variable name="x2">
      <distribution>UniDist</distribution>
    </variable>
    <ROM class = 'Models' type = 'ROM'>hdmrrom</ROM>
    <TargetEvaluation class = 'DataObjects' type = 'PointSet'>solns</TargetEvaluation>
  </AdaptiveSobol>
  ...
</Samplers>
\end{lstlisting}


   \section{Optimizers} \label{sec:Optimizers} The optimizer is another important entity in the
  RAVEN framework. It performs the driving of a specific ``goal function'' or ``objective function''
  over the model for value optimization. The Optimizer can be used almost anywhere a Sampler can be
  used, and is only distinguished from other AdaptiveSampler strategies for clarity.

\subsection{GradientDescent}
  The \xmlNode{GradientDescent} optimizer represents an a la carte option
  for performing gradient-based optimization with a variety of gradient
  estimation techniques, stepping strategies, and acceptance criteria. \hspace{12pt}
  Gradient descent optimization generally behaves as a ball rolling down a hill;
  the algorithm estimates the local gradient at a point, and attempts to move
  ``downhill'' in the opposite direction of the gradient (if minimizing; the
  opposite if maximizing). Once the lowest point along the iterative gradient search
  is discovered, the algorithm is considered converged. \hspace{12pt}
  Note that gradient descent algorithms are particularly prone to being trapped
  in local minima; for this reason, depending on the model, multiple trajectories
  may be needed to obtain the global solution.
\vspace{7pt} \\When used as part of a \xmlNode{MultiRun} step, this entity provides
        additional information through the \xmlNode{SolutionExport} DataObject. The
        following variables can be requested within the \xmlNode{SolutionExport}:
        \begin{itemize}
          \item \texttt{trajID}: integer identifier for different optimization starting locations and paths
             \item \texttt{iteration}: integer identifying which iteration (or step, or generation) a trajectory is on
             \item \texttt{accepted}: string acceptance status of the potential optimal point (algorithm dependent)
             \item \texttt{\{VAR\}}: any variable from the \xmlNode{TargetEvaluation} input or output; gives the value of that variable at the optimal candidate for this iteration.
             \item \texttt{stepSize}: the size of step taken in the normalized input space to arrive at each optimal point
             \item \texttt{conv\_\{CONV\}}: status of each given convergence criteria
             \item \texttt{CG\_task}: for ConjugateGradient, current task of line search. FD suggests continuing the search, and CONV indicates the line search converged and will pivot.
           
         \end{itemize}

  The \xmlNode{GradientDescent} node recognizes the following parameters:
    \begin{itemize}
      \item \xmlAttr{verbosity}: \xmlDesc{[silent, quiet, all, debug], optional}, 
        Desired verbosity of messages coming from this entity
      \item \xmlAttr{name}: \xmlDesc{string, required}, 
        User-defined name to designate this entity in the RAVEN input file.
  \end{itemize}

  The \xmlNode{GradientDescent} node recognizes the following subnodes:
  \begin{itemize}
    \item \xmlNode{objective}: \xmlDesc{string}, 
      Name of the response variable (or ``objective function'') that should be optimized
      (minimized or maximized).

    \item \xmlNode{variable}:
      defines the input space variables to be sampled through various means.
      The \xmlNode{variable} node recognizes the following parameters:
        \begin{itemize}
          \item \xmlAttr{name}: \xmlDesc{string, optional}, 
            user-defined name of this Sampler. \nb As for the other objects,               this is
            the name that can be used to refer to this specific entity from other input blocks
          \item \xmlAttr{shape}: \xmlDesc{comma-separated integers, optional}, 
            determines the number of samples and shape of samples               to be taken.  For
            example, \xmlAttr{shape}=``2,3'' will provide a 2 by 3               matrix of values,
            while \xmlAttr{shape}=``10'' will produce a vector of 10 values.               Omitting
            this optional attribute will result in a single scalar value instead.               Each
            of the values in the matrix or vector will be the same as the single sampled value.
            \nb A model interface must be prepared to handle non-scalar inputs to use this option.
      \end{itemize}

      The \xmlNode{variable} node recognizes the following subnodes:
      \begin{itemize}
        \item \xmlNode{distribution}: \xmlDesc{string}, 
          name of the distribution that is associated to this variable.               Its name needs
          to be contained in the \xmlNode{Distributions} block explained               in Section
          \ref{sec:distributions}. In addition, if NDDistribution is used,               the
          attribute \xmlAttr{dim} is required. \nb{Alternatively, this node must be omitted
          if the \xmlNode{function} node is supplied.}
          The \xmlNode{distribution} node recognizes the following parameters:
            \begin{itemize}
              \item \xmlAttr{dim}: \xmlDesc{integer, optional}, 
                for an NDDistribution, indicates the dimension within the NDDistribution that
                corresponds               to this variable.
          \end{itemize}

        \item \xmlNode{function}: \xmlDesc{string}, 
          name of the function that               defines the calculation of this variable from
          other distributed variables.  Its name               needs to be contained in the
          \xmlNode{Functions} block explained in Section               \ref{sec:functions}. This
          function must implement a method named ``evaluate''.               \nb{Each
          \xmlNode{variable} must contain only one \xmlNode{Function} or
          \xmlNode{Distribution}, but not both.}

        \item \xmlNode{initial}: \xmlDesc{comma-separated floats}, 
          indicates the initial values where independent trajectories for this optimization
          effort should begin. The number of entries should be the same for all variables, unless
          a variable is initialized with a sampler (see \xmlNode{samplerInit} below). Note these
          entries are ordered; that is, if the optimization variables are $x$ and $y$, and the
          initial               values for $x$ are \xmlString{1, 2, 3, 4} and initial values for $y$
          are \xmlString{5, 6, 7, 8},               then there will be four starting trajectories
          beginning at the locations (1, 5), (2, 6),               (3, 7), and (4, 8).
      \end{itemize}

    \item \xmlNode{TargetEvaluation}: \xmlDesc{string}, 
      name of the DataObject where the sampled outputs of the Model will be collected.
      This DataObject is the means by which the sampling entity obtains the results of requested
      samples, and so should require all the input and output variables needed for adaptive
      sampling.
      The \xmlNode{TargetEvaluation} node recognizes the following parameters:
        \begin{itemize}
          \item \xmlAttr{class}: \xmlDesc{string, required}, 
            RAVEN class for this entity (e.g. Samplers, Models, DataObjects)
          \item \xmlAttr{type}: \xmlDesc{string, required}, 
            RAVEN type for this entity; a subtype of the class (e.g. MonteCarlo, Code, PointSet)
      \end{itemize}

    \item \xmlNode{samplerInit}:
      collection of nodes that describe the initialization of the optimization algorithm.

      The \xmlNode{samplerInit} node recognizes the following subnodes:
      \begin{itemize}
        \item \xmlNode{limit}: \xmlDesc{integer}, 
          limits the number of Model evaluations that may be performed as part of this optimization.
          For example, a limit of 100 means at most 100 total Model evaluations may be performed.

        \item \xmlNode{writeSteps}: \xmlDesc{[final, every]}, 
          delineates when the \xmlNode{SolutionExport} DataObject should be written to. In case
          of \xmlString{final}, only the final optimal solution for each trajectory will be written.
          In case of \xmlString{every}, the \xmlNode{SolutionExport} will be updated with each
          iteration               of the Optimizer.

        \item \xmlNode{initialSeed}: \xmlDesc{integer}, 
          seed for random number generation. Note that by default RAVEN uses an internal seed,
          so this seed must be changed to observe changed behavior. \default{RAVEN-determined}

        \item \xmlNode{type}: \xmlDesc{[min, max]}, 
          the type of optimization to perform. \xmlString{min} will search for the lowest
          \xmlNode{objective} value, while \xmlString{max} will search for the highest value.
      \end{itemize}

    \item \xmlNode{gradient}:
      a required node containing the information about which gradient approximation algorithm to
      use, and its settings if applicable. Exactly one of the gradient approximation algorithms
      below may be selected for this Optimizer.

      The \xmlNode{gradient} node recognizes the following subnodes:
      \begin{itemize}
        \item \xmlNode{FiniteDifference}:
          if node is present, indicates that gradient approximation should be performed
          using Finite Difference approximation. Finite difference makes use of orthogonal
          perturbations         in each dimension of the input space to estimate the local gradient,
          requiring a total of $N$         perturbations, where $N$ is dimensionality of the input
          space. For example, if the input space         $\mathbf{i} = (x, y, z)$ for objective
          function $f(\mathbf{i})$, then FiniteDifference chooses         three perturbations
          $(\alpha, \beta, \gamma)$ and evaluates the following perturbation points:
          \begin{itemize}           \item $f(x+\alpha, y, z)$,           \item $f(x, y+\beta, z)$,
          \item $f(x, y, z+\gamma)$         \end{itemize}         and evaluates the gradient $\nabla
          f = (\nabla^{(x)} f, \nabla^{(y)} f, \nabla^{(z)} f)$ as         \begin{equation*}
          \nabla^{(x)}f \approx \frac{f(x+\alpha, y, z) - f(x, y, z)}{\alpha},
          \end{equation*}         and so on for $ \nabla^{(y)}f$ and $\nabla^{(z)}f$.

        \item \xmlNode{CentralDifference}:
          if node is present, indicates that gradient approximation should be performed
          using Central Difference approximation. Central difference makes use of pairs of
          orthogonal perturbations         in each dimension of the input space to estimate the
          local gradient, requiring a total of $2N$         perturbations, where $N$ is
          dimensionality of the input space. For example, if the input space         $\mathbf{i} =
          (x, y, z)$ for objective function $f(\mathbf{i})$, then CentralDifference chooses
          three perturbations $(\alpha, \beta, \gamma)$ and evaluates the following perturbation
          points:         \begin{itemize}           \item $f(x\pm\alpha, y, z)$,           \item
          $f(x, y\pm\beta, z)$,           \item $f(x, y, z\pm\gamma)$         \end{itemize}
          and evaluates the gradient $\nabla f = (\nabla^{(x)} f, \nabla^{(y)} f, \nabla^{(z)} f)$
          as         \begin{equation*}           \nabla^{(x)}f \approx \frac{f(x+\alpha, y, z) -
          f(x-\alpha, y, z)}{2\alpha},         \end{equation*}         and so on for $
          \nabla^{(y)}f$ and $\nabla^{(z)}f$.

        \item \xmlNode{SPSA}:
          if node is present, indicates that gradient approximation should be performed
          using the Simultaneous Perturbation Stochastic Approximation (SPSA).         SPSA makes
          use of a single perturbation as a zeroth-order gradient approximation,         requiring
          exactly $1$         perturbation regardless of the dimensionality of the input space. For
          example, if the input space         $\mathbf{i} = (x, y, z)$ for objective function
          $f(\mathbf{i})$, then SPSA chooses         a single perturbation point $(\epsilon^{(x)},
          \epsilon^{(y)}, \epsilon^{(z)})$ and evaluates         the following perturbation point:
          \begin{itemize}           \item $f(x+\epsilon^{(x)}, y+\epsilon^{(y)}, z+\epsilon^{(z)})$
          \end{itemize}         and evaluates the gradient $\nabla f = (\nabla^{(x)} f, \nabla^{(y)}
          f, \nabla^{(z)} f)$ as         \begin{equation*}           \nabla^{(x)}f \approx
          \frac{f(x+\epsilon^{(x)}, y+\epsilon^{(y)}, z+\epsilon^{(z)})) -               f(x, y,
          z)}{\epsilon^{(x)}},         \end{equation*}         and so on for $ \nabla^{(y)}f$ and
          $\nabla^{(z)}f$. This approximation is much less robust         than FiniteDifference or
          CentralDifference, but has the benefit of being dimension agnostic.
      \end{itemize}

    \item \xmlNode{stepSize}:
      a required node containing the information about which iterative stepping algorithm to
      use, and its settings if applicable. Exactly one of the stepping algorithms
      below may be selected for this Optimizer.

      The \xmlNode{stepSize} node recognizes the following subnodes:
      \begin{itemize}
        \item \xmlNode{GradientHistory}:
          if this node is present, indicates that the iterative steps in the gradient
          descent algorithm should be determined by the sequential change in gradient. In
          particular, rather         than using the magnitude of the gradient to determine step
          size, the directional change of the         gradient versor determines whether to take
          larger or smaller steps. If the gradient in two successive         steps changes
          direction, the step size shrinks. If the gradient instead continues in the same
          direction, the step size grows. The rate of shrink and growth are controlled by the
          \xmlNode{shrinkFactor}         and \xmlNode{growthFactor}. Note these values have a large
          impact on the optimization path taken.         Large growth factors converge slowly but
          explore more of the input space; large shrink factors         converge quickly but might
          converge before arriving at a local minimum.

          The \xmlNode{GradientHistory} node recognizes the following subnodes:
          \begin{itemize}
            \item \xmlNode{growthFactor}: \xmlDesc{float}, 
              specifies the rate at which the step size should grow if the gradient continues in
              same direction through multiple iterative steps. For example, a growth factor of 2
              means               that if the gradient is identical twice, the step size is doubled.
              \default{1.25}

            \item \xmlNode{shrinkFactor}: \xmlDesc{float}, 
              specifies the rate at which the step size should shrink if the gradient changes
              direction through multiple iterative steps. For example, a shrink factor of 2 means
              that if the gradient completely flips direction, the step size is halved. Note that
              for               stochastic surfaces or low-order gradient approximations such as
              SPSA, a small value               for the shrink factor is recommended. If an
              optimization path appears to be converging               early, increasing the shrink
              factor might improve the search. \default{1.15}
          \end{itemize}

        \item \xmlNode{ConjugateGradient}:
          Base class for Step Manipulation algorithms in the GradientDescent Optimizer.
      \end{itemize}

    \item \xmlNode{acceptance}:
      a required node containing the information about the acceptability criterion for iterative
      optimization steps, i.e. when a potential new optimal point should be rejected and when
      it can be accepted. Exactly one of the acceptance criteria               below may be selected
      for this Optimizer.

      The \xmlNode{acceptance} node recognizes the following subnodes:
      \begin{itemize}
        \item \xmlNode{Strict}:
          if this node is present, indicates that a Strict acceptance policy for         potential
          new optimal points should be enforced; that is, for a potential optimal point to
          become the new point from which to take another iterative optimizer step, the new response
          value         must be improved over the old response value. Otherwise, the potential opt
          point is rejected         and the search continues with the previously-discovered optimal
          point.
      \end{itemize}

    \item \xmlNode{convergence}:
      a node containing the desired convergence criteria for the optimization algorithm.
      Note that convergence is met when any one of the convergence criteria is met. If no
      convergence               criteria are given, then nominal convergence on gradient value is
      used.

      The \xmlNode{convergence} node recognizes the following subnodes:
      \begin{itemize}
        \item \xmlNode{gradient}: \xmlDesc{float}, 
          provides the desired value for the local estimated of the gradient
          for convergence. \default{1e-6, if no criteria specified}

        \item \xmlNode{objective}: \xmlDesc{float}, 
          provides the maximum relative change in the objective function for convergence.

        \item \xmlNode{stepSize}: \xmlDesc{float}, 
          provides the maximum size in relative step size for convergence.

        \item \xmlNode{terminateFollowers}: \xmlDesc{[yes, y, true, t, si, vero, dajie, oui, ja, yao, verum, evet, dogru, 1, on, no, n, false, f, nono, falso, nahh, non, nicht, bu, falsus, hayir, yanlis, 0, off, Yes, Y, True, T, Si, Vero, Dajie, Oui, Ja, Yao, Verum, Evet, Dogru, 1, On, No, N, False, F, Nono, Falso, Nahh, Non, Nicht, Bu, Falsus, Hayir, Yanlis, 0, Off]}, 
          indicates whether a trajectory should be terminated when it begins following the path
          of another trajectory.
          The \xmlNode{terminateFollowers} node recognizes the following parameters:
            \begin{itemize}
              \item \xmlAttr{proximity}: \xmlDesc{float, optional}, 
                provides the normalized distance at which a trajectory's head should be proximal to
                another trajectory's path before terminating the following trajectory.
          \end{itemize}

        \item \xmlNode{persistence}: \xmlDesc{integer}, 
          provides the number of consecutive times convergence should be reached before a trajectory
          is considered fully converged. This helps in preventing early false convergence.

        \item \xmlNode{constraintExplorationLimit}: \xmlDesc{integer}, 
          provides the number of consecutive times a functional constraint boundary can be explored
          for an acceptable sampling point before aborting search. Only apples if using a
          \xmlNode{Constraint}. \default{500}
      \end{itemize}

    \item \xmlNode{constant}: \xmlDesc{comma-separated strings, integers, and floats}, 
      allows variables that do not change value to be part of the input space.
      The \xmlNode{constant} node recognizes the following parameters:
        \begin{itemize}
          \item \xmlAttr{name}: \xmlDesc{string, required}, 
            variable name for this constant, which will be provided to the Model.
          \item \xmlAttr{shape}: \xmlDesc{comma-separated integers, optional}, 
            determines the shape of samples of the constant value.               For example,
            \xmlAttr{shape}=``2,3'' will shape the values into a 2 by 3               matrix, while
            \xmlAttr{shape}=``10'' will shape into a vector of 10 values.               Unlike the
            \xmlNode{variable}, the constant requires each value be entered; the number
            of required values is equal to the product of the \xmlAttr{shape} values, e.g. 6 entries
            for shape ``2,3'').               \nb A model interface must be prepared to handle non-
            scalar inputs to use this option.
          \item \xmlAttr{source}: \xmlDesc{string, optional}, 
            the name of the DataObject containing the value to be used for this constant.
            Requires \xmlNode{ConstantSource} node with a \xmlNode{DataObject} identified for this
            Sampler/Optimizer.
          \item \xmlAttr{index}: \xmlDesc{integer, optional}, 
            the index of the realization in the \xmlNode{ConstantSource} \xmlNode{DataObject}
            containing the value for this constant. Requires \xmlNode{ConstantSource} node with
            a \xmlNode{DataObject} identified for this Sampler/Optimizer.
      \end{itemize}

    \item \xmlNode{ConstantSource}: \xmlDesc{string}, 
      identifies a \xmlNode{DataObject} to provide \xmlNode{constant} values to the input
      space of this entity while sampling. As an alternative to providing predefined values
      for constants, the \xmlNode{ConstantSource} provides a dynamic means of always providing
      the same value for a constant. This is often used as part of a larger multi-workflow
      calculation.
      The \xmlNode{ConstantSource} node recognizes the following parameters:
        \begin{itemize}
          \item \xmlAttr{class}: \xmlDesc{string, optional}, 
            The RAVEN class for this source. Options include \xmlString{DataObject}.
          \item \xmlAttr{type}: \xmlDesc{string, optional}, 
            The RAVEN type for this source. Options include any valid \xmlNode{DataObject} type,
            such as HistorySet or PointSet.
      \end{itemize}

    \item \xmlNode{Constraint}: \xmlDesc{string}, 
      name of \xmlNode{Function} which contains explicit constraints for the sampling of
      the input space of the Model. From a practical point of view, this XML node must contain
      the name of a function defined in the \xmlNode{Functions} block (see
      Section~\ref{sec:functions}).               This external function must contain a method
      called ``constrain'', which returns 1 for               inputs satisfying the constraints and
      0 otherwise.
      The \xmlNode{Constraint} node recognizes the following parameters:
        \begin{itemize}
          \item \xmlAttr{class}: \xmlDesc{string, required}, 
            RAVEN class for this entity (e.g. Samplers, Models, DataObjects)
          \item \xmlAttr{type}: \xmlDesc{string, required}, 
            RAVEN type for this entity; a subtype of the class (e.g. MonteCarlo, Code, PointSet)
      \end{itemize}

    \item \xmlNode{Sampler}: \xmlDesc{string}, 
      name of a Sampler that can be used to initialize the starting points for the trajectories
      of some of the variables. From a practical point of view, this XML node must contain the
      name of a Sampler defined in the \xmlNode{Samplers} block (see
      Section~\ref{subsec:onceThroughSamplers}).               The Sampler will be used to
      initialize the trajectories' initial points for some or all               of the variables.
      For example, if the Sampler selected samples only 2 of the 5 optimization
      variables, the \xmlNode{initial} XML node is required only for the remaining 3 variables.
      The \xmlNode{Sampler} node recognizes the following parameters:
        \begin{itemize}
          \item \xmlAttr{class}: \xmlDesc{string, required}, 
            RAVEN class for this entity (e.g. Samplers, Models, DataObjects)
          \item \xmlAttr{type}: \xmlDesc{string, required}, 
            RAVEN type for this entity; a subtype of the class (e.g. MonteCarlo, Code, PointSet)
      \end{itemize}

    \item \xmlNode{Restart}: \xmlDesc{string}, 
      name of a DataObject. Used to leverage existing data when sampling a model. For
      example, if a Model has               already been sampled, but some samples were not
      collected, the successful samples can               be stored and used instead of rerunning
      the model for those specific samples. This RAVEN               entity definition must be a
      DataObject with contents including the input and output spaces               of the Model
      being sampled.
      The \xmlNode{Restart} node recognizes the following parameters:
        \begin{itemize}
          \item \xmlAttr{class}: \xmlDesc{string, optional}, 
            The RAVEN class for this source. Options include \xmlString{DataObject}.
          \item \xmlAttr{type}: \xmlDesc{string, optional}, 
            The RAVEN type for this source. Options include any valid \xmlNode{DataObject} type,
            such as HistorySet or PointSet.
      \end{itemize}

    \item \xmlNode{restartTolerance}: \xmlDesc{float}, 
      specifies how strictly a matching point from a \xmlNode{Restart} DataObject must match
      the desired sample point in order to be used. If a potential restart point is within a
      relative Euclidean distance (as specified by the value in this node) of a desired sample
      point,               the restart point will be used instead of sampling the Model.
      \default{1e-15}

    \item \xmlNode{variablesTransformation}:
      Allows transformation of variables via translation matrices. This defines two spaces,
      a ``latent'' transformed space sampled by RAVEN and a ``manifest'' original space understood
      by the Model.
      The \xmlNode{variablesTransformation} node recognizes the following parameters:
        \begin{itemize}
          \item \xmlAttr{distribution}: \xmlDesc{string, optional}, 
            the name for the distribution defined in the XML node \xmlNode{Distributions}.
            This attribute indicates the values of \xmlNode{manifestVariables} are drawn from
            \xmlAttr{distribution}.
      \end{itemize}

      The \xmlNode{variablesTransformation} node recognizes the following subnodes:
      \begin{itemize}
        \item \xmlNode{latentVariables}: \xmlDesc{comma-separated strings}, 
          user-defined latent variables that are used for the variables transformation.
          All the variables listed under this node should be also mentioned in \xmlNode{variable}.

        \item \xmlNode{manifestVariables}: \xmlDesc{comma-separated strings}, 
          user-defined manifest variables that can be used by the \xmlNode{Model}.

        \item \xmlNode{manifestVariablesIndex}: \xmlDesc{comma-separated strings}, 
          user-defined manifest variables indices paired with \xmlNode{manifestVariables}.
          These indices indicate the position of manifest variables associated with multivariate
          normal               distribution defined in the XML node \xmlNode{Distributions}.
          The indices should be postive integer. If not provided, the code will use the positions
          of manifest variables listed in \xmlNode{manifestVariables} as the indices.

        \item \xmlNode{method}: \xmlDesc{string}, 
          the method that is used for the variables transformation. The currently available method
          is \xmlString{pca}.
      \end{itemize}
  \end{itemize}

\hspace{24pt}
Gradient Descent Example:
\begin{lstlisting}[style=XML]
<Optimizers>
  ...
  <GradientDescent name="opter">
    <objective>ans</objective>
    <variable name="x">
      <distribution>x_dist</distribution>
      <initial>-2</initial>
    </variable>
    <variable name="y">
      <distribution>y_dist</distribution>
      <initial>2</initial>
    </variable>
    <samplerInit>
      <limit>100</limit>
    </samplerInit>
    <gradient>
      <FiniteDifference/>
    </gradient>
    <stepSize>
      <GradientHistory/>
    </stepSize>
    <acceptance>
      <Strict/>
    </acceptance>
    <TargetEvaluation class="DataObjects" type="PointSet">optOut</TargetEvaluation>
  </GradientDescent>
  ...
</Optimizers>
\end{lstlisting}



\subsection{SimulatedAnnealing}
  The \xmlNode{SimulatedAnnealing} optimizer is a metaheuristic approach
  to perform a global search in large design spaces. The methodology rose
  from statistical physics and was inspitred by metallurgy where                             it was
  found that fast cooling might lead to smaller and defected crystals,
  and that reheating and slowly controling cooling will lead to better states.
  This allows climbing to avoid being stuck in local minima and hence facilitates
  finding the global minima for non-convex probloems.                             More information
  can be found in: Kirkpatrick, S.; Gelatt Jr, C. D.; Vecchi, M. P. (1983).
  ``Optimization by Simulated Annealing". Science. 220 (4598): 671–680.
\vspace{7pt} \\When used as part of a \xmlNode{MultiRun} step, this entity provides
        additional information through the \xmlNode{SolutionExport} DataObject. The
        following variables can be requested within the \xmlNode{SolutionExport}:
        \begin{itemize}
          \item \texttt{trajID}: integer identifier for different optimization starting locations and paths
             \item \texttt{iteration}: integer identifying which iteration (or step, or generation) a trajectory is on
             \item \texttt{accepted}: string acceptance status of the potential optimal point (algorithm dependent)
             \item \texttt{\{VAR\}}: any variable from the \xmlNode{TargetEvaluation} input or output; gives the value of that variable at the optimal candidate for this iteration.
             \item \texttt{conv\_\{CONV\}}: status of each given convergence criteria
             \item \texttt{amp\_\{VAR\}}: amplitued associated to each variable used to compute step size based on cooling method and the corresponding next neighbour
             \item \texttt{delta\_\{VAR\}}: step size associated to each variable
             \item \texttt{Temp}: temperature at current state
             \item \texttt{fraction}: current fraction of the max iteration limit
           
         \end{itemize}

  The \xmlNode{SimulatedAnnealing} node recognizes the following parameters:
    \begin{itemize}
      \item \xmlAttr{verbosity}: \xmlDesc{[silent, quiet, all, debug], optional}, 
        Desired verbosity of messages coming from this entity
      \item \xmlAttr{name}: \xmlDesc{string, required}, 
        User-defined name to designate this entity in the RAVEN input file.
  \end{itemize}

  The \xmlNode{SimulatedAnnealing} node recognizes the following subnodes:
  \begin{itemize}
    \item \xmlNode{objective}: \xmlDesc{string}, 
      Name of the response variable (or ``objective function'') that should be optimized
      (minimized or maximized).

    \item \xmlNode{variable}:
      defines the input space variables to be sampled through various means.
      The \xmlNode{variable} node recognizes the following parameters:
        \begin{itemize}
          \item \xmlAttr{name}: \xmlDesc{string, optional}, 
            user-defined name of this Sampler. \nb As for the other objects,               this is
            the name that can be used to refer to this specific entity from other input blocks
          \item \xmlAttr{shape}: \xmlDesc{comma-separated integers, optional}, 
            determines the number of samples and shape of samples               to be taken.  For
            example, \xmlAttr{shape}=``2,3'' will provide a 2 by 3               matrix of values,
            while \xmlAttr{shape}=``10'' will produce a vector of 10 values.               Omitting
            this optional attribute will result in a single scalar value instead.               Each
            of the values in the matrix or vector will be the same as the single sampled value.
            \nb A model interface must be prepared to handle non-scalar inputs to use this option.
      \end{itemize}

      The \xmlNode{variable} node recognizes the following subnodes:
      \begin{itemize}
        \item \xmlNode{distribution}: \xmlDesc{string}, 
          name of the distribution that is associated to this variable.               Its name needs
          to be contained in the \xmlNode{Distributions} block explained               in Section
          \ref{sec:distributions}. In addition, if NDDistribution is used,               the
          attribute \xmlAttr{dim} is required. \nb{Alternatively, this node must be omitted
          if the \xmlNode{function} node is supplied.}
          The \xmlNode{distribution} node recognizes the following parameters:
            \begin{itemize}
              \item \xmlAttr{dim}: \xmlDesc{integer, optional}, 
                for an NDDistribution, indicates the dimension within the NDDistribution that
                corresponds               to this variable.
          \end{itemize}

        \item \xmlNode{function}: \xmlDesc{string}, 
          name of the function that               defines the calculation of this variable from
          other distributed variables.  Its name               needs to be contained in the
          \xmlNode{Functions} block explained in Section               \ref{sec:functions}. This
          function must implement a method named ``evaluate''.               \nb{Each
          \xmlNode{variable} must contain only one \xmlNode{Function} or
          \xmlNode{Distribution}, but not both.}

        \item \xmlNode{initial}: \xmlDesc{comma-separated floats}, 
          indicates the initial values where independent trajectories for this optimization
          effort should begin. The number of entries should be the same for all variables, unless
          a variable is initialized with a sampler (see \xmlNode{samplerInit} below). Note these
          entries are ordered; that is, if the optimization variables are $x$ and $y$, and the
          initial               values for $x$ are \xmlString{1, 2, 3, 4} and initial values for $y$
          are \xmlString{5, 6, 7, 8},               then there will be four starting trajectories
          beginning at the locations (1, 5), (2, 6),               (3, 7), and (4, 8).
      \end{itemize}

    \item \xmlNode{TargetEvaluation}: \xmlDesc{string}, 
      name of the DataObject where the sampled outputs of the Model will be collected.
      This DataObject is the means by which the sampling entity obtains the results of requested
      samples, and so should require all the input and output variables needed for adaptive
      sampling.
      The \xmlNode{TargetEvaluation} node recognizes the following parameters:
        \begin{itemize}
          \item \xmlAttr{class}: \xmlDesc{string, required}, 
            RAVEN class for this entity (e.g. Samplers, Models, DataObjects)
          \item \xmlAttr{type}: \xmlDesc{string, required}, 
            RAVEN type for this entity; a subtype of the class (e.g. MonteCarlo, Code, PointSet)
      \end{itemize}

    \item \xmlNode{samplerInit}:
      collection of nodes that describe the initialization of the optimization algorithm.

      The \xmlNode{samplerInit} node recognizes the following subnodes:
      \begin{itemize}
        \item \xmlNode{limit}: \xmlDesc{integer}, 
          limits the number of Model evaluations that may be performed as part of this optimization.
          For example, a limit of 100 means at most 100 total Model evaluations may be performed.

        \item \xmlNode{writeSteps}: \xmlDesc{[final, every]}, 
          delineates when the \xmlNode{SolutionExport} DataObject should be written to. In case
          of \xmlString{final}, only the final optimal solution for each trajectory will be written.
          In case of \xmlString{every}, the \xmlNode{SolutionExport} will be updated with each
          iteration               of the Optimizer.

        \item \xmlNode{initialSeed}: \xmlDesc{integer}, 
          seed for random number generation. Note that by default RAVEN uses an internal seed,
          so this seed must be changed to observe changed behavior. \default{RAVEN-determined}

        \item \xmlNode{type}: \xmlDesc{[min, max]}, 
          the type of optimization to perform. \xmlString{min} will search for the lowest
          \xmlNode{objective} value, while \xmlString{max} will search for the highest value.
      \end{itemize}

    \item \xmlNode{convergence}:
      a node containing the desired convergence criteria for the optimization algorithm.
      Note that convergence is met when any one of the convergence criteria is met. If no
      convergence               criteria are given, then the defaults are used.

      The \xmlNode{convergence} node recognizes the following subnodes:
      \begin{itemize}
        \item \xmlNode{objective}: \xmlDesc{float}, 
          provides the desired value for the convergence criterion of the objective function
          ($\epsilon^{obj}$), i.e., convergence is reached when: $$ |newObjevtive - oldObjective|
          \le \epsilon^{obj}$$.                        \default{1e-6}, if no criteria specified

        \item \xmlNode{temperature}: \xmlDesc{float}, 
          provides the desired value for the convergence creiteron of the system temperature,
          ($\epsilon^{temp}$), i.e., convergence is reached when: $$T \le \epsilon^{temp}$$.
          \default{1e-10}, if no criteria specified

        \item \xmlNode{persistence}: \xmlDesc{integer}, 
          provides the number of consecutive times convergence should be reached before a trajectory
          is considered fully converged. This helps in preventing early false convergence.
      \end{itemize}

    \item \xmlNode{coolingSchedule}:
      The function governing the cooling process. Currently, user can select
      between,\xmlString{exponential},                  \xmlString{cauchy},
      \xmlString{boltzmann},or \xmlString{veryfast}.\\ \\In case of \xmlString{exponential} is
      provided, The cooling process will be governed by: $$ T^{k} = T^0 * \alpha^k$$
      In case of \xmlString{boltzmann} is provided, The cooling process will be governed by: $$
      T^{k} = \frac{T^0}{log(k + d)}$$                  In case of \xmlString{cauchy} is provided,
      The cooling process will be governed by: $$ T^{k} = \frac{T^0}{k + d}$$In case of
      \xmlString{veryfast} is provided, The cooling process will be governed by: $$ T^{k} =  T^0 *
      \exp(-ck^{1/D}),$$                  where $D$ is the dimentionality of the problem (i.e.,
      number of optimized variables), $k$ is the number of the current iteration
      $T^{0} = \max{(0.01,1-\frac{k}{\xmlNode{limit}})}$ is the initial temperature, and $T^{k}$ is
      the current temperature                  according to the specified cooling schedule.
      \default{exponential}.

      The \xmlNode{coolingSchedule} node recognizes the following subnodes:
      \begin{itemize}
        \item \xmlNode{exponential}: \xmlDesc{string}, 
          exponential cooling schedule

          The \xmlNode{exponential} node recognizes the following subnodes:
          \begin{itemize}
            \item \xmlNode{alpha}: \xmlDesc{float}, 
              slowing down constant, should be between 0,1 and preferable very close to 1.
              \default{0.94}
          \end{itemize}

        \item \xmlNode{veryfast}: \xmlDesc{string}, 
          veryfast cooling schedule

          The \xmlNode{veryfast} node recognizes the following subnodes:
          \begin{itemize}
            \item \xmlNode{c}: \xmlDesc{float}, 
              decay constant, \default{1.0}
          \end{itemize}

        \item \xmlNode{cauchy}: \xmlDesc{string}, 
          cauchy cooling schedule

          The \xmlNode{cauchy} node recognizes the following subnodes:
          \begin{itemize}
            \item \xmlNode{d}: \xmlDesc{float}, 
              bias, \default{1.0}
          \end{itemize}

        \item \xmlNode{boltzmann}: \xmlDesc{string}, 
          boltzmann cooling schedule

          The \xmlNode{boltzmann} node recognizes the following subnodes:
          \begin{itemize}
            \item \xmlNode{d}: \xmlDesc{float}, 
              bias, \default{1.0}
          \end{itemize}
      \end{itemize}

    \item \xmlNode{constant}: \xmlDesc{comma-separated strings, integers, and floats}, 
      allows variables that do not change value to be part of the input space.
      The \xmlNode{constant} node recognizes the following parameters:
        \begin{itemize}
          \item \xmlAttr{name}: \xmlDesc{string, required}, 
            variable name for this constant, which will be provided to the Model.
          \item \xmlAttr{shape}: \xmlDesc{comma-separated integers, optional}, 
            determines the shape of samples of the constant value.               For example,
            \xmlAttr{shape}=``2,3'' will shape the values into a 2 by 3               matrix, while
            \xmlAttr{shape}=``10'' will shape into a vector of 10 values.               Unlike the
            \xmlNode{variable}, the constant requires each value be entered; the number
            of required values is equal to the product of the \xmlAttr{shape} values, e.g. 6 entries
            for shape ``2,3'').               \nb A model interface must be prepared to handle non-
            scalar inputs to use this option.
          \item \xmlAttr{source}: \xmlDesc{string, optional}, 
            the name of the DataObject containing the value to be used for this constant.
            Requires \xmlNode{ConstantSource} node with a \xmlNode{DataObject} identified for this
            Sampler/Optimizer.
          \item \xmlAttr{index}: \xmlDesc{integer, optional}, 
            the index of the realization in the \xmlNode{ConstantSource} \xmlNode{DataObject}
            containing the value for this constant. Requires \xmlNode{ConstantSource} node with
            a \xmlNode{DataObject} identified for this Sampler/Optimizer.
      \end{itemize}

    \item \xmlNode{ConstantSource}: \xmlDesc{string}, 
      identifies a \xmlNode{DataObject} to provide \xmlNode{constant} values to the input
      space of this entity while sampling. As an alternative to providing predefined values
      for constants, the \xmlNode{ConstantSource} provides a dynamic means of always providing
      the same value for a constant. This is often used as part of a larger multi-workflow
      calculation.
      The \xmlNode{ConstantSource} node recognizes the following parameters:
        \begin{itemize}
          \item \xmlAttr{class}: \xmlDesc{string, optional}, 
            The RAVEN class for this source. Options include \xmlString{DataObject}.
          \item \xmlAttr{type}: \xmlDesc{string, optional}, 
            The RAVEN type for this source. Options include any valid \xmlNode{DataObject} type,
            such as HistorySet or PointSet.
      \end{itemize}

    \item \xmlNode{Constraint}: \xmlDesc{string}, 
      name of \xmlNode{Function} which contains explicit constraints for the sampling of
      the input space of the Model. From a practical point of view, this XML node must contain
      the name of a function defined in the \xmlNode{Functions} block (see
      Section~\ref{sec:functions}).               This external function must contain a method
      called ``constrain'', which returns 1 for               inputs satisfying the constraints and
      0 otherwise.
      The \xmlNode{Constraint} node recognizes the following parameters:
        \begin{itemize}
          \item \xmlAttr{class}: \xmlDesc{string, required}, 
            RAVEN class for this entity (e.g. Samplers, Models, DataObjects)
          \item \xmlAttr{type}: \xmlDesc{string, required}, 
            RAVEN type for this entity; a subtype of the class (e.g. MonteCarlo, Code, PointSet)
      \end{itemize}

    \item \xmlNode{Sampler}: \xmlDesc{string}, 
      name of a Sampler that can be used to initialize the starting points for the trajectories
      of some of the variables. From a practical point of view, this XML node must contain the
      name of a Sampler defined in the \xmlNode{Samplers} block (see
      Section~\ref{subsec:onceThroughSamplers}).               The Sampler will be used to
      initialize the trajectories' initial points for some or all               of the variables.
      For example, if the Sampler selected samples only 2 of the 5 optimization
      variables, the \xmlNode{initial} XML node is required only for the remaining 3 variables.
      The \xmlNode{Sampler} node recognizes the following parameters:
        \begin{itemize}
          \item \xmlAttr{class}: \xmlDesc{string, required}, 
            RAVEN class for this entity (e.g. Samplers, Models, DataObjects)
          \item \xmlAttr{type}: \xmlDesc{string, required}, 
            RAVEN type for this entity; a subtype of the class (e.g. MonteCarlo, Code, PointSet)
      \end{itemize}

    \item \xmlNode{Restart}: \xmlDesc{string}, 
      name of a DataObject. Used to leverage existing data when sampling a model. For
      example, if a Model has               already been sampled, but some samples were not
      collected, the successful samples can               be stored and used instead of rerunning
      the model for those specific samples. This RAVEN               entity definition must be a
      DataObject with contents including the input and output spaces               of the Model
      being sampled.
      The \xmlNode{Restart} node recognizes the following parameters:
        \begin{itemize}
          \item \xmlAttr{class}: \xmlDesc{string, optional}, 
            The RAVEN class for this source. Options include \xmlString{DataObject}.
          \item \xmlAttr{type}: \xmlDesc{string, optional}, 
            The RAVEN type for this source. Options include any valid \xmlNode{DataObject} type,
            such as HistorySet or PointSet.
      \end{itemize}

    \item \xmlNode{restartTolerance}: \xmlDesc{float}, 
      specifies how strictly a matching point from a \xmlNode{Restart} DataObject must match
      the desired sample point in order to be used. If a potential restart point is within a
      relative Euclidean distance (as specified by the value in this node) of a desired sample
      point,               the restart point will be used instead of sampling the Model.
      \default{1e-15}

    \item \xmlNode{variablesTransformation}:
      Allows transformation of variables via translation matrices. This defines two spaces,
      a ``latent'' transformed space sampled by RAVEN and a ``manifest'' original space understood
      by the Model.
      The \xmlNode{variablesTransformation} node recognizes the following parameters:
        \begin{itemize}
          \item \xmlAttr{distribution}: \xmlDesc{string, optional}, 
            the name for the distribution defined in the XML node \xmlNode{Distributions}.
            This attribute indicates the values of \xmlNode{manifestVariables} are drawn from
            \xmlAttr{distribution}.
      \end{itemize}

      The \xmlNode{variablesTransformation} node recognizes the following subnodes:
      \begin{itemize}
        \item \xmlNode{latentVariables}: \xmlDesc{comma-separated strings}, 
          user-defined latent variables that are used for the variables transformation.
          All the variables listed under this node should be also mentioned in \xmlNode{variable}.

        \item \xmlNode{manifestVariables}: \xmlDesc{comma-separated strings}, 
          user-defined manifest variables that can be used by the \xmlNode{Model}.

        \item \xmlNode{manifestVariablesIndex}: \xmlDesc{comma-separated strings}, 
          user-defined manifest variables indices paired with \xmlNode{manifestVariables}.
          These indices indicate the position of manifest variables associated with multivariate
          normal               distribution defined in the XML node \xmlNode{Distributions}.
          The indices should be postive integer. If not provided, the code will use the positions
          of manifest variables listed in \xmlNode{manifestVariables} as the indices.

        \item \xmlNode{method}: \xmlDesc{string}, 
          the method that is used for the variables transformation. The currently available method
          is \xmlString{pca}.
      \end{itemize}
  \end{itemize}

\hspace{24pt}
Simulated Annealing Example:
\begin{lstlisting}[style=XML]
  <Optimizers>
    ...
    <SimulatedAnnealing name="simOpt">
      <samplerInit>
        <limit>2000</limit>
        <initialSeed>42</initialSeed>
        <writeSteps>every</writeSteps>
        <type>min</type>
      </samplerInit>
      <convergence>
        <objective>1e-6</objective>
        <temperature>1e-20</temperature>
        <persistence>1</persistence>
      </convergence>
      <coolingSchedule>
        <exponential>
          <alpha>0.94</alpha>
        </exponential>
      </coolingSchedule>
      <variable name="x">
        <distribution>beale_dist</distribution>
        <initial>-2.5</initial>
      </variable>
      <variable name="y">
        <distribution>beale_dist</distribution>
        <initial>3.5</initial>
      </variable>
      <objective>ans</objective>
      <TargetEvaluation class="DataObjects" type="PointSet">optOut</TargetEvaluation>
    </SimulatedAnnealing>
    ...
  </Optimizers>
\end{lstlisting}


\input{database_data.tex}
\input{OutStreamSystem.tex}
\input{model.tex}
\input{functions.tex}
\input{metrics.tex}
\input{step.tex}
\section{Existing Interfaces}
\label{sec:existingInterface}
%%%%%%%%%%%%%%%%%%%%%%%%%%%
%%%%%% Generic  INTERFACE  %%%%%%
%%%%%%%%%%%%%%%%%%%%%%%%%%%
\subsection{Generic Interface}
\label{subsec:genericInterface}
The GenericCode interface is meant to handle a wide variety of generic codes
that take straightforward input files and produce output CSV files.  There are
some limitations for this interface.
If a code: \vspace{-20pt}
\begin{itemize}
\item accepts a keyword-based input file with no cross-dependent inputs,
\item has no more than one filetype extension per command line flag,
\item and returns a CSV with the input parameters and output parameters,
\end{itemize}\vspace{-20pt}
the GenericCode interface should cover the code for RAVEN.

The GenericCode interface leverages a wildcard-based approach to editing input files. Using the
special wildcard format \texttt{\$RAVEN-\$}, RAVEN parses text-based inputs and replaces the
wildcards with sampled values. For example, consider RAVEN sampling variables named
\texttt{initial\_velocity} and \texttt{initial\_angle}. Assume we're using a projectile tracking model
with keyword based entry input files; for example,
\begin{lstlisting}[language=python]
  initial_height = 0     # starting height, m
  initial_angle = 35     # starting angle, degrees
  initial_velocity = 40  # starting velocity, m/s
  gravity = 9.8          # accel due to grav, m/s/s
  auxfile = gen.two      # additional properties file
  case = myOut           # output name (adds .csv)
 \end{lstlisting}
Since we want to sample \texttt{initial\_velocity} and \texttt{initial\_angle}, we create a new
template input and replace the values where samples should go with the wildcard and the variable
name:
\begin{lstlisting}[language=python]
  initial_height = 0     # starting height, m
  initial_angle = $RAVEN-initial_angle$ # starting angle, degrees
  initial_velocity = $RAVEN-initial_velocity$ # starting velocity, m/s
  gravity = 9.8          # accel due to grav, m/s/s
  auxfile = gen.two      # additional properties file
  case = myOut           # output name (adds .csv)
\end{lstlisting}
See more discussion of replacing the output case and auxiliary file names below. When RAVEN samples
values for the initial height and velocity, it will generate a new input file with those values in
place, for example,
\begin{lstlisting}[language=python]
  initial_height = 0     # starting height, m
  initial_angle = 22.7589 # starting angle, degrees
  initial_velocity = 47.2076 # starting velocity, m/s
  gravity = 9.8          # accel due to grav, m/s/s
  auxfile = gen.two      # additional properties file
  case = myOut           # output name (adds .csv)
\end{lstlisting}

If a code contains cross-dependent data, the generic interface is not able to
edit the correct values.  For example, if a geometry-building script specifies
inner\_radius, outer\_radius, and thickness, the generic interface cannot
calculate the thickness given the outer and inner radius, or vice versa.
In this case, the \textit{function} method explained in the Samplers (see \ref{sec:Samplers})
and Optimizers (see \ref{sec:Optimizers}) sections can be used.

 An example of the code interface is shown here.  The input parameters are read
 from the input files \texttt{gen.one} and \texttt{gen.two} respectively.
 The code is run using \texttt{python}, so that is part of the \xmlNode{clargs} node with the \xmlAttr{type} equal \xmlString{prepend}.
 The command line entry to normally run the code is
\begin{lstlisting}[language=bash]
python poly_inp.py -i gen.one -a gen.two -o myOut
\end{lstlisting}
and produces the output \texttt{myOut.csv}.

Example:
\begin{lstlisting}[style=XML]
    <Code name="poly" subType="GenericCode">
      <executable>GenericInterface/poly_inp.py</executable>
      <inputExtentions>.one,.two</inputExtentions>
      <clargs type='prepend' arg='python'/>
      <clargs type='input'   arg='-i' extension='.one'/>
      <clargs type='input'   arg='-a' extension='.two'/>
      <clargs type='output'  arg='-o'/>
    </Code>
\end{lstlisting}

If a code doesn't accept necessary Raven-editable auxiliary input files
or output filenames through the command line, the GenericCode interface
can also edit the input files and insert the filenames there.  For example,
in the previous example, say instead of \texttt{-a gen.two} and \texttt{-o myOut}
in the command line, \texttt{gen.one} has the following lines:
\begin{lstlisting}[language=python]
...
auxfile = gen.two
case = myOut
...
\end{lstlisting}
Then, our example XML for the code would be

Example:
\begin{lstlisting}[style=XML]
    <Code name="poly" subType="GenericCode">
      <executable>GenericInterface/poly_inp.py</executable>
      <inputExtentions>.one,.two</inputExtentions>
      <clargs   type='prepend' arg='python'/>
      <clargs   type='input'   arg='-i'  extension='.one'/>
      <fileargs type='input'   arg='two' extension='.two'/>
      <fileargs type='output'  arg='out'/>
    </Code>
\end{lstlisting}
and the corresponding template input file lines would be changed to read
\begin{lstlisting}[language=python]
...
auxfile = $RAVEN-two$
case = $RAVEN-out$
...
\end{lstlisting}


%%%%
If a code has hard-coded output file names that are not changeable,
the GenericCode interface can be invoked using the \xmlNode{outputFile}
node in which the output file name (CSV only) must be specified.
For example, in the previous example, say instead of \texttt{-a gen.two} and \texttt{-o myOut}
in the command line, the code always produce a CSV file named ``fixed\_output.csv'';

Then, our example XML for the code would be

Example:
\begin{lstlisting}[style=XML]
    <Code name="poly" subType="GenericCode">
      <executable>GenericInterface/poly_inp.py</executable>
      <inputExtentions>.one,.two</inputExtentions>
      <clargs   type='prepend' arg='python'/>
      <clargs   type='input'   arg='-i'  extension='.one'/>
      <fileargs type='input'   arg='two' extension='.two'/>
      <outputFile>fixed_output.csv</outputFile>
    </Code>
\end{lstlisting}

In addition, the ``wild-cards'' above can contain two special and optional symbols:
\begin{itemize}
  \item  \texttt{:}, that defines an eventual default value;
  \item  \texttt{|}, that defines the format of the value. The  Generic Interface currently supports the following formatting options (* in the examples means blank space):
    \begin{itemize}
       \item \textbf{plain integer}, in this case  the value that is going to be replaced by the Generic Interface, will be left-justified with a string length equal to the integer value specified here (e.g. ``\texttt{|}6'', the value is left-justified with a string length of 6);
      \item \textbf{d}, signed integer decimal, the value is going to be formatted as an integer (e.g.  if the value is 9 and the format ``\texttt{|}10d'', the replaced value will be formatted as follows: ``*********9'');
      \item \textbf{e}, floating point exponential format (lowercase), the value is going to be formatted as a float in scientific notation (e.g. if the value is 9.1234 and the format ``\texttt{|}10.3e'', the replaced value will be formatted as follows: ``*9.123e+00'' );
      \item \textbf{E}, floating point exponential format (uppercase), the value is going to be formatted as a float in scientific notation (e.g. if the value is 9.1234 and the format ``\texttt{|}10.3E'', the replaced value will be formatted as follows: ``*9.123E+00'' );
      \item \textbf{f or F}, floating point decimal format, the value is going to be formatted as a float in decimal notation (e.g. if the value is 9.1234 and the format ``\texttt{|}10.3f'', the replaced value will be formatted as follows: ``*****9.123'' );
      \item \textbf{g}, floating point format. Uses lowercase exponential format if exponent is less than -4 or not less than precision, decimal format otherwise (e.g. if the value is 9.1234 and the format ``\texttt{|}10.3g'', the replaced value will be formatted as follows: ``******9.12'' );
      \item \textbf{G}, floating point format. Uses uppercase exponential format if exponent is less than -4 or not less than precision, decimal format otherwise (e.g. if the value is 0.000009 and the format ``\texttt{|}10.3G'', the replaced value will be formatted as follows: ``*****9E-06'' ).
    \end{itemize}|
\end{itemize}
For example:
\begin{lstlisting}[language=python]
...
auxfile = $RAVEN-two:3$
case = $RAVEN-out:5|10$
...
\end{lstlisting}
Where,
\begin{itemize}
  \item  \texttt{:}, in case the variable ``two'' is not defined in the RAVEN XML input file, the Parser, will replace it with the value ``3''.;
  \item  \texttt{|}, the value that is going to be replaced by the Generic Interface, will be left- justified with a string length of ``10'';
\end{itemize}

%%%%%%%%%%%%%%%%%%%%%%%%%%%%%%%%%%%%%
%%%%%% RAVEN  INTERFACE  (RAVEN running RAVEN) %%%%%%
%%%%%%%%%%%%%%%%%%%%%%%%%%%%%%%%%%%%%
\subsection{RAVEN Interface}
\label{subsec:RAVENInterface}
The RAVEN interface is meant to provide the possibility to execute a RAVEN input file
driving a set of SLAVE RAVEN calculations. For example, if the user wants to optimize the parameters
of a surrogate model (e.g. minimizing the distance between the surrogate predictions and the real data), he
can achieve this task by setting up  a RAVEN input file (master) that performs an optimization on the feature
space characterized by the surrogate model parameters, whose training and validation assessment  is performed in the SLAVE
RAVEN runs.
\\ There are some limitations for this interface:
\begin{itemize}
\item only one  sub-level of RAVEN can be executed (i.e. if the SLAVE RAVEN input file contains the run of another RAVEN SLAVE, the MASTER RAVEN will error out)
\item only data from Outstreams of type Print can be collected by the MASTER RAVEN
\item only a maximum of two Outstreams can be collected (1 PointSet and 1 HistorySet)
\end{itemize}


Like for every other interface, most of the RAVEN workflow stays the same independently of which type of Model (i.e. Code) is used.
\\ Similarly to any other code interface, the user provides paths to executables and aliases for sampled variables within the
\xmlNode{Models} block.  The \xmlNode{Code} block will contain attributes \xmlAttr{name} and
\xmlAttr{subType}.  \xmlAttr{name} identifies that particular \xmlNode{Code} model within RAVEN, and
\xmlAttr{subType} specifies which code interface the model will use (In this case \xmlAttr{subType}=``RAVEN'').
The \xmlNode{executable}
block should contain the absolute or relative (with respect to the current working
directory) path to the RAVEN framework script (\textbf{raven\_framework}).
\\ In addition to the attributes and xml nodes reported above, the RAVEN accepts the following XML nodes (required and optional):
\begin{itemize} % nodes for code
  \item  \xmlNode{outputExportOutStreams}, \xmlDesc{comma separated list,
    required parameter} will specify the  \xmlNode{OutStreams} that will be loaded as outputs of the SLAVE RAVEN.
    Maximum two  \xmlNode{OutStreams} can be listed here (1 for PointSet and/or 1 for HistorySet).
  \item  \xmlNode{conversion}, \xmlDesc{Node,optional parameter} will specify details of conversion scripts to
    be used in creating the inner RAVEN input file.  This node contains the following nodes:
    \begin{itemize} % nodes for conversion
      \item \xmlNode{module}, \xmlDesc{Node, optional parameter} a module for directly manipulating the xml
        structure of perturbed input files. This can be used to modify the template input file in arbitrary
        ways; however, it should be used with caution, and is considered an advanced method.
        This node has the following attribute:
        \begin{itemize} % attributes for module
          \item \xmlAttr{source}, \xmlDesc{string, required} provides the path to the manipulation module including
              the module file itself.  The following method should be defined in order to perform the input
              manipulation:
              \begin{itemize} % functions for module
                 \item \textbf{\textit{modifyInput}}, manipulates the input in arbitrary ways. This method
                   takes two arguments. The first is the root \xmlNode{Simulation} node of the template input
                   file that has already been modified with the perturbed samples (the object is a Python
                   \texttt{xml.etree.ElementTree.Element} object). The second input is a dictionary with all
                   the modification information used to previously modify the template xml. The method should
                   return the modified root.
                 Example:
                  \begin{lstlisting}[language=python]
import xml.etree.ElementTree as ET
def modifyInput(root, modDict):
  """
    Manipulate the inner RAVEN xml input.
    @ In, root, ET.Element, perturbed RAVEN input
    @ In, modDict, dictionary, modifications made to the input
    @ Out, root, ET.Element, modified RAVEN input
  """
  # adds the file <Input name='aux_inp'>auxfile.txt</Input> to the <Files> node
  filesNode = root.find('Files')
  newNode = ET.Element('Input')
  newNode.text = 'auxfile.txt'
  newNode.attrib['name'] = 'aux_inp'
  filesNode.append(newNode)
  return root
                   \end{lstlisting}
              \end{itemize} % end functions for module
        \end{itemize} % end attributes for module

      \item \xmlNode{module}, \xmlDesc{Node, optional parameter} contains the information about a specific
        conversion module (python file).  This node can be repeated multiple times.
        This node has the following attribute:
        \begin{itemize} % attributes for module
          \item \xmlAttr{source}, \xmlDesc{string, required} provides the path to the conversion module including
              the module file itself.  There are two methods that can be placed in the conversion module:
              \begin{itemize} % functions for module
                 \item \textbf{\textit{manipulateScalarSampledVariables}}, a method that is aimed to manipulate sampled variables and to create more in case needed.
                 Example:
                  \begin{lstlisting}[language=python]
def manipulateScalarSampledVariables(sampledVariables):
  """
  This method is aimed to manipulate scalar variables.
  The user can create new variables based on the
  variables sampled by RAVEN
   @ In, sampledVariables, dict, dictionary of
       sampled variables ({"var1":value1,"var2":value2})
   @ Out, None, the new variables should be
                 added in the "sampledVariables" dictionary
  """
  newVariableValue =
    sampledVariables['Distributions|Uniform@name:a_dist|lowerBound']
    + 1.0
  sampledVariables['Distributions|Uniform@name:a_dist|upperBound'] =
    newVariableValue
  return
                   \end{lstlisting}

                 \item \textbf{\textit{convertNotScalarSampledVariables}}, a method that is aimed to convert not scalar variables (e.g. 1D arrays) into multiple scalar variables
                 (e.g.  \xmlNode{constant}(s) in a sampling strategy).
                  This method is going to be required in case not scalar variables are detected by the interface.
                  Example:
                  \begin{lstlisting}[language=python]
 def convertNotScalarSampledVariables(noScalarVariables):
   """
   This method is aimed to convert not scalar
   variables into multiple scalar variables. The user MUST
    create new variables based on the not Scalar Variables
     sampled (and passed in) by RAVEN
   @ In, noScalarVariables, dict, dictionary of sampled
        variables that are not scalar ({"var1":1Darray1,"var2":1Darray2})
   @ Out, newVars, dict,  the new variables that have
        been created based on the not scalar variables
        contained in "noScalarVariables" dictionary
   """
   oneDimensionalArray =
       noScalarVariables['temperatureHistory']
   newVars = {}
   for cnt, value in enumerate(oneDimensionalArray):
     newVars['Samplers|MonteCarlo@name:myMC|constant'+
                '@name=temperatureHistory'+str(cnt)] =
                oneDimensionalArray[cnt]
   return newVars
                  \end{lstlisting}
              \end{itemize} % end functions for module
        \end{itemize} % end attributes for module
        The \xmlNode{module} node also takes the following node:
        \begin{itemize} % nodes of module
          \item \xmlNode{variables}, \xmlDesc{comma-separated list, required} provides a comma-separated list of
            the variables from the MASTER RAVEN that need to be accessed by the conversion script module.  The
            variables listed here use the pipe naming system (un-aliased names).
        \end{itemize} % end nodes of module
    \end{itemize} % end nodes for conversion
\end{itemize} % end nodes for code

Code input example:
\begin{lstlisting}[style=XML]
<Code name="RAVENrunningRAVEN" subType="RAVEN">
  <executable>../../../raven_framework</executable>
  <outputExportOutStreams>
     HistorySetOutStream,PointSetOutStream
  </outputExportOutStreams>
  <conversion>
    <module source=/Users/username/whateverConversionModule.py>
      <variables>a,b,x,y</variables>
    </module>
  </conversion>
</Code>
\end{lstlisting}

Like for every other interface,  the syntax of the variable names is important to make the parser understand how to perturb an input file.
\\ For the RAVEN interface, a syntax inspired by the XPath nomenclature is used.
\begin{lstlisting}[style=XML]
<Samplers>
  <MonteCarlo name="MC_external">
     ...
    <variable name="Models|ROM@subType:SciKitLearn@name:ROM1|C">
      <distribution>C_distrib</distribution>
    </variable>
    <variable name="Models|ROM@subType:SciKitLearn@name:ROM1|tol">
      <distribution>toll_distrib</distribution>
    </variable>
    <variable name="Samplers|Grid@name:'+
          'GridName|variable@name:var1|grid@construction:equal@type:value@steps">
      <distribution>categorical_step_distrib</distribution>
    </variable>
    ...
  </MonteCarlo>
</Samplers>
\end{lstlisting}
In the above example, it can be inferred that each XML node (subnode) needs to be separated by a ``|'' separator. In addition,
every time an XML node has attributes, the user can specify them using the ``@'' separator to specify a value for them.
The first variable above will be pointing to the following XML sub-node ( \xmlNode{C}):
\begin{lstlisting}[style=XML]
<Models>
  <ROM name="ROM1" subType="SciKitLearn">
     ...
     <C>10.0</C>
    ...
 </ROM>
</Models>
\end{lstlisting}
The second variable above will be pointing to the following XML sub-node ( \xmlNode{tol}):
\begin{lstlisting}[style=XML]
<Models>
  <ROM name="ROM1" subType="SciKitLearn">
     ...
     <tol>0.0001</tol>
    ...
 </ROM>
</Models>
\end{lstlisting}
The third variable above will be pointing to the following XML attribute ( \xmlAttr{steps}):
\begin{lstlisting}[style=XML]
<Samplers>
  <Grid name="GridName">
     ...
    <variable name="var1">
       ...
      <grid construction="equal" type="value" steps="1">0 1</grid>
       ...
    </variable>

    ...
  </MonteCarlo>
</Samplers>
\end{lstlisting}

The above nomenclature must be used for all the variables to be sampled and for the variables generated by the two methods contained, in case, in
the module that gets specified by the \xmlNode{conversionModule} in the \xmlNode{Code} section.
\\ Finally the SLAVE RAVEN input file (s) must be ``tagged'' with the attribute  \xmlAttr{type="raven"} in the Files section. For example,

\begin{lstlisting}[style=XML]
<Files>
    <Input name="slaveRavenInputFile" type="raven" >
      test_rom_trainer.xml
    </Input>
</Files>
\end{lstlisting}

\subsubsection{ExternalXML and RAVEN interface}
Care must be taken if the SLAVE RAVEN uses \xmlNode{ExternalXML} nodes.  In this case, each file containing
external XML nodes must be added in the \xmlNode{Step} as an \xmlNode{Input} class \xmlAttr{Files} to make sure it gets copied to
the individual run directory.  The type for these files can be anything, with the exception of type
\xmlString{raven}.

%%%%%%%%%%%%%%%%%%%%%%%%%%%%
%%%%%% RELAP5  INTERFACE  %%%%%%
%%%%%%%%%%%%%%%%%%%%%%%%%%%%
\subsection{RELAP5 Interface}
\label{subsec:RELAP5Interface}

\subsubsection{Sequence}
In the \xmlNode{Sequence} section, the names of the steps declared in the
\xmlNode{Steps} block should be specified.
%
As an example, if we called the first multirun ``Grid\_Sampler'' and the second
multirun ``MC\_Sampler'' in the sequence section we should see this:
\begin{lstlisting}[style=XML]
<Sequence>Grid_Sampler,MC_Sampler</Sequence>
\end{lstlisting}
%%%%%%%%%%%%%%%%%%%%%%%%%%%%%%%%%%%%%%%%%%%%%%%%%%%

\subsubsection{batchSize and mode}
For the \xmlNode{batchSize} and \xmlNode{mode} sections please refer to the
\xmlNode{RunInfo} block in the previous chapters.
%
%%%%%%%%%%%%%%%%%%%%%%%%%%%%%%%%%%%%%%%%%%%%%%%%%%%%
\subsubsection{RunInfo}
After all of these blocks are filled out, a standard example RunInfo block may
look like the example below:
\begin{lstlisting}[style=XML]
<RunInfo>
  <WorkingDir>~/workingDir</WorkingDir>
  <Sequence>Grid_Sampler,MC_Sampler</Sequence>
  <batchSize>1</batchSize>
  <mode>mpi</mode>
  <expectedTime>1:00:00</expectedTime>
  <ParallelProcNumb>1</ParallelProcNumb>
</RunInfo>
\end{lstlisting}
%%%%%%%%%%%%%%%%%%%%%%%%%%%%%%%%%%%%%%%%%%%%%%%%%%%%%%%%%%%
\subsubsection{Files}
In the \xmlNode{Files} section, as specified before, all of the files needed for
the code to run should be specified.
%
In the case of RELAP5, the files typically needed are:
\begin{itemize}
  \item RELAP5 Input file
  \item Table file or files that RELAP needs to run
\end{itemize}
Example:
\begin{lstlisting}[style=XML]
<Files>
  <Input name='tpfh2o' type=''>tpfh2o</Input>
  <Input name='inputrelap.i' type=''>X10.i</Input>
</Files>
\end{lstlisting}

It is a good practice to put inside the working directory all of these files and
also:
\begin{itemize}
  \item the RAVEN input file
  \item the license for the executable of RELAP5
\end{itemize}
\textcolor{red}{
\textbf{It is important to notice that the interface output collection relies on the MINOR EDITS. The user must specify the MINOR
EDITS block and those variables are the only one the INTERFACE will read and make available to RAVEN. In addition, it is important to notice that:}
\begin{itemize}
  \item \textbf{the simulation time is stored in a variable called \textit{``time''}};
  \item \textbf{all the variables specified in the MINOR EDIT block are going to be converted using underscores (e.g.  an edit such as
  $301 \:\:\: p \:\:\: 345010000$ will be named in the converted CSVs as $p\_345010000$).In addition, if a variable contains spaces, the trailing spaces
   are going to be removed and internal spaces are replaced with underscores (e.g. $HTTEMP 1131008 12$ will become $HTTEMP\_1131008\_12$}.
\end{itemize}
}

Remember also that a RELAP5 simulation run is considered successful (i.e., the simulation did not crash) if it terminates with the following
message:
\textcolor{red}{Transient terminated by end of time step cards}
or
\textcolor{red}{Transient terminated by trip}

If the a RELAP5 simulation run stops with messages other than this one (e.g., `` Transient terminated by failure.'') than the simulation is considered as
crashed, i.e., it will not be saved.
Hence, it is strongly recommended to set up the RELAP5 input file so that the simulation exiting conditions are set through control logic trip variables
(e.g., simulation mission time and clad temperature equal to clad failure temperature).

%%%%%%%%%%%%%%%%%%%%%%%%%%%%%%%%%%%%%%%%%%%%%%%%%%%%
\subsubsection{Models}
\label{subsubsection:Relap5Models}
For the \xmlNode{Models} block here is a standard example of how it would look
when using RELAP5 as the external model:
\begin{lstlisting}[style=XML]
<Models>
  <Code name='MyRELAP' subType='Relap5'>
    <executable>~/path_to_the_executable</executable>
  </Code>
</Models>
\end{lstlisting}
In case the \textbf{multi-deck} approach is used in RELAP5, the interface is going to load all the outputs in one CSV RAVEN is
going to read. This means that all the decks' outputs are going to be loaded in one of the Output of RAVEN. In case the user
wants to select the outputs coming from only one deck, the following XML node needs to be specified:
\begin{itemize}
   \item \xmlNode{outputDeckNumber}, \xmlDesc{integer, optional parameter}, the deck number from
   which the results needs to be retrieved. \default{all}.
\end{itemize}
In addition, if some command line parameters need to be passed to RELAP5 \\(e.g. ``-r
$\: restartFileWithCustomName.r$''), the user might use (optionally) the \xmlNode{clargs} XML nodes.
\begin{lstlisting}[style=XML]
<Models>
  <Code name='MyRELAP' subType='Relap5'>
    <executable>~/path_to_the_executable</executable>
    <outputDeckNumber>1</outputDeckNumber>
    <clargs type="text" arg="-r restartFileWithCustomName.r"/>
  </Code>
</Models>
\end{lstlisting}

An additional feature of the RELAP5 Code interface is the possibility to specify operation based on the value of user-inputted
cards. For example, let's assume the values in cards 1180801:2 and 1180802:2 must come from a calculation based on sampled variables (e.g. 20100154:2 and 20100155:2), the user can specify the following XML node:

\begin{itemize}
   \item \xmlNode{operator}, \xmlDesc{XML node, optional parameter}, The operator block.
    This XML node must contain the following attribute:
     \begin{itemize}
      \item \textit{variables}, \xmlDesc{comma separated list, required parameter}, The list of variables
       (coming from a Sampler) that will be used in the  \xmlNode{expression} XML node.
      \end{itemize}
     Within the  \xmlNode{operator} the following XML sub-nodes must be specified:
     \begin{itemize}
        \item \xmlNode{expression}, \xmlDesc{string, required parameter}, The string representing the expression to be
        performed. The ``card'' (if needed to be used) must be identified with the token \%card\% and it
         will be replaced with the values of the cards
        (specified in the XML node \xmlNode{cards} ) from the original input file. In this expression, all the functions available in
        the Python $math$ module can be used (e.g. $sqrt$, $exp$, $sin$, etc.).
         \item \xmlNode{cards}, \xmlDesc{comma separated list, required parameter},
          The list of cards in the original input file whose values need to be replaced by the value resulting from the expression
          contained in \xmlNode{expression}.
     \end{itemize}
    \nb the user can specify as many \xmlNode{operator} nodes as needed.
\end{itemize}
An example is reported below:
\begin{lstlisting}[style=XML]
<Models>
  <Code name='MyRELAP' subType='Relap5'>
    <executable>~/path_to_the_executable</executable>
    ...
    <operator variables="20100154:2,20100155:2">
      <expression> %card%*20100155:2*2./20100155:2</expression>
      <cards>1180801:2,1180802:2,1180901:3</cards>
    </operator>
    ...
  </Code>
</Models>
\end{lstlisting}


%%%%%%%%%%%%%%%%%%%%%%%%%%%%%%%%%%%%%%%%%%%%%%%%%%%%%%%%%
\subsubsection{Distributions}
The \xmlNode{Distribution} block defines the distributions that are going
to be used for the sampling of the variables defined in the \xmlNode{Samplers}
block.
%
For all the possibile distributions and all their possible inputs please see the
chapter about Distributions (see~\ref{sec:distributions}).
%
Here we give a general example of three different distributions:
\begin{lstlisting}[style=XML,morekeywords={name,debug}]
<Distributions verbosity='debug'>
  <Triangular name='BPfailtime'>
    <apex>5.0</apex>
    <min>4.0</min>
    <max>6.0</max>
  </Triangular>
  <LogNormal name='BPrepairtime'>
    <mean>0.75</mean>
    <sigma>0.25</sigma>
  </LogNormal>
  <Uniform name='ScalFactPower'>
    <lowerBound>1.0</lowerBound>
    <upperBound>1.2</upperBound>
  </Uniform>
 </Distributions>
\end{lstlisting}

It is good practice to name the distribution something similar to what kind of
variable is going to be sampled, since there might be many variables with the
same kind of distributions but different input parameters.
%
%%%%%%%%%%%%%%%%%%%%%%%%%%%%%%%%%%%%%%%%%%%%%%%%%%%%%%%%%
\subsubsection{Samplers}
In the \xmlNode{Samplers} block we want to define the variables that are going
to be sampled.
%
\textbf{Example}:
We want to do the sampling of 3 variables:
\begin{itemize}
  \item Battery Fail Time
  \item Battery Repair Time
  \item Scaling Factor Power Rate
\end{itemize}

We are going to sample these 3 variables using two different sampling methods:
grid and MonteCarlo.

In RELAP5, the sampler reads the variable as, given the name, the first number
is the card number and the second number is the word number.
%
In this example we are sampling:
\begin{itemize}
  \item For card 0000588 (trip) the word 6 (battery failure time)
  \item For card 0000575 (trip) the word 6 (battery repair time)
  \item For card 20210000 (reactor power) the word 4 (reactor scaling factor)
\end{itemize}

We proceed to do so for both the Grid sampling and the MonteCarlo sampling.

\begin{lstlisting}[style=XML,morekeywords={name,type,construction,lowerBound,steps,limit,initialSeed}]
<Samplers verbosity='debug'>
  <Grid name='Grid_Sampler' >
    <variable name='0000588:6'>
      <distribution>BPfailtime</distribution>
      <grid type='value' construction='equal'  steps='10'>0.0 28800</grid>
    </variable>
    <variable name='0000575:6'>
      <distribution>BPrepairtime</distribution>
      <grid type='value' construction='equal' steps='10'>0.0 28800</grid>
    </variable>
    <variable name='20210000:4'>
      <distribution>ScalFactPower</distribution>
      <grid type='value' construction='equal' steps='10'>1.0 1.2</grid>
    </variable>
  </Grid>
  <MonteCarlo name='MC_Sampler'>
     <samplerInit>
       <limit>1000</limit>
     </samplerInit>
    <variable name='0000588:6'>
      <distribution>BPfailtime</distribution>
    </variable>
    <variable name='0000575:6'>
      <distribution>BPrepairtime</distribution>
    </variable>
    <variable name='20210000:4'>
      <distribution>ScalFactPower</distribution>
    </variable>
  </MonteCarlo>
</Samplers>
\end{lstlisting}

In case the RELAP5 input file is a multi-deck, the user can specify the deck to which each sampled variable
corresponds to. As an example, the following sampling strategy:

\begin{lstlisting}[style=XML,morekeywords={name,type,construction,lowerBound,steps,limit,initialSeed}]
<MonteCarlo name='MC_Sampler'>
   <samplerInit>
     <limit>1000</limit>
   </samplerInit>
  <variable name='1|0000588:6'>
    <distribution>BPfailtime</distribution>
  </variable>
  <variable name='2|0000575:6'>
    <distribution>BPrepairtime</distribution>
  </variable>
</MonteCarlo>
</Samplers>
\end{lstlisting}
performs:
\begin{itemize}
  \item the sampling of the distribution \\\xmlNode{BPfailtime} and it provides the sampled value
        to the 6th word of card 0000588 for the first deck
  \item the sampling of the distribution \\\xmlNode{BPrepairtime} and it provides the sampled value
        to the 6th word of card 0000575 for the second deck
\end{itemize}

It can be seen that each variable is connected with a proper distribution
defined in the \\\xmlNode{Distributions} block (from the previous example).
%
The following demonstrates how the input for the first variable is read.

We are sampling a a variable situated in word 6 of the card 0000588 using a Grid
sampling method.
%
The distribution that this variable is following is a Triangular distribution
(see section above).
%
We are sampling this variable beginning from 0.0 in 10 \textit{equal} steps of
2880.
%
%%%%%%%%%%%%%%%%%%%%%%%%%%%%%%%%%%%%%%%%%%%%%%%%%%%%%%%%%%%
\subsubsection{Steps}
For a RELAP interface, the \xmlNode{MultiRun} step type will most likely be
used.
%
First, the step needs to be named: this name will be one of the names used in
the \xmlNode{Sequence} block.
%
In our example, \texttt{Grid\_Sampler} and \texttt{MC\_Sampler}.
%
\begin{lstlisting}[style=XML,morekeywords={name,debug,re-seeding}]
     <MultiRun name='Grid_Sampler' verbosity='debug'>
\end{lstlisting}

With this step, we need to import all the files needed for the simulation:
\begin{itemize}
  \item RELAP input file
  \item element tables -- tpfh2o
\end{itemize}
\begin{lstlisting}[style=XML,morekeywords={name,class,type}]
    <Input   class='Files' type=''>inputrelap.i</Input>
    <Input   class='Files' type=''>tpfh2o</Input>
\end{lstlisting}
We then need to define which model will be used:
\begin{lstlisting}[style=XML]
    <Model  class='Models' type='Code'>MyRELAP</Model>
\end{lstlisting}
We then need to specify which Sampler is used, and this can be done as follows:
\begin{lstlisting}[style=XML]
    <Sampler class='Samplers' type='Grid'>Grid_Sampler</Sampler>
\end{lstlisting}
And lastly, we need to specify what kind of output the user wants.
%
For example the user might want to make a database (in RAVEN the database
created is an HDF5 file).
%
Here is a classical example:
\begin{lstlisting}[style=XML,morekeywords={class,type}]
    <Output  class='Databases' type='HDF5'>Grid_out</Output>
\end{lstlisting}
Following is the example of two MultiRun steps which use different sampling
methods (grid and Monte Carlo), and creating two different databases for each
one:
\begin{lstlisting}[style=XML]
<Steps verbosity='debug'>
  <MultiRun name='Grid_Sampler' verbosity='debug'>
    <Input   class='Files' type=''>inputrelap.i</Input>
    <Input   class='Files'     type=''    >tpfh2o</Input>
    <Model   class='Models'    type='Code'>MyRELAP</Model>
    <Sampler class='Samplers'  type='Grid'>Grid_Sampler</Sampler>
    <Output  class='Databases' type='HDF5'>Grid_out</Output>
  </MultiRun>
  <MultiRun name='MC_Sampler' verbosity='debug' re-seeding='210491'>
    <Input   class='Files' type=''>inputrelap.i</Input>
    <Input   class='Files'     type=''          >tpfh2o</Input>
    <Model   class='Models'    type='Code'      >MyRELAP</Model>
    <Sampler class='Samplers'  type='MonteCarlo'>MC_Sampler</Sampler>
    <Output  class='Databases' type='HDF5'      >MC_out</Output>
  </MultiRun>
</Steps>
\end{lstlisting}
%%%%%%%%%%%%%%%%%%%%%%%%%%%%%%%%%%%%%%%%%%%%%%%%%%%%%%
\subsubsection{Databases}
As shown in the \xmlNode{Steps} block, the code is creating two database objects
called \texttt{Grid\_out} and \texttt{MC\_out}.
%
So the user needs to input the following:
\begin{lstlisting}[style=XML]
<Databases>
  <HDF5 name="Grid_out" readMode="overwrite"/>
  <HDF5 name="MC_out" readMode="overwrite"/>
</Databases>
\end{lstlisting}
As listed before, this will create two databases.
%
The files will have names corresponding to their \xmlAttr{name} appended with
the .h5 extension (i.e. \texttt{Grid\_out.h5} and \texttt{MC\_out.h5}).

\subsubsection{Modified Version of the Institute of Nuclear Safety System Incorporated (Japan)}
The Institute of Nuclear Safety System Incorporated (Japan) has modified the \textbf{RELAP5}  source code
in order to be able to control some additional parameters from an auxiliary input file (\textbf{modelPar.inp}).
\\In order to use this interface, the user needs to input the $subType$ attribute\textbf{Relap5inssJp}:
\begin{lstlisting}[style=XML]
<Models>
  <Code name='MyRELAP' subType='Relap5'>
    <executable>~/path_to_the_executable</executable>
    <!-- here is taking the output from the first deck only -->
    <outputDeckNumber>1</outputDeckNumber>
  </Code>
</Models>
\end{lstlisting}
For perturbing such input file, the approach presented in section \ref{subsec:genericInterface} (Generic Interface)
has been employed. For the standard \textbf{RELAP5} input, the same approach previously in this section is used.
\\For example, in the following Sampler block, the card $9100101$ is perturbed with the same approach used in standard \textbf{RELAP5}; in addition, the variable $modelParTest$  is going to be perturbed in the \textbf{modelPar.inp} input file.
\begin{lstlisting}[style=XML]
    <MonteCarlo name="mc_loca">
      <samplerInit>
        <limit>1</limit>
      </samplerInit>
      <variable name="9100101:3">
        <distribution>break_size</distribution>
      </variable>
      <variable name="modelParTest">
          <distribution>break_size</distribution>
      </variable>
    </MonteCarlo>
\end{lstlisting}

%%%%%%%%%%%%%%%%%%%%%%%%%%%
%%%%%% RELAP7 INTERFACE  %%%%%%
%%%%%%%%%%%%%%%%%%%%%%%%%%%
\subsection{RELAP7 Interface}
This section covers the input specifications for running RELAP7 through RAVEN.
It is important to notice that this short explanation assumes that the reader already knows
how to use the control logic system in RELAP7.
Since the presence of the control logic system in RELAP7, this code interface is different with respect to the others
and uses some special keyword available in RAVEN (see the following).

\subsubsection{Files}
In the \xmlNode{Files} section, as specified before, all of the files needed for
the code to run should be specified.
%
In the case of RELAP7, the files typically needed are the following:
\begin{itemize}
  \item RELAP7 Input file
  \item Control Logic file
\end{itemize}
Example:
\begin{lstlisting}[style=XML]
<Files>
  <Input name='nat_circ.i' type=''>nat_circ.i</Input>
  <Input name='control_logic.py' type=''>control_logic.py</Input>
</Files>
\end{lstlisting}
The RAVEN/RELAP7 interface recognizes as RELAP7 inputs the files with the extensions  ``*.i'', ``*.inp'' and ``*.in''.

%%%%%%%%%%%%%%%%%%%%%%%%%%%%%%%%%%%%%%%%%%%%%%%%%%%%%%%%
\subsubsection{Models}
For the \xmlNode{Models} block RELAP7 uses the RAVEN executable, since through this executable the stochastic
environment gets activated (possibility to sample parameters directly in the control logic system)
%
Here is a standard example of what can be used to use RELAP7 as the model:
\begin{lstlisting}[style=XML]
<Models>
    <Code name='MyRAVEN' subType='RAVEN'><executable>~path/to/RAVEN-opt</executable></Code>
</Models>
\end{lstlisting}
%%%%%%%%%%%%%%%%%%%%%%%%%%%%%%%%%%%%%%%%%%%%%%%%%%%%%%%%
\subsubsection{Distributions}
As for all the other codes interfaces  the \xmlNode{Distributions} block needs to be specified in order to employ
as sampling strategy (e.g. MonteCarlo, Stratified, etc.). In this block, the user specifies the distributions that need to be used.
Once the user defines the distributions in this block, RAVEN activates the Distribution environment in the RAVEN/RELAP7 control logic
system. The sampling of the parameters is then performed directly in the control logic input file.

%
For example, let's consider the sampling of a normal distribution for the primary pressure in
RELAP7:
%
\begin{lstlisting}[style=XML]
<Distributions>
 <Normal name="Prim_Pres">
 <mean>1000000</mean>
 <sigma>100<sigma/>
 </Normal>
</Distributions>
\end{lstlisting}
In order to change a parameter (independently on the sampling strategy), the control logic input file should be modified as follows:
%\lstset{margin=1.5cm}
\begin{lstlisting}[language=Python]
def initial_function(monitored, controlled, auxiliary)
    print("monitored",monitored,"controlled",
    controlled,"auxiliary",auxiliary)

    controlled.pressureInPressurizer =
     distributions.Prim_Pres.getDistributionRandom()
    return
\end{lstlisting}

%%%%%%%%%%%%%%%%%%%%%%%%%%%%%%%%%%%%%%%%%%%%%%%
\subsubsection{Samplers}
In the \xmlNode{Samplers} block, all the variables that needs to be sampled must be specified.
In case some of these variables are directly sampled in the Control Logic system, the
\xmlNode{variable} needs to be replaced with \xmlNode{Distribution}. In this way, RAVEN is able
to understand which variables needs to be directly modified through input file (i.e. modifying the original
input file *.i)  and which variables are going to be ``sampled'' through the control logic system.
%
For the example, we are performing Grid Sampling.
%
The global initial pressure wasn't specified in the control logic so it is going to be specified
using the node \xmlNode{variable}. The ``pressureInPressurizer'' variable is instead sampled in the
control logic system; for this reason, it is going to be specified using the node  \xmlNode{Distribution}.
%
For example,
%
\begin{lstlisting}[style=XML]
<Samplers>
 <Grid name="MC_samp">
   <samplerInit> <limit>500</limit> </samplerInit>
   <variable name="GlobalParams|global_init_P">
      <distribution>Prim_Pres</distribution>
      <grid construction="equal" steps="10" type="CDF">0.0 1.0</grid>
   </variable>
   <Distribution name="pressureInPressurizer">
      <distribution>Prim_Pres</distribution>
      <grid construction="equal" steps="10" type="CDF">0.0 1.0</grid>
   </Distribution>
 </Grid>
</Samplers>
\end{lstlisting}


%%%%%%%%%%%%%%%%%%%%%%%%%%%%%%%%%%
%%%%%% MooseBasedApp INTERFACE  %%%%%%
%%%%%%%%%%%%%%%%%%%%%%%%%%%%%%%%%%
\subsection{MooseBasedApp Interface}
\subsubsection{Files}
In the \xmlNode{Files} section, as specified before, all of the files needed for
the code to run should be specified.
%
In the case of any MooseBasedApp, the files typically needed are the following:
\begin{itemize}
  \item MooseBasedApp GetPot input file
  \item Restart Files (if the calculation is instantiated from a restart point)
  \item Mesh Files (in case the mesh is externally specified)
  \item Any other generic input file (CSVs with Power histories, boundary conditions files, etc.)
\end{itemize}
Example:
\begin{lstlisting}[style=XML]
<Files>
  <Input name='mooseBasedApp.i' type=''>mooseBasedApp.i</Input>
  <Input name='0020_mesh.cpr' type=''>0020_mesh.cpr</Input>
  <Input name='0020.xdr.0000' type="">0020.xdr.0000</Input>
  <Input name='0020.rd-0' type="">0020.rd-0</Input>
  <Input name='exodus_mesh.e' type="">exodus_mesh.e</Input>
  <Input name='a_generic_additional_input_file.csv' type="Generic">a_generic_additional_input_file.csv</Input>
</Files>
\end{lstlisting}
If any file is tagged with the type \texttt{Generic}, it will be perturbable with the approach (wildcards)
 explained in the generic code interface (see  \ref{subsec:genericInterface}).



%%%%%%%%%%%%%%%%%%%%%%%%%%%%%%%%%%%%%%%%%%%%%%%%%%%%%%%%
\subsubsection{Models}
In the \xmlNode{Models} block particular MooseBasedApp executable needs to be specified.
%
Here is a standard example of what can be used to use with a typical MooseBasedApp (Bison) as the model:
\begin{lstlisting}[style=XML]
<Models>
    <Code name='MyMooseBasedApp' subType='MooseBasedApp'><executable>~path/to/Bison-opt</executable></Code>
</Models>
\end{lstlisting}
%%%%%%%%%%%%%%%%%%%%%%%%%%%%%%%%%%%%%%%%%%%%%%%%%%%%%%%%
%%%%%%%%%%%%%%%%%%%%%%%%%%%%%%%%%%%%%%%%%%%%%%%%%%%%%%%%%
\subsubsection{Distributions}
The \xmlNode{Distributions} block defines the distributions that are going
to be used for the sampling of the variables defined in the \xmlNode{Samplers}
block.
%
For all the possible distributions and all their possible inputs please see the
chapter about Distributions (see~\ref{sec:distributions}).
%
Here we give a general example of three different distributions:
\begin{lstlisting}[style=XML,morekeywords={name,debug}]
<Distributions>
    <Normal name='ThermalConductivity1'>
        <mean>1</mean>
        <sigma>0.001</sigma>
        <lowerBound>0.5</lowerBound>
        <upperBound>1.5</upperBound>
    </Normal>
    <Normal name='SpecificHeat'>
        <mean>1</mean>
        <sigma>0.4</sigma>
        <lowerBound>0.5</lowerBound>
        <upperBound>1.5</upperBound>
    </Normal>
    <Triangular name='ThermalConductivity2'>
        <apex>1</apex>
        <min>0.1</min>
        <max>4</max>
    </Triangular>
</Distributions>
\end{lstlisting}

It is good practice to name the distribution something similar to what kind of
variable is going to be sampled, since there might be many variables with the
same kind of distributions but different input parameters.
%
%%%%%%%%%%%%%%%%%%%%%%%%%%%%%%%%%%%%%%%%%%%%%%%%%%%%%%%%%
\subsubsection{Samplers}
In the \xmlNode{Samplers} block we want to define the variables that are going
to be sampled.
%
\textbf{Example}:
We want to do the sampling of 3 variables:
\begin{itemize}
  \item Thermal Conductivity of the Fuel;
  \item Specific Heat Transfer Ratio of the Cladding;
  \item Thermal Conductivity of the Cladding.
\end{itemize}

We are going to sample these 3 variables using two different sampling methods:
Grid and Monte-Carlo.

In order to perturb any MooseBasedApp, the user needs to specify the variables to be
sampled indicating the path to the value separated with the symbol ``$|$''. For example,
if the variable that we want to perturb is specified in the input as follows:
\begin{lstlisting}[style=XML]
[Materials]
  ...
  [./heatStructure]
     ...
     thermal_conductivity = 1.0
     ...
  [../]
  ...
[]
\end{lstlisting}
the variable name in the Sampler input block needs to be named as follows:
\begin{lstlisting}[style=XML]
...
<Samplers>
  <aSampler name='aUserDefinedName' >
    <variable name='Materials|heatStructure|thermal_conductivity'>
      ...
    </variable>
  </aSampler>
</Samplers>
...
\end{lstlisting}
%
In case some variables in external (\texttt{Generic} input files) need to be perturbed, 
the wildcard approach can be used (for those variables):
\begin{lstlisting}[style=XML]
...
<Samplers>
  <aSampler name='aUserDefinedName' >
    <variable name='aWildCard1'>
      ...
    </variable>
        <variable name='aWildCard2'>
      ...
    </variable>
    <variable name='Materials|heatStructure|thermal_conductivity'>
      ...
    </variable>
  </aSampler>
</Samplers>
...
\end{lstlisting}
In this case the tagged file (\texttt{Generic}) will be parsed to find the variables
\texttt{\$RAVEN-aWildCard1\$} and \texttt{\$RAVEN-aWildCard1\$} and to replace their values
with the corresponding sampled variables (for more details, see  \ref{subsec:genericInterface})

In this example, we proceed to do so for both the Grid sampling and the Monte-Carlo sampling.
\begin{lstlisting}[style=XML,morekeywords={name,type,construction,lowerBound,steps,limit,initialSeed}]
<Samplers verbosity='debug'>
    <Grid name='myGrid'>
      <variable name='Materials|heatStructure1|thermal_conductivity' >
        <distribution>ThermalConductivity1</distribution>
        <grid         type='value' construction='custom' >0.6 0.7 0.8</grid>
      </variable>
      <variable name='Materials|heatStructure1|specific_heat' >
        <distribution >SpecificHeat</distribution>
        <grid         type='CDF'    construction='custom'>0.5 1.0 0.0</grid>
      </variable>
      <variable name='Materials|heatStructure2|thermal_conductivity'>
        <distribution  >ThermalConductivity2</distribution>
        <grid type='value' upperBound='4' construction='equal' steps='1'>0.5</grid>
      </variable>
      <variable name='aWildCard1'>
        <distribution  >ThermalConductivity2</distribution>
        <grid type='value' upperBound='4' construction='equal' steps='1'>0.5</grid>
      </variable>
    </Grid>
  <MonteCarlo name='MC_Sampler' limit='1000'>
      <variable name='Materials|heatStructure1|thermal_conductivity' >
        <distribution>ThermalConductivity1</distribution>
      </variable>
      <variable name='Materials|heatStructure1|specific_heat' >
        <distribution >SpecificHeat</distribution>
      </variable>
      <variable name='Materials|heatStructure2|thermal_conductivity'>
        <distribution  >ThermalConductivity2</distribution>
      </variable>
      <variable name='aWildCard1'>
        <distribution  >ThermalConductivity2</distribution>
      </variable>
  </MonteCarlo>
</Samplers>
\end{lstlisting}
%%%%%%%%%%%%%%%%%%%%%%%%%%%%%%%%%%%%%%%%%%%%%%%%%%%%%%%%%%%
\subsubsection{Steps}
For a MooseBasedApp, the \xmlNode{MultiRun} step type will most likely be
used, as first step.
%
First, the step needs to be named: this name will be one of the names used in
the \xmlNode{Sequence} block.
%
In our example, \texttt{Grid\_Sampler} and \texttt{MC\_Sampler}.
%
\begin{lstlisting}[style=XML,morekeywords={name,debug,re-seeding}]
     <MultiRun name='Grid_Sampler' >
\end{lstlisting}

With this step, we need to import all the files needed for the simulation:
\begin{itemize}
  \item MooseBasedApp YAML input file;
  \item eventual restart files (optional);
  \item other auxiliary files (e.g., powerHistory tables, etc.).
\end{itemize}
\begin{lstlisting}[style=XML,morekeywords={name,class,type}]
    <Input   class='Files' type=''>mooseBasedApp.i</Input>
    <Input   class='Files' type=''>0020_mesh.cpr</Input>
    <Input   class='Files' type=''>0020.xdr.0000</Input>
    <Input   class='Files' type=''>0020.rd-0</Input>
\end{lstlisting}
We then need to define which model will be used:
\begin{lstlisting}[style=XML]
    <Model  class='Models' type='Code'>MyMooseBasedApp</Model>
\end{lstlisting}
We then need to specify which Sampler is used, and this can be done as follows:
\begin{lstlisting}[style=XML]
    <Sampler class='Samplers' type='Grid'>Grid_Sampler</Sampler>
\end{lstlisting}
And lastly, we need to specify what kind of output the user wants.
%
For example the user might want to make a database (in RAVEN the database
created is an HDF5 file) and a DataObject of type PointSet, to use in sub-sequential
post-processing.
%
Here is a classical example:
\begin{lstlisting}[style=XML,morekeywords={class,type}]
    <Output  class='Databases' type='HDF5'>MC_out</Output>
    <Output  class='DataObjects' type='PointSet'>MCOutData</Output>
\end{lstlisting}

Following is the example of two MultiRun steps which use different sampling
methods (grid and Monte Carlo), and creating two different databases for each
one:
\begin{lstlisting}[style=XML]
<Steps verbosity='debug'>
  <MultiRun name='Grid_Sampler' verbosity='debug'>
    <Input  class='Files' type=''>mooseBasedApp.i</Input>
    <Input  class='Files' type=''>0020_mesh.cpr</Input>
    <Input  class='Files' type='' >0020.xdr.0000</Input>
    <Input  class='Files' type=''>0020.rd-0</Input>
    <Model  class='Models'    type='Code'>MyMooseBasedApp</Model>
    <Sampler class='Samplers'  type='Grid'>Grid_Sampler</Sampler>
    <Output  class='Databases' type='HDF5'>Grid_out</Output>
    <Output  class='DataObjects' type='PointSet'>gridOutData</Output>
  </MultiRun>
  <MultiRun name='MC_Sampler' verbosity='debug' re-seeding='210491'>
    <Input  class='Files' type=''>mooseBasedApp.i</Input>
    <Input  class='Files' type=''>0020_mesh.cpr</Input>
    <Input  class='Files' type='' >0020.xdr.0000</Input>
    <Input  class='Files' type=''>0020.rd-0</Input>
    <Model  class='Models'    type='Code'>MyMooseBasedApp</Model>
    <Sampler class='Samplers'  type='MonteCarlo' >MC_Sampler</Sampler>
    <Output  class='Databases' type='HDF5' >MC_out</Output>
    <Output  class='DataObjects' type='PointSet'>MCOutData</Output>
  </MultiRun>
</Steps>
\end{lstlisting}
%%%%%%%%%%%%%%%%%%%%%%%%%%%%%%%%%%%%%%%%%%%%%%%%%%%%%%
\subsubsection{Databases}
As shown in the \xmlNode{Steps} block, the code is creating two database objects
called \texttt{Grid\_out} and \texttt{MC\_out}.
%
So the user needs to input the following:
\begin{lstlisting}[style=XML]
<Databases>
  <HDF5 name="Grid_out" readMode="overwrite"/>
  <HDF5 name="MC_out" readMode="overwrite"/>
</Databases>
\end{lstlisting}
As listed before, this will create two databases.
%
The files will have names corresponding to their \xmlAttr{name} appended with
the .h5 extension (i.e. \texttt{Grid\_out.h5} and \texttt{MC\_out.h5}).
%%%%%%%%%%%%%%%%%%%%%%%%%%%%%%%%%%%%%%%%%%%%%%%%%%%%%%
\subsubsection{DataObjects}
As shown in the \xmlNode{Steps} block, the code is creating two DataObjects of type PointSet
called \texttt{gridOutData} and \texttt{MCOutData}.
%
So the user needs to input the following:
\begin{lstlisting}[style=XML]
<DataObjects>
    <PointSet name='gridOutData'>
      <Input>
          Materials|heatStructure2|thermal_conductivity,
          Materials|heatStructure1|specific_heat,
          Materials|heatStructure2|thermal_conductivity
      </Input>
      <Output>aveTempLeft</Output>
    </PointSet>
    <PointSet name='MCOutData'>
      <Input>
          Materials|heatStructure2|thermal_conductivity,
          Materials|heatStructure1|specific_heat,
          Materials|heatStructure2|thermal_conductivity
      </Input>
      <Output>aveTempLeft</Output>
    </PointSet>
</DataObjects>
\end{lstlisting}
As listed before, this will create two DataObjects that can be used in sub-sequential post-processing.
%%%%%%%%%%%%%%%%%%%%%%%%%%%%%%%%%%%%%%%%%%%%%%%%%%%%%%
\subsubsection{OutStreams}
As fully explained in section~\ref{sec:outstream}, if the user want to print out or plot the content of a \textbf{DataObjects},
he needs to create an \textbf{OutStream} in the \xmlNode{OutStreams} XML block.
\\As it shown in the example below, for MooseBasedApp (and any other Code interface that might use the symbol $|$
for the Sampler's variable syntax), in the Plot \xmlNode{x} and \xmlNode{y} specification, the user needs to
utilize curly brackets.
\begin{lstlisting}[style=XML]
<OutStreams>
  <Print name='gridOutDataDumpCSV'>
    <type>csv</type>
    <source>gridOutData</source>
  </Print>
   <Plot verbosity='debug' name='test'   overwrite='False'>
    <plotSettings>
       <plot>
        <type>line</type>
        <x>MCOutData|Input|{Materials|heatStructure2|thermal_conductivity}</x>
        <y>MCOutData|Output|aveTempLeft</y>
        <kwargs><color>blue</color></kwargs>
      </plot>
    </plotSettings>
    <actions><how>screen,png</how></actions>
  </Plot>
</OutStreams>
\end{lstlisting}

%%%%%%%%%%%%%%%%%%%%%%%%%%%%%%%%%%
%%%%%% Moose VectorPostProcessor INTERFACE  %%%%%%
%%%%%%%%%%%%%%%%%%%%%%%%%%%%%%%%%%
\subsection{MooseVPP Interface}

The Moose Vector Post Processor is used mainly in the solid mechanics analysis.
This interface loads the values of the vector ouput processor to a \xmlNode{DataObjects} object.

To use this interface the [DomainIntegral] needs to be present in the MooseBasedApp's
 input file and the subnode \xmlNode{fileargs} should be defined in the subnode \xmlNode{Code} in
 the \xmlNode{Models} block of the RAVEN input file. The \xmlNode{fileargs} is required to have attributes with
the below specified values:

\begin{itemize}
    \item \xmlAttr{type}, \xmlDesc{string, required field}, must be "MooseVPP"
    \item \xmlAttr{arg}, \xmlDesc{string, required field}, the string value attached to the vector post processor action
         while creating the output files.
\end{itemize}

This interface is actually identical to the MooseBasedApp interface, however there is
 few constraints on defining the output values of the post processor.
The definition of these outputs in the \xmlNode{DataObjects} depends on the definition of
 the [DomainIntegral].

The location of the value outputted is defined as \textit{ID\#} and the value is as
\textit{value\#}. The ''\#'' defines
the number of the location. The example below contains 3 locations in the [DomainIntegral]
where the values are outputted.

%
Example:
\begin{lstlisting}[style=XML]
 ...
  <Models>
    <Code name="MOOSETestApp" subType="MooseBasedApp">
      <executable>%FRAMEWORK_DIR%/../../moose/
        modules/combined/modules-%METHOD%</executable>
      <fileargs type = "MooseVPP" arg = "_J_1_" />
      <alias variable = "poissonsRatio" >
        Materials|stiffStuff|poissons_ratio</alias>
      <alias variable = "youngModulus"  >
        Materials|stiffStuff|youngs_modulus</alias>
    </Code>
  </Models>
 ...
  <DataObjects>
    <PointSet name="collset">
      <Input>youngModulus,poissonsRatio</Input>
      <Output>ID1,ID2,ID3,value1,value2,value3</Output>
    </PointSet>
  </DataObjects>
 ...
\end{lstlisting}




%%%%%%%%%%%%%%%%%%%%%%%%%%%%%%%%%%%%%
%%%%%% OPENMODELICA INTERFACE  %%%%%%
%%%%%%%%%%%%%%%%%%%%%%%%%%%%%%%%%%%%%
\subsection{OpenModelica Interface}
OpenModelica (\url{http://www.openmodelica.org}) is an open souce implementation of the Modelica simulation language.  Modelica is "a non-proprietary,
object-oriented, equation based language to conveniently model complex physical systems containing, e.g., mechanical, electrical, electronic, hydraulic,
thermal, control, electric power or process-oriented subcomponents."\footnote{\url{http://www.modelica.org}}.  Modelica models are specified in text files
with a file extension of .mo.  A standard Modelica example called BouncingBall which simulates the trajectory of an object falling in one dimension from a
height is shown as an example:
\begin{lstlisting}
model BouncingBall
  parameter Real e=0.7 "coefficient of restitution";
  parameter Real g=9.81 "gravity acceleration";
  Real h(start=1) "height of ball";
  Real v "velocity of ball";
  Boolean flying(start=true) "true, if ball is flying";
  Boolean impact;
  Real v_new;
  Integer foo;

equation
  impact = h <= 0.0;
  foo = if impact then 1 else 2;
  der(v) = if flying then -g else 0;
  der(h) = v;

  when {h <= 0.0 and v <= 0.0,impact} then
    v_new = if edge(impact) then -e*pre(v) else 0;
    flying = v_new > 0;
    reinit(v, v_new);
  end when;

end BouncingBall;
\end{lstlisting}

\subsubsection{Files}
An OpenModelica installation specific to the operating system is used to create a stand-alone executable program that performs the model calculations.
A separate XML file containing model parameters and initial conditions is also generated as part of the build process.  The RAVEN OpenModelica interface
modifies input parameters by changing copies of this file.  Both the executable and XML parameter file names must be provided to RAVEN.  In the case of
the BouncingBall model previously mentioned on the Windows operating system, the \textless Files\textgreater  specification would look like:
\begin{lstlisting}[style=XML]
<Files>
  <Input name='BouncingBall_init.xml' type=''>BouncingBall_init.xml</Input>
  <Input name='BouncingBall.exe' type=''>BouncingBall.exe</Input>
</Files>
\end{lstlisting}
\subsubsection{Models}
OpenModelica models may provide simulation output in a number of formats.  The particular format used is specified during the model generation
process.  RAVEN works best with Comma-Separated Value (CSV) files, which is one of the possible output format options.  Models are generated
using the OpenModelica Shell (OMS) command-line interface, which is part of the OpenModelica installation.  To generate an executable that provides
CSV-formatted output, use OMSl commands as follows:
\lstset{
    frame=single,
    breaklines=true,
    postbreak=\raisebox{0ex}[0ex][0ex]{\ensuremath{\color{red}\hookrightarrow\space}}
}
 \begin{enumerate}
\item Change to the directory containing the .mo file to generate an executable for:
\begin{lstlisting}
>> cd("C:/MinGW/msys/1.0/home/bobk/projects/raven/framework/CodeInterfaces/OpenModelica")
"C:/MinGW/msys/1.0/home/bobk/projects/raven/framework/CodeInterfaces/OpenModelica"
\end{lstlisting}
\item Load the model file into memory:
\begin{lstlisting}
>> loadFile("BouncingBall.mo")
true
\end{lstlisting}
\item Create the model executable, specifying CSV output format:
\begin{lstlisting}
>> buildModel(BouncingBall, outputFormat="csv")
{"C:/MinGW/msys/1.0/home/bobk/projects/raven/framework/CodeInterfaces/OpenModelica/BouncingBall","BouncingBall_init.xml"}
Warning: The initial conditions are not fully specified. Use +d=initialization for more information.
\end{lstlisting}
At this point the model executable and XML initialization file should have been created in the same directory as the original model file.
\end{enumerate}
The model executable is specified to RAVEN using the \textless Models\textgreater  section of the input file as follows:
\begin{lstlisting}[style=XML]
<Simulation>
    ...
  <Models>
    <Code name="BouncingBall" subType = "OpenModelica">
      <executable>BouncingBall.exe</executable>
    </Code>
  </Models>
    ...
</Simulation>
\end{lstlisting}
\subsubsection{CSV Output}
The CSV files produced by OpenModelica model executables require adjustment before it may be read by RAVEN.
The first few lines of original CSV output from the
BouncingBall example is shown below:
\begin{lstlisting}
"time","h","v","der(h)","der(v)","v_new","foo","flying","impact",
0,1,0,0,-9.810000000000001,0,2,1,0,
  ...
\end{lstlisting}
RAVEN will not properly read this file as-generated for two reasons:
\begin{itemize}
  \item The variable names in the first line are each enclosed in double-quotes.
  \item Each line has a trailing comma.
\end{itemize}
 The OpenModelica inteface will automatically remove the double-quotes and trailing commas through its implementation of the
finalizeCodeOutput function.


%%%%%%%%%%%%%%%%%%%%%%%%%%%%%%%%%%%%%
%%%%%% DYMOLA INTERFACE  %%%%%%%%%%%%
%%%%%%%%%%%%%%%%%%%%%%%%%%%%%%%%%%%%%
\subsection{Dymola Interface}
Modelica is "a non-proprietary, object-oriented, equation-based language to conveniently model complex physical systems containing, e.g., mechanical, electrical, electronic, hydraulic,
thermal, control, electric power or process-oriented subcomponents."\footnote{\url{http://www.modelica.org}}.  Modelica models (with a file extension of .mo) are built, translated (compiled), and simulated in Dymola (http://www.modelon.com/p-
roducts/dymola/), which is a commercial modeling and simulation environment based on the Modelica modeling language.
A standard Modelica example called BouncingBall, which simulates the trajectory of an object falling in one dimension from a height, is shown as an example:
\begin{lstlisting}
model BouncingBall
  parameter Real e=0.7 "coefficient of restitution";
  parameter Real g=9.81 "gravity acceleration";
  parameter Real hstart = 10 "height of ball at time zero";
  parameter Real vstart = 0 "velocity of ball at time zero";
  Real h(start=hstart,fixed=true) "height of ball";
  Real v(start=vstart,fixed=true) "velocity of ball";
  Boolean flying(start=true) "true, if ball is flying";
  Boolean impact;
  Real v_new;
  Integer foo;

equation
  impact = h <= 0.0;
  foo = if impact then 1 else 2;
  der(v) = if flying then -g else 0;
  der(h) = v;

  when {h <= 0.0 and v <= 0.0,impact} then
    v_new = if edge(impact) then -e*pre(v) else 0;
    flying = v_new > 0;
    reinit(v, v_new);
  end when;

  annotation (uses(Modelica(version="3.2.1")),
    experiment(StopTime=10, Interval=0.1),
    __Dymola_experimentSetupOutput);

end BouncingBall;
\end{lstlisting}

\subsubsection{Files}
When a modelica model, e.g., BouncingBall model, is implemented in Dymola, the platform dependent C-code from a Modelica model and the corresponding executable code
(i.e., by default dymosim.exe on the Windows operating system) are generated for simulation.  After the executable is generated, it may be run multiple times (with Dymola license).
A separate TEXT file (by default dsin.txt) containing model parameters and initial conditions are also generated as part of the build process.  The RAVEN Dymola interface
modifies input parameters by changing copies of this file.  Both the executable and TEXT parameter file (or simulation initialization file) names must be provided to RAVEN. The TEXT parameter file must be of type 'DymolaInitialisation'.  In the case of
the BouncingBall model previously mentioned on the Windows operating system, the \textless Files\textgreater  specification would look like:
\begin{lstlisting}[style=XML]
<Files>
  <Input name='dsin.txt' type='DymolaInitialisation'>dsin.txt</Input>
</Files>
\end{lstlisting}

The Dymola interface can only pass scalar values into the TEXT parameter file. If the user wants to pass vector information to Dymola, he can do so by providing an optional TEXT vector file to Dymola. This file must have the type 'DymolaVectors'. This additional file can then be read by the Dymola model. If vecor data is passed from RAVEN to the Dymola interface and the TEXT vector file is not specified, the interface will display an error and stop the Dymola execution. If the TEXT vector file is specified (and vector data is passed to the interface), the interface will write the datd into the specified file, but also display a warning, saying that the Dymola interface found vector data to be passed and if this data is supposed to go into the simulation initialisation file of type 'DymolaInitialisation' the array must be split into scalars. The \textless Files\textgreater specification for the vector data look as follows:
\begin{lstlisting}[style=XML]
<Files>
  <Input name='timeSeriesData.txt' type='DymolaVectors'>timeSeriesData.txt</Input>
</Files>
\end{lstlisting}

\subsubsection{Models}
An executable (dymosim.exe) and a simulation initialization file (dsin.txt) can be generated after either translating or simulating the
Modelica model (BouncingBall.mo) using the Dymola Graphical User Interface (GUI) or Dymola Application Programming Interface (API)-routines.
To generate an executable and a simulation initialization file, use the Dymola API-routines (or Dymola GUI) to translate the model as follows:
\lstset{
    frame=single,
    breaklines=true,
    postbreak=\raisebox{0ex}[0ex][0ex]{\ensuremath{\color{red}\hookrightarrow\space}}
}
\begin{enumerate}
\item Change to the directory containing the .mo file to generate an executable.  In Dymola GUI, this corresponds to File/Change Directory in menus:
\begin{lstlisting}
>> cd("C:/msys64/home/KIMJ/projects/raven/framework/CodeInterfaces/Dymola");
C:/msys64/home/KIMJ/projects/raven/framework/CodeInterfaces/Dymola
 = true
\end{lstlisting}
\item Reads the specified file and displays its window.  In Dymola GUI, this corresponds to File/Open in the menus:
\begin{lstlisting}
>> openModel("BouncingBall.mo")
 = true
\end{lstlisting}
\item Compile the model (with current settings), and create the model executable and the corresponding simulation initialization file.  In Dymola GUI, this corresponds to Translate Model in the menus:
\begin{lstlisting}
>> translateModel("BouncingBall");
 = true
\end{lstlisting}
At this point the model executable and the simulation initialization file should have been created in the same directory as the original model file.
Additionally, they could be created by simulating the model.  The following command corresponds to Simulate in the menus in Dymola GUI:
\begin{lstlisting}
>> simulateModel("BouncingBall", stopTime=10, numberOfIntervals=0, outputInterval=0.1, method="dassl", resultFile="BouncingBall");
 = true
\end{lstlisting}
The file extension (.mat) is automatically added to a output file (resultFile), e.g., BouncingBall.mat.  If the generated executable code is triggered directly from a
command prompt, the output file is always named as "dsres.mat".
\end{enumerate}
The model executable is specified to RAVEN using the \textless Models\textgreater  section of the input file as follows:
\begin{lstlisting}[style=XML]
<Simulation>
    ...
  <Models>
    <Code name="BouncingBall" subType = "Dymola">
      <executable>dymosim.exe</executable>
    </Code>
  </Models>
    ...
</Simulation>
\end{lstlisting}
RAVEN works best with Comma-Separated Value (CSV) files.  Therefore, the default
.mat output type needs to be converted to .csv output.
The Dymola interface will automatically convert the .mat output to human-readable
forms, i.e., .csv output, through its implementation of the finalizeCodeOutput function.
\\In order to speed up the reading and conversion of the .mat file, the user can specify
the list of variables (in addition to the Time variable) that need to be imported and
converted into a csv file minimizing
the IO memory usage as much as possible. Within the \xmlNode{Code} the following
XML
node (in addition ot the \xmlNode{executable} one) can be inputted:

\begin{itemize}
   \item \xmlNode{outputVariablesToLoad}, \xmlDesc{space separated list, optional
   parameter}, a space separated list of variables that need be exported from the .mat
   file (in addition to the Time variable). \default{all the variables in the .mat file}.
\end{itemize}
For example:
\begin{lstlisting}[style=XML]
<Simulation>
    ...
  <Models>
    <Code name="BouncingBall" subType = "Dymola">
      <executable>dymosim.exe</executable>
      <outputVariablesToLoad>var1 var2 var3</outputVariablesToLoad>
    </Code>
  </Models>
    ...
</Simulation>
\end{lstlisting}


%%%%%%%%%%%%%%%%%%%%%%%%%%%%%%%%%%%%%%%%%%%%%%%%%
%%%%%% MESH GENERATION COUPLED INTERFACES %%%%%%%
%%%%%%%%%%%%%%%%%%%%%%%%%%%%%%%%%%%%%%%%%%%%%%%%%
\subsection{Mesh Generation Coupled Interfaces}
Some software requires a provided mesh that requires a separate code run to generate.
In these cases, we use sampled geometric
variables to generate a new mesh for each perturbation of the original problem, then run the input with
the remainder of the perturbed parameters and the perturbed mesh.  RAVEN currently provides two interfaces for
this type of calculation, listed below.

%%%%%%%%%% CUBIT MOOSE INTERFACE %%%%%%%%%%
\subsubsection{MooseBasedApp and Cubit Interface}
Many MOOSE-based applications use Cubit (\url{https://cubit.sandia.gov}) to generate Exodus II files as
geometry and meshing for calculations.  To use the developed interface, Cubit's
bin directory must be added to the user's PYTHONPATH.  Input parameters for Cubit can be listed in a journal
(\texttt{.jou}) file.  Parameter values
are typically hardcoded into the Cubit command syntax, but variables may be
predefined in a journal file through Aprepro syntax.  This is an example of a journal
file that generates a rectangle of given height and width, meshes it, defines its
volume and sidesets, lists its element type, and writes it as an Exodus file:

%Cubit (\url{https://cubit.sandia.gov}) is a toolkit developed at Sandia National
%Laboratory used to create two- and three-dimensional finite element meshes with
%various options for defining geometric properties as a part of the grid. It is
%capable of reading and writing a variety of standard mesh file types, including
%Genesis or Exodus II (*.e) files.  As MOOSE applications use Exodus II files for
%meshes and results, Cubit is commonly used to generate meshes for problems of
%interest.  Cubit commands are used to create the geometry, mesh the object, and
%identify volumes, sidesets, and nodesets for a mesh.  These commands may be
%placed in journal files (*.jou) to be used as input to Cubit.  Parameter values
%are typically hardcoded into the Cubit command syntax, but variables may be
%predefined in a journal file through Aprepro syntax.  An example of a journal
%file that generates a rectangle of given height and width, meshes it, defines its
%volume and sidesets, lists its element type, and writes it as an Exodus file is given:

\begin{lstlisting}
#{x = 3}
#{y = 3}
#{out_name = "'out_mesh.e'"}
create surface rectangle width {x} height {y} zplane
mesh surface 1
set duplicate block elements off
block 1 surface 1
Sideset 1 curve 3
Sideset 2 curve 4
Sideset 3 curve 1
Sideset 4 curve 2
Block all element type QUAD4
export genesis {out_name} overwrite
\end{lstlisting}

The first three lines are the Aprepro variable definitions that RAVEN requires to
insert sampled variables.  All variables that RAVEN samples
need to be defined as Aprepro variables in the journal
file.
%These are typically geometric parameters, though almost anything that Cubit
%quantifies in a command may be defined as a variable and sampled through RAVEN
%such as internal mesh refinement values.
One essential caveat to running
this interface is that an Aprepro variable MUST be defined with the name "out\_name".
In order to run this script without RAVEN inserting the correct syntax for the
output file name and properly generate the Exodus file for a mesh, the output file
name is REQUIRED to be in both single and double quotation marks with the file
extension appended to the end of the file base name (e.g. '"output\_file.e"').

%%%%%%%%%%%%%%%%%%%%%%%%%%%%%%%%%%%%%%%%%%%%%%%%%%
\paragraph{Files}
\xmlNode{Files} works the same as in other interfaces with name and type
attributes for each node entry.  The \xmlAttr{name} attribute is a user-chosen internal
name for the file contained in the node, and \xmlAttr{type} identifies which base-level
interface the file is used within.  \xmlNode{type} should only be specified for inputs
that RAVEN will perturb.  For Moose input files, \xmlNode{type} should be \xmlString{MooseInput} and for
Cubit journal files, the \xmlNode{type} should be \xmlString{CubitInput}.  The node should contain the
path to the file from the working directory.  The following is an example
of a typical \xmlNode{Files} block.

\begin{lstlisting}[style=XML]
<Files>
  <Input name='moose_test' type='MooseInput'>simple_diffusion.i</Input>
  <Input name='mesh_in'    type='CubitInput'>rectangle.jou</Input>
  <Input name='other_file' type=''          >some_file_moose_input_needs.ext</Input>
</Files>
\end{lstlisting}

%%%%%%%%%%%%%%%%%%%%%%%%%%%%%%%%%%%%%%%%%%%%%%%%%%
\paragraph{Models}
A user provides paths to executables and aliases for sampled variables within the
\xmlNode{Models} block.  The \xmlNode{Code} block will contain attributes name and
subType.  Name identifies that particular \xmlNode{Code} model within RAVEN, and
subType specifies which code interface the model will use. The \xmlNode{executable}
block should contain the absolute or relative (with respect to the current working
directory) path to the MooseBasedApp that RAVEN will use to run generated input
files.  The absolute or relative path to the Cubit executable is specified within
\xmlNode{preexec}.  If the \xmlNode{preexec} block is not needed, the
MooseBasedApp interface is probably preferable to the Cubit-Moose interface.

Aliases are defined by specifying the variable attribute in an \xmlNode{alias} node with
the internal RAVEN variable name chosen with the node containing the model
variable name.  The Cubit-Moose interface uses the same syntax as the
MooseBasedApp to refer to model variables, with pipes separating terms starting
with the highest YAML block going down to the individual parameter that RAVEN
will change.  To specify variables that are going to be used in the Cubit
journal file, the syntax is "Cubit|aprepro\_var".  The Cubit-Moose interface
will look for the Cubit tag in all variables passed to it and upon finding it,
send it to the Cubit interface.  If the model variable does not begin with \xmlString{Cubit},
the variable MUST be specified in the MooseBasedApp input file.  While the model
variable names are not required to have aliases defined (the \xmlNode{alias}
blocks are optional), it is highly suggested to do so not only to ensure brevity
throughout the RAVEN input, but to easily identify where variables are being sent
in the interface.

An example \xmlNode{Models} block follows.

\begin{lstlisting}[style=XML]
<Models>
  <Code name="moose-modules" subType="CubitMoose">
    <executable>%FRAMEWORK_DIR%/../../moose/modules/combined/...
      modules-%METHOD%</executable>
    <preexec>/hpc-common/apps/local/cubit/13.2/bin/cubit</preexec>
    <alias variable="length">Cubit@y</alias>
    <alias variable="bot_BC">BCs|bottom|value</alias>
  </Code>
</Models>
\end{lstlisting}

%%%%%%%%%%%%%%%%%%%%%%%%%%%%%%%%%%%%%%%%%%%%%%%%%%
\paragraph{Distributions}
The \xmlNode{Distributions} block defines all distributions used to
sample variables in the current RAVEN run.

For all the possible distributions and their possible inputs please
refer to the Distributions chapter (see~\ref{sec:distributions}).
%
It is good practice to name the distribution something similar to what kind of
variable is going to be sampled, since there might be many variables with the
same kind of distributions but different input parameters.

%%%%%%%%%%%%%%%%%%%%%%%%%%%%%%%%%%%%%%%%%%%%%%%%%%
\paragraph{Samplers}
The \xmlNode{Samplers} block defines the variables to be sampled.

After defining a sampling scheme, the variables to be sampled and
their distributions are identified in the \xmlNode{variable} blocks.
The name attribute in the \xmlNode{variable} block must either be the
full MooseBasedApp model variable name or the alias name specifed in
\xmlNode{Models}.  If the sampled variable is a geometric property
that will be used to generate a mesh with Cubit, remember the syntax for
variables being passed to journal files (Cubit|aprepro\_var).

For listings of available samplers
refer to the Samplers chapter (see~\ref{sec:Samplers}).

See the following for an example of a grid based sampler for
length and the bottom boundary condition (both of which have aliases
defined in \xmlNode{Models}).

\begin{lstlisting}[style=XML]
<Samplers>
  <Grid name="Grid_sampling">
    <variable name="length" >
      <distribution>length_dist</distribution>
      <grid type="value" construction="custom">1.0 2.0</grid>
    </variable>
    <variable name="bot_BC">
      <distribution>bot_BC_dist</distribution>
      <grid type="value" construction="custom">3.0 6.0</grid>
    </variable>
  </Grid>
</Samplers>
\end{lstlisting}

%%%%%%%%%%%%%%%%%%%%%%%%%%%%%%%%%%%%%%%%%%%%%%%%%%
\paragraph{Steps,OutStreams,DataObjects}
This interface's \xmlNode{Steps}, \xmlNode{OutStreams}, and
\xmlNode{DataObjects} blocks do not deviate significantly from
other interfaces' respective nodes.  Please refer to previous
entries for these blocks if needed.

%%%%%%%%%%%%%%%%%%%%%%%%%%%%%%%%%%%%%%%%%%%%%%%%%%
\paragraph{File Cleanup}
The Cubit-Moose interface automatically removes files that are commonly
unwanted after the RAVEN run reaches completion. Cubit has been described as
"talkative" due to additional journal files with execution information
being generated by the program after every completed journal file run.
The quantity of these files can quickly become unwieldly if the working
directory is not kept clean; thus these files are removed.  In addition, some users
may wish to remove Exodus files after the RAVEN run is complete as
the typical size of each file is quite large and it is assumed that any
output quantities of interest will be collected by appropriate postprocessors
and the OutStreams.  Exodus files are not automatically removed,
but by using the \xmlNode{deleteOutExtension} node in \xmlNode{RunInfo}, one
may specify the Exodus extension to save a fair amount of storage space
after RAVEN completes a sequence. For example:

\begin{lstlisting}[style=XML]
<RunInfo>
  ...
  <deleteOutExtension>e</deleteOutExtension>
  ...
</RunInfo>
\end{lstlisting}

%%%%%%%%%% BISON MESH SCRIPT MOOSE INTERFACE %%%%%%%%%%
\subsubsection{MooseBasedApp and Bison Mesh Script Interface}
For BISON users, a Python mesh generation script is included in
the \%BISON\_DIR\%/tools/UO2/ directory.  This script generates
3D or 2D (RZ) meshes for nuclear fuel rods using Cubit with
templated commands.  The BISON Mesh Script (BMS) is capable of
generating rods with discrete fuel pellets of various size in
assorted configurations.  To use this interface, Cubit's bin
directory must be added to the user's PYTHONPATH.

%%%%%%%%%%%%%%%%%%%%%%%%%%%%%%%%%%%%%%%%%%%%%%%%%%
\paragraph{Files}
Similar to the Cubit-Moose interface, the BisonAndMesh interface
requires users to specify all files required to run their input
so that these file may be copied into the respective sequence's
working directory.  The user will give each file an internal
RAVEN designation with the name attribute, and the MooseBasedApp
and BISON Mesh Script inputs must be assigned their respective types
in another attribute of the \xmlNode{Input} node.  An example follows.

\begin{lstlisting}[style=XML]
<Files>
  <Input name='bison_test' type='MooseInput'>simple_bison_test.i</Input>
  <Input name='mesh_in'    type='BisonMeshInput'>coarse_input.py</Input>
  <Input name='other_file' type=''>some_file_moose_input_needs.ext</Input>
</Files>
\end{lstlisting}

%%%%%%%%%%%%%%%%%%%%%%%%%%%%%%%%%%%%%%%%%%%%%%%%%%
\paragraph{Models}
A user provides paths to executables and aliases for sampled variables within the
\xmlNode{Models} block.  The \xmlNode{Code} block will contain attributes \xmlAttr{name} and
\xmlAttr{subType}.  \xmlAttr{name} identifies that particular \xmlNode{Code} model within RAVEN, and
\xmlAttr{subType} specifies which code interface the model will use. The \xmlNode{executable}
block should contain the absolute or relative (with respect to the current working
directory) path to the MooseBasedApp that RAVEN will use to run generated input
files.  The absolute or relative path to the mesh script python file is specified within
\xmlNode{preexec}.  If the \xmlNode{preexec} block is not needed, use the
MooseBasedApp interface.

Aliases are defined by specifying the variable attribute in an \xmlNode{alias} node with
the internal RAVEN variable name chosen with the node containing the model
variable name.  The BisonAndMesh interface uses the same syntax as the
MooseBasedApp to refer to model variables, with pipes separating terms starting
with the highest YAML block going down to the individual parameter that RAVEN
will change.  To specify variables that are going to be used in the BISON Mesh Script
python input, the syntax is "Cubit|dict\_name|var\_name".  The interface
will look for the Cubit tag in all variables passed to it and upon finding the tag,
send it to the BISON Mesh Script interface.  If the model variable does not begin with Cubit,
the variable MUST be specified in the MooseBasedApp input file.  While the model
variable names are not required to have aliases defined (the \xmlNode{alias}
blocks are optional), it is highly suggested to do so not only to ensure brevity
throughout the RAVEN input, but to easily identify where variables are being sent
in the interface.

An example \xmlNode{Models} block follows.

\begin{lstlisting}[style=XML]
<Models>
  <Code name="Bison-opt" subType="BisonAndMesh">
    <executable>%FRAMEWORK_DIR%/../../bison/bison-%METHOD%</executable>
    <preexec>%FRAMEWORK_DIR%/../../bison/tools/UO2/mesh_script.py</preexec>
    <alias variable="pellet_radius" >Cubit@Pellet1|outer_radius</alias>
    <alias variable="clad_thickness">Cubit@clad|clad_thickness</alias>
    <alias variable="fuel_k"        >Materials|fuel_thermal|thermal_conductivity</alias>
    <alias variable="clad_k"        >Materials|clad_thermal|thermal_conductivity</alias>
  </Code>
</Models>
\end{lstlisting}

%%%%%%%%%%%%%%%%%%%%%%%%%%%%%%%%%%%%%%%%%%%%%%%%%%
\paragraph{Distributions}
The \xmlNode{Distributions} block defines all distributions used to
sample variables in the current RAVEN run.

For all the possible distributions and their possible inputs please
refer to the Distributions chapter (see~\ref{sec:distributions}).
%
It is good practice to name the distribution something similar to what kind of
variable is going to be sampled, since there might be many variables with the
same kind of distributions but different input parameters.

%%%%%%%%%%%%%%%%%%%%%%%%%%%%%%%%%%%%%%%%%%%%%%%%%%
\paragraph{Samplers}
The \xmlNode{Samplers} block defines the variables to be sampled.

After defining a sampling scheme, the variables to be sampled and
their distributions are identified in the \xmlNode{variable} blocks.
The name attribute in the \xmlNode{variable} block must either be the
full MooseBasedApp model variable name or the alias name specified in
\xmlNode{Models}.  If the sampled variable is a geometric property
that will be used to generate a mesh with Cubit, remember the syntax for
variables being passed to journal files (Cubit|aprepro\_var).

For listings of available samplers
refer to the Samplers chapter (see~\ref{sec:Samplers}).

See the following for an example of a grid based sampler for
length and the bottom boundary condition (both of which have aliases
defined in \xmlNode{Models}).

\begin{lstlisting}[style=XML]
<Samplers>
  <Grid name="Grid_sampling">
    <variable name="length" >
      <distribution>length_dist</distribution>
      <grid type="value" construction="custom">1.0 2.0</grid>
    </variable>
    <variable name="bot_BC">
      <distribution>bot_BC_dist</distribution>
      <grid type="value" construction="custom">3.0 6.0</grid>
    </variable>
  </Grid>
</Samplers>
\end{lstlisting}

%%%%%%%%%%%%%%%%%%%%%%%%%%%%%%%%%%%%%%%%%%%%%%%%%%
\paragraph{Steps,OutStreams,DataObjects}
This interface's \xmlNode{Steps}, \xmlNode{OutStreams}, and
\xmlNode{DataObjects} blocks do not deviate significantly from
other interfaces' respective nodes.  Please refer to previous
entries for these blocks if needed.

%%%%%%%%%%%%%%%%%%%%%%%%%%%%%%%%%%%%%%%%%%%%%%%%%%
\paragraph{File Cleanup}
The BisonAndMesh interface automatically removes files that are commonly
unwanted after the RAVEN run reaches completion. Cubit has been described as
"talkative" due to additional journal files with execution information
being generated by the program after every completed journal file run.
The BISON Mesh Script creates a journal file to run with cubit after reading input parameters;
so Cubit will generate its "redundant" journal files, and .pyc files will
litter the working directory as artifacts of the python mesh script
reading from the .py input files.  The quantity of these files can quickly
become unwieldly if the working directory is not kept clean, thus these
files are removed.  Some users
may wish to remove Exodus files after the RAVEN run is complete as
the typical size of each file is quite large and it is assumed that any
output quantities of interest will be collected by appropriate postprocessors
and the OutStreams.  Exodus files are not automatically removed,
but by using the \xmlNode{deleteOutExtension} node in \xmlNode{RunInfo}, one
may specify the Exodus extension (*.e) to save a fair amount of storage space
after RAVEN completes a sequence. For example:

\begin{lstlisting}[style=XML]
<RunInfo>
  ...
  <deleteOutExtension>e</deleteOutExtension>
  ...
</RunInfo>
\end{lstlisting}


%%%%%%%%%%%%%%%%%%%%%%%%%%%%%%%%%%%%%%%%%%%%%%%%%

%%%%%%%%%%%%%%%%%%%%%%%%%%%%%%%%%%%%%%%%%%%%%%%%%
%%%%%%%%%%%%% RATTLESNAKE INTERFACE %%%%%%%%%%%%%
%%%%%%%%%%%%%%%%%%%%%%%%%%%%%%%%%%%%%%%%%%%%%%%%%
\subsection{Rattlesnake Interfaces} \label{RattlesnakeInterfaces}
%
This section covers the input specification for running Rattlesnake through RAVEN. It is important
to notice that this short explanation assumes that the reader already knows how to use Rattlesnake.
The interface can be used to perturb the Rattlesnake MOOSE-based input file as well as the Yak
cross section libraries XML input files (e.g. multigroup cross section libraries) and Instant format
cross section libraries.
%
%%%%%%%%%%%%%%%%%%%%%%%%%%%%%%%%%%%%%%%%%%%%%%%%%%
\subsubsection{Files}
\xmlNode{Files} works the same as in other interfaces with name and type
attributes for each node entry.  The \xmlAttr{name} attribute is a user-chosen internal
name for the file contained in the node, and \xmlAttr{type} identifies which base-level
interface the file is used within.  \xmlAttr{type} should only be specified for inputs
that RAVEN will perturb. Take Rattlesnake input files for example, \xmlAttr{type} should
be \xmlString{RattlesnakeInput}.

\paragraph{Perturb Yak Multigroup Cross Section Libraries}
If the user would like to perturb the Yak multigroup cross section libraries, the user need to use the
\xmlString{YakXSInput} for the \xmlAttr{type} of the libaries. In addition, the \xmlAttr{type} of the
alias files that are used to perturb the Yak multigroup cross section libraries should be
\xmlString{YakXSAliasInput}. The following is an example of a typical \xmlNode{Files} block.
%
\begin{lstlisting}[style=XML]
<Files>
  <Input name='rattlesnakeInput' type='RattlesnakeInput'>simple_diffusion.i</Input>
  <Input name='crossSection'    type='YakXSInput'>xs.xml</Input>
  <Input name='alias' type='YakXSAliasInput'>alias.xml</Input>
</Files>
\end{lstlisting}
%
The alias files are employed to define the variables that will be used to perturb Yak multigroup cross section
libraries. The following is an example of a typical alias file:
%
\begin{lstlisting}[style=XML]
<Multigroup_Cross_Section_Libraries Name="twigl" NGroup="2" Type="rel">
    <Multigroup_Cross_Section_Library ID="1">
        <Fission gridIndex="1" mat="pseudo-seed1" gIndex="1">f11</Fission>
        <Capture gridIndex="1" mat="pseudo-seed1" gIndex="1">c11</Capture>
        <TotalScattering gridIndex="1" mat="pseudo-seed1" gIndex="1">t11</TotalScattering>
        <Nu gridIndex="1" mat="pseudo-seed1" gIndex="1">n11</Nu>
        <Fission gridIndex="1" mat="pseudo-seed2" gIndex="2">f22</Fission>
        <Capture gridIndex="1" mat="pseudo-seed2" gIndex="2">c22</Capture>
        <TotalScattering gridIndex="1" mat="pseudo-seed2" gIndex="2">t22</TotalScattering>
        <Nu gridIndex="1" mat="pseudo-seed2" gIndex="2">n22</Nu>
        <Fission gridIndex="1" mat="pseudo-seed1-dup" gIndex="1">f11</Fission>
        <Capture gridIndex="1" mat="pseudo-seed1-dup" gIndex="1">c11</Capture>
        <TotalScattering gridIndex="1" mat="pseudo-seed1-dup" gIndex="1">t11</TotalScattering>
        <Nu gridIndex="1" mat="pseudo-seed1-dup" gIndex="1">n11</Nu>
        <Transport gridIndex="1" mat="pseudo-seed1-dup" gIndex="1">d11</Transport>
    </Multigroup_Cross_Section_Library>
</Multigroup_Cross_Section_Libraries>
\end{lstlisting}
%
In the above alias file, the \xmlAttr{Name} of \xmlNode{Multigroup\_Cross\_Section\_Libraries} are used to indicate
which Yak multigroup cross section library input file will be perturbed.
The \xmlAttr{NGroup}, \xmlAttr{ID}, and \xmlNode{Multigroup\_Cross\_Section\_Library}
should be consistent with the Yak multigroup cross section library input files.
The \xmlNode{Fission}, \xmlNode{Capture}, \xmlNode{TotalScattering}, \xmlNode{Nu}, \xmlAttr{gridIndex},
\xmlAttr{mat}, and \xmlAttr{gIndex} are used to find the corresponding cross sections in the Yak multigroup cross
section library input files. For example:
%
\begin{lstlisting}[style=XML]
<Fission gridIndex="1" mat="pseudo-seed1" gIndex="1">f11</Fission>
\end{lstlisting}
%
This node defines an alias with name \xmlString{f11} used to represent the fission cross section at energy group \xmlString{1}
for material with name 'pseudo-seed1' at grid index \xmlString{1} in the Yak multigroup cross section library input files.

\nb The attribute \xmlAttr{Type="rel"} indicates that the cross sections will be perturbed relatively (i.e. perturbed by
percents). In this case, the user also needs to specify a relative covariance matrix for \xmlNode{covaraince \xmlAttr{type="rel"}} in
\xmlNode{MultivariateNormal} distribution, and the values for \xmlNode{mu} should be `ones'. In the other case, if
the user choose \xmlAttr{Type="abs"}, the cross sections will be perturbed absolutely (i.e. perturbed by values), and
the user needs to provide an absolute covariance matrix and specify `zeros' for \xmlNode{mu} in \xmlNode{MultivariateNormal}
distribution.

\nb Currently, only the following cross sections can be perturbed by the user: Fission, Capture, Nu, TotalScattering,
and Transport.

\paragraph{Perturb Instant format Cross Section Libraries}
If the user would like to perturb the Instant cross section libraries, the user need to use the
\xmlString{InstantXSInput} for the \xmlAttr{type} of the libaries. In addition, the \xmlAttr{type} of the
alias files that are used to perturb the Instant format cross section libraries should be
\xmlString{InstantXSAliasInput}. The following is an example of a typical \xmlNode{Files} block.
%
\begin{lstlisting}[style=XML]
<Files>
  <Input name='rattlesnakeInput' type='RattlesnakeInput'>iaea2d_ls_sn.i</Input>
  <Input name='crossSection'    type='InstantXSInput'>iaea2d_materials.xml</Input>
  <Input name='alias' type='InstantXSAliasInput'>alias.xml</Input>
</Files>
\end{lstlisting}
%
The alias files are employed to define the variables that will be used to perturb Instant format cross section
libraries. The following is an example of a typical alias file:
%
\begin{lstlisting}[style=XML]
<Materials>
  <Macros NG="2" Type="rel">
    <material ID="1">
      <FissionXS gIndex="1">f11</FissionXS>
      <CaptureXS gIndex="1">c11</CaptureXS>
      <TotalScatteringXS gIndex="1">t11<TotalScatteringXS>
      <Nu gIndex="1">n11</Nu>
      <DiffusionCoefficient gIndex="1">d11</DiffusionCoefficient>
    </material>
  </Macros>
</Materials>
\end{lstlisting}

%
In the above alias file, the \xmlAttr{NG} and \xmlAttr{ID} should be consistent with the Instant format cross
section library input files. The \xmlNode{FissionXS}, \xmlNode{CaptureXS}, \xmlNode{TotalScatteringXS}, \xmlNode{Nu}, \xmlAttr{gIndex},
are used to find the corresponding cross sections in the Instant format cross
section library input files. For example, the variable \xmlString{f11} used to represent the fission cross section at energy group \xmlString{1}
for material with \xmlString{ID} equal \xmlString{1} in the given cross section library.

\nb The attribute \xmlAttr{Type="rel"} indicates that the cross sections will be perturbed relatively (i.e. perturbed by
percents). In this case, the user also needs to specify a relative covariance matrix for \xmlNode{covaraince \xmlAttr{type="rel"}} in
\xmlNode{MultivariateNormal} distribution, and the values for \xmlNode{mu} should be `ones'. In the other case, if
the user choose \xmlAttr{Type="abs"}, the cross sections will be perturbed absolutely (i.e. perturbed by values), and
the user needs to provide an absolute covariance matrix and specify `zeros' for \xmlNode{mu} in \xmlNode{MultivariateNormal}
distribution.

\nb Currently, only the following cross sections can be perturbed by the user: FissionXS, CaptureXS, Nu, TotalScatteringXS,
and DiffusionCoefficient.

%%%%%%%%%%%%%%%%%%%%%%%%%%%%%%%%%%%%%%%%%%%%%%%%%%
\subsubsection{Models}
A user provides paths to executables and aliases for sampled variables within the
\xmlNode{Models} block.  The \xmlNode{Code} block will contain attributes \xmlNode{name} and
\xmlNode{subType}. The \xmlNode{name} identifies that particular \xmlNode{Code} model within RAVEN, and
\xmlNode{subType} specifies which code interface the model will use. The \xmlNode{executable}
block should contain the absolute or relative (with respect to the current working
directory) path to Rattlesnake that RAVEN will use to run generated input
files.

An example \xmlNode{Models} block follows.

\begin{lstlisting}[style=XML]
<Models>
  <Code name="Rattlesnake" subType="Rattlesnake">
    <executable>%FRAMEWORK_DIR%/../../rattlesnake/
     rattlesnake-%METHOD%</executable>
  </Code>
</Models>
\end{lstlisting}

%%%%%%%%%%%%%%%%%%%%%%%%%%%%%%%%%%%%%%%%%%%%%%%%%%
\subsubsection{Distributions}
The \xmlNode{Distributions} block defines all distributions used to
sample variables in the current RAVEN run.

For all the possible distributions and their possible inputs please
refer to the Distributions chapter (see~\ref{sec:distributions}).
%
It is good practice to name the distribution something similar to what kind of
variable is going to be sampled, since there might be many variables with the
same kind of distributions but different input parameters.

%%%%%%%%%%%%%%%%%%%%%%%%%%%%%%%%%%%%%%%%%%%%%%%%%%
\paragraph{Samplers}
The \xmlNode{Samplers} block defines the variables to be sampled.
After defining a sampling scheme, the variables to be sampled and
their distributions are identified in the \xmlNode{variable} blocks.
The name attribute in the \xmlNode{variable} block must either be the
full MooseBasedApp (Rattlesnake) model variable name, the alias name specifed in
\xmlNode{Models}, or the variable name specified in the provided alias files.

For listings of available samplers, please refer to the Samplers chapter (see~\ref{sec:Samplers}).
See the following for an example of a grid based sampler for
the first energy group fission and capture cross sections  (both of which have
defined in alias files provided in \xmlNode{Files}).

\begin{lstlisting}[style=XML]
<Samplers>
  <Grid name="Grid_sampling">
    <variable name="fission_group_1" >
      <distribution>fission_dist</distribution>
      <grid type="value" construction="custom">1.0 2.0</grid>
    </variable>
    <variable name="capture_group_1">
      <distribution>capture_dist</distribution>
      <grid type="value" construction="custom">3.0 6.0</grid>
    </variable>
  </Grid>
</Samplers>
\end{lstlisting}

%%%%%%%%%%%%%%%%%%%%%%%%%%%%%%%%%%%%%%%%%%%%%%%%%%
\subsubsection{Steps}
For a Rattlesnake interface, the \xmlNode{MultiRun} step type will most likely be used. First, the step needs
to be named: this name will be one of the names used in the \xmlNode{Sequence} block. In our example, \xmlString{Grid\_Rattlesnake}.
%
\begin{lstlisting}[style=XML]
<MultiRun name='Grid_Rattlesnake' verbosity='debug'>
    <Input   class='Files' type=''>RattlesnakeInput.i</Input>
    <Input   class='Files' type=''>xs.xml</Input>
    <Input   class='Files' type=''>alias.xml</Input>
    <Model   class='Models' type='Code'>Rattlesnake</Model>
    <Sampler class='Samplers' type='Grid'>Grid_Samplering</Sampler>
    <Output  class='DataObjects' type='PointSet'>solns</Ouput>
\end{lstlisting}
%
With this step, we need to import all the files needed for the simulation:
%
\begin{itemize}
  \item Rattlesnake MOOSE-based input file;
  \item Yak multigroup cross section libraries input files (XML);
  \item Yak alias files used to define the perturbed variables (XML).
\end{itemize}
We then need to define \xmlNode{Model}, \xmlNode{Sampler} and \xmlNode{Output}. The \xmlNode{Output} can be
\xmlNode{DataObjects} or \xmlNode{OutStreams}.

%%%%%%%%%%%%%%%%%%%%%%%%%%%%%%%%%%%%%%%%%%%%%%%%%%%%%%%%%%%%%%%%%%%%%%%%%%%%%%%%%%
%%%%%%%%%%%%%%%%%%%%%%%%%%%%%%%% MAAP5  INTERFACE %%%%%%%%%%%%%%%%%%%%%%%%%%%%%%%%
%%%%%%%%%%%%%%%%%%%%%%%%%%%%%%%%%%%%%%%%%%%%%%%%%%%%%%%%%%%%%%%%%%%%%%%%%%%%%%%%%%
\subsection{MAAP5 Interface}
This section presents the main aspects of the interface coupling RAVEN with MAAP5,
the consequent RAVEN input adjustments and the modifications of the MAAP5
files required to run the two coupled codes.
The interface works both for forward sampling and the DET,
however there are some differences depending on the selected sampling strategy.
%%%%%%%%%%%%%%%%%%%%%%%%%%%%%%%%%%%%%%%%%%%%%%%%%%%%%%%%%%%%%%%%%%%%%%%%%%%%%%%%%%
\subsubsection{RAVEN Input file}
%%%%%%%%%%%%%%%%%%%%%%%%%%%%%%%%%%%%%%%%%%%%%%%%%%%%%%%%%%%%%%%%%%%%%%%%%%%%%%%%%%
\paragraph{Files}
MAAP5 requires more than one file to run a simulation.
This means that, since the \xmlNode{Files} section has to contain all the files required by
the external model (MAAP5) to be run, all these files need to be included within this node.
This involves not only the input file (.inp) but also the include file, the parameter file, all the
files defining the different ``PLOTFILs'', if any, and the other files which could
result useful for the MAAP5 simulation run.

Example:
\begin{lstlisting}[style=XML]
<Files>
  <Input name="test.inp" type="">test.inp</Input>
  <Input name="include" type="">include</Input>
  <Input name="plot.txt" type="">plot.txt</Input>
  <Input name="plant.par" type="">plant.par</Input>
</Files>
\end{lstlisting}
The files mentioned in this section
 need, then, to be put into the working directory specified
by the \xmlNode{workingDir} node into the \xmlNode{RunInfo} block.
%%%%%%%%%%%%%%%%%%%%%%%%%%%%%%%%%%%%%%%%%%%%%%%%%%%%%%%%%%%%%%%%%%%%%%%%%%%%%%%%%%
\paragraph{Models}
The \xmlNode{Models} block contains the name of the executable file of MAAP5
(with the path, if necessary),
and the name of the interface (e.g. MAAP5\_GenericV7).
The block has also some required nodes:
\begin{itemize}
  \item \xmlNode{boolMaapOutputVariables}: containing the number of the MAAP5 IEVNT corresponding to the boolean events of interest;
  \item \xmlNode{contMaapOutputVariables}: containing the list of all the continuous variables we are interested at,
  and that we want to monitor;
  \item \xmlNode{stopSimulation}: this node is required only in case of DET sampling strategy.
The user needs to specify
  if the MAAP5 simulation run stops due to the reached END TIME, specifying ''mission\_time'',
or due to the occurrence of a specific event by
  inserting the number of the corresponding MAAP5 IEVNT (e.g IEVNT(691) for core uncovery)
 \item \xmlNode{includeForTimer}: also this node is required only in case of DET sampling
 strategy and it contains the name of the
 MAAP5 include file where the TIMERS for the different variables are defined
(see paragraph ''MAAP5 include file below'' for more information about timers).
\end{itemize}

A \xmlNode{Models} block is shown as an example below:
\begin{lstlisting}[style=XML]
<Models>
  <Code name="MyMAAP" subType="MAAP5\_GenericV7">
    <executable>MAAP5.exe</executable>
    <clargs type='input' extension='.inp'/>
    <boolMaapOutputVariables>691</boolMaapOutputVariables>
    <contMaapOutputVariables>PPS,PSGGEN(1),ZWDC2SG(1)
    </contMaapOutputVariables>
    <stopSimulation>mission_time</stopSimulation>
    <includeForTimer>include</includeForTimer>
  </Code>
</Models>
\end{lstlisting}
%%%%%%%%%%%%%%%%%%%%%%%%%%%%%%%%%%%%%%%%%%%%%%%%%%%%%%%%%%%%%%%%%%%%%%%%%%%%%%%%%%
\paragraph{Other blocks}
All the other blocks (e.g. \xmlNode{Distributions}, \xmlNode{Samplers}, \xmlNode{Steps},
 \xmlNode{Databases}, \xmlNode{OutStream}, etc.)
do not require any particular arrangements than already provided by a RAVEN input.
User can, therefore, refer to the corresponding sections of the User's Manual.
This is valid for both forward sampling and DET.
%%%%%%%%%%%%%%%%%%%%%%%%%%%%%%%%%%%%%%%%%%%%%%%%%%%%%%%%%%%%%%%%%%%%%%%%%%%%%%%%%%
\subsubsection{MAAP5 Input files}
%%%%%%%%%%%%%%%%%%%%%%%%%%%%%%%%%%%%%%%%%%%%%%%%%%%%%%%%%%%%%%%%%%%%%%%%%%%%%%%%%%
The coupling of RAVEN and MAAP5 requires modifications to some
MAAP5 files in order to work. This is particularly true when a DET analysis is performed.
The MAAP5 input files that need to be modified are:
\begin{itemize}
  \item MAAP5 include file
  \item MAAP5 input file (.inp)
  \item PLOTFIL blocks
\end{itemize}
%%%%%%%%%%%%%%%%%%%%%%%%%%%%%%%%%%%%%%%%%%%%%%%%%%%%%%%%%%%%%%%%%%%%%%%%%%%%%%%%%%
\paragraph{MAAP5 include file}
Usually MAAP5 simulation provides the presence of some include files, for example,
containing the user-defined variables, timers, definition of the plotfil, etc.
The adjustments explained in this section are required only in case of a DET analysis.
The user needs to modify the include file containing the set of the
timers used into the run, by adding the definition of the different timers,
one for each variable that causes a branching.
The include file to be modified should correspond to that one defined in the \xmlNode{includeForTimer}
block of the RAVEN xml input.

User is supposed to check that the numbers used for the different timers definition
are not already used in any of the other MAAP5 files.
These timers should be preceeded by a line reporting ''C Branching + name of the variable
sampled by RAVEN causing the branching''.

For example, we assume that DIESEL is the name of the variable corresponding to the failure time
of the Diesel generators (user defined). User has to firstly ensure that, for example,
''TIMER 100'' is not already used into the model, then the following lines
need to be added into the selected include file for the set of the timer corresponding
to the Diesel generators failure:
\begin{lstlisting}[style=XML]
C Branching DIESEL
WHEN (TIM>DIESEL)
   SET TIMER 100
END
\end{lstlisting}
It is worth mentioning that at this step a TIMER should be defined
also for the event IEVNT specified into the \xmlNode{stopSimulation},
 if this is the stop condition for the MAAP5 run:
\begin{lstlisting}[style=XML]
WHEN IEVNT(691) == 1.0
  SET TIMER 10
END
\end{lstlisting}
The interface will check that one timer is defined for each variable
of the DET. If not, an error arises suggesting to user the name of the variable having no
timer defined.
%%%%%%%%%%%%%%%%%%%%%%%%%%%%%%%%%%%%%%%%%%%%%%%%%%%%%%%%%%%%%%%%%%%%%%%%%%%%%%%%%%
\paragraph{MAAP5 input file}
In the ''parameter change'' section of the MAAP5 input file, the user should declare
the name of the variables sampled by RAVEN according to the following statement:
\begin{lstlisting}[style=XML]
 variable = $ RAVEN-variable:default$
\end{lstlisting}
where the dafault value is optional.

For example:
\begin{lstlisting}[style=XML]
DIESEL = $RAVEN-DIESEL:-1$
\end{lstlisting}
This is valid for both forward and DET sampled variables.
In particular, in case of DET analysis, the variables causing the occurrence of the branch should be
assigned within a block identified by the comment ''C DET Sampled variables'':
\begin{lstlisting}[style=XML]
C DET Sampled Variables
DIESEL = $RAVEN-DIESEL:-1$
C End DET Sampled Variables
\end{lstlisting}
If the sampled variables are user-defined, then the user shall ensure that they are initialized
(to the default value) and set within the user-defined variables section of one of the include
file.
As usual, a distribution and a sampling strategy should then correspond to each of these variables
into the RAVEN xml input file.

Only for the DET analysis, then, the occurence of a branch will be identified by a comment before. This comment is
 ''C BRANCHING + name of the variable determining the branch'' and acts as a sort of branching marker.
Looking for these markers, indeed, the interface (in case of DET sampler) verifies that at least
one branching exists, and furthermore, that one branching is defined for each of the
variables contained into ''DET sampled variables''.

Within the block, the occurrence of the branching leads the value of a variable (user-defined)
called ''TIM+number of the corresponding timer set into
the include file'' to switch to 1.0. The code, in fact, detects if a branch has occurred by monitoring
the value of these kind of variables. SInce these variables are user-defined, they need to be
initialized to a value (different from 1.0), into the ''user-defined variables'' section of one of the include
files.

Therefore following the previous example, if we want that, when the diesels failure occurs it leads
to the event ''Loss of AC Power'' (IEVNT(205) of MAAP5), we will have:
\begin{lstlisting}[style=XML]
C Branching TIMELOCA
WHEN TIM > DIESEL
 TIM100=1.0
 IEVNT(205)=1.0
END
\end{lstlisting}
It is worth noticing that no comments should be contained within the line of assignment
(i.e. IEVNT(205)=1.0 //LOSS OF AC POWER is not allowed).

Finally, only in case of DET analysis, a stop simulation condition (provided by the comment
''C Stop Simulation condition'') needs to be put into the input.
The original input should have all the timers (linked with the branching) separated by an OR
condition, even including that one of the event that stops the simulation (e.g. IEVNT(691)),
if any.
\begin{lstlisting}[style=XML]
C Stop Simulation condition
IF (TIMER 10 > 0) OR (TIMER 100 > 0) OR ... (TIMER N > 0)
 TILAST=TIM
END
\end{lstlisting}
This allows the simulation run to stop when a branch condition occurs, creating the restart file that will
be used by the two following branches.

For each branch, then, the interface will automatically update the name of the RESTART FILE to be used and
of the RESTART TIME that will be equal to the difference between the END TIME of the ''parent'' simulation
and the PRINT INTERVAL (which specifies the interval at which the restart output is written).
%%%%%%%%%%%%%%%%%%%%%%%%%%%%%%%%%%%%%%%%%%%%%%%%%%%%%%%%%%%%%%%%%%%%%%%%%%%%%%%%%%
\paragraph{MAAP5 PLOTFIL blocks}
This section refers to the ''PLOTFIL blocks'' used to modify the plot file (.csv) defined into the parameter file.
These blocks need to be modified in order to include some variables.
It is important, indeed, that the MAAP5 csv PLOTFIL files contain the evolution of:
\begin{itemize}
  \item RAVEN sampled variables (e.g. DIESEL) (both for Forward and DET sampling)
  \item the variables whose value is modified by the occurrence of one of the branches, either continuous or boolean (e.g. IEVNT(225))
  \item the variables of interest defined within \xmlNode{boolMaapOutputVariables}
  and \\ \xmlNode{contMaapOutputVariables} blocks (both for Forward and DET sampling)
\end{itemize}
If one of these variables is not contained into one of csv files, RAVEN will give an error.

%%%%%%%%%%%%%%%%%%%%%%%%%%%%%%%%%%%%%%%%%%%%%%%%%
%%%%%%%%%%%%% MAMMOTH INTERFACE %%%%%%%%%%%%%
%%%%%%%%%%%%%%%%%%%%%%%%%%%%%%%%%%%%%%%%%%%%%%%%%
\subsection{MAMMOTH Interface}
%
This section covers the input specification for running MAMMOTH through RAVEN.
It is important to notice that this short explanation assumes that the reader already knows how to use MAMMOTH.
The interface can be used to perturb Bison, Rattlesnake, RELAP-7, and general MOOSE input files that utilize
MOOSE's standard YAML input structure as well as Yak multigroup cross section library XML input files.
%

%%%%%%%%%%%%%%%%%%%%%%%%%%%%%%%%%%%%%%%%%%%%%%%%%%
\subsubsection{Files}
\xmlNode{Files} works the same as in other interfaces with name and type
attributes for each node entry.  The \xmlAttr{name} attribute is a user-chosen internal
name for the file contained in the node, and \xmlAttr{type} identifies which base-level
interface the file is used within.  \xmlAttr{type} should be specified for all inputs
used in RAVEN's MultiRun for MAMMOTH (including files not perturbed by RAVEN).
The MAMMOTH input file's \xmlAttr{type} should have \xmlString{MAMMOTHInput} prepended
to the driver app's input specification (e.g. \xmlString{MAMMOTHInput|appNameInput}).
Any other app's input file needs a \xmlAttr{type} with the app's name prepended to \xmlString{Input}
(e.g. \xmlString{BisonInput}, \xmlString{Relap7Input}, etc.).  In addition, the \xmlAttr{type} for any mesh
input is the app in which that mesh is utilized prepended to \xmlString{|Mesh}; so a Bison mesh would have
a \xmlAttr{type} of \xmlString{Bison|Mesh} and similarly a mesh for Rattlesnake would have \xmlString{Rattlesnake|Mesh}
as its \xmlAttr{type}. In cases where a file needs to be copied to each perturbed run
directory (to be used as function input, control logic, etc.), one can use the \xmlAttr{type}
\xmlString{AncillaryInput} to make it clear in the RAVEN input file that this is file
is required for the simulation to run but contains no perturbed parameters.
For Yak multigroup cross section libraries,
the \xmlAttr{type} should be \xmlString{YakXSInput}, and for the Yak
alias files that are used to perturb the Yak multigroup cross section libraries, the \xmlAttr{type} should be
\xmlString{YakXSAliasInput}.

The node should contain the path to the file from the working directory.
The following is an example of a typical \xmlNode{Files} block.
%
\begin{lstlisting}[style=XML]
<Files>
  <Input name='mammothInput' type='MAMMOTHInput|RattlesnakeInput'>test_mammoth.i</Input>
  <Input name='crossSection'    type='YakXSInput'>xs.xml</Input>
  <Input name='alias' type='YakXSAliasInput'>alias.xml</Input>
  <Input name='bisonInput'    type='BisonInput'>test_bison.xml</Input>
  <Input name='bisonMesh'    type='Bison|Mesh'>bisonMesh.e</Input>
  <Input name='fuelCTEfunct' type='AncillaryInput'>uo2_CTE.csv</Input>
  <Input name='rattlesnakeMesh'    type='Rattlesnake|Mesh'>rattlesnakeMesh.e</Input>
</Files>
\end{lstlisting}
%
The alias files are employed to define the variables that will be used to perturb Yak multigroup cross section
libraries. Please see the section \ref{RattlesnakeInterfaces} for the example.
%
%%%%%%%%%%%%%%%%%%%%%%%%%%%%%%%%%%%%%%%%%%%%%%%%%%
\subsubsection{Models}
A user provides paths to executables and aliases for sampled variables within the
\xmlNode{Models} block.  The \xmlNode{Code} block will contain \xmlAttr{name} and
\xmlAttr{subType}.  The attribute \xmlAttr{name} identifies that particular \xmlNode{Code} model within RAVEN, and
\xmlAttr{subType} specifies which code interface the model will use. The \xmlNode{executable}
block should contain the absolute or relative (with respect to the current working
directory) path to MAMMOTH that RAVEN will use to run generated input
files.

An example \xmlNode{Models} block follows.

\begin{lstlisting}[style=XML]
<Models>
  <Code name="Mammoth" subType="MAMMOTH">
    <executable>\%FRAMEWORK_DIR\%/../../mammoth/
     mammoth-%METHOD%</executable>
  </Code>
</Models>
\end{lstlisting}

%%%%%%%%%%%%%%%%%%%%%%%%%%%%%%%%%%%%%%%%%%%%%%%%%%
\subsubsection{Distributions}
The \xmlNode{Distributions} block defines all distributions used to
sample variables in the current RAVEN run.

For all the possible distributions and their possible inputs please
refer to the Distributions chapter (see~\ref{sec:distributions}).
%
It is good practice to name the distribution something similar to what kind of
variable is going to be sampled, since there might be many variables with the
same kind of distributions but different input parameters.

%%%%%%%%%%%%%%%%%%%%%%%%%%%%%%%%%%%%%%%%%%%%%%%%%%
\paragraph{Samplers}
The \xmlNode{Samplers} block defines the variables to be sampled.
After defining a sampling scheme, the variables to be sampled and
their distributions are identified in the \xmlNode{variable} blocks.
The \xmlAttr{name} attribute in the \xmlNode{variable} block must either be the
app's name prepended to the full MooseBasedApp model variable name, the alias name specifed in
\xmlNode{Models}, or the variable name specified in the provided alias files.

For listings of available samplers, please refer to the Samplers chapter (see~\ref{sec:Samplers}).
See the following for an example of a grid based sampler used to generate the samples for
the first energy group fission and capture cross sections (both of which have
defined in alias files provided in \xmlNode{Files}), the initial condition temperature defined in Rattlesnake
input file and the poissons ratio, clad thickness, and gap width defined in Bison input files with clad
and gap parameters calculated using an external function with sampled clad inner and outer diameters
as inputs.
%
\begin{lstlisting}[style=XML]
<Samplers>
  <Grid name="Grid_sampling">
    <variable name="Rattlesnake@fission_group_1" >
      <distribution>fission_dist</distribution>
      <grid type="value" construction="custom">1.0 2.0</grid>
    </variable>
    <variable name="Rattlesnake@capture_group_1">
      <distribution>capture_dist</distribution>
      <grid type="value" construction="custom">3.0 6.0</grid>
    </variable>
    <variable name="Rattlesnake@AuxVariables|Temp|initial_condition">
      <distribution>uniform</distribution>
      <grid type="value" construction="custom">3.0 6.0</grid>
    </variable>
    <variable name="Bison@Materials|fuel_solid_mechanics_elastic|poissons_ratio">
      <distribution>normal</distribution>
      <grid type="value" construction="custom">3.0 6.0</grid>
    </variable>
    <variable name='clad_outer_diam'>
      <distribution>clad_outer_diam_dist</distribution>
      <grid construction='equal' steps='144' type='CDF'>0.02275 0.97725</grid>
    </variable>
    <variable name='clad_inner_diam'>
      <distribution>clad_inner_diam_dist</distribution>
      <grid construction='equal' steps='144' type='CDF'>0.02275 0.97725</grid>
    </variable>
    <variable name='Bison@Mesh|clad_thickness'>
      <function>clad_thickness_calc</function>
    </variable>
    <variable name='Bison@Mesh|clad_gap_width'>
      <function>clad_gap_width_calc</function>
    </variable>
  </Grid>
</Samplers>
\end{lstlisting}
%
In order to make the input variables of one application distinct from input variables of another,
an app's name followed by the '@' symbol is prepended to the variable name (e.g. \xmlString{appName@varName}).
Each variable to be used in an app's input file and sampled in the MAMMOTH interface is required
to have a destination app specified. All variables utilizing Rattlesnake's executable (whether
they are in the Rattlesnake input file or not) are listed as Rattlesnake variables
as that application's interface will sort input file and cross section
variables itself.  Notice that the clad inner and outer diameter sampled parameters have no app
name specified.  These parameters are utilized to sample values used as inputs
for the clad thickness and gap width variables in BISON, so by not specifying a destination
app, these are passed through the interface having only been used in an external function
to calculate parameters usable in an app's input.
%%%%%%%%%%%%%%%%%%%%%%%%%%%%%%%%%%%%%%%%%%%%%%%%%%
\subsubsection{Steps}
For a MAMMOTH interface run, the \xmlNode{MultiRun} step type will most likely be used. First, the step needs
to be named: this name will be one of the names used in the \xmlNode{Sequence} block. In our example, \xmlString{Grid\_Mammoth}.
%
\begin{lstlisting}[style=XML]
<MultiRun name='Grid_Mammoth' verbosity='debug'>
    <Input   class='Files' type=''>mammothInput</Input>
    <Input   class='Files' type=''>crossSection</Input>
    <Input   class='Files' type=''>alias</Input>
    <Input   class='Files' type=''>bisonInput</Input>
    <Input   class='Files' type=''>bisonMesh</Input>
    <Input   class='Files' type=''>fuelCTEfunct</Input>
    <Input   class='Files' type=''>rattlesnakeMesh</Input>
    <Model   class='Models' type='Code'>Mammoth</Model>
    <Sampler class='Samplers' type='Grid'>Grid_Samplering</Sampler>
    <Output  class='DataObjects' type='PointSet'>solns</Output>
</MultiRun>
\end{lstlisting}
%
With this step, we need to import all the files needed for the simulation:
%
\begin{itemize}
  \item MAMMOTH|Rattlesnake YAML input file;
  \item Yak multigroup cross section libraries input files (XML);
  \item Yak alias files used to define the perturbed variables (XML);
  \item Bison YAML input file;
  \item Bison mesh file;
  \item Bison function file for the fuel's coefficient of thermal expansion as a function of temperature;
  \item Rattlesnake mesh file.
\end{itemize}
As well as \xmlNode{Model}, \xmlNode{Sampler} and outputs, such as \xmlNode{OutStreams} and \xmlNode{DataObjects}.

%%%%%%%%%%%%%%%%%%%%%%%%%%%%
%%%%%% MELCOR  INTERFACE  %%%%%%
%%%%%%%%%%%%%%%%%%%%%%%%%%%%
\subsection{MELCOR Interface}
\label{subsec:MELCORInterface}

The current implementation of MELCOR interface is valid for MELCOR 2.1/2.2; its validity for MELCOR
1.8 is \textbf{not been tested}.

\subsubsection{Sequence}
In the \xmlNode{Sequence} section, the names of the steps declared in the
\xmlNode{Steps} block should be specified.
%
As an example, if we called the first multirun ``Grid\_Sampler'' and the second
multirun ``MC\_Sampler'' in the sequence section we should see this:
\begin{lstlisting}[style=XML]
<Sequence>Grid_Sampler,MC_Sampler</Sequence>
\end{lstlisting}
%%%%%%%%%%%%%%%%%%%%%%%%%%%%%%%%%%%%%%%%%%%%%%%%%%%

\subsubsection{batchSize and mode}
For the \xmlNode{batchSize} and \xmlNode{mode} sections please refer to the
\xmlNode{RunInfo} block in the previous chapters.
%
%%%%%%%%%%%%%%%%%%%%%%%%%%%%%%%%%%%%%%%%%%%%%%%%%%%%
\subsubsection{RunInfo}
After all of these blocks are filled out, a standard example RunInfo block may
look like the example below:
\begin{lstlisting}[style=XML]
<RunInfo>
  <WorkingDir>~/workingDir</WorkingDir>
  <Sequence>Grid_Sampler,MC_Sampler</Sequence>
  <batchSize>8</batchSize>
</RunInfo>
\end{lstlisting}
In this example, the \xmlNode{batchSize} is set to $8$; this means that 8 simultaneous (parallel) instances
of MELCOR are going to be executed when a sampling strategy is employed.
%%%%%%%%%%%%%%%%%%%%%%%%%%%%%%%%%%%%%%%%%%%%%%%%%%%%%%%%%%%
\subsubsection{Files}
In the \xmlNode{Files} section, as specified before, all of the files needed for
the code to run should be specified.
%
In the case of MELCOR, the files typically needed are:
\begin{itemize}
  \item MELCOR Input file (file extension ``.i'' or ``.inp'')
  \item Restart file (if present)
\end{itemize}
Example:
\begin{lstlisting}[style=XML]
<Files>
  <Input name='melcorInputFile' type=''>inputFileMelcor.i</Input>
  <Input name='aRestart' type=''>restartFile</Input>
</Files>
\end{lstlisting}

It is a good practice to put inside the working directory (\xmlNode{WorkingDir}) all of these files.

\textcolor{red}{
\textbf{It is important to notice that the interface output collection  (i.e. the parser of the MELCOR output)
currently is able to extract \textit{CONTROL VOLUME HYDRODYNAMICS EDIT AND CONTROL FUNCTION EDIT} data only. Only those
variables are going to be exported and make available to RAVEN.
In addition, it is important to notice that:}
\begin{itemize}
  \item \textbf{the simulation time is stored in a variable called \textit{``time''}};
  \item \textbf{all the variables specified in the \textit{CONTROL VOLUME HYDRODYNAMICS EDIT}
   block are going to be converted using underscores. For example, the following EDITs:}
    \begin{table}[h]
    \centering
    \begin{tabular}{ccccc}
        VOLUME & PRESSURE & TLIQ   & TVAP   & MASS     \\
                & PA       & K      & K      & KG       \\
             1      & 1.00E+07 & 584.23 & 584.23 & 1.66E+03
     \end{tabular}
    \end{table}
    \\\textbf{will be converted in the following way (CSV):}
    \begin{table}[h]
    \centering
    \begin{tabular}{ccccc}
         $time$ & $volume\_1\_PRESSURE$& $volume\_1\_TLIQ$ & $volume\_1\_TVAP$   & $volume\_1\_MASS$     \\
             1.0   & 1.00E+07 & 584.23 & 584.23 & 1.66E+03
     \end{tabular}
    \end{table}
\end{itemize}
}

CONTROL FUNCTION EDIT data will not be converted in this manner. All data will be labeled using a label identical to what was entered in the MELCOR input file, with no changes.

Remember also that a MELCOR simulation run is considered successful (i.e., the simulation did not crash) if it terminates with the
following message:

\textcolor{red}{Normal termination}

If the a MELCOR simulation run stops with messages other than this one than the simulation is considered as
crashed, i.e., it will not be saved.
Hence, it is strongly recommended to set up the MELCOR input file so that the simulation exiting conditions are set through control logic
trip variables.

%%%%%%%%%%%%%%%%%%%%%%%%%%%%%%%%%%%%%%%%%%%%%%%%%%%%
\subsubsection{Models}
For the \xmlNode{Models} block here is a standard example of how it would look
when using MELCOR 2.1/2.2 as the external code:
\begin{lstlisting}[style=XML]
<Models>
  <Code name='MyMELCOR' subType='Melcor'>
    <executable>~/path_to_the_executable_of_melcor</executable>
    <preexec>~/path_to_the_executable_of_melgen</preexec>
  </Code>
</Models>
\end{lstlisting}
As it can be seen above, the \xmlNode{preexec} node must be specified, since MELCOR 2.1/2.2 must run the MELGEN utility
code before executing. Once the \xmlNode{preexec} node is inputted, the execution of MELGEN is performed automatically by the Interface.
\\In addition, if some command line parameters need to be passed to MELCOR, the user might use (optionally) the \xmlNode{clargs} XML nodes.
\begin{lstlisting}[style=XML]
<Models>
  <Code name='MyMELCOR' subType='Melcor'>
    <executable>~/path_to_the_executable_of_melcor</executable>
    <preexec>~/path_to_the_executable_of_melgen</preexec>
    <clargs type="text" arg="-r whatever command line instruction"/>
  </Code>
</Models>
\end{lstlisting}

%%%%%%%%%%%%%%%%%%%%%%%%%%%%%%%%%%%%%%%%%%%%%%%%%%%%%%%%%
\subsubsection{Distributions}
The \xmlNode{Distribution} block defines the distributions that are going
to be used for the sampling of the variables defined in the \xmlNode{Samplers}
block.
%
For all the possible distributions and all their possible inputs please see the
chapter about Distributions (see~\ref{sec:distributions}).
%
Here we report an example of a Normal distribution:
\begin{lstlisting}[style=XML,morekeywords={name,debug}]
<Distributions verbosity='debug'>
    <Normal name="temper">
      <mean>1.E+7</mean>
      <sigma>1.5</sigma>
      <upperBound>9.E+6</upperBound>
      <lowerBound>1.1E+7</lowerBound>
    </Normal>
 </Distributions>
\end{lstlisting}

It is good practice to name the distribution something similar to what kind of
variable is going to be sampled, since there might be many variables with the
same kind of distributions but different input parameters.
%
%%%%%%%%%%%%%%%%%%%%%%%%%%%%%%%%%%%%%%%%%%%%%%%%%%%%%%%%%
\subsubsection{Samplers}
In the \xmlNode{Samplers} block we want to define the variables that are going
to be sampled.
%
\textbf{Example}:
We want to do the sampling of 1 single variable:
\begin{itemize}
  \item The in pressure ($P\_in$) of a control volume regulated by a Tabular Function $TF\_TAB$
\end{itemize}

We are going to sample this variable using two different sampling methods:
Grid and MonteCarlo.

The interface of MELCOR uses the \textbf{\textit{GenericCode}} (see section \ref{subsec:genericInterface})
interface for the input perturbation; this means that the original input file (listed in the \xmlNode{Files} XML block)
needs to implement wild-cards.
%
In this example we are sampling the variable:
\begin{itemize}
  \item \textit{PRE}, which acts on the Tabular Function $TF\_TAB$ whose $TF\_ID $ is $P\_in$.
\end{itemize}

We proceed to do so for both the Grid sampling and the MonteCarlo sampling.

\begin{lstlisting}[style=XML,morekeywords={name,type,construction,lowerBound,steps,limit,initialSeed}]
<Samplers verbosity='debug'>
  <Grid name='Grid_Sampler' >
    <variable name='PRE'>
      <distribution>temper</distribution>
      <grid type='CDF' construction='equal'  steps='10'>0.001 0.999</grid>
    </variable>
  </Grid>
  <MonteCarlo name='MC_Sampler'>
     <samplerInit>
       <limit>1000</limit>
     </samplerInit>
    <variable name='PRE'>
      <distribution>temper</distribution>
  </MonteCarlo>
</Samplers>
\end{lstlisting}

It can be seen that each variable is connected with a proper distribution
defined in the \\\xmlNode{Distributions} block (from the previous example).
%
The following demonstrates how the input for the variable is read.

We are sampling a variable whose wild-card in the original input file is named $\$RAVEN-PRE\$$
using a Grid sampling method.
%
The distribution that this variable is following is a Normal distribution
(see section above).
%
We are sampling this variable beginning from 0.001 (CDF) in 10 \textit{equal} steps of
0.0998 (CDF).
%
%%%%%%%%%%%%%%%%%%%%%%%%%%%%%%%%%%%%%%%%%%%%%%%%%%%%%%%%%%%
\subsubsection{Steps}
For a MELCOR interface, the \xmlNode{MultiRun} step type will most likely be
used.
%
First, the step needs to be named: this name will be one of the names used in
the \xmlNode{sequence} block.
%
In our example, \texttt{Grid\_Sampler} and \texttt{MC\_Sampler}.
%
\begin{lstlisting}[style=XML,morekeywords={name,debug,re-seeding}]
     <MultiRun name='Grid_Sampler' verbosity='debug'>
\end{lstlisting}

With this step, we need to import all the files needed for the simulation:
\begin{itemize}
  \item MELCOR input file
  \item any other file needed by the calculation (e.g. restart file)
\end{itemize}
\begin{lstlisting}[style=XML,morekeywords={name,class,type}]
    <Input   class='Files' type=''>inputFileMelcor.i</Input>
    <Input   class='Files' type=''>restartFile</Input>
\end{lstlisting}
We then need to define which model will be used:
\begin{lstlisting}[style=XML]
    <Model  class='Models' type='Code'>MyMELCOR</Model>
\end{lstlisting}
We then need to specify which Sampler is used, and this can be done as follows:
\begin{lstlisting}[style=XML]
    <Sampler class='Samplers' type='Grid'>Grid_Sampler</Sampler>
\end{lstlisting}
And lastly, we need to specify what kind of output the user wants.
%
For example the user might want to make a database (in RAVEN the database
created is an HDF5 file).
%
Here is a classical example:
\begin{lstlisting}[style=XML,morekeywords={class,type}]
    <Output  class='Databases' type='HDF5'>Grid_out</Output>
\end{lstlisting}
Following is the example of two MultiRun steps which use different sampling
methods (Grid and Monte Carlo), and creating two different databases for each
one:
\begin{lstlisting}[style=XML]
<Steps verbosity='debug'>
  <MultiRun name='Grid_Sampler' verbosity='debug'>
    <Input   class='Files' type=''>inputFileMelcor.i</Input>
    <Input   class='Files' type=''>restartFile</Input>
    <Model   class='Models'    type='Code'>MyMELCOR</Model>
    <Sampler class='Samplers'  type='Grid'>Grid_Sampler</Sampler>
    <Output  class='Databases' type='HDF5'>Grid_out</Output>
    <Output  class='DataObjects' type='PointSet'   >GridMelcorPointSet</Output>
    <Output  class='DataObjects' type='HistorySet'>GridMelcorHistorySet</Output>
  </MultiRun>
  <MultiRun name='MC_Sampler' verbosity='debug' re-seeding='210491'>
    <Input   class='Files' type=''>inputFileMelcor.i</Input>
    <Input   class='Files' type=''>restartFile</Input>
    <Model   class='Models'    type='Code'>MyMELCOR</Model>
    <Sampler class='Samplers'  type='MonteCarlo'>MC_Sampler</Sampler>
    <Output  class='Databases' type='HDF5'      >MC_out</Output>
    <Output  class='DataObjects' type='PointSet'   >MonteCarloMelcorPointSet</Output>
    <Output  class='DataObjects' type='HistorySet'>MonteCarloMelcorHistorySet</Output>
  </MultiRun>
</Steps>
\end{lstlisting}
%%%%%%%%%%%%%%%%%%%%%%%%%%%%%%%%%%%%%%%%%%%%%%%%%%%%%%
\subsubsection{Databases}
As shown in the \xmlNode{Steps} block, the code is creating two database objects
called \texttt{Grid\_out} and \texttt{MC\_out}.
%
So the user needs to input the following:
\begin{lstlisting}[style=XML]
<Databases>
  <HDF5 name="Grid_out" readMode="overwrite"/>
  <HDF5 name="MC_out" readMode="overwrite"/>
</Databases>
\end{lstlisting}
As listed before, this will create two databases.
%
The files will have names corresponding to their \xmlAttr{name} appended with
the .h5 extension (i.e. \texttt{Grid\_out.h5} and \texttt{MC\_out.h5}).
%%%%%%%%%%%%%%%%%%%%%%%%%%%%%%%%%%%%%%%%%%%%%%%%%%%%%%
\subsubsection{DataObjects}
As shown in the \xmlNode{Steps} block, the code is creating $4$ data objects ($2$ HistorySet and $2$ PointSet)
called \texttt{GridMelcorPointSet} \texttt{GridMelcorHistorySet} \texttt{MonteCarloMelcorPointSet} and
 \texttt{MonteCarloMelcorHistorySet}.
%
So the user needs to input the following block as well, where the Input and Output variables are listed:
\begin{lstlisting}[style=XML]
  <DataObjects>
    <PointSet name="GridMelcorPointSet">
      <Input>PRE</Input>
      <Output>
        time,volume_1_PRESSURE,volume_1_TLIQ,
        volume_1_TVAP,volume_1_MASS
      </Output>
    </PointSet>
    <HistorySet name="GridMelcorHistorySet">
      <Input>PRE</Input>
      <Output>
        time,volume_1_PRESSURE,volume_1_TLIQ,
        volume_1_TVAP,volume_1_MASS
      </Output>
    </HistorySet>
    <PointSet name="MonteCarloMelcorPointSet">
      <Input>PRE</Input>
      <Output>
        time,volume_1_PRESSURE,volume_1_TLIQ,
        volume_1_TVAP,volume_1_MASS
      </Output>
    </PointSet>
    <HistorySet name="MonteCarloMelcorHistorySet">
      <Input>PRE</Input>
      <Output>
        time,volume_1_PRESSURE,volume_1_TLIQ,
        volume_1_TVAP,volume_1_MASS
      </Output>
    </HistorySet>
  </DataObjects>
\end{lstlisting}
As mentioned before, this will create $4$ DataObjects.
%
%%%%%%%%%%%%%%%%%%%%%%%
%%%%%% SCALE  INTERFACE %%%%%%
%%%%%%%%%%%%%%%%%%%%%%%
\subsection{SCALE Interface}
This section presents the main aspects of the interface between RAVEN and SCALE system,
the consequent RAVEN input adjustments and the modifications of the SCALE
files required to run the two coupled codes.
\\ \textcolor{red}{
\textbf{\textit{\nb Considering the large amount of SCALE sequences, this interface is
currently limited in driving the following SCALE calculation codes:}}
\begin{itemize}
  \item \textbf{\textit{ORIGEN}}
  \item \textbf{\textit{TRITON (using NEWT as transport solver)}}
\end{itemize}
}

In the following sections a short explanation on how to use RAVEN coupled with SCALE is reported.

%%%%%%%%%%%%%%%%%%%%%%%%%%%%%%%%%%%%%%%%%%%%%%%%%%%%%%%%%

\subsubsection{Models}
As for any other Code, in order to activate the SCALE interface, a  \xmlNode{Code} XML node needs to be inputted, within the
main XML node \xmlNode{Models}.
\\The  \xmlNode{Code} XML node contains the
information needed to execute the specific External Code.

\attrsIntro
%
\vspace{-5mm}
\begin{itemize}
  \itemsep0em
  \item \nameDescription
  \item \xmlAttr{subType}, \xmlDesc{required string attribute}, specifies the
  code that needs to be associated to this Model.
  %
  \nb See Section~\ref{sec:existingInterface} for a list of currently supported
  codes.
  %
\end{itemize}
\vspace{-5mm}

\subnodesIntro
%
\begin{itemize}
  \item \xmlNode{executable} \xmlDesc{string, required field} specifies the path
  of the executable to be used.
  %
  \nb Either an absolute or relative path can be used.
  \item \aliasSystemDescription{Code}
  %
\end{itemize}

In addition (and specifc for the SCALE interface), the  \xmlNode{Code} can contain the following optional nodes:

\begin{itemize}
  \item \xmlNode{sequence}, optional, comma separated list. In this node the user can specify a list of sequences that need to be
  executed in sequence. For example, if a TRITON calculation needs to be followed by an ORIGEN decay heat calculation the user
  would input here the sequence ``\textit{triton,origen}''. \default{triton}.
  \\\nb Currently only the following entries are supported:
    \begin{itemize}
     \item  ``\textit{triton}''
     \item  ``\textit{origen}''
     \item  ``\textit{triton,origen}''
    \end{itemize}
  \item \xmlNode{timeUOM}, optional, string. In this node the user can specify  the \textit{units} for the independent variable ``time''.
   If the outputs are exported by SCALE in a different unit (e.g days, years, etc.), the SCALE interface will convert all the different
   time scales into the unit here specified (in order to have a consistent  (and unique) time scale). Available are:
    \begin{itemize}
     \item ``\textit{s}'', seconds
     \item ``\textit{m}'', minutes
     \item ``\textit{h}'', hours
     \item ``\textit{d}'', days
     \item ``\textit{y}'', years
    \end{itemize}
    \default{s}
\end{itemize}

An example  is shown  below:
\begin{lstlisting}[style=XML]
<Models>
    <Code name="MyScale" subType="Scale">
      <executable>path/to/scalerte</executable>
      <sequence>triton,origen</sequence>
      <timeUOM>d</timeUOM>
    </Code>
</Models>
\end{lstlisting}

%%%%%%%%%%%%%%%%%%%%%%%%%%%%%%%%%%%%%%%%%%%%%%%%%%%%%%%%%%%%%%%%%%%%%%%%%%%%%%%%%%
\subsubsection{Files}
%%%%%%%%%%%%%%%%%%%%%%%%%%%%%%%%%%%%%%%%%%%%%%%%%%%%%%%%%%%%%%%%%%%%%%%%%%%%%%%%%%
The \xmlNode{Files} XML node has to contain all the files required by the particular
sequence (s) of the external code  (SCALE) to be run.
This involves not only the input file(s) (.inp) but also the auxiliary files that might be needed (e.g. binary initial compositions, etc.).
As mentioned, the current SCALE interface only supports TRITON and ORIGEN sequences. For this reason, depending on the
type of sequence (see previous section) to be run, the relative input files need to be marked with the sequence they are associated
with. This means that the type of the input file must be either ``triton'' or ``origen''. The auxiliary files that might be needed by
a particular sequence (e.g. binary initial compositions, etc.) should not be marked with any specific type (i.e. \textit{type=``''}).
Example:
\begin{lstlisting}[style=XML]
<Files>
  <Input name="triton_input" type="triton">pwr_depletion.inp</Input>
  <Input name="origen_input" type="origen">decay_calc.inp</Input>
  <Input name="binary_comp" type="">pwr_depletion.f71</Input>
</Files>
\end{lstlisting}
The files mentioned in this section
 need, then, to be placed into the working directory specified
by the \xmlNode{workingDir} node in the \xmlNode{RunInfo} XML node.

\paragraph{Output Files conversion}
Since RAVEN expects to receive a CSV file containing the outputs of the simulation, the results in the SCALE output
files need to be converted by the code interface.
\\As mentioned, the current interface \textcolor{red}{ is able to collect data from TRITON and ORIGEN sequences only}.
%% TRITON
\\The following information is collected from TRITON output:
\begin{itemize}
  \item \textit{\textbf{k-eff and k-inf time-dep information}}
  \begin{lstlisting}[basicstyle=\tiny]
  Outer   Eigenvalue Eigenvalue Max Flux   Max Flux     Max Fuel   Max Fuel     Wall   Elapsed   Iteration  CPU   Inners
 Iter. #              Delta      Delta   Location(r,g)   Delta   Location(r,g) Clock   CPU Time   CPU Time Usage Converged
 - - - - - - - - - - - - - - - - - - - - - - - - - - - - - - - - - - - - - - - - - - - - - - - - - - - - - - - - - - - -
     1    1.00000   0.000E+00 6.480E+09 (    4,252)   1.000E+00 (  614,  0) 14:16:42   89.9 s    89.9 s  92.7%    F
     2    0.35701   1.801E+00 4.149E+01 (  319,  4)   2.673E+00 ( 7035,  0) 14:18:16  182.8 s    92.9 s  98.8%    F
 k-eff =       0.94724509     Time=      0.00d Nominal conditions

   Four-Factor Estimate of k-infinity.  Fast/Thermal boundary:   0.6250 eV
      Fiss. neutrons/thermal abs. (eta):          1.279827
      Thermal utilization (f):                    0.960903
      Resonance Escape probability (p):           0.706209
      Fast-fission factor (epsilon):              1.091716
                                            --------------
      Infinite neutron multiplication             0.948143

\end{lstlisting}
   that will be converted in the following way (CSV):
   \begin{table}[h]
    \centering
    \caption{CSV transport info}
    \label{CSVkeff}
    \tabcolsep=0.11cm
    \tiny
    \begin{tabular}{|c|c|c|c|c|c|c|c|c|c|}
     time & keff       & iter\_number & keff\_delta & max\_flux\_delta & kinf     & kinf\_epsilon & kinf\_p  & kinf\_f  & kinf\_eta \\
     0.00 & 0.94724509 & 2            & 1.801E+00   & 4.149e+01        & 0.948143 & 1.091716      & 0.706209 & 0.960903 & 1.279827
    \end{tabular}
   \end{table}

  \item \textit{\textbf{material powers}}
  \begin{lstlisting}[basicstyle=\tiny]
  --- Material powers for depletion pass no.   1 (MW/MITHM) ---
       Time =     0.00 days (   0.000 y), Burnup =    0.000     GWd/MTIHM, Transport k=  0.9473

                    Total    Fractional  Mixture     Mixture       Mixture
         Mixture    Power      Power      Power    Thermal Flux  Total Flux
          Number (MW/MTIHM)    (---)   (MW/MTIHM)  n/(cm^2*sec)  n/(cm^2*sec)
            13      32.985    0.99054     32.985    5.3666e+13    1.2574e+14
             6       0.252    0.00757     N/A       2.7587e+13    9.1781e+13
         Total      33.300    1.00000
\end{lstlisting}
   that will be converted in the following way (CSV):
   \begin{table}[h]
     \centering
     \caption{CSV material powers}
     \label{CSVmatPowers}
     \tabcolsep=0.11cm
     \tiny
     \begin{tabular}{|c|c|c|c|c|c|c|c|c|c|l}
     \cline{1-10}
     time    & bu  & tot\_power\_mix\_13 & fract\_power\_mix\_13 & th\_flux\_mix\_13 & tot\_flux\_mix\_13 & tot\_power\_mix\_6 & fract\_power\_mix\_6 & th\_flux\_mix\_6 & tot\_flux\_mix\_6 &  \\ \cline{1-10}
     1.0E-06 & 0.0 & 32.985              & 0.99054               & 5.3666e+13        & 1.2574e+14         & 0.252              & 0.00757              & 2.7587e+13       & 9.1781e+13        &  \\ \cline{1-10}
     \end{tabular}
   \end{table}


 \item \textit{\textbf{nuclide/element tables}}
  \begin{lstlisting}[basicstyle=\tiny]
            | nuclide concentrations
            | time: days
      grams |    0.00e+00d
------------+--------------------
       u235 |   2.9619e+04
       u238 |   9.6993e+05
   subtotal |   1.0010e+06
      total |   1.1858e+06
\end{lstlisting}
   that will be converted in the following way (CSV):
   \begin{table}[h]
    \centering
    \caption{CSV Nuclide/element Tables}
    \label{CSVnuclideTables}
    \tabcolsep=0.11cm
    \tiny
    \begin{tabular}{|c|c|c|}
     time & u235\_conc       & u238\_conc   \\
     0.00 & 2.9619e+04  & 9.6993e+05
    \end{tabular}
   \end{table}
\end{itemize}
%% ORIGEN
The following information is collected from ORIGEN output:
\begin{itemize}
  \item \textit{\textbf{history overview}}
  \begin{lstlisting}[basicstyle=\tiny]
=========================================================================================================================
=   History overview for case 'decay' (#1/1)                                                                            =
-------------------------------------------------------------------------------------------------------------------------
   step          t0          t1          dt           t        flux     fluence       power      energy
    (-)       (sec)       (sec)         (s)         (s)   (n/cm2-s)     (n/cm2)        (MW)       (MWd)
      1  0.0000E+00  1.0000E-06  1.0000E-06  1.0000E-06  0.0000E+00  0.0000E+00  0.0000E+00  0.0000E+00
\end{lstlisting}
   that will be converted in the following way (CSV):
    \begin{table}[h]
    \centering
    \caption{CSV History Overview}
    \label{CSVhistoryOverview}
    \tabcolsep=0.11cm
    \tiny
    \begin{tabular}{|c|c|c|c|c|c|c|c|}
    \hline
     time    & t0  & t1      & dt      & flux & fluence & power & energy \\
     1.0E-06 & 0.0 & 1.0E-06 & 1.0E-06 & 0.0  & 0.0     & 0.0   & 0.0
    \end{tabular}
   \end{table}

   \item \textit{\textbf{concentration tables}}
  \begin{lstlisting}[basicstyle=\tiny]
=========================================================================================================================
=   Nuclide concentrations in watts, actinides for case 'decay' (#1/1)                                                  =
-------------------------------------------------------------------------------------------------------------------------
  (relative cutoff; integral of concentrations over time >   1.00E-04 % of integral of all concentrations over time)
.
                0.0E+00sec  1.0E-06sec
  th231       8.6167E-08  8.6167E-08
  th234       7.7763E-09  7.7763E-09
------------
  totals       4.6831E+03  4.6831E+03
=========================================================================================================================
.
.
=========================================================================================================================
=   Nuclide concentrations in watts, fission products for case 'decay' (#1/1)                                           =
-------------------------------------------------------------------------------------------------------------------------
  (relative cutoff; integral of concentrations over time >   1.00E-04 % of integral of all concentrations over time)
.
                0.0E+00sec  1.0E-06sec
  ga74        2.4264E-01  2.4264E-01
  ga75        1.8106E+00  1.8106E+00
------------
  totals       1.2266E+06  1.2266E+06
  \end{lstlisting}
  that will be converted in the following way (CSV):
   \begin{table}[h]
    \centering
    \caption{CSV Concentration Tables}
    \label{CSVconcentrationTables}
    \tabcolsep=0.11cm
    \tiny
    \begin{tabular}{|c|c|c|c|c|c|c|c|}
    \hline
     time    & ga74\_watts  & ga75\_watts      & subtotals\_fission\_products      & th231\_watts & th234\_watts & subtotals\_actinides & totals\_watts \\ \hline
     0.0E+00 & 2.4264E-01 & 1.8106E+00 & 1.2266E+06 & 8.6167E-08  & 7.7763E-09     & 4.6831E+03   & 1.2313E+06    \\
     1.0E-06 & 2.4264E-01 & 1.8106E+00 & 1.2266E+06 & 8.6167E-08  & 7.7763E-09     & 4.6831E+03  & 1.2313E+06
    \end{tabular}
   \end{table}
\end{itemize}

\textbf{Remember also that a SCALE simulation run is considered successful (i.e., the simulation did not crash) if it does not contain, in
the last 20 lines, the following message:}

\textcolor{red}{terminated due to errors}

\textbf{If the a SCALE simulation terminates with this message, the simulation is considered ``failed'', i.e., it will not be saved.}

%%%%%%%%%%%%%%%%%%%%%%%%%%%%%%%%%%%%%%%%%%%%%%%%%%%%%%%%%%%%%%%%%%%%%%%%%%%%%%%%%%
\subsubsection{Samplers or Optimizers}
In the \xmlNode{Samplers} or  \xmlNode{Optimizers} block we want to define the variables that are going
to be sampled or optimized.
%
\\The perturbation or optimization of the input of any SCALE sequence is performed using the approach detailed in the \textit{Generic Interface} section (see \ref{subsec:genericInterface}). Briefly, this approach uses
 ``wild-cards'' (placed in the original input files) for injecting the perturbed values.
 For example, if the original input file (that needs to be perturbed) is the following:
\begin{lstlisting}[language=python]
=origen
case(actual_mass){
  lib{ file="end7dec" }
  mat{ iso=[zr-95=1.0] units="moles" }
  time=[1.0] %1 day
}
end
\end{lstlisting}
and  the initial moles of ``zr-95'' need to be perturbed, a RAVEN ``wild-card'' will be defined:
\begin{lstlisting}[language=python]
=origen
case(actual_mass){
  lib{ file="end7dec" }
  mat{ iso=[zr-95=$RAVEN-zrMoles$] units="moles" }
  time=[1.0] %1 day
}
end
\end{lstlisting}

Finally, the variable \textbf{\textit{zrMoles}} needs to be specified in the specific Sampler or Optimizer that will be used:

\begin{lstlisting}[style=XML]
...
<Samplers>
  <aSampler name='aUserDefinedName' >
    <variable name='zrMoles'>
      ...
    </variable>
  </aSampler>
</Samplers>
...
<Optimizers>
  <anOptimizer name='aUserDefinedName' >
    <variable name='zrMoles'>
      ...
    </variable>
  </anOptimizer>
</Samplers>
...
\end{lstlisting}
%
%%%%%%%%%%%%%%%%%%%%%%%
%%%%%% CTF  INTERFACE %%%%%%
%%%%%%%%%%%%%%%%%%%%%%%
\subsection{CTF Interface}
This section presents the main aspects of the interface between RAVEN and CTF (COBRA-TF) system,
the consequent RAVEN input adjustments and the modifications of the CTF
files required to run the two coupled codes. \noindent

\noindent \textcolor{red}{
\textbf{\textit{\nb This interface is currently working only with the specific type of CTF output file (.ctf.out or deck.out (if input file name is deck.inp)) }}
}

\noindent In the following sections a short explanation on how to use RAVEN coupled with CTF is reported.
%%%%%%%%%%%%%%%%%%%%%%%%%%%%%%%%%%%%%%%%%%%%%%%%%%%
\subsubsection{Sequence}
%%%%%%%%%%%%%%%%%%%%%%%%%%%%%%%%%%%%%%%%%%%%%%%%%%%
In the \xmlNode{Sequence} section, the names of the steps declared in the
\xmlNode{Steps} block should be specified.
%
As an example, if we called the first MultiRun ``Grid\_Sampler'' and the second
MultiRun ``MC\_Sampler'' in the sequence section we should see this:

\begin{lstlisting}[style=XML]
<Sequence>Grid_Sampler, MC_Sampler</Sequence>
\end{lstlisting}

%%%%%%%%%%%%%%%%%%%%%%%%%%%%%%%%%%%%%%%%%%%%%%%%%%%
\subsubsection{batchSize and mode}
%%%%%%%%%%%%%%%%%%%%%%%%%%%%%%%%%%%%%%%%%%%%%%%%%%%
For the \xmlNode{batchSize} and \xmlNode{mode} sections please refer to the
\xmlNode{RunInfo} block in the previous chapters.

%%%%%%%%%%%%%%%%%%%%%%%%%%%%%%%%%%%%%%%%%%%%%%%%%%%%
\subsubsection{RunInfo}
%%%%%%%%%%%%%%%%%%%%%%%%%%%%%%%%%%%%%%%%%%%%%%%%%%%%
After all of these blocks are filled out, a standard example RunInfo block may
look like the example below: \\

\begin{lstlisting}[style=XML]
<RunInfo>
  <WorkingDir>~/workingDir</WorkingDir>
  <Sequence>Grid_Sampler,MC_Sampler</Sequence>
  <batchSize>8</batchSize>
</RunInfo>
\end{lstlisting}
In this example, the \xmlNode{batchSize} is set to $8$; this means that 8 simulatenous (parallel) instances
of CTF are going to be executed when a sampling strategy is employed.

%%%%%%%%%%%%%%%%%%%%%%%%%%%%%%%%%%%%%%%%%%%%%%%%%%%%%%%%%
\subsubsection{Models}
%%%%%%%%%%%%%%%%%%%%%%%%%%%%%%%%%%%%%%%%%%%%%%%%%%%%%%%%%
As any other Code, in order to activate the CTF interface, a \xmlNode{Code} XML node needs to be inputted, within the
main XML node \xmlNode{Models}.
\\The  \xmlNode{Code} XML node contains the
information needed to execute the specific External Code.

\attrsIntro
%
\vspace{-5mm}
\begin{itemize}
  \itemsep0em
  \item \nameDescription
  \item \xmlAttr{subType}, \xmlDesc{required string attribute}, specifies the
  code that needs to be associated to this Model.
  %
  \nb See Section~\ref{sec:existingInterface} for a list of currently supported
  codes.
  %
\end{itemize}
\vspace{-5mm}

\subnodesIntro
%
\begin{itemize}
  \item \xmlNode{executable} \xmlDesc{string, required field} specifies the path
  of the executable to be used.
  %
  \nb Either an absolute or relative path can be used.
  \item \aliasSystemDescription{Code}
  %
\end{itemize}

An example  is shown  below:

\begin{lstlisting}[style=XML]
<Models>
    <Code name="MyCobraTF" subType="CTF">
      <executable>path/to/cobratf</executable>
    </Code>
</Models>
\end{lstlisting}

%%%%%%%%%%%%%%%%%%%%%%%%%%%%%%%%%%%%%%%%%%%%%%%%%%%%%%%%%%%%%%%%%%%%%%%%%%%%%%%%%%
\subsubsection{Files}
%%%%%%%%%%%%%%%%%%%%%%%%%%%%%%%%%%%%%%%%%%%%%%%%%%%%%%%%%%%%%%%%%%%%%%%%%%%%%%%%%%
The \xmlNode{Files} XML node has to contain all the files required to run the external code  (CTF).
For RAVEN coupled with CTF, there are three input files. CTF input file (.inp) is required by the code. This input file includes all the geometry, boundary and calculation definitions.

\noindent There are two additional files (\textit{optional}) that can be used for model parameter perturbation(s) (vuq$\_$param.txt, vuq$\_$mult.txt). These two files can be used to change variables of models embedded in CTF. The ''vuq$\_$param.txt'' file includes parameter values, and ''vuq$\_$mult.txt'' file includes multipliers or additions to parameters. These files are not required by CTF unless a parameter exposure is desired. One, both or neither of them can be included in the simulation folder. The code first controls if these files exist and does modifications accordingly if needed.

The  \xmlNode{Files} XML node contains the information needed to execute CTF.

\attrsIntro
%
\vspace{-5mm}
\begin{itemize}
  \itemsep0em
  \item \nameDescription
  \item \xmlAttr{type}, \xmlDesc{required string attribute}, specifies the
  input type used by CTF (ctf, vuq$\_$mult, vuq$\_$param). Accepted types are as follows.
  \begin{itemize}
    \item CTF, \xmlDesc{required string attribute}, identifies the CTF input file (geometry, boundary, calculation options, etc.) and the code currently accept any name for input. One CTF input file is required.
    \item vuq\_mult, \xmlDesc{optional string attribute if closure modifiers are used}, identifies the closure term multiplier input file. If user needs to alter closure terms this file should be used and named \textcolor{red}{vuq$\_$mult.txt}. No other file name is accepted.
    \item vuq\_param, \xmlDesc{optional string attribute if model parameter modifiers are used}, identifies the model parameter input file. If user needs to change model parameters this file this model should be used and named \textcolor{red}{vuq$\_$params.txt}. No other file name is accepted.
    \item "", Empty type is also accepted by RAVEN input to perturb. Currently, CTF does not use any other input file that is not mentioned above, but to sample auxiliary files, this option can be used.
  \end{itemize}
\end{itemize}

\noindent Example:
\begin{lstlisting}[style=XML]
<Files>
  <Input name="CTF_input" type="ctf">case1.inp</Input>
  <Input name="vuq_param_input" type="vuq_param">vuq_param.txt</Input>
  <Input name="vuq_mult_input" type="vuq_mult">vuq_mult.txt</Input>
  <Input name="auxiliary_input" type="">auxiliaryInput</Input>
</Files>
\end{lstlisting}
The files mentioned in this section
 need, then, to be placed into the working directory specified
by the \xmlNode{workingDir} node in the \xmlNode{RunInfo} XML node.

\paragraph{Output Files Conversion}
Since RAVEN expects to receive a CSV file containing the outputs of the simulation, the results in the CTF output
files (.ctf.out or deck.out) need to be converted by the code interface.

\noindent \textcolor{red}{
\textbf{It is important to note that the interface output collection (i.e., the parser of the CTF output) is currently able to extract
major edit data (.ctf.out or deck.out) only. Only those variables printed in the "major edit" output files are exported and made available to RAVEN.} } \\
\\The following information is collected from CTF output file (.ctf.out or deck.out):
\begin{itemize}
  \item \textit{\textbf{average properties for channels}}
  \begin{lstlisting}[basicstyle=\tiny]


 ************************************************************************************************************************
          simulation time =      1.03030  seconds           aver. properties for channels
 node  dist.  quality    void fraction            mass flow               enthalpy incr.    enthalpy    heat added
  no.   (ft.)                                     (lbm/s)                    (btu/s)         (btu/s)      (btu/s)
                      liq.  vapor  entr.  liquid vapor entr.  integr.  liquid vapor integr.  mixture  liquid vapor integr.

  50   12.00  -.119   1.000 0.000 0.000   16.39  0.00  0.00  16.39     32.41  0.00  32.41   10683.98  32.37  0.00  32.37

\end{lstlisting}

   that will be converted in the following way (CSV):
   \begin{table}[h]
    \centering
    \caption{CSV transport info (average properties for channels)}
    \label{CSVaverageProperties}
    \tabcolsep=0.11cm
    \tiny
    \begin{tabular}{|c|c|c|c|c|c|c|c|c|c|c|}
     time & AVG\_ch\_ax50\_quality  & AVG\_ch\_ax50\_voidFractionLiquid & AVG\_ch\_ax50\_voidFractionVapor & AVG\_ch\_ax50\_volumeEntrainFraction & ...\\
     1.03030 & -.119 & 1.000  & 0.000   & 0.000  & ...
    \end{tabular}
   \end{table}

  \item \textit{\textbf{fluid properties for each sub-channel}}
  \begin{lstlisting}[basicstyle=\tiny]
              simulation time =      0.00000  seconds           fluid properties for channel   19
 node  dist. pressure  velocity             void fraction           flow rate          flow    heat added         gama
  no.  (ft.) (psi)     (ft/sec)                                      (lbm/s)           reg.     (btu/s)          (lbm/s)
                     liquid vapor entr. liquid  vapor  entr.   liquid  vapor   entr.         liquid    vapor


 155 0.00  1251.687  2.66   2.66  0.01  1.0000 0.0000 0.0000  0.12456  0.0000  0.00000  0   0.595E-01  0.000E+00   0.00

  \end{lstlisting}
   that will be converted in the following way (CSV):
   \begin{table}[h]
    \centering
    \caption{CSV transport info (fluid properties for channels)}
    \label{CSVfluidProperties}
    \tabcolsep=0.11cm
    \tiny
    \begin{tabular}{|c|c|c|c|c|c|c|c|c|c|c|}
     time & ch19\_ax155\_pressure  & ch19\_ax155\_velocityLiquid & ch19\_ax155\_velocityVapor & ch19\_ax155\_velocityEntrain & ch19\_ax155\_voidFractionLiquid & ...\\
     0.00 & 1251.687 & 2.66           & 2.66   & 0.01        & 1.00 & ...
    \end{tabular}
   \end{table}

  \item \textit{\textbf{nuclear fuel rod}}
  \begin{lstlisting}[basicstyle=\tiny]
          nuclear fuel rod no.  1                         simulation time =    0.00 seconds
             surface no.  1 of  1
          -----------------------        conducts heat to channels  1  0  0  0  0  0        geometry type =  1
                                         and azimuthally to surfaces   1 and   1            no. of radial nodes = 13

 **********************************************************************************************

   rod    axial    fluid temperatures  surface      heat       -clad temperatures-     gap        -fuel temperatures-
   node  location      (deg-f)         heat flux   transfer          (deg-f)        conductance         (deg-f)
   no.    (in.)    liquid  vapor       (b/h-ft2)    mode       outside  inside      (b/h-ft2-f)    surface   center
   ----  --------  ------  -----      ---------    --------    -------  ------      -----------    -------   ------

    10   22.80     464.1   467.1     0.5929E+04     spl        466.08   592.98       1594.2        859.58     2946.22
\end{lstlisting}
   that will be converted in the following way (CSV):
   \begin{table}[h]
     \centering
     \caption{CSV transport info (nuclear fuel rod)}
     \label{CSVfuelRod}
     \tabcolsep=0.11cm
     \tiny
     \begin{tabular}{|c|c|c|c|}
     time    & fuelRod10\_surface1\_ax10\_fluidTemperatureLiquid  & fuelRod10\_surface1\_ax10\_fluidTemperatureVapor & ... \\
     0.00 & 464.1 & 467.1  &  ...
     \end{tabular}
   \end{table}

   \item \textit{\textbf{cyclindrical tube}}

      \noindent \textcolor{red}{
   \textbf{Warning: CTF reports results for cylindrical tubes based on the flow type. Not every result will be available depending on the \underline{internal} or \underline{external} flow type. Check output file and see if the flow around heat slab is internal or external. If the user requests values that are not in the output file reported values will be from different columns and wrong. For example there is no outside surface liquid temperature when flow is internal.} } \\

   \begin{lstlisting}[basicstyle=\tiny]
    cylindrical tube rod no.  5                         simulation time =    2.00 seconds
           surface no.  1 of  4
    ------------------------        conducts heat to channels 10  0  0  0  0  0               geometry type =  2
                                    and azimuthally to surfaces   4 and   2                   no. of radial nodes =  2


 **********************************************************************************************************************

 rod    axial    *-------------- outside surface ----------------*   *---------------- inside surface ----------------*
 node  location   heat flux   h.t.  **** temperatures (deg-f) ****   **** temperatures (deg-f) ****    h.t.   heat flux
 no.    (in.)     (b/h-ft2)   mode     wall     vapor     liquid       liquid     vapor     wall       mode   (b/h-ftl)
 ----  --------   ---------   ----   -------   -------   -------       -------   -------   -------

  51    144.00  -0.4424E+03   spl    629.71     653.31    629.77                           629.68           0.0000E+00
  \end{lstlisting}
    that will be converted in the following way (CSV):
    \begin{table}[h]
      \centering
      \caption{CSV transport info (cylindrical tube)}
      \label{CSVcylTube}
      \tabcolsep=0.11cm
      \tiny
      \begin{tabular}{|c|c|c|c|}
      time    & cylRod10\_surface1\_ax51\_outsideSurfaceHeatFlux  & cylRod10\_surface1\_ax51\_outsideSurfaceWallTemperature &  ...  \\
      0.00 & -0.4424E+03 & 629.71  & ...
      \end{tabular}
    \end{table}

  \item \textit{\textbf{heat slab (tube)}}

   \noindent \textcolor{red}{
   \textbf{Warning: CTF reports results for heat slabs based on the flow type. Not every result will be available depending on the \underline{internal} or \underline{external} flow type. Check output file and see if the flow around heat slab is internal or external. If the user requests values that are not in the output file reported values will be from different columns and wrong. For example there is no outside surface liquid temperature when flow is internal.} } \\

   \begin{lstlisting}[basicstyle=\tiny]
          heat slab no.  1  (tube)                  simulation time =   20.00 seconds
                                                    fluid channel on inside surface =  1
                                                    fluid channel on outside surface =  0
                                                    geometry type =  1
                                                    no. of nodes =   2


 ***************************************************************************************************************

   rod    axial    *------------- outside surface ------------* *-------------- inside surface ----------------*
   node  location   heat flux  h.t.  ** temperatures (deg-f) ** ** temperatures (deg-f) ****    h.t.   heat flux
   no.    (in.)     (b/h-ft2)  mode   wall     vapor     liquid liquid     vapor     wall       mode   (b/h-ftl)
   ----  --------   ---------  ----  ------   -------   ------- -------   -------   -------

    21     19.69    999666E+00       200.00                     212.00    212.00    247.85     tran    0.2641E+02
  \end{lstlisting}
    that will be converted in the following way (CSV):
    \begin{table}[h]
      \centering
      \caption{CSV transport info (heat slab (tube) tube)}
      \label{CSVheatSlab}
      \tabcolsep=0.11cm
      \tiny
      \begin{tabular}{|c|c|c|c|}
      time    & heatSlab1\_ax21\_outsideSurfaceHeatFlux &  heatSlab1\_ax21\_outsideSurfaceWallTemperature &  ...  \\
      0.00 & 999666E+00 & 200.00  & ...
      \end{tabular}
    \end{table}

 \item \textit{\textbf{CTF's Output Variables and Corresponding Names in CSV file}}

   In CSV file, the output results obtained from the CTF output file (.ctf.out) will be saved with the names as described in Table \ref{CSVvariableNames}.

\end{itemize}

\captionsetup{justification=centering}
\captionof{table}{Variables Name List in CSV File \hspace{\textwidth} \textcolor{red}{NN: Axial Node Number; CN: Channel Number; \\ RN: Rod Number; SN: Surface Number: HN: Heat Slab Number}}
\label{CSVvariableNames}
\begingroup
% header and footer information
\scriptsize
\begin{longtable}{|l|l|}
\hline
\textbf{Output Variable} & \textbf{Name in CSV file} \\
\hline
\endhead
\hline
\endfoot
% body of table
      \scriptsize {simulation time  }&\scriptsize{ time  } \\
      channels' average height  & AVG\_ch\_ax\textcolor{red}{NN}\_height \\
      channels' average quality  & AVG\_ch\_ax\textcolor{red}{NN}\_quality \\
      channels' average void fraction (liquid)  & AVG\_ch\_ax\textcolor{red}{NN}\_voidFractionLiquid \\
      channels' average void fraction (vapor)  & AVG\_ch\_ax\textcolor{red}{NN}\_voidFractionVapor \\
      channels' average entrainment (volumetric) fraction  & AVG\_ch\_ax\textcolor{red}{NN}\_volumeEntrainFraction \\
      channels' average mass flow rate (liquid)  & AVG\_ch\_ax\textcolor{red}{NN}\_massFlowRateLiquid \\
      channels' average mass flow rate (vapor)  & AVG\_ch\_ax\textcolor{red}{NN}\_massFlowRateVapor \\
      channels' average entrainment rate (mass flow rate)  & AVG\_ch\_ax\textcolor{red}{NN}\_massFlowRateEntrain \\
      channels' average mass flow rate (integrated)  & AVG\_ch\_ax\textcolor{red}{NN}\_massFlowRateIntegrated \\
      channels' average enthalpy increase (liquid)  & AVG\_ch\_ax\textcolor{red}{NN}\_enthalpyIncreaseLiquid \\
      channels' average enthalpy increase (vapor)  & AVG\_ch\_ax\textcolor{red}{NN}\_enthalpyIncreaseVapor \\
      channels' average enthalpy increase (integrated)  & AVG\_ch\_ax\textcolor{red}{NN}\_enthalpyIncreaseIntegrated \\
      channels' average mixture enthalpy  & AVG\_ch\_ax\textcolor{red}{NN}\_enthalpyMixture \\
      channels' average heat added to liquid & AVG\_ch\_ax\textcolor{red}{NN}\_heatAddedToLiquid \\
      channels' average heat added to vapor & AVG\_ch\_ax\textcolor{red}{NN}\_heatAddedToVapor \\
      channels' average heat added (integrated)  & AVG\_ch\_ax\textcolor{red}{NN}\_heatAddedIntegrated' \\
      channel height & ch\textcolor{red}{CN}\_ax\textcolor{red}{NN}\_height \\
      channel pressure & ch\textcolor{red}{CN}\_ax\textcolor{red}{NN}\_pressure \\
      channel liquid velocity  & ch\textcolor{red}{CN}\_ax\textcolor{red}{NN}\_velocityLiquid \\
      channel vapor velocity  & ch\textcolor{red}{CN}\_ax\textcolor{red}{NN}\_velocityVapor \\
      channel entrainment rate (velocity) & ch\textcolor{red}{CN}\_ax\textcolor{red}{NN}\_velocityEntrain \\
      channel void fraction (liquid) & ch\textcolor{red}{CN}\_ax\textcolor{red}{NN}\_voidFractionLiquid \\
      channel void fraction (vapor)  & ch\textcolor{red}{CN}\_ax\textcolor{red}{NN}\_voidFractionVapor \\
      channel volume fraction of entrainment liquid  & ch\textcolor{red}{CN}\_ax\textcolor{red}{NN}\_volumeEntrainFraction \\
      channel mass flow rate (liquid) & ch\textcolor{red}{CN}\_ax\textcolor{red}{NN}\_massFlowRateLiquid \\
      channel mass flow rate (vapor)  & ch\textcolor{red}{CN}\_ax\textcolor{red}{NN}\_massFlowRateVapor \\
      channel entrainment rate (mass flow rate) & ch\textcolor{red}{CN}\_ax\textcolor{red}{NN}\_massFlowRateEntrain \\
      channel flow regime ID & ch\textcolor{red}{CN}\_ax\textcolor{red}{NN}\_flowRegimeID \\
      channel heat added to liquid & ch\textcolor{red}{CN}\_ax\textcolor{red}{NN}\_heatAddedToLiquid \\
      channel heat added to vapor  & ch\textcolor{red}{CN}\_ax\textcolor{red}{NN}\_heatAddedToVapor \\
      channel evaporation rate  & ch\textcolor{red}{CN}\_ax\textcolor{red}{NN}\_evaporationRate \\
      channel enthalpy of vapor  & ch\textcolor{red}{CN}\_ax\textcolor{red}{NN}\_enthalpyVapor \\
      channel enthalpy of saturated vapor & ch\textcolor{red}{CN}\_ax\textcolor{red}{NN}\_enthalpySaturatedVapor \\
      channel enthalpy difference between vapor and saturated vapor & ch\textcolor{red}{CN}\_ax\textcolor{red}{NN}\_enthalpyVapor-SaturatedVapor \\
      channel enthalpy of liquid & ch\textcolor{red}{CN}\_ax\textcolor{red}{NN}\_enthalpyLiquid \\
      channel enthalpy of saturated liquid & ch\textcolor{red}{CN}\_ax\textcolor{red}{NN}\_enthalpySaturatedLiquid \\
      channel enthalpy difference between liquid and saturated liquid & ch\textcolor{red}{CN}\_ax\textcolor{red}{NN}\_enthalpyLiquid-SaturatedLiquid \\
      channel enthalpy of mixture & ch\textcolor{red}{CN}\_ax\textcolor{red}{NN}\_enthalpyMixture \\
      channel density of liquid  & ch\textcolor{red}{CN}\_ax\textcolor{red}{NN}\_densityLiquid \\
      channel density of vapor  & ch\textcolor{red}{CN}\_ax\textcolor{red}{NN}\_densityVapor \\
      channel density of mixture & ch\textcolor{red}{CN}\_ax\textcolor{red}{NN}\_densityMixture \\
      channel net entrainment rate & \\ (difference between entrainment rate and de-entrainment rate)  & ch\textcolor{red}{CN}\_ax\textcolor{red}{NN}\_netEntrainRate \\
      % gas volumetric analysis
      channel enthalpy of the mixture of non-condensable gases & ch\textcolor{red}{CN}\_ax\textcolor{red}{NN}\_enthalpyNonCondensableMixture \\
      channel density of the mixture of non-condensable gases & ch\textcolor{red}{CN}\_ax\textcolor{red}{NN}\_densityNonCondensableMixture \\
      channel steam volume fraction [0-100] & ch\textcolor{red}{CN}\_ax\textcolor{red}{NN}\_volumeFractionSteam \\
      channel air volume fraction [0-100] & ch\textcolor{red}{CN}\_ax\textcolor{red}{NN}\_volumeFractionAir \\
      channel total equivalent diameter of the liquid droplets & \\ (all droplets as a single big one) (diam-ld) & ch\textcolor{red}{CN}\_ax\textcolor{red}{NN}\_equiDiameterLiquidDroplet \\
      channel averaged diameter of liquid droplets field (diam-sd) & ch\textcolor{red}{CN}\_ax\textcolor{red}{NN}\_avgDiameterLiquidDroplet \\
      channel averaged flow rate of liquid droplets field (flow-sd) & ch\textcolor{red}{CN}\_ax\textcolor{red}{NN}\_avgFlowRateLiquidDroplet \\
      channel averaged velocity of liquid droplets field (veloc-sd) & ch\textcolor{red}{CN}\_ax\textcolor{red}{NN}\_avgVelocityLiquidDroplet \\
      channel evaporation rate of liquid droplets field (gamsd) & ch\textcolor{red}{CN}\_ax\textcolor{red}{NN}\_evaporationRateLiquidDroplet \\
      % fuel rod
      fuel rod height & fuelRod\textcolor{red}{RN}\_surface\textcolor{red}{SN}\_ax\textcolor{red}{NN}\_height \\
      fuel rod fluid temperatures (liquid) & fuelRod\textcolor{red}{RN}\_surface\textcolor{red}{SN}\_ax\textcolor{red}{NN}\_fluidTemperatureLiquid \\
      fuel rod fluid temperatures (vapor)  & fuelRod\textcolor{red}{RN}\_surface\textcolor{red}{SN}\_ax\textcolor{red}{NN}\_fluidTemperatureVapor \\
      fuel rod surface heat flux & fuelRod\textcolor{red}{RN}\_surface\textcolor{red}{SN}\_ax\textcolor{red}{NN}\_surfaceHeatflux \\
      %fuel rod surface heat transfer mode [-] & fuelRod\textcolor{red}{(Rod Node Number)}_surfaceN\textcolor{red}{SN}_ax\textcolor{red}{NN}\_heatTransferMode \\
      clad outer surface temperature  & fuelRod\textcolor{red}{RN}\_surface\textcolor{red}{SN}\_ax\textcolor{red}{NN}\_cladOutTemperature \\
      clad iRNer surface temperature  & fuelRod\textcolor{red}{RN}\_surface\textcolor{red}{SN}\_ax\textcolor{red}{NN}\_cladInTemperature \\
      gap conductance  & fuelRod\textcolor{red}{RN}\_surface\textcolor{red}{SN}\_ax\textcolor{red}{NN}\_gapConductance \\
      fuel outer suface temperature  & fuelRod\textcolor{red}{RN}\_surface\textcolor{red}{SN}\_ax\textcolor{red}{NN}\_fuelTemperatureSurface \\
      fuel center temperature  & fuelRod\textcolor{red}{RN}\_surface\textcolor{red}{SN}\_ax\textcolor{red}{NN}\_fuelTemperatureCenter \\
      % cylindrical rod
      cylindrical tube height & cylRod\textcolor{red}{RN}\_surface\textcolor{red}{SN}\_ax\textcolor{red}{NN}\_height \\
      cylindrical tube outside surface heat flux & cylRod\textcolor{red}{RN}\_surface\textcolor{red}{SN}\_ax\textcolor{red}{NN}\_outsideSurfaceHeatFlux \\
      cylindrical tube outside surface wall temperature & cylRod\textcolor{red}{RN}\_surface\textcolor{red}{SN}\_ax\textcolor{red}{NN}\_outsideSurfaceWallTemperature \\
      cylindrical tube outside surface vapor temperature & cylRod\textcolor{red}{RN}\_surface\textcolor{red}{SN}\_ax\textcolor{red}{NN}\_outsideSurfaceVaporTemperature \\
      cylindrical tube outside surface liquid temperature & cylRod\textcolor{red}{RN}\_surface\textcolor{red}{SN}\_ax\textcolor{red}{NN}\_outsideSurfaceLiquidTemperature \\
      cylindrical tube inside surface wall temperature & cylRod\textcolor{red}{RN}\_surface\textcolor{red}{SN}\_ax\textcolor{red}{NN}\_insideSurfaceWallTemperature \\
      cylindrical tube inside surface vapor temperature & cylRod\textcolor{red}{RN}\_surface\textcolor{red}{SN}\_ax\textcolor{red}{NN}\_insideSurfaceVaporTemperature \\
      cylindrical tube inside surface liquid temperature & cylRod\textcolor{red}{RN}\_surface\textcolor{red}{SN}\_ax\textcolor{red}{NN}\_insideSurfaceLiquidTemperature \\
      cylindrical tube inside surface heat flux & cylRod\textcolor{red}{RN}\_surface\textcolor{red}{SN}\_ax\textcolor{red}{NN}\_insideSurfaceHeatFlux \\
      % heat slab
      heat slab (tube) height & heatSlab\textcolor{red}{HN}\_ax\textcolor{red}{NN}\_height \\
      heat slab (tube) outside surface heat flux & heatSlab\textcolor{red}{HN}\_ax\textcolor{red}{NN}\_outsideSurfaceHeatFlux\\
      heat slab (tube) outside surface wall temperature & heatSlab\textcolor{red}{HN}\_ax\textcolor{red}{NN}\_outsideSurfaceWallTemperature\\
      heat slab (tube) outside surface vapor temperature & heatSlab\textcolor{red}{HN}\_ax\textcolor{red}{NN}\_outsideSurfaceVaporTemperature\\
      heat slab (tube) outside surface liquid temperature & heatSlab\textcolor{red}{HN}\_ax\textcolor{red}{NN}\_outsideSurfaceLiquidTemperature\\
      heat slab (tube) inside surface wall temperature & heatSlab\textcolor{red}{HN}\_ax\textcolor{red}{NN}\_insideSurfaceWallTemperature\\
      heat slab (tube) inside surface vapor temperature & heatSlab\textcolor{red}{HN}\_ax\textcolor{red}{NN}\_insideSurfaceVaporTemperature\\
      heat slab (tube) inside surface liquid temperature & heatSlab\textcolor{red}{HN}\_ax\textcolor{red}{NN}\_insideSurfaceLiquidTemperature\\
      heat slab (tube) inside surface heat flux & heatSlab\textcolor{red}{HN}\_ax\textcolor{red}{NN}\_insideSurfaceHeatFlux
\end{longtable}
\endgroup

\textbf{\textit{\nb RAVEN, regonizes failed or crashed CTF runs and no data will be saved from those.}}

%%%%%%%%%%%%%%%%%%%%%%%%%%%%%%%%%%%%%%%%%%%%%%%%%%%%%%%%%
\subsubsection{Distributions}
%%%%%%%%%%%%%%%%%%%%%%%%%%%%%%%%%%%%%%%%%%%%%%%%%%%%%%%%%
The \xmlNode{Distribution} block defines the distributions that are going
to be used for the sampling of the variables defined in the \xmlNode{Samplers} block.
%
For all the possibile distributions and all their possible inputs please see the
chapter about Distributions (see~\ref{sec:distributions}).
%
Here we report an example of a Normal distribution:

\begin{lstlisting}[style=XML,morekeywords={name,debug}]
<Distributions verbosity='debug'>
    <Normal name="GridLossCoeff">
      <mean>0.7</mean>
      <sigma>0.1</sigma>
      <upperBound>0.9</upperBound>
      <lowerBound>0.6</lowerBound>
    </Normal>
    <Uniform name="DB1dist">
      <upperBound>0.026</upperBound>
      <lowerBound>0.020</lowerBound>
    </Uniform>
    <Uniform name="DB2dist">
      <upperBound>0.9</upperBound>
      <lowerBound>0.7</lowerBound>
    </Uniform>
    <Uniform name="DB3dist">
      <upperBound>0.5</upperBound>
      <lowerBound>0.3</lowerBound>
    </Uniform>
 </Distributions>
\end{lstlisting}

\noindent
It is good practice to name the distribution something similar to what kind of
variable is going to be sampled, since there might be many variables with the
same kind of distributions but different input parameters.

%%%%%%%%%%%%%%%%%%%%%%%%%%%%%%%%%%%%%%%%%%%%%%%%%%%%%%%%%
\subsubsection{Samplers}
%%%%%%%%%%%%%%%%%%%%%%%%%%%%%%%%%%%%%%%%%%%%%%%%%%%%%%%%%
In the \xmlNode{Samplers} block we want to define the variables that are going to be sampled.

\noindent The perturbation or optimization of the input of any CTF sequence is performed using the approach detailed in the \textit{Generic Interface} section (see \ref{subsec:genericInterface}). Briefly, this approach uses
 ``wild-cards'' (placed in the original input files) for injecting the perturbed values.
 For example, if the original input file (that needs to be perturbed) is the following:

\textbf{Example}:
We want to do the sampling of 1 single variable:
\begin{itemize}
  \item The Grid Loss Coefficient Data is used from sampled values.
\end{itemize}

\noindent We are going to sample this variable using two different sampling methods: Grid and MonteCarlo.
The RAVEN input is then written as follows:

\begin{lstlisting}[style=XML,morekeywords={name,type,construction,lowerBound,steps,limit,initialSeed}]
<Samplers verbosity='debug'>
  <Grid name='Grid_Sampler' >
    <variable name='GrdLss'>
      <distribution>GridLossCoeff</distribution>
      <grid type='CDF' construction='equal'  steps='10'>0.001 0.999</grid>
    </variable>
  </Grid>
  <MonteCarlo name='MC_Sampler'>
     <samplerInit>
       <limit>1000</limit>
     </samplerInit>
    <variable name='GrdLss'>
      <distribution>GridLossCoeff</distribution>
    </variable>
    <variable name='DB1'>
      <distribution>DB1dist</distribution>
    </variable>
    <variable name='DB2'>
      <distribution>DB2dist</distribution>
    </variable>
    <variable name='DB3'>
      <distribution>DB3dist</distribution>
    </variable>
  </MonteCarlo>
</Samplers>
\end{lstlisting}

CTF input file should be modified with wild-cards in the following way.
\begin{lstlisting}[basicstyle=\tiny]
***********************************************************************************************
*GROUP 7 - Grid Loss Coefficient Data                                                         *
***********************************************************************************************
**NGR
    7
*Card 7.1
**  NCD NGT  IFGQF IFSDRP  IFESPV  IFTPE  IGTEMP  NFBS  NDM9 NDM10 NDM11 NDM12 NDM13 NDM14
     21   0      0      0       0      0       0     0     0     0     0     0     0     0
*Card 7.2
**       CDL      J   CD1   CD2   CD3   CD4   CD5   CD6   CD7   CD8   CD9  CD10  CD11  CD12
 $RAVEN-GrdLss$   1     1     2     3     4     5     6     7     8     9    10    11    12
     0.90700      1    13    14    15    16    17    18    19    20    21    22    23    24
     0.90700      1    25    26    27    28    29    30    31    32    33    34    35    36

\end{lstlisting}

It is also possible to modify input values in parameter exposure input files.\\

\textbf{Example}:
We want to do the sampling of 3 correlation parameters (Dittus-Boelter parameter modification, DB1 $\times$ Re$^{DB2}$ $\times$ Pr$^{DB3}$):

\begin{itemize}
  \item DB1, DB2, DB3 values will be sampled and vuq\_param.txt will be modified with sampled values.
\end{itemize}

vuq$\_$param.txt and vuq$\_$mult.txt files are modified similarly with defined variable names.

\begin{lstlisting}[basicstyle=\tiny]
k_chen_1 = 0.24
k_chen_2 = 0.75
k_db_1 = $RAVEN-DB1$
k_db_2 = $RAVEN-DB2$
k_db_3 = $RAVEN-DB3$
k_db_4 = 7.86
k_wf_1 = 1.691
k_wf_2 = 0.43
\end{lstlisting}

\noindent It can be seen that each variable is connected with a proper distribution
defined in the \\\xmlNode{Distributions} block (from the previous example).

%%%%%%%%%%%%%%%%%%%%%%%%%%%%%%%%%%%%%%%%%%%%%%%%%%%%%%%%%%%
\subsubsection{Steps}
For a CTF interface, the \xmlNode{MultiRun} step type will most likely be
used. But \xmlNode{SingleRun} step can also be used for plotting and data extraction purposes.
%
First, the step needs to be named: this name will be one of the names used in
the \xmlNode{sequence} block.
%
In our example, \texttt{Grid\_Sampler} and \texttt{MC\_Sampler}.
%
\begin{lstlisting}[style=XML,morekeywords={name,debug,re-seeding}]
     <MultiRun name='Grid_Sampler' verbosity='debug'>
\end{lstlisting}

With this step, we need to import all the files needed for the simulation:
\begin{itemize}
  \item CTF input files
\end{itemize}
\begin{lstlisting}[style=XML,morekeywords={name,class,type}]
    <Input class="Files" type="ctf">CTFinput</Input>
    <Input class="Files" type="vuq_param">vuq_param</Input>
    <Input class="Files" type="vuq_mult" >vuq_mult</Input>
\end{lstlisting}
We then need to define which model will be used:
\begin{lstlisting}[style=XML]
    <Model  class='Models' type='Code'>MyCobraTF</Model>
\end{lstlisting}
We then need to specify which Sampler is used, and this can be done as follows:
\begin{lstlisting}[style=XML]
    <Sampler class='Samplers' type='Grid'>Grid_Sampler</Sampler>
\end{lstlisting}
And lastly, we need to specify what kind of output the user wants.
%
For example the user might want to make a database (in RAVEN the database
created is an HDF5 file).
%
Here is a classical example:
\begin{lstlisting}[style=XML,morekeywords={class,type}]
    <Output  class='Databases' type='HDF5'>Grid_out</Output>
\end{lstlisting}
Following is the example of two MultiRun steps which use different sampling
methods (Grid and Monte Carlo), and creating two different databases for each
one:
\begin{lstlisting}[style=XML]
<Steps verbosity='debug'>
  <MultiRun name='Grid_Sampler' verbosity='debug'>
    <Input   class='Files' type="ctf">CTFinput</Input>
    <Input   class="Files" type="vuq_param">vuq_param</Input>
    <Input   class="Files" type="vuq_mult" >vuq_mult</Input>
    <Model   class='Models'    type='Code'>MyCobraTF</Model>
    <Sampler class='Samplers'  type='Grid'>Grid_Sampler</Sampler>
    <Output  class='Databases' type='HDF5'>Grid_out</Output>
    <Output  class='DataObjects' type='PointSet'   >GridCTFPointSet</Output>
    <Output  class='DataObjects' type='HistorySet'>GridCTFHistorySet</Output>
  </MultiRun>
  <MultiRun name='MC_Sampler' verbosity='debug' re-seeding='210491'>
    <Input   class='Files' type=''>CTFinput</Input>
    <Input   class="Files" type="">vuq_param</Input>
    <Input   class="Files" type="" >vuq_mult</Input>
    <Model   class='Models'    type='Code'>MyCobraTF</Model>
    <Sampler class='Samplers'  type='MonteCarlo'>MC_Sampler</Sampler>
    <Output  class='Databases' type='HDF5'      >MC_out</Output>
    <Output  class='DataObjects' type='PointSet'   >MonteCarloCobraPointSet</Output>
    <Output  class='DataObjects' type='HistorySet'>MonteCarloCobraHistorySet</Output>
  </MultiRun>
</Steps>
\end{lstlisting}

%%%%%%%%%%%%%%%%%%%%%%%%%%%%%%%%%%%%%%%%%%%%%%%%%%%%%%
%%%%%%%%%%%%%%%%% Saphire Interface %%%%%%%%%%%%%%%%%%
%%%%%%%%%%%%%%%%%%%%%%%%%%%%%%%%%%%%%%%%%%%%%%%%%%%%%%
\subsection{SAPHIRE Interface}
\label{subsec:saphireInterface}
This section covers the input specification for running SAPHIRE through RAVEN. It is important to notice
that this short explanation assumes that the reader already knows how to use SAPHIRE.

\subsubsection{Files}
In the \xmlNode{Files} section, as specified before, all the files needed for the code to
run should be specified. In the case of SAPHIRE, the files typically needed are the following:
\begin{itemize}
  \item SAPHIRE compressed project inputs with file extension `.zip';
  \item SAPHIRE macro input file with file extension `.mac'.
\end{itemize}

Example:
\begin{lstlisting}[style=XML]
  <Files>
    <Input name="macro" type="">changeSet.mac</Input>
    <Input name="saphireInput" type="">saphireInput.zip</Input>
  </Files>
\end{lstlisting}

%%%%%%%%%%%%%%%%%%%%%%%%%%%%%%%%%%%%%%%%%%%%%%%%%%%%%X
\subsubsection{Models}
In the \xmlNode{Models} block SAPHIRE executable needs to be specified. Here is a standard example of what
can be used:
\begin{lstlisting}[style=XML]
  <Models>
    <Code name="saphire" subType="Saphire">
      <executable>"C:\Saphire 8\tools\SAPHIRE.exe"</executable>
      <clargs arg="macro" extension=".mac" type="input" delimiter="="/>
      <clargs arg="project" extension=".zip" type="input" delimiter="="/>
      <outputFile>fixed_output.csv</outputFile>
      <codeOutput type="uncertainty">et_uq.csv</codeOutput>
      <codeOutput type="uncertainty">ft_uq.csv</codeOutput>
    </Code>
  </Models>
\end{lstlisting}

The \xmlNode{Code} XML node contains the information needed to execute the specific External Code. This
XML node accepts the following attributes:
\begin{itemize}
  \item \xmlAttr{name}, \xmlDesc{required string attribute}, user-defined identifier of this model.
    \nb As with other objects, this identifier can be used to reference this specific entity from other input
    blocks in the XML.
  \item \xmlAttr{subType}, \xmlDesc{required string attribute}, specifies the code that needs to be
    associated to this Model.
\end{itemize}
This model can be initialized with the following children:
\begin{itemize}
  \item \xmlNode{executable}, \xmlDesc{string, required field}, specifies the path of the executable to
    be used; \nb Either an absolute or relative path can be used.
  \item \xmlNode{clargs}, \xmlDesc{string, required field}, allows addition of command-line arguments to
    the execution command. This node is used to specify the input files that are required by SAPHIRE.
    This node accepts the following attributes:
    \begin{itemize}
      \item \xmlAttr{type}, \xmlDesc{require string attribute}, specifies the type of command-line argument
        to add. The current option is \xmlString{input}
      \item \xmlAttr{arg}, \xmlDesc{string, required field} specifies the flag to be used before the entry.
      \item \xmlAttr{extension}, \xmlDesc{string, required field}, specifies the type of file extension
        to use. This links the \xmlNode{Input} file in the \xmlNode{Steps} to this location in the execution
        command. Currently only accepts `.zip' and `.mac'.
      \item \xmlAttr{delimiter}, \xmlDesc{string, required field}, uses to link the \xmlAttr{arg} and the
        \xmlNode{Input} with the extension given by \xmlAttr{extension}
    \end{itemize}
    \nb As shown in previous example, the following command will be generated:
    \begin{lstlisting}
      "C:\Saphire 8\tools\SAPHIRE.exe" project=path/to/saphireInput.zip macro=path/to/changeSet.mac
    \end{lstlisting}
  \item \xmlNode{outputFile}, \xmlDesc{string, optional field}, uses to specify the output file name (CSV only). In this case, the code
    interface always produce a CSV file named ``fixed\_output.csv''.
  \item \xmlNode{codeOutput}, \xmlDesc{string, required field}, uses to specify output file generated by SAPHIRE that will be processed
    via the code interface. The following attributes can be specified:
    \begin{itemize}
      \item \xmlAttr{type}, \xmlDesc{required string attribute}, the actual type of the provided file. The
        only type accepted here is \xmlString{uncertainty}
    \end{itemize}
\end{itemize}

In this example, two output files ``eq\_uq.csv'' amd ``ft\_uq.csv'' will be processed by the SAPHIRE code
interface, and the results will be saved in output file with name ``fixed\_output.csv''.

%%%%%%%%%%%%%%%%%%%%%%%%%%%%%%%%%%%%%%%%%%%%%%%%%%%%%%
\subsubsection{Distributions}
The \xmlNode{Distributions} block defines the distributions that are going to be used for the sampling
of the variables defined in the \xmlNode{Samplers} block. For all the possible distributions and all
their possible inputs, please see the chapter about Distributions (see~\ref{sec:distributions}).
%
Here we give a general example:
\begin{lstlisting}[style=XML]
  <Distributions>
    <Normal name="allEvents">
        <mean>0.1</mean>
        <sigma>0.025</sigma>
        <lowerBound>0.05</lowerBound>
        <upperBound>0.15</upperBound>
    </Normal>
    <Normal name="mov1Event">
        <mean>0.5</mean>
        <sigma>0.1</sigma>
        <lowerBound>0.3</lowerBound>
        <upperBound>0.8</upperBound>
    </Normal>
    <Normal name="single1">
        <mean>0.2</mean>
        <sigma>0.05</sigma>
        <lowerBound>0.1</lowerBound>
        <upperBound>0.3</upperBound>
    </Normal>
  </Distributions>
\end{lstlisting}

It is good practice to name the distribution similar to what kind of variable is going to be sampled.
%
%%%%%%%%%%%%%%%%%%%%%%%%%%%%%%%%%%%%%%%%%%%%%%%%%%%%%%
\subsubsection{Samplers}
In the \xmlNode{Samplers} block we want to define the variables that are going to be sampled.
The perturbation of the input of SAPHIRE MACRO is performed using the approach detailed in the
\textit{Generic Interface} section (see \ref{subsec:genericInterface}). This approach uses the
``wild-cards'' (placed in the original input files) for injecting the perturbed values. For example,
if one wants to perturb the event tree probabilities of the original input file, i.e.
\begin{lstlisting}[style=XML]
<change set>
  <unmark></unmark>
  <delete>
    <name>ALL-EVENTS</name>
  </delete>
  <add>
    <name>ALL-EVENTS</name>
    <description>Class change all events Change Set</description>
    <class>
      <event name>*</event name>
      <suscept>1</suscept>
      <probability>1.E-2</probability>
    </class>
  </add>
  <mark name>ALL-EVENTS</mark name>
  <generate></generate>
</change set>
...
<change set>
  <unmark></unmark>
  <delete>
    <name>MOV-1-EVENTS</name>
  </delete>
  <add>
    <name>MOV-1-EVENTS</name>
    <description>Class change subset events Change Set</description>
    <class>
      <event name>?-MOV-CC-1</event name>
      <calc type>1</calc type>
      <probability>5E-3</probability>
    </class>
  </add>
    <mark name>MOV-1-EVENTS</mark name>
    <generate></generate>
</change set>
...

<change set>
  <unmark></unmark>
  <delete>
    <name>SINGLE-1</name>
  </delete>
  <add>
    <name>SINGLE-1</name>
    <description>Single Event Change Set</description>
    <single>
      <event name>E-MOV-CC-A</event name>
      <calc type>1</calc type>
      <probability>4E-1</probability>
    </single>
  </add>
  <mark name>SINGLE-1</mark name>
  <generate></generate>
</change set>
\end{lstlisting}

One need to use the RAVEN ``wild-cards`` to inject the perturbed values, i.e.
\begin{lstlisting}[style=XML]
<change set>
  <unmark></unmark>
  <delete>
    <name>ALL-EVENTS</name>
  </delete>
  <add>
    <name>ALL-EVENTS</name>
    <description>Class change all events Change Set</description>
    <class>
      <event name>*</event name>
      <suscept>1</suscept>
      <probability>$RAVEN-allEventsPb$</probability>
    </class>
  </add>
  <mark name>ALL-EVENTS</mark name>
  <generate></generate>
</change set>
...
<change set>
  <unmark></unmark>
  <delete>
    <name>MOV-1-EVENTS</name>
  </delete>
  <add>
    <name>MOV-1-EVENTS</name>
    <description>Class change subset events Change Set</description>
    <class>
      <event name>?-MOV-CC-1</event name>
      <calc type>1</calc type>
      <probability>$RAVEN-mov1EventPb$</probability>
    </class>
  </add>
    <mark name>MOV-1-EVENTS</mark name>
    <generate></generate>
</change set>
...

<change set>
  <unmark></unmark>
  <delete>
    <name>SINGLE-1</name>
  </delete>
  <add>
    <name>SINGLE-1</name>
    <description>Single Event Change Set</description>
    <single>
      <event name>E-MOV-CC-A</event name>
      <calc type>1</calc type>
      <probability>$RAVEN-single1Pb$</probability>
    </single>
  </add>
  <mark name>SINGLE-1</mark name>
  <generate></generate>
</change set>
\end{lstlisting}

The RAVEN \xmlNode{Samplers} input will be

\textbf{Example}:
\begin{lstlisting}[style=XML]
  <Samplers>
    <MonteCarlo name="mcSaphire">
        <samplerInit>
            <limit>2</limit>
        </samplerInit>
        <variable name="allEventsPb">
            <distribution>allEvents</distribution>
        </variable>
        <variable name="mov1EventPb">
            <distribution>mov1Event</distribution>
        </variable>
        <variable name="single1Pb">
            <distribution>single1</distribution>
        </variable>
    </MonteCarlo>
\end{lstlisting}

%%%%%%%%%%%%%%%%%%%%%%%%%%%%%%%%%%%%%%%%%%%%%%%%%%%%%%
\subsubsection{Steps}
In this section, the \xmlNode{MultiRun} will be used. As shown in the following, two SAPHIRE
input files listed in \xmlNode{Files} are linked here using \xmlNode{Input}, the \xmlNode{Model}
and \xmlNode{Sampler} defined in previous sections will be used in this \xmlNode{MultiRun}. The
outputs will be saved in the \textbf{DataObject} ``saphireDump'', and will be printed via
\textbf{OutStreams}.

\begin{lstlisting}[style=XML]
  <Steps>
    <MultiRun name="sample">
      <Input class="Files" type="">macro</Input>
      <Input class="Files" type="">saphireInput</Input>
      <Model class="Models" type="Code">saphire</Model>
      <Sampler class="Samplers" type="MonteCarlo">mcSaphire</Sampler>
      <Output class="DataObjects" type="PointSet">saphireDump</Output>
      <Output class="OutStreams" type="Print">saphirePrint</Output>
    </MultiRun>
  </Steps>
\end{lstlisting}
%%%%%%%%%%%%%%%%%%%%%%%%%%%%%%%%%%%%%%%%%%%%%%%%%
%%%%%%%%%%%%% PHISICS INTERFACE %%%%%%%%%%%%%%%%%
%%%%%%%%%%%%%%%%%%%%%%%%%%%%%%%%%%%%%%%%%%%%%%%%%
\subsection{PHISICS Interface}
\label{subsec:PhisicsInterface}
%
\subsubsection{General Information}
%
This section covers the input specification for running PHISICS through RAVEN.
The interface can be used to perturb the PHISICS input files including the INSTANT and MRTAU input decks
and the following libraries: cross sections, fission yield, decay, fission Q-values, decay Q-values and the XML material input.
The interface also the cabablity to work in MRTAU standlone calculations, in INSTANT/MRTAU mode (PHISICS) and
in PHISICS/RELAP5 coupled mode (see~\ref{subsec:PhisicsRelap5Interface}).
%
\subsubsection{Files}
\label{subsubsec:PhisicsInterfaceFiles}
\xmlNode{Files} includes two attributes \xmlAttr{name} and \xmlAttr{type} entries, identically as other interfaces.
It also includes two optional attributes \xmlAttr{perturbable} and \xmlAttr{subDirectory}.

The \xmlAttr{name} attribute is a user-defined internal name for the file contained in the node.
Default: None (required entry).

The \xmlAttr{type} attribute identifies which base-level parser an input file is used within.  The \xmlAttr{type} has to be specified as long as the
file is parsed by the interface or an interface's parser. \xmlAttr{type} is hardcoded for this speciific inputs, in order to assign each input to its corresponding RAVEN parser.
Default: None (required entry if parsed).

The corresponding hardcoded flags accepted by RAVEN are given in Table~\ref{TypeTable}. The type attributes are case-incensitive.
\begin{table}[]
  \centering
  \caption{Correspondance between the type attributes required and the PHISICS input files}\label{TypeTable}
    \begin{tabular}{l|l|l}
\textit{Type Attribute} & \textit{Corresponding PHISICS input}                           & \textit{Perturbable} \\
\hline
decay                   & MrTau decay library                          & Yes                  \\
inp                     & XML Instant input                            & No                   \\
path                    & XML MrTau path-to-libraries file             & No                   \\
material                & XML MrTau material input                     & Yes                  \\
depletion\_input        & XML MrTau depletion input                    & No                   \\
Xs-Library              & XML MrTau library input                      & No                   \\
FissionYield            & MrTau fission yield library                  & Yes                  \\
FissQValue              & MrTau fission Q-values                       & Yes                  \\
AlphaDecay              & MrTau $\alpha$ decay library                 & Yes                  \\
Beta+Decay              & MrTau $\beta^+$ decay library                & Yes                  \\
Beta+xDecay             & MrTau $\beta^{+*}$ decay library             & Yes                  \\
BetaDecay               & MrTau $\beta$ decay library                  & Yes                  \\
BetaxDecay              & MrTau $\beta^*$ decay library                & Yes                  \\
IntTraDecay             & MrTau internal transition decay library      & Yes                  \\
XS                      & XML XS scaling factors file                  & Yes                  \\
N,2N                    & MrTau n,2n library                           & No                   \\
N,ALPHA                 & MrTau n,$\alpha$ library                     & No                   \\
N,G                     & MrTau n,$\gamma$ library                     & No                   \\
N,Gx                    & MrTau n,$\gamma^*$ library                   & No                   \\
N,P                     & MrTau n,proton library                       & No                   \\
budep                   & MrTau burn-up history                        & No                   \\
CRAM\_coeff\_PF         & MrTau CRAM coefficients                      & No                   \\
IsotopeList             & MrTau isotope list input                     & No                   \\
mass                    & MrTau mass input                             & No                   \\
tabMap             & XML tabulation mapping                       & No
    \end{tabular}
\end{table}

The cross section libraries files can be defined with any \xmlAttr{type} attribute.
The tabulation mapping file is optional. If a \xmlAttr{type}=\xmlString{tabMap} is found in an \xmlNode{Input} node, the cross section
parser will be based on the tabulation points provided in the tabulation mapping file. If no tabulation mapping file is provided,
the cross section parser will print the perturbed cross section in one single tabulation point.
The \xmlAttr{perturbable} attribute indicates whether the input file can be perturbed. It is an optional boolean attribute.
Default: False

The \xmlAttr{subDirectory} indicates the subdirectory to which RAVEN search an input file. It is an optional attribute.
Default: . / (relative path of the working directory)

The \xmlString{Input File} string is the user-defined input file name.
The XML file specifying the library input paths corresponding to the decay, fission yields, the Q-values and the Mrtau standalone inputs will be automatically populated according to the user-input file names.
The Instant-MrTau input files material, library, instant control, library path and depletion are also user-defined in the input section. The user does not have to use the default file names.
Example:
\begin{lstlisting}[style=XML]
<File>
  <Input name="path"  type="path"  perturbable="False"                    >pathMrTau.xml</Input>
  <Input name="dec"   type="decay" perturbable="True" subDirectory="libF" >decayLibrary.dat</Input>
  <Input name="input" type="inp"   perturbable="False"                    >inpInstant.xml</Input>
</File>
\end{lstlisting}
in the example, the path-to-MrTau-libraries is pointed by the \xmlAttr{type}="path". The file name associated to the path-to-MrTau-libraries file is then user-defined as \xmlString{pathMrTau.xml}.
It is located in the working directory and it cannot be perturbed.

The decay library is pointed by the \xmlAttr{type}="decay". The file name associated to the decay library is then user-defined as \xmlString{decayLibrary.dat}.
The decay library is located at the relative path "libF". This path and name will be populated in the \xmlString{pathMrTau.xml} file automatically. The decay library can be perturbed.

The Instant XML input is pointed by the \xmlAttr{type}="inp". The file name associated to the Instant input is then user-defined as \xmlString{inpInstant.xml}.
it is located in the working directory and it cannot be perturbed.
%
%%%%%%%%%%%%%%%%%%%%%%%%%%%%%%%%%%%%%%%%%%%%%%%%%%
\subsubsection{Models}
\label{subsubsection:PhisicsModel}
The user provides paths to executables for sampled variables within the \xmlNode{Models} block.

The \xmlNode{Code} block will contain attributes \xmlAttr{name} and
\xmlAttr{subType}. The \xmlAttr{name} identifies that particular \xmlNode{Code} model within RAVEN, and
\xmlAttr{subType} specifies which code interface the model will use.

The \xmlNode{executable} block contains the absolute or relative path (with respect to the current working directory)
to PHISICS that RAVEN will use to run generated input files.

The \xmlNode{mrtauStandAlone} node informs whether or not MrTau is ran in standalone mode. The
\xmlNode{mrtauStandAlone} accepts only a boolean entry (\xmlString{true}, \xmlString{t}, \xmlString{false}
,\xmlString{f}). It is case insensitive.
Default: false.
If \xmlNode{mrtauStandAlone} is false, a coupled INSTANT+MrTau calculation is ran, using the Phisics executable.

The \xmlNode{printSpatialRR} node indicates if the spatial reaction rates computed by PHISICS are to be included
in the RAVEN csv output. The entry is case insensitive.
. If False, the total reaction rates are printed instead. Default: False (spatial reaction rate not printed).

The \xmlNode{printSpatialFlux} node indicates if the spatial neutron fluxes computed by PHISICS are to be included in the RAVEN csv output. The entry is case insensitive. Default: False (spatial fluxes not printed).

An example of the \xmlNode{Models} block is given below:

\begin{lstlisting}[style=XML]
  <Models>
    <Code name="PHISICS" subType="Phisics">
       <executable>./path/to/instant/executable</executable>
       <mrtauStandAlone>F</mrtauStandAlone>
       <printSpatialRR>F</printSpatialRR>
       <printSpatialFlux>T</printSpatialFlux>
    </Code>
  </Models>
\end{lstlisting}
In the example, note that because \xmlNode{mrtauStandAlone} is false.
If  \xmlNode{mrtauStandAlone} is changed to \xmlString{true}, the path \xmlString{/path/to/mrtau/executable} will be read to get the MrTau executable.
In the example, RAVEN is used in PHISICS standalone mode. Spatial neutron fluxes are printed a and no spatial reaction rates are printed in the RAVEN output.

%%%%%%%%%%%%%%%%%%%%%%%%%%%%%%%%%%%%%%%%%%%%%%%%%%
\subsubsection{Distributions}
The \xmlNode{Distributions} block defines all distributions used to
sample variables in the current RAVEN run.

For all the possible distributions and their possible inputs please
refer to the Distributions chapter (see~\ref{sec:distributions}).
%
It is good practice to name a distribution and its corresponding sampled variable with identical root names, and appending sufix to the distribution name, since there might be many variables with the
same kind of distributions but different input parameters.

%%%%%%%%%%%%%%%%%%%%%%%%%%%%%%%%%%%%%%%%%%%%%%%%%%
\subsubsection{Samplers}
\label{subsubsection:SamplersPhisics}
The \xmlNode{Samplers} block defines the variables to be sampled.
After defining a sampling scheme, the variables to be sampled and
their distributions are identified in the \xmlNode{variable} blocks.
The \xmlAttr{name} must be formatted according to library which the variable belongs to.
The description of the \xmlString{variable} template is detailed in the next sub-sections for the decay
 constants (\ref{decayPara}), the fission yields (\ref{FYPara}), the number densities (\ref{NDPara}),
the fission Q-values (\ref{FQPara}), the $\alpha$ decay Q-values (\ref{AlphaPara}), the $\beta^{+}$ Q-values
 (\ref{BetaPlusPara}), $\beta^{+*}$ Q-values (\ref{BetaPlusStarPara}), $\beta$ Q-values (\ref{BetaPara}),
$\beta^{*}$ Q-values (\ref{BetaStarPara}), the internal transition decay Q-values (\ref{IntTraPara}) and the cross
section scaling factors (\ref{XSPara}).

\paragraph{Decay constant variable} \label{decayPara}

The \xmlString{variable} template is: DECAY$\vert$TYPE\_OF\_DECAY$\vert$ISOTOPE.
The type of decay (TYPE\_OF\_DECAY) is the decay mode relative to the isotope's decay constant perturbed.
The type of decay depends on the isotope perturbed.
If the isotope is an actinide, the available decay modes are:
\begin{enumerate}
  \item [$-$]BETA;
  \item [$-$]BETA+;
  \item [$-$]ALPHA.
\end{enumerate}
If the isotope is a fission product, the available decay modes are:
\begin{enumerate}
  \item [$-$]BETA;
  \item [$-$]BETA*;
  \item [$-$]BETA+;
  \item [$-$]BETA+*;
  \item [$-$]ALPHA;
  \item [$-$]INTER\_TRAN.
\end{enumerate}

The decay types are immediately parsed from the MRTAU decay library. Hence, If the user modifies the decay labels
 in the decay library,
the user will have to modify the her/his decay type in the RAVEN input. The isotope defined in the variable has
 to originally exist in the decay library.

\paragraph{Fission yield variable} \label{FYPara}
The \xmlString{variable} template is FY$\vert$SPECTRUM$\vert$FISSION\_ISOTOPE$\vert$FISSION\_PRODUCT.

The types of spectrum (SPECTRUM) available are: FAST; THERMAL.
The fission isotopes (FISSION\_ISOTOPE) and fission products (FISSION\_PRODUCT) in the variable have to
originally exist in the fission yield library.

\paragraph{Number density variable} \label{NDPara}
The \xmlString{variable} template is DENSITY$\vert$MATERIAL\_ID$\vert$ISOTOPE.
The Material ID (MATERIAL\_ID) has to originally exist in the material XML input. The isotope in the variable has to
be originally defined within the material ID aforementioned.

\paragraph{Fission Q-values variables} \label{FQPara}
The \xmlString{variable} template is QVALUES$\vert$ ISOTOPE.

The isotope (ISOTOPE) in the variable has to originally exist in the fission Q-values library.

\paragraph{$\alpha$ decay variable} \label{AlphaPara}
The \xmlString{variable} template is ALPHADECAY$\vert$ISOTOPE.

The isotope (ISOTOPE) in the variable has to originally exist in the $\alpha$ decay Q-values library.

\paragraph{$\beta^{+}$ decay variable} \label{BetaPlusPara}
The \xmlString{variable} template is BETA+DECAY$\vert$ ISOTOPE.

The isotope (ISOTOPE) in the variable has to originally exist in the $\beta^+$ decay Q-values library.

\paragraph{$\beta^{+*}$ decay variable} \label{BetaPlusStarPara}
The \xmlString{variable} template is BETA+XDECAY$\vert$ISOTOPE.

The isotope (ISOTOPE) in the variable has to originally exist in the $\beta^{+*}$ decay Q-values library.

\paragraph{$\beta$ decay variable} \label{BetaPara}
The \xmlString{variable} template is BETADECAY$\vert$ISOTOPE.

The isotope (ISOTOPE) in the variable has to originally exist in the $\beta$ decay Q-values library.

\paragraph{$\beta^{*}$ decay variable} \label{BetaStarPara}
The \xmlString{variable} template is BETAXDECAY$\vert$ISOTOPE.

The isotope (ISOTOPE) in the variable has to originally exist in the $\beta^*$ decay Q-values library.

\paragraph{Internal transition decay variable} \label{IntTraPara}
The \xmlString{variable} template is INTTRADECAY$\vert$ISOTOPE.

The isotope (ISOTOPE) in the variable has originally to exist in the internal transition decay Q-value library.

\paragraph{Cross section scaling factors} \label{XSPara}
The \xmlString{variable} template is:

XS$\vert$TABULATION\_POINT$\vert$MATERIAL\_ID$\vert$ISOTOPE$\vert$OPERATOR$\vert$XS\_TYPE$\vert$GROUP\_NUMBER.
\begin{enumerate}
  \item [$\cdot$]The tabulation point (TABULATION\_POINT) is the integer referrencing to the tabulation point.
  The tabulation numbering is given by the XML input file 'tabMapping.xml' (Section \ref{additionalInput}).
   If there are no tabulations, the tabulation number has to be 1.
  \item [$\cdot$]The material ID (MATERIAL\_ID) is the string referring to the material in which the isotope is defined.
  The Material ID  has to originally exist in the material XML file.
   \item [$\cdot$]The isotope (ISOTOPE) is the ISOTOPE that the user desires to perturbed. The isotope perturbed
   has to originally exist in the material ID.
   \item [$\cdot$]The operator (OPERATOR) determines how the cross section is perturbed from its nominal value.
   The operators available are:
   \begin{enumerate}
      \item[$-$]ADDITIVE; The additive operator adds the user-defined value to the nominal cross section value.
      \item[$-$]MULTIPLIER; the multiplier operator multiplies the user-defined factor to the nominal value.
      \item[$-$]ABSOLUTE; the absolute operator replaces the nominal value by the user-defined value.
  \end{enumerate}
  \item [$\cdot$]The types of cross sections (CROSS\_SECTION\_TYPE) available are:
  \begin{enumerate}
     \item[$-$]FISSIONXS; Fission cross section.
     \item[$-$]NPXS; Neutron to proton capture cross section.
     \item[$-$]NGXS; Neutron to gamma capture cross section.
     \item[$-$]NUFISSIONXS; $\nu$*fission cross section. The $\nu$*fission is coordinated with the
        fission cross section so that only the coefficient $\nu$ is perturbed.
      \item[$-$]SCATTERINGXS. Total scattering cross section.
      \item[$-$]N2NXS. n,2n cross section.
      \item[$-$]NALPHAXS. Neutron to alpha capture cross section.
      \item[$-$]KAPPAXS. Kappa coefficient.
   \end{enumerate}
   \item The group number (GROUP\_NUMBER) is the group number of the cross section perturbed.
    The group number is an integer and has to be inferior or equal to the number of groups used in the cross section library.
\end{enumerate}
An example is given for each type of variable in Table \ref{VariableExampleTable}.

\begin{table}[]
\centering
\caption{Examples of possible variable names}\label{VariableExampleTable}
\begin{tabular}{l|l|l}
\textit{Input file perturbed} & \textit{Variable perturbed}  & \textit{Example}  \\
\hline
Decay lib.               & Decay constant              & DECAY$\vert$BETAX$\vert$SE78     \\
Fission yield lib.       & Fission yield               & FY$\vert$FAST$\vert$U235$\vert$NB93       \\
XML Material             & Number densitiy             & DENSITY$\vert$FUEL1$\vert$PU239     \\
Fission Q-values lib.    & Fission Q-value              & QVALUES$\vert$U235            \\
$\alpha$ decay lib.      & $\alpha$ decay Q-value       & ALPHADECAY$\vert$U234         \\
$\beta^{+}$ decay lib.     & $\beta^{+}$ decay Q-value      & BETA+DECAY$\vert$U235         \\
$\beta^{+*}$ decay lib.  & $\beta^{+*}$ decay Q-value   & BETA+XDECAY$\vert$U236        \\
$\beta$ decay lib.       & $\beta$ decay Q-value       & BETADECAY$\vert$CM242         \\
$\beta^{*}$ decay lib.     & $\beta^{*}$ decay Q-value      & BETAXDECAY$\vert$PU240        \\
Internal transition lib. & Internal transition Q-value  & INTTRADECAY$\vert$U238        \\
XS scaling factors       & XS scaling factor           & XS$\vert$1$\vert$FUEL1$\vert$U238$\vert$ABSOLUTE$\vert$N2NXS$\vert$4 \\
\end{tabular}
\end{table}

The following example is a Monte Carlo-based sampler with two variables. The $\beta$ decay constant of
 uranium $^{235}$U and the $^{135}$Xe n,p cross section in group 12 at tabulation point 1 within the material
  ID "F1" are to be perturbed. The absolute operator is chosen for the n,p cross section, which means the
  nominal value will be replaced by the averaged value defined in the distribution block, with its corresponding distribution.

\begin{lstlisting}[style=XML]
  <Samplers>
    <MonteCarlo name="MC_samp">
      <samplerInit>
        <limit>100</limit>
      </samplerInit>
      <variable name="DECAY|BETA|U235">
        <distribution>DECAY|BETA|U235_dis</distribution>
      </variable>
      <variable name="XS|1|FUEL1|XE135|ABSOLUTE|NPXS|12">
        <distribution>XS|1|FUEL1|XE135|ABSOLUTE|NPXS|12_dis                   </distribution>
      </variable>
    </MonteCarlo>
  </Samplers>
\end{lstlisting}

%%%%%%%%%%%%%%%%%%%%%%%%%%%%%%%%%%%%%%%%%%%%%%%%%
\subsubsection{Steps}
For a PHISICS interface, the \xmlNode{MultiRun} step type will most likely be used. First, the step needs
to be named: this name will be one of the names used in the \xmlNode{Sequence} block.
In the example, the step is called \xmlString{testDummyStep}.
%
\begin{lstlisting}[style=XML]
<Steps>
  <MultiRun name='testDummyStep' verbosity='debug'>
    <Input   class='Files' type='decay'>decay.dat</Input>
    <Input   class='Files' type='inp'>inp.xml</Input>
    <Input   class='Files' type='XS'>xs.xml</Input>
    <Model   class='Models' type='Code'>PHISICS</Model>
    <Sampler class='Samplers' type='MonteCarlo'>MC_samp</Sampler>
    <Output  class='Databases' type='HDF5'>DataB_REL5_1</Ouput>
  </MultiRun>
</Steps>
\end{lstlisting}
%
%%%%%%%%%%%%%%%%%%%%%%%%%%%%%%%%%%%%%%%%%%%%%%%%%
\subsubsection{Additional Input}
\label{subsubsection:PhisicsAdditionalInput}
In addition to the usual PHISICS inputs required (INSTANT input, depletion input, material
 input, library input, path input) and the regular MrTau libraries, additional inputs may be required, depending on the user's needs.
\begin{enumerate}
\item [$\cdot$]A file 'tabMapping.xml' is (optional) maps the cross section tabulation points for interpolation purposes.
The tabulation mapping assigns an integer to a given tabulation in order to identify it in the RAVEN variable definition.
The format of the tabulation mapping is the following:
\begin{lstlisting}[style=XML]
<mapping>
  <tabulation set="1">
    <tab name="mod_temperature">559.0</tab>
    <tab name="BURN-UP">0.0</tab>
  </tabulation>
  <tabulation set="2">
    <tab name="mod_temperature">1000</tab>
    <tab name="BURN-UP">0.0</tab>
  </tabulation>
  <tabulation set="3">
    <tab name="mod_temperature">1000</tab>
    <tab name="BURN-UP">100</tab>
  </tabulation>
</mapping>
\end{lstlisting}
In \xmlNode{tabulation}, the \xmlAttr{set} refers to the user-defined number assigned to a given tabulation point.
This \xmlAttr{set} number corresponds to the second argument of the cross section scaling factor variable.
The user enters the tabulation parameters and corresponding tabulation values with the \xmlNode{tabulation},
 using the sub-node \xmlNode{tab}.

\item [$\cdot$]If the user perturbs cross sections, an XML file 'scaled\_xs.xml' will be generated at each perturbation
 in the output folder.
The file 'scaled\_xs.xml' is automatically created from the cross section variables defined in the RAVEN
\xmlNode{Sampler} block (see \ref{XSPara}). The cross section file 'scaled\_xs.xml' has the following format:

\begin{lstlisting}[style=XML]
<scaling_library>
  <tabulation>
    <tab name="mod_temperature">559.0</tab>
    <tab name="BURN-UP">0.0</tab>
    <library lib_name="fuel1" >
      <isotope id="xe135" type="absolute">
        <npxs g="8,3,12">8.889E+02,3.333E+02,1.212E+05</npxs>
      </isotope>
      <isotope id="u235" type="multiplier">
        <fissionxs g="1">2.019E-02</fissionxs>
      </isotope>
    </library>
  </tabulation>
<scaling_library>
\end{lstlisting}
The tabulation points \xmlNode{tab} are optional have to agree with the tabulation points defined in the
 ``tabMapping.xml'' if they are provided.
The \xmlNode{library} has one required attribute \xmlAttr{lib\_name} corresponding to one of the libraries
listed in the PHISICS library input.
The \xmlNode{isotope} provides the information related to an isotope included in the library aforementioned.
The \xmlAttr{id} gives the isotope ID (no dash allowed).
The \xmlAttr{type} specifies the type of operator used (\xmlString{additive}, \xmlString{multiplier} or
 \xmlString{absolute}).
The sub-node \xmlNode{XS} (where 'XS' is the perturb-able type of cross sections listed in section~\ref{XSPara})
 provides the cross section information.
The \xmlAttr{g} attribute refers to the group numbers to be perturbed, separated by commas.
The \xmlString{XS} provides the scaling factors or the new cross section values.
\end{enumerate}
%
%%%%%%%%%%%%%%%%%%%%%%%%%%%%%%%%%%%%%%%%%%%%%%%%%
\subsubsection{Output Files Conversion}
\label{subsubsection:PhisicsOutFileConv}

The PHISICS output available for RAVEN post-processing are described in this section. The PHISICS outputs
 are by convention separated by '$\vert$' if they are contained in a matrix form such as group-wise or region-wise values.
In the PHISICS mode (i.e. \xmlNode{mrtauStandalone} is \xmlString{False}) The variables available for RAVEN
 post-processing are:
\begin{enumerate}
  \item [$-$]the MrTau time;
  \item [$-$]the multiplication factor;
  \item [$-$]the multiplication factor error;
  \item [$-$]the spatial reaction rates (only if \xmlNode{printSpatialRR} is \xmlString{True});
  \item [$-$]the spatial power (only if \xmlNode{printSpatialRR} is \xmlString{True});
  \item [$-$]the spatial fluxes by region (only if \xmlNode{printSpatialRR} is \xmlString{True});
  \item [$-$]the neutron fluxes by cell (only if \xmlNode{printSpatialFlux} is \xmlString{True});
  \item [$-$]the neutron fluxes by material (only if \xmlNode{printSpatialFlux} is \xmlString{True});
  \item [$-$]the total reaction rates (only if \xmlNode{printSpatialRR} is \xmlString{False});
  \item [$-$]the decay heat (only if decay heat flag is on in the PHISICS input);
  \item [$-$]the burnup;
  \item [$-$]the cross section values (only for perturbed cross sections);
  \item [$-$]the PHISICS cpu time.
\end{enumerate}

In the MrTau standalone mode (i.e \xmlNode{mrtauStandalone} is \xmlString{True}) The variables available for RAVEN post-processing are:
\begin{enumerate}
  \item [$-$]the MRTAU time;
  \item [$-$]the isotope number densities;
  \item [$-$]the decay heat.
\end{enumerate}

The variable template is provided in Table \ref{VariableTemplateTable}. In the table, the region number is taken equal to 4,
 the group number is taken equal to 7, the cell number equal to 2 and the material number equal to 3.
Those values are only examples and can be adapted to the user's convenience.

\begin{table}[]
\centering
\caption{template of the RAVEN output variables}\label{VariableTemplateTable}
\begin{tabular}{l|l|l}
\textit{Variable} & \textit{Variable template}  & \textit{Comment}  \\
\hline
MrTau Time                     & timeMrTau                            & \\
Multiplication Factor          & keff                                 & \\
Multiplication Factor Error    & errorKeff                            & \\
n2n Reaction Rate              & n2n$\vert$gr7$\vert$reg4             & only if \xmlNode{printSpatialRR} is \xmlString{True})    \\
Power                          & power$\vert$gr7$\vert$reg4           & only if \xmlNode{printSpatialRR} is \xmlString{True})    \\
Absorption Reaction Rate       & absorption$\vert$gr7$\vert$reg4      & only if \xmlNode{printSpatialRR} is \xmlString{True})    \\
Fission Reaction Rate          & fission$\vert$gr7$\vert$reg4         & only if \xmlNode{printSpatialRR} is \xmlString{True})    \\
Neutron Flux                   & flux$\vert$gr7$\vert$reg4            & only if \xmlNode{printSpatialRR} is \xmlString{True})    \\
$\nu$ Fission Reaction Rate    & neutron$\vert$gr7$\vert$reg4         & only if \xmlNode{printSpatialRR} is \xmlString{True})    \\
Neutron Flux by Cell           & flux$\vert$cell2$\vert$gr7           & only if \xmlNode{printSpatialFlux} is \xmlString{True})  \\
Neutron Flux by Material       & flux$\vert$mat3$\vert$gr4            & only if \xmlNode{printSpatialFlux} is \xmlString{True})  \\
Total n2n Reaction Rate        & n2n$\vert$Total                      & only if \xmlNode{printSpatialRR} is \xmlString{False})  \\
Total Power reaction Rate      & power$\vert$Total                    & only if \xmlNode{printSpatialRR} is \xmlString{False})  \\
Total Absorption Reaction Rate & absorption$\vert$Total               & only if \xmlNode{printSpatialRR} is \xmlString{False})  \\
Total Fission Reaction Rate    & fission$\vert$Total                  & only if \xmlNode{printSpatialRR} is \xmlString{False})  \\
Total Neutron Flux             & flux$\vert$Total                     & only if \xmlNode{printSpatialRR} is \xmlString{False})  \\
Total $\nu$ Fis. Reaction Rate & neutron$\vert$Total                  & only if \xmlNode{printSpatialRR} is \xmlString{False})  \\
Decay Heat                     & decay$\vert$Fuel1$\vert$gr4          & only if the decay heat flag is turned on in PHISICS  \\
Cross sections                 & fuel1$\vert$xe135$\vert$npxs$\vert$8 & only if cross sections are perturbed  \\
PHISIC CPU time                & cpuTime                              & \\
\end{tabular}
\end{table}
$\nu$ is the average number of neutrons generated after fission. Note that the material number in the neutron
 flux by material corresponds to the material ID number in the PHISICS csv output,
while the material string ID in the decay heat corresponds to the material name given by the user in the xml
material file. Hence, the neutron flux by material will always have the format matX, where X is an integer.
The decay heat material is user-defined in the xml material file via the attribute \xmlAttr{id} of the \xmlNode{mat} node.
%%%%%%%%%%%%%%%%%%%%%%%%%%%%%%%%%%%%%%%%%%%%%%%%%
%%%%%%%%%%%%%%%%%%%%%%%%%%%%%%%%%%%%%%%%%%%%%%%%%
%%%%%%%%%%%%% PHISICS/RELAP5 INTERFACE %%%%%%%%%%
%%%%%%%%%%%%%%%%%%%%%%%%%%%%%%%%%%%%%%%%%%%%%%%%%
\subsection{PHISICS/RELAP5 Interface}
\label{subsec:PhisicsRelap5Interface}
%
\subsubsection{General Information}
%
This section covers the input specification for running PHISICS/RELAP5 through RAVEN.
This interface can be used to perturb the PHISICS and/or RELAP5 input files. This interface is strongly built around the PHISICS and RELAP5
standalone interfaces, hence this sections covers the additional cautions to take care of to run the coupled PHISICS/RELAP5 code. The
user will find additional information regarding PHISICS in section \ref{subsec:PhisicsInterface} or RELAP5 in section \ref{subsec:RELAP5Interface}.
%
\subsubsection{Files}
\label{subsec:PhisicsRelap5Files}
\xmlNode{Files} includes two attributes \xmlAttr{name} and \xmlAttr{type} entries, identically as other interfaces.
It also includes two optional attributes \xmlAttr{perturbable} and \xmlAttr{subDirectory}.

The \xmlAttr{name} attribute is a user-defined internal name for the file contained in the node.
Default: None (required entry).

For the files parsed by the PHISICS interface or the PHISICS interface's parsers, some of the \xmlAttr{type} attributes are hardcoded.
The accepted PHISICS \xmlAttr{type} attributes are given in Table \ref{TypeTable} and are not repeated here. Additional information can be found in section \ref{subsubsec:PhisicsInterfaceFiles}.
All the necessary RELAP5 input files need to have a \xmlAttr{type} attribute starting with the string 'relap'.
The necessary RELAP5 files for use in the coupled PHISICS/RELAP mode within RAVEN are given in Table~\ref{PhisicsRelap5TypeTable}. The files associated
with a  \xmlAttr{type} that does not start with the string 'relap' will be treated by the PHISICS interface.
The  \xmlAttr{type} attributes are case-incensitive.
\begin{table}[]
  \centering
  \caption{Example of RELAP5 type attributes in coupled PHISICS/RELAP5 mode}\label{PhisicsRelap5TypeTable}
    \begin{tabular}{l|l|l}
\textit{Type Attribute} & \textit{Corresponding RELAP5 input} & \textit{Perturbable} \\
\hline
relapFluid              & fluid properties                  & No  \\
relapInp                & Relap input File                  & yes \\
relapLicense            & license for the RELAP5 executable & No  \\
    \end{tabular}
\end{table}

Example of acceptable RELAP5 entries within PHISICS/RELAP5:
\begin{lstlisting}[style=XML]
<File>
  <Input name="H2O"       type="relaph2o"     perturbable="False">tph2o</Input>
  <Input name="H2"        type="relaph2"      perturbable="False">tph2</Input>
  <Input name="inputDeck" type="relapInput"   perturbable="True" >inp.i</Input>
  <Input name="lic"       type="relapLicence" perturbable="False">license.bin</Input>
</File>
\end{lstlisting}
%
\subsubsection{Models}
The user has to provide the paths to executables for the sampled variables within the \xmlNode{Models} block.

The \xmlNode{Code} block will contain attributes \xmlAttr{name} and \xmlAttr{subType}.
The \xmlAttr{name} identifies the particular \xmlNode{Code} model within RAVEN, and \xmlAttr{subType}
specifies which code interface the model will use. \xmlAttr{subType}=\xmlString{PhisicsRelap5} is the class name
currently used for PHISICS/RELAP5 coupled calculations.

The \xmlNode{executable} block contains the absolute or relative path (with respect to the current working directory) to PHISICS/RELAP5
that RAVEN will use to run the code. The additional nodes in the \xmlNode{Models} applicable to PHISICS standalone and RELAP5 standalone
are valid in coupled mode and can be consulted in section~\ref{subsubsection:PhisicsModel} and section~\ref{subsubsection:Relap5Models} respectively.
Exception: the use of MrTau in standalone mode (i.e. \xmlNode{mrtauStandAlone} set to \xmlString{True}) is not allowed in PHISICS/RELAP5 coupled
calculations.

An example of the \xmlNode{Models} block is given below:

\begin{lstlisting}[style=XML]
  <Models>
    <Code name="PHISICS_RELAP5" subType="PhisicsRelap5">
       <executable>./path/to/instant/executable</executable>
    </Code>
  </Models>
\end{lstlisting}
%%%%%%%%%%%%%%%%%%%%%%%%%%%%%%%%%%%%%%%%%%%%%%%%%%
\subsubsection{Distributions}
The \xmlNode{Distributions} block defines all distributions used to
sample variables in the current RAVEN run.

For all the possible distributions and their possible inputs please
refer to the Distributions chapter (see~\ref{sec:distributions}).
%
%%%%%%%%%%%%%%%%%%%%%%%%%%%%%%%%%%%%%%%%%%%%%%%%%%
\subsubsection{Samplers}\label{SamplerPhisicsRelap5}
The \xmlNode{Samplers} block defines the variables to be sampled.
After defining a sampling scheme, the variables to be sampled and
their distributions are identified in the \xmlNode{variable} blocks.
The \xmlAttr{name} must be formatted according to the PHISICS library which the variable belongs to.
Information relative to PHISICS distributions are in section~\ref{subsubsection:SamplersPhisics},
as well as specifications on PHISICS variable names.
An example of a \xmlNode{Samplers} block is given below:
\begin{lstlisting}[style=XML]
 <Samplers>
    <MonteCarlo name="MC_samp">
      <samplerInit>
        <limit>10</limit>
      </samplerInit>
      <variable name="DENSITY|FUEL1|U238">
        <distribution>DENSITY|FUEL1|U238_distrib</distribution>
      </variable>
      <variable name="20100154:2">
        <distribution>heat_capacity_154</distribution>
      </variable>
    </MonteCarlo>
  </Samplers>
\end{lstlisting}
In this example, the variable \xmlString{DENSITY$\vert$FUEL1$\vert$U238} is relative to PHISICS and the variable \xmlString{20100154:2} is relative to RELAP5.
%%%%%%%%%%%%%%%%%%%%%%%%%%%%%%%%%%%%%%%%%%%%%%%%%
\subsubsection{Steps}
The tasks performed by RAVEN need to be defined in the \xmlNode{Steps} block. Each task needs to be defined with a \xmlAttr{name}. This \xmlAttr{name} is later on used in the
the \xmlNode{Sequence} block. In the example, the step is called \xmlString{testDummyStep}.
%
\begin{lstlisting}[style=XML]
<Steps>
  <MultiRun name='testDummyStep' verbosity='debug'>
    <Input class='Files' type='decay'>decay.dat</Input>
    <Input class='Files' type='inp'>inp.xml</Input>
    <Input class='Files' type='XS'>xs.xml</Input>
    <Input class="Files" type="relapFluid">tpfhe</Input>
    <Input class="Files" type="relapExec">relap5Exec.x</Input>
    <Model class="Models" type="Code">PHISICS_RELAP5</Model>
    <Sampler class="Samplers" type="MonteCarlo">MC_samp</Sampler>
    <Output class="Databases" type="HDF5">DataB_REL5_1</Output>
    <Output class="DataObjects" type="PointSet">collset</Output>
    </MultiRun>
  </MultiRun>
</Steps>
\end{lstlisting}
%
%%%%%%%%%%%%%%%%%%%%%%%%%%%%%%%%%%%%%%%%%%%%%%%%%
\subsubsection{Additional Input}
\label{subsubsection:PhisicsRelap5AdditionalInput}
The PHISICS additional inputs are described in section~\ref{subsubsection:PhisicsAdditionalInput}. The RELAP5 additional inputs are described in
section ~\ref{subsubsection:Relap5Models}.
%
%%%%%%%%%%%%%%%%%%%%%%%%%%%%%%%%%%%%%%%%%%%%%%%%%
\subsubsection{Output Files Conversion}
\label{subsubsection:PhisicsRelap5OutputFileConversion}

The PHISICS output available for RAVEN post-processing are described in section~\ref{subsubsection:PhisicsOutFileConv}.
The output printed from PHISICS and RELAP5 are synchronized in the RAVEN csv output. The synchronization scheme is explained in this section.

At t = 0 seconds, the RELAP5 initialized output are printed in the csv output, while the output variables from PHISICS are taken equal to 0.
Then, the RAVEN/PHISICS/RELAP5 post-processor finds the time step number at the end of each PHISICS
burn step based on the \xmlNode{tab\_time\_step} values, and prints the RELAP minor edits according to the \xmlNode{TH\_between\_BURN} values.

Let's consider the following example:
\begin{enumerate}
\item [$-$]\xmlNode{tab\_time\_step} \xmlString{5 3 2} \xmlNode{/tab\_time\_step} (in the PHISICS depletion file);
\item [$-$]\xmlNode{TH\_between\_BURN} \xmlString{1.0 2.0} \xmlNode{TH\_between\_BURN}() in the PHISICS input file);
\item [$-$]\xmlNode{tabulation\_boundaries} \xmlString{5.0 35.0 45.0}\xmlNode{tabulation\_boundaries} (upper burn step boundaries in the PHISICS depletion file).
\item [$-$]in the RELAP5 input, a 3.0 seconds steady state is considered, with minor edits every 0.5 seconds.
\end{enumerate}
The first PHISICS/RELAP5 output line printed will be at t = 0 seconds. The PHISICS outputs are set to 0.0, the RELAP5 values are obtained at the end of the initialization.
The second line printed will be at the PHISICS time step \xmlString{5} (end of the first burn step, corresponding to t = 5.0 seconds),
and prints the RELAP5 minor edits as long as the time from the minor edits is lower than the first \xmlNode{TH\_between\_BURN} value \xmlString{1.0}.
The RELAP5 minor edits are printed along with the PHISICS burn step \xmlString{5} as long as time in the minor edits is smaller or equal to \xmlString{1.0}.
When the RELAP5 time in the minor edits time is greater than \xmlString{1.0}, the end of the second PHISICS burn step is targetted, from the
\xmlNode{tab\_time\_step}: \xmlString{3}. This corresponds to a PHISICS time equal to 35.0 seconds. The RELAP5 minor edits are printed along with the PHISICS values at t = 35.0 s,
as long as the minor data time is smaller than the \xmlNode{TH\_between\_BURN} equal to \xmlString{2.0}.
Finally, the last PHISICS time step at 45.0 seconds is printed along with the RELAP5 minor edits.
Overall, the output variables printed will be:

mrtauTime, n2n$\vert$gr1$\vert$reg4, httemp\_3001010\_1, time

0.000, 0.000, 1.526, 0.000

5.000, 1.111, 1.859, 0.500

5.000, 1.111, 2.369, 1.000

35.00, 7.800, 3.666, 1.500

35.00, 7.800, 4.789, 2.000

45.00, 9.330, 4.225, 3.000
%%%%%%%%%%%%%%%%%%%%%%%%%%%%%%%%%%%%%%%%%%%%%%%%%%%%%%
%%%%%%%%%%%%%%%%% Neutrino Interface %%%%%%%%%%%%%%%%%%
%%%%%%%%%%%%%%%%%%%%%%%%%%%%%%%%%%%%%%%%%%%%%%%%%%%%%%
\subsection{Neutrino Interface}
\label{subsec:neutrinoInterface}
This section covers the input specification for running Neutrino through RAVEN. It is important to notice
that this explanation assumes that the reader already knows how to use Neutrino. The existing inteface can be used to modify the particle size.
However, the interface can be modified to alter other parameters by using a similar method to the
existing particle size modification included in the interface.

\subsubsection{Files}
In the \xmlNode{Files} section, as specified before, all the files needed for the code to
run should be specified. In the case of Neutrino, the file needed is the following:
\begin{itemize}
  \item Neutrino input file with file extension `.nescene';
\end{itemize}
The Neutrino input file name must be NeutrinoInput.nescene. Otherwise, the Neutrino interface must be modified.
%
Example:
\begin{lstlisting}[style=XML]
  <Files>
    <Input name="neutrinoInput" type="">NeutrinoInput.nescene</Input>
  </Files>
\end{lstlisting}

%%%%%%%%%%%%%%%%%%%%%%%%%%%%%%%%%%%%%%%%%%%%%%%%%%%%%X
\subsubsection{Models}
In the \xmlNode{Models} block, the Neutrino executable needs to be specified. The entire path to the Neutrino executable must be included.
 Here is a standard example of what
can be used:
\begin{lstlisting}[style=XML]
  <Models>
    <Code name="neutrinoCode" subType="Neutrino">
      <executable>"C:\Program Files\Neutrino_02_22_19\Neutrino.exe"</executable>
    </Code>
  </Models>
\end{lstlisting}

The \xmlNode{Code} XML node contains the information needed to execute the specific External Code. This
XML node accepts the following attributes:
\begin{itemize}
  \item \xmlAttr{name}, \xmlDesc{required string attribute}, user-defined identifier of this model.
    \nb As with other objects, this identifier can be used to reference this specific entity from other input
    blocks in the XML.
  \item \xmlAttr{subType}, \xmlDesc{required string attribute}, specifies the code that needs to be
    associated to this Model.
\end{itemize}
This model can be initialized with the following children:
\begin{itemize}
  \item \xmlNode{executable}, \xmlDesc{string, required field}, specifies the path of the executable to
    be used; \nb Either an absolute or relative path can be used.
\end{itemize}


%%%%%%%%%%%%%%%%%%%%%%%%%%%%%%%%%%%%%%%%%%%%%%%%%%%%%%
\subsubsection{Distributions}
The \xmlNode{Distributions} block defines the distributions that are going to be used for the sampling
of the variables defined in the \xmlNode{Samplers} block. For all the possible distributions and all
their possible inputs, please see the chapter about Distributions (see~\ref{sec:distributions}). Here is an example of a
Uniform distribution:
\begin{lstlisting}[style=XML]
  <Distributions>
    <Uniform name="uni">
        <lowerBound>0.1</lowerBound>
        <upperBound>0.2</upperBound>
    </Uniform>
  </Distributions>
\end{lstlisting}
%
%%%%%%%%%%%%%%%%%%%%%%%%%%%%%%%%%%%%%%%%%%%%%%%%%%%%%%
\subsubsection{Samplers}
The \xmlNode{Samplers} block defines the variables to be sampled. After defining a sampling scheme, the variables to be sampled and
their distributions are identified in the \xmlNode{variable} blocks.
The \xmlAttr{name} must be formatted according to the Neutrino parameter name, which for the particle size is
'ParticleSize'.
An example of a \xmlNode{Samplers} block is given below:
\begin{lstlisting}[style=XML]
 <Samplers>
    <MonteCarlo name="myMC">
      <samplerInit>
        <limit>5</limit>
      </samplerInit>
      <variable name='ParticleSize'>
        <distribution>uni</distribution>
      </variable>
    </MonteCarlo>
  </Samplers>
\end{lstlisting}

%%%%%%%%%%%%%%%%%%%%%%%%%%%%%%%%%%%%%%%%%%%%%%%%%%%%%%
\subsubsection{Steps}
In this section, the \xmlNode{MultiRun} will be used. As shown in the following, a Neutrino
input file is listed in \xmlNode{Files} and is linked here using \xmlNode{Input}, the \xmlNode{Model}
and \xmlNode{Sampler} defined in previous sections will be used in this \xmlNode{MultiRun}. The
outputs will be saved in the \textbf{DataObject} 'resultPointSet'.

\begin{lstlisting}[style=XML]
  <Steps>
    <MultiRun name="run">
      <Input class="Files" type="">neutrinoInput</Input>
      <Model class="Models" type="Code">neutrinoCode</Model>
      <Sampler class="Samplers" type="MonteCarlo">myMC</Sampler>
      <Output class="DataObjects" type="PointSet">resultPointSet</Output>
    </MultiRun>
  </Steps>
\end{lstlisting}

%%%%%%%%%%%%%%%%%%%%%%%%%%%%%%%%%%%%%%%%%%%%%%%%%%%%%%
\subsubsection{Output File Conversion}
The Neutrino measurement field output is a CSV output. However, labels must be added to the Neutrino output
and it must be moved for RAVEN. These are both done in the Neutrino interface. The labels that are added to the
output file are 'time' and 'result'. These labels would be used in the \xmlNode{DataObjects} specification.
If different labels are wanted, they would need to be changed directly in the Neutrino interface.

%%%%%%%%%%%%%%%%%%%%%%%%%%%%%%%%%%%%%%%%%%%%%%%%%%%%%%
\subsubsection{Additional Information}
The Neutrino interface is used to alter the particle size by modifying the Neutrino input file. The Neutrino interface
searches for the default SPH solver parameter name: '\texttt{NIISphSolver\_1}'. If the SPH solver name is changed in the
Neutrino input file, the Neutrino interface must also be changed. Similarly, the Neutrino interface searches for the
output in the default Measurement field name: '\texttt{MeasurementField\_1}'. Again, this would need to modified in the interface
if the measurement field name was changed.

\input{couplingAcode.tex}
\input{advanced_users_plugins.tex}
\input{advanced_users_templates.tex}
\input{examplesPrimer.tex}
\section*{Document Version Information}
\input{../version.tex}


    % ---------------------------------------------------------------------- %
    % References
    %
    \clearpage
    % If hyperref is included, then \phantomsection is already defined.
    % If not, we need to define it.
    \providecommand*{\phantomsection}{}
    \phantomsection
    \addcontentsline{toc}{section}{References}
    \bibliographystyle{ieeetr}
    \bibliography{raven_user_manual}


    % ---------------------------------------------------------------------- %
    %

    % \printindex

    %\include{distribution}

\end{document}
