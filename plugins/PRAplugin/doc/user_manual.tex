%
\documentclass[pdf,12pt]{article}

%\usepackage{times}
%\usepackage[FIGBOTCAP,normal,bf,tight]{subfigure}
\usepackage{amsmath}
\usepackage{amssymb}
%\usepackage{pifont}
\usepackage{enumerate}
\usepackage{listings}
\usepackage{fullpage}
\usepackage{xcolor}          % Using xcolor for more robust color specification
%\usepackage{ifthen}          % For simple checking in newcommand blocks
%\usepackage{textcomp}
%\usepackage{authblk}         % For making the author list look prettier
%\renewcommand\Authsep{,~\,}

% Custom colors
\definecolor{deepblue}{rgb}{0,0,0.5}
\definecolor{deepred}{rgb}{0.6,0,0}
\definecolor{deepgreen}{rgb}{0,0.5,0}
\definecolor{forestgreen}{RGB}{34,139,34}
\definecolor{orangered}{RGB}{239,134,64}
\definecolor{darkblue}{rgb}{0.0,0.0,0.6}
\definecolor{gray}{rgb}{0.4,0.4,0.4}

\lstset {
  basicstyle=\ttfamily,
  frame=single
}

\lstdefinestyle{XML} {
    language=XML,
    extendedchars=true,
    breaklines=true,
    breakatwhitespace=true,
%    emph={name,dim,interactive,overwrite},
    emphstyle=\color{red},
    basicstyle=\ttfamily,
%    columns=fullflexible,
    commentstyle=\color{gray}\upshape,
    morestring=[b]",
    morecomment=[s]{<?}{?>},
    morecomment=[s][\color{forestgreen}]{<!--}{-->},
    keywordstyle=\color{cyan},
    stringstyle=\ttfamily\color{black},
    tagstyle=\color{darkblue}\bf\ttfamily,
    morekeywords={name,type},
%    morekeywords={name,attribute,source,variables,version,type,release,x,z,y,xlabel,ylabel,how,text,param1,param2,color,label},
}
\lstset{language=xml}

\usepackage{titlesec}
\newcommand{\sectionbreak}{\clearpage}
\setcounter{secnumdepth}{4}


%%%%%%%% Begin comands definition to input python code into document
\usepackage[utf8]{inputenc}

% Default fixed font does not support bold face
\DeclareFixedFont{\ttb}{T1}{txtt}{bx}{n}{9} % for bold
\DeclareFixedFont{\ttm}{T1}{txtt}{m}{n}{9}  % for normal

\usepackage{listings}

% Python style for highlighting
%\newcommand\pythonstyle{\lstset{
%language=Python,
%basicstyle=\ttm,
%otherkeywords={self, none, return},             % Add keywords here
%keywordstyle=\ttb\color{deepblue},
%emph={MyClass,__init__},          % Custom highlighting
%emphstyle=\ttb\color{deepred},    % Custom highlighting style
%stringstyle=\color{deepgreen},
%frame=tb,                         % Any extra options here
%showstringspaces=false            %
%}}


% Python environment
%\lstnewenvironment{python}[1][]
%{
%$\pythonstyle
%\lstset{#1}
%}
%{}

% Python for external files
%\newcommand\pythonexternal[2][]{{
%\pythonstyle
%\lstinputlisting[#1]{#2}}}
%
%\lstnewenvironment{xml}
%{}
%{}

% Python for inline
%\newcommand\pythoninline[1]{{\pythonstyle\lstinline!#1!}}

%\def\DRAFT{} % Uncomment this if you want to see the notes people have been adding
% Comment command for developers (Should only be used under active development)
%\ifdefined\DRAFT
%  \newcommand{\nameLabeler}[3]{\textcolor{#2}{[[#1: #3]]}}
%\else
%  \newcommand{\nameLabeler}[3]{}
%\fi
% Commands for making the LaTeX a bit more uniform and cleaner
%\newcommand{\TODO}[1]    {\textcolor{red}{\textit{(#1)}}}
\newcommand{\xmlAttrRequired}[1] {\textcolor{red}{\textbf{\texttt{#1}}}}
\newcommand{\xmlAttr}[1] {\textcolor{cyan}{\textbf{\texttt{#1}}}}
\newcommand{\xmlNodeRequired}[1] {\textcolor{deepblue}{\textbf{\texttt{<#1>}}}}
\newcommand{\xmlNode}[1] {\textcolor{darkblue}{\textbf{\texttt{<#1>}}}}
\newcommand{\xmlString}[1] {\textcolor{black}{\textbf{\texttt{'#1'}}}}
\newcommand{\xmlDesc}[1] {\textbf{\textit{#1}}} % Maybe a misnomer, but I am
                                                % using this to detail the data
                                                % type and necessity of an XML
                                                % node or attribute,
                                                % xmlDesc = XML description
\newcommand{\default}[1]{~\\*\textit{Default: #1}}
\newcommand{\nb} {\textcolor{deepgreen}{\textbf{~Note:}}~}

% The bm package provides \bm for bold math fonts.  Apparently
% \boldsymbol, which I used to always use, is now considered
% obsolete.  Also, \boldsymbol doesn't even seem to work with
% the fonts used in this particular document...
\usepackage{bm}

% Define tensors to be in bold math font.
\newcommand{\tensor}[1]{{\bm{#1}}}

% Override the formatting used by \vec.  Instead of a little arrow
% over the letter, this creates a bold character.
\renewcommand{\vec}{\bm}

% Define unit vector notation.  If you don't override the
% behavior of \vec, you probably want to use the second one.
\newcommand{\unit}[1]{\hat{\bm{#1}}}

% Use this to refer to a single component of a unit vector.
\newcommand{\scalarunit}[1]{\hat{#1}}

% \toprule, \midrule, \bottomrule for tables
\usepackage{booktabs}

% \llbracket, \rrbracket
\usepackage{stmaryrd}

\usepackage{hyperref}

\usepackage{graphicx}
\graphicspath{{./figures/}}

% Compress lists of citations like [33,34,35,36,37] to [33-37]
\usepackage{cite}

% If you want to relax some of the SAND98-0730 requirements, use the "relax"
% option. It adds spaces and boldface in the table of contents, and does not
% force the page layout sizes.
% e.g. \documentclass[relax,12pt]{SANDreport}
%
% You can also use the "strict" option, which applies even more of the
% SAND98-0730 guidelines. It gets rid of section numbers which are often
% useful; e.g. \documentclass[strict]{SANDreport}

% The INLreport class uses \flushbottom formatting by default (since
% it's intended to be two-sided document).  \flushbottom causes
% additional space to be inserted both before and after paragraphs so
% that no matter how much text is actually available, it fills up the
% page from top to bottom.  My feeling is that \raggedbottom looks much
% better, primarily because most people will view the report
% electronically and not in a two-sided printed format where some argue
% \raggedbottom looks worse.  If we really want to have the original
% behavior, we can comment out this line...
\raggedbottom
\setcounter{secnumdepth}{5} % show 5 levels of subsection
\setcounter{tocdepth}{5} % include 5 levels of subsection in table of contents

% ---------------------------------------------------------------------------- %
%
% Set the title, author, and date
%
\title{The RAVEN PRA Plugin \\ - User Manual -}
%\author{%
%\begin{tabular}{c} Author 1 \\ University1 \\ Mail1 \\ \\
%Author 3 \\ University3 \\ Mail3 \end{tabular} \and
%\begin{tabular}{c} Author 2 \\ University2 \\ Mail2 \\ \\
%Author 4 \\ University4 \\ Mail4\\
%\end{tabular} }


\author{D. Mandelli, C. Wang, A. Alfonsi}

% There is a "Printed" date on the title page of a SAND report, so
% the generic \date should [WorkingDir:]generally be empty.
\date{\today}

%\def\component#1{\texttt{#1}}

% ---------------------------------------------------------------------------- %
%\newcommand{\systemtau}{\tensor{\tau}_{\!\text{SUPG}}}

% Added by Sonat
%\usepackage{placeins}
%\usepackage{array}

%\newcolumntype{L}[1]{>{\raggedright\let\newline\\\arraybackslash\hspace{0pt}}m{#1}}
%\newcolumntype{C}[1]{>{\centering\let\newline\\\arraybackslash\hspace{0pt}}m{#1}}
%\newcolumntype{R}[1]{>{\raggedleft\let\newline\\\arraybackslash\hspace{0pt}}m{#1}}

% end added by Sonat
% ---------------------------------------------------------------------------- %
%
% Start the document
%

\begin{document}
    \maketitle

    % ------------------------------------------------------------------------ %
    % The table of contents and list of figures and tables
    % Comment out \listoffigures and \listoftables if there are no
    % figures or tables. Make sure this starts on an odd numbered page
    %
    \cleardoublepage		% TOC needs to start on an odd page
    \tableofcontents
    %\listoffigures
    %\listoftables
    % ---------------------------------------------------------------------- %

    % ---------------------------------------------------------------------- %
    % This is where the body of the report begins; usually with an Introduction
    %
    \input{Introduction.tex}
    \input{include/ETmodel.tex}
    \input{include/FTmodel.tex}
    \input{include/MarkovModel.tex}
    \input{include/RBDmodel.tex}
    \input{include/DataClassifier.tex}
    \input{include/ETdataImporter.tex}
    \input{include/FTdataImporter.tex}
    \section{MCSSolver}
\label{sec:MCSSolver}

This model is designed to read from file a list of Minimal Cut Sets (MCSs) and to import such Boolean logic structure as a RAVEN model.
Provided the sampled values of Basic Events (BEs) probabilities, the MCSSolver determines the probability of Top Event (TE), i.e., the union of the MCSs.
The list of MCS must be provided through a CSV file with the following format:

\begin{table}
  \begin{center}
    \caption{MCS file format.}
    \label{tab:table1}
    \begin{tabular}{c|c|c} 
      \textbf{ID} & \textbf{Prob} & \textbf{MCS}\\
      \hline
      1, & 0.01, & BE1\\
      2, & 0.02, & BE3\\
      3, & 0.03, & BE2,BE4\\
    \end{tabular}
  \end{center}
\end{table}

In this example:
\begin{itemize}
  \item three MCSs are defined: MCS1 = BE1, MCS2 = BE3 and MCS3 = BE2 and BE4 
  \item four BEs are defined: BE1, BE2, BE3 and BE4
  \item probability of TE, i.e. P(TE), is equal to: $P(TE) = P(MCS1 \cup MCS2 \cup MCS3)$
\end{itemize}

Note that the MCSSolver considers only the list of MCSs and it discards the rest of data contained in the csv file.

All the specifications of the MCSSolver model are given in the \xmlNode{ExternalModel} block. 
Inside the \xmlNode{ExternalModel} block, the XML nodes that belong to this models are:
\begin{itemize}
  \item  \xmlNode{variables}, \xmlDesc{string, required parameter}, a list containing the names of both the input and output variables of the model
  \item  \xmlNode{solverOrder},\xmlDesc{integer, required parameter}, solver order for $P(TE)$ 
  \item  \xmlNode{topEventID},\xmlDesc{string, required parameter}, the name of the alias variable for the Top Event
  \item  \xmlNode{map},\xmlDesc{string, required parameter}, the name ID of the ET branching variable
	  \begin{itemize}
	    \item \xmlAttr{var}, \xmlDesc{required string attribute}, the ALIAS name ID of the basic event
	  \end{itemize}
\end{itemize}

An example of RAVEN input file is the following:

\begin{lstlisting}[style=XML,morekeywords={anAttribute},caption=MCSSolver model input example., label=lst:MCSSolver_InputExample]
  <Models> 
    ...
    <Models>
      <ExternalModel name="MCSmodel" subType="PRAplugin.MCSSolver">
        <variables>statusBE1,statusBE2,statusBE3,statusBE4,TOP</variables>
        <solverOrder>3</solverOrder>
        <topEventID>TOP</topEventID>>
        <map var='statusBE1'>BE1</map>
        <map var='statusBE2'>BE2</map>
        <map var='statusBE3'>BE3</map>
        <map var='statusBE4'>BE4</map>
      </ExternalModel>
    </Models>
    ...
  </Models>
\end{lstlisting}

If $solverOrder=1$ then: $P(TE) = P(MCS1)*P(MCS2)*P(MCS3)$.  
If $solverOrder=2$ then: $P(TE) = P(MCS1)*P(MCS2)*P(MCS3) - P(MCS1 MCS2) - P(MCS1 MCS3) - P(MCS2 MCS3)$.  
If $solverOrder=3$ then: $P(TE) = P(MCS1)*P(MCS2)*P(MCS3) - P(MCS1 MCS2) - P(MCS1 MCS3) - P(MCS2 MCS3) + P(MCS1 MCS2 MCS3)$

\subsection{MCSSolver model reference tests}
The following is the provided analytic test:
\begin{itemize}
	\item test\_MCSSolver.xml
\end{itemize}





    \section*{Document Version Information}
    This document has been compiled using the following version of the plug-in git repository:
    \newline
    \input{version.tex}

    % ---------------------------------------------------------------------- %
    % References
    %
    \clearpage
    % If hyperref is included, then \phantomsection is already defined.
    % If not, we need to define it.
    \providecommand*{\phantomsection}{}
    \phantomsection
    \addcontentsline{toc}{section}{References}
    \bibliographystyle{ieeetr}
    \bibliography{user_manual}


    % ---------------------------------------------------------------------- %

\end{document}
